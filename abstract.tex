\abstract
The discipline of Computer Theory is one of the bases of support
of the area of computation, for this reason, of fundamental
importance in the formation of the students of the area.
Its content has a high level of abstraction and requires a
big effort from students and educators in the teaching and
learning process.
The students' perception that they are not understanding the
concepts and the demotivation of these are the main challenges
in this process.
In turn, the use of active learning methodologies such as
the \ac{PBL} approach has been presented as an interesting
alternative to reduce problems related to student motivation
and learning.
In this sense the objective of this master's thesis is
to describe the implementation of \ac{PBL} in classes
of Computer Theory in the course of System
Information in \ac{UFBA}.
For the evaluation of the results and validation
of the hypotheses, this work carried out a descriptive
research in the form of opinion research, where the
students were invited to answer questionnaires with
profile characterization, Likert scale affirmations
and open space regarding the adopted approach.
The results of this study suggest that the use
of \ac{PBL} is promising as an alternative to theoretical
subjects of Computation, as in the discipline in which this
study was applied, especially with respect to
students' perception about motivation and learning.
\begin{keywords}
\ac{PBL}, Computer Theory, pedagogical approach,
pedagogical methodology.
\end{keywords}
