\newcommand{\NewItem}[1]{%
$\bullet$ #1
}

\newcommand{\AfirmacaoSimNao}[1]{\item{#1}
\begin{tabular}{ll}
\NewItem{Não} & \NewItem{Sim}
\end{tabular}
                                }

\newcommand{\AfirmacaoLikertBase}[1]{%
\begin{tabular}{lll}
\NewItem{Discordo} & \NewItem{Discordo parcialmente} & \NewItem{Indiferente}\\
\NewItem{Concordo parcialmente} & \NewItem{Concordo} &
{\ifthenelse{\equal{#1}{}}{}{\NewItem{#1}}}
\end{tabular}
                                }

\newcommand{\AfirmacaoLikert}[1]{\item{#1}\\
				\AfirmacaoLikertBase{}
}

\newcommand{\AfirmacaoLikertA}[1]{\item{#1}\\
				\AfirmacaoLikertBase{Não tive trabalho neste período}
}

% form problemas
\newcommand{\LikertPA}{A construção da solução envolveu
a elaboração, análise e exposição clara e racional de argumentos.}
\newcommand{\LikertPB}{Foram identificadas ideias alternativas para uma mesma situação (ou seja, partes ou etapas individuais do problema).}
\newcommand{\LikertPC}{Foram utilizadas, ao menos, uma das referências bibliográficas indicadas no texto do problema para estudo extraclasse.}
\newcommand{\LikertPD}{Foi necessário recorrer a livros, vídeos, apostilas ou outros recursos não indicados no problema para chegar à solução.}
\newcommand{\LikertPE}{O problema lhe deixou motivado para descobrir uma possível solução.}
\newcommand{\LikertPF}{Você acredita que cumpriu com os objetivos de aprendizagem do problema.}
\newcommand{\LikertPG}{Você teve que aprender novos conhecimentos (conceitos, habilidades ou atitudes) para chegar à solução do problema.}
\newcommand{\LikertPH}{As situações abordadas pelo problema se aproximam de um cenário real e atual.}
\newcommand{\LikertPI}{O conhecimento aprendido é útil para um profissional da área de Computação.}
\newcommand{\LikertPJ}{O texto do problema estava claro e bem escrito.}
\newcommand{\LikertPK}{Havia no texto informações suficientes para direcionar a investigação.}
\newcommand{\LikertPL}{O tempo disponibilizado para o desenvolvimento da solução foi adequado.}
\newcommand{\LikertPM}{O problema estimulou o trabalho em grupo.}
\newcommand{\LikertPN}{Você utilizou conhecimentos prévios (de um problema anterior ou mesmo de outro componente curricular) para chegar à solução do problema.}
\newcommand{\LikertPO}{O problema exigiu o estudo individual de seus conteúdos fora das sessões tutoriais.}
\newcommand{\LikertPP}{O método PBL me motiva para ir em busca do meu próprio conhecimento.}
\newcommand{\LikertPQ}{As sessões tutoriais contribuem para o processo de resolução do problema.}
\newcommand{\LikertPR}{Os participantes geralmente apresentam uma relação interpessoal boa e produtiva.}
\newcommand{\LikertPS}{A quantidade de pessoas em cada Grupo Tutorial é apropriada.}
\newcommand{\LikertPT}{Os problemas são úteis no processo de ensino-aprendizagem.}
\newcommand{\LikertPU}{Os tutores contribuem, quando necessário, para a evolução das sessões tutoriais.}
\newcommand{\LikertPV}{Os tutores deixam claros os critérios de avaliação do produto.}
\newcommand{\LikertPW}{Os tutores dão \textit{feedbacks} sobre o desempenho do grupo tutorial a cada sessão.}
\newcommand{\LikertPX}{Eu gosto do método PBL.}

% form concluintes
\newcommand{\LikertCA}{A metodologia da disciplina me agradou.}
\newcommand{\LikertCB}{Gosto que a minha presença em sala faça parte da avaliação.}
\newcommand{\LikertCC}{Gosto de ser avaliado pela participação durante as aulas.}
\newcommand{\LikertCD}{Gosto de ter várias avaliações.}
\newcommand{\LikertCE}{Gosto de avaliações em equipe.}
\newcommand{\LikertCF}{Acredito que os tutores me motivaram suficientemente.}
\newcommand{\LikertCG}{Consigo conciliar a disciplina com o trabalho profissional.}
\newcommand{\LikertCH}{Eu entendi a metodologia PBL.}
\newcommand{\LikertCI}{Falar em público é um grande problema para mim.}
\newcommand{\LikertCJ}{Eu gostaria de experimentar outras abordagens de ensino
durante o meu curso além das aulas tradicionais.}
\newcommand{\LikertCK}{A metodologia da disciplina me ajudou nas avaliações
escritas (provas).}
\newcommand{\LikertCL}{A metodologia da disciplina me ajudou a entender melhor
a maioria dos conceitos estudados na disciplina.}
\newcommand{\LikertCM}{A carga horária entre sessões tutorias e
aulas expositivas foi suficientemente balanceada.}
\newcommand{\LikertCN}{Tenho interesse em cursar outras
disciplinas que possam utilizar esta abordagem.}

% form desistentes
\newcommand{\LikertDA}{Desisti da disciplina porque a metodologia não me agradou.}
\newcommand{\LikertDB}{Desisti da disciplina porque não gosto que a minha
presença em sala faça parte da avaliação.}
\newcommand{\LikertDC}{Desisti da disciplina porque não gosto de ser avaliado
pela participação durante as aulas.}
\newcommand{\LikertDD}{Desisti da disciplina porque não gosto de ter várias avaliações.}
\newcommand{\LikertDE}{Desisti da disciplina porque não gosto de avaliações em equipe.}
\newcommand{\LikertDF}{Desisti da disciplina porque acredito que os tutores
não me motivaram suficientemente.}
\newcommand{\LikertDG}{Desisti da disciplina porque não consigo conciliar
a disciplina com o trabalho profissional.}
\newcommand{\LikertDGa}{Desisti da disciplina porque tive
questões pessoais (exceto trabalho).}
\newcommand{\LikertDH}{Eu entendi a metodologia PBL.}
\newcommand{\LikertDI}{Falar em público é um grande problema para mim.}
\newcommand{\LikertDJ}{Eu gostaria de experimentar outras abordagens de ensino
durante o meu curso além das aulas tradicionais.}
\newcommand{\LikertDO}{Eu gostaria de experimentar a metodologia
PBL, mas com mais carga horária de aulas expositivas.}
\newcommand{\LikertDOa}{Tenho interesse em cursar em breve a disciplina com esta abordagem.}
