% Agradecimentos
\acknowledgements
Eu tenho que agradecer a Deus por todas as oportunidades
que tive durante a vida.

No trabalho tenho que agradecer ao Natanael Soares,
ao Evaldo Machado, a Mariana Taís e ao Luís Soares que
foram gestores imediatos na Petróleo Brasileiro aos quais
eu estive subordinado durante esse tempo de estudo e
construção deste trabalho.
Os gestores flexibilizaram para que eu pudesse
assistir aulas do mestrado e apresentasse
um trabalho no Congresso.

O meu co-orientador, David, agradeço especialmente
pois foi a partir da entrada dele que percebemos
a oportunidade de fazer este trabalho utilizando a
metodologia baseada em problemas.

A minha orientadora, Laís, que ao longo deste tempo descobri
uma amizade, de extrema compreensão e confiança
que me permitiu ter segurança na caminhada, mesmo no
começo em que eu não tinha muita certeza do que iria
fazer.
Foi a minha orientadora que descobriu essa oportunidade
de eu trabalhar mais envolvido com ensino e aprendizagem,
uma oportunidade que eu espero levar para
além deste trabalho.

Ao meu amigo, Daniel Cason, agradeço pela ajuda na
revisão do artigo, na revisão da apresentação
para o Congresso e por toda ajuda em São Paulo.
Não menos importante tenho que agradecer ao Daniel
pelas tantas gargalhadas que este meu amigo me
proporcionou mesmo quando estávamos conversando
(sério) de assuntos como a política brasileira.
Pode rir?

Agradeço para a Adriana Ribeiro pelas parcerias
de estudo, pelas revisões em textos e também
por ter sido uma oportunidade de amizade durante
este período de mestrado.

Ao amigo Thiago Sena agradeço pela ajuda
nas necessidades de revisão dos textos em
inglês.

Aos revisores anônimos dos trabalhos submetidos,
agradeço pelas contribuições que ficam
refletidas neste trabalho.

Ao amigo, Gabriel Luiz, e as amigas,
Priscilla Crhistina, Karine, Katyane
e Nailan Janaína, o meu agradecimento é
por estarem presentes em minha vida,
pelo carinho que demonstram ter
por mim e por terem muitas vezes dado
oportunidades para as minhas complexas
reflexões.
Outras pessoas estiveram próximas de mim e
foram úteis também para este meu passo e para
a minha caminhada como um todo, mesmo não citando
nominalmente, também agradeço para elas.

Agradeço para a minha família.
Minha irmã, Leidiane, que sempre foi uma das minhas
maiores incentivadoras na vida e que sempre
demonstrou extrema felicidade pelas oportunidades
que existiram em minha vida.
Meu pai, Roque, por estar próximo de mim,
mesmo quando esteve fisicamente distante.
Minha mãe, Julinda, agradeço pelas renúncias que
fez ao longo da vida para que eu conseguisse realizar
alguns dos meus sonhos.
