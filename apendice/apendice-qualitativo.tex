\newcommand{\discursivaA}{%
\item{\textit{\textbf{Espaço para exposição livre sobre o problema:}}}}

\newcommand{\discursivaB}{%
\item{\textit{\textbf{Espaço para exposição livre sobre percepção do método PBL:}}}}

\newcommand{\discursivaC}{%
\item{\textit{\textbf{Espaço para exposição livre sobre a disciplina:}}}}

\newcommand{\discursivaProblema}[4]{%
\item{\textbf{Problema #1 - #2}}
\begin{itemize}
		{
		\discursivaA
		\begin{enumerate}
			#3
		\end{enumerate}
		}
		{
		\discursivaB
		\begin{enumerate}
			#4
		\end{enumerate}
		}
\end{itemize}}

\newcommand{\discursivaProblemaA}[3]{%
\item{\textbf{Problema #1 - #2}}
\begin{itemize}
		{
		\discursivaA
		\begin{enumerate}
			#3
		\end{enumerate}
		}
\end{itemize}}

\newcommand{\discursivaDisciplina}[3]{%
\item{\textbf{Estudantes #1}}
\begin{itemize}
		{
		\discursivaB
		\begin{enumerate}
			#2
		\end{enumerate}
		}
		{
		\discursivaC
		\begin{enumerate}
			#3
		\end{enumerate}
		}
\end{itemize}}

\xchapter{Respostas discursivas}{}
\acresetall
\label{apendice-qualitativo}

\begin{itemize}
\item{Semestre 2016.1}
	\begin{itemize}
	\discursivaProblema{1}{\ProblemaA}%
	{
	\item{O problema escolhido é muito complexo para
	uma tarefa inicial do curso e que tem dependencia real com a musica.}
	\item{Talvez, por ser o primeiro problema, o desempenho na
	maioria dos grupos foi regular.}
	}%
	{
	\item{Fazer um trabalho com uma equipe ruim foi a para mais complicada.}
	\item{A metodologia é interessante porém, como estamos tentando
	fazer, necessita de aprimoramentos. A alternância entre aula teórica
	e sessões tutorial me pareceu mais eficaz do que somente sessões
	tutorial.
	No mais a metodologia parece ser bem promissora se bem trabalhada.}
	\item{Gostei do método de ensino.
	Já sabia como funcionava, mas foi minha primeira participação em um.
	As reuniões foram bem úteis e o ambiente foi muito colaborativo.}
	\item{O PBL é muito interessante, pois, estimula o trabalho
	colaborativo que por consequência leva a soluções mais rápidas.}
	}
	
	\discursivaProblema{2}{\ProblemaB}%
	{
	\item{O problema deve ter mais regras como preços da
	maquina ou valores aceitos para amarrar o problema ao conteúdo.}
	\item{Achei que a descrição do problema não ficou claro,
	um exemplo bem claro é que fala somente sobre o troco aleatório
	e não sobre o troco quando a entrada é acima do valor do produto.}
	}%
	{
	\item{o método está de acordo, mas o \textit{feedback} dos trabalhos está insuficiente.}
	\item{Essa segunda atividade foi melhor que a primeira, acho que a
	turma está conseguindo entender e discutir melhor.}
	}

	\discursivaProblema{3}{\ProblemaC}%
	{
	\item{o tempo e dificuldade estavam bem equilibrado.
	Poderia se dificultar um pouco se houvesse mais tempo ou uma equipe. Se a proporção fosse independente do carro.}
	}
	{
	\item{Os alunos têm participado mais das aulas.}
	}

	\discursivaProblemaA{4}{\ProblemaD}%
	{
	\item{O problema foi útil para aprendizagem, no entanto é
	necessário realizar a construção de outras máquinas antes
	de pedir uma de forma avaliativa.
	A máquina do problema não era difícil, porém, antes de
	fazer uma máquina que abrange tantas possibilidades,
	seria mais produtivo construir algumas mais simples em
	sala, para que todos tivessem a oportunidade de adquirir
	o conhecimento mais básico, além da teoria passada nas
	aulas expositivas.}
	}

	\discursivaProblema{5}{\ProblemaE}%
	{
	\item{O problema da parada é muito complicado para um
	período de tempo curto, além de ter poucos encontros.
	Em vez de fazer encontros para cada problema e fixar
	na resolução individual e com encontros programados,
	poderia fazer uma apresentação de alguns problemas
	maiores e ao longo do semestre encontros mais
	distantes para a resolução do problema.}
	}
	{
	\item{O PBL é muito bom quando bem aplicado.
	Acho que a quantidade de problemas poderia ser menor
	e a dificuldade dos problemas poderiam ser maiores,
	ex: passaria um problema de autômato finito até autômato
	com pilha, com mais encontros, ou até maquina de turing...
	Para que assim a resolução do problema envolva diversos
	assuntos, e que depois venha o conteúdo fazendo o
	fechamento do método PBL.
	E a mesma coisa poderia ser utilizada para os
	outros assuntos.}
	}

	\discursivaDisciplina{concluintes}%
	{
	\item{Acho que seria muito útil a criação de máquinas em sala,
	não apenas sua exposição e explicação de como funciona.}
	\item{O método PBL reque mais esforço do aluno e um trabalho
	em conjunto de todos os participantes.
	O ideal para usar o método PBL é que os participantes estejam
	motivados para resolver o problema e achar este motivador é
	o que pode fazer a diferença.}
	}
	{
	\item{Acho que mais aulas expositivas seria uma boa ideia.
	Foram muitas atividades avaliativas, concordo em número de
	atividades, mas não que todas sejam pontuadas.
	Para última prova, como disse em sala, aconselho que seja
	menos conteúdo ou que tenha um peso maior.}
	\item{A disciplina foi ótima.}
	}
	
	\discursivaDisciplina{desistentes}%
	{
	\item{Achei interessante o PBL, mas achei que deveria ter
	sempre antes de expor um problema, fornecer uma base/start
	para que o estudante não perca muito tempo buscando algo sem
	sentido à ser aplicado ao problema e nos debates em sala,
	pois facilmente acaba fugindo do assunto, com isso a expansão
	do cenário.
	Sobre este último ponto, estas atividades de resolução de
	problemas, costumam demandar um longo periodo de análise.
	Sei sobre isto na prática, porque 2h ou 3h para mim não é
	nada quando se trata de problemas para se resolver.
	Como muitos estudantes possuem outras matérias na maioria
	das vezes, meio que é preciso ter algum resultado no tempo
	restrito de estudo, isto principalmente para quem trabalha,
	que é no caso do turno noturno.}
	}%
	{
	\item{Gosto da disciplina, mas acho que um caso geral que é
	normal de acontecer é de não saber o perfil da turma no momento
	de aplicar uma metodologia, as cobranças dia/aula foi o mais
	complicado mesmo, pois era algo que demandava de você ter sempre
	algo produzido, o que poucas vezes acontecia, porque como eu
	citei no quadro anterior, as 3h de dedicação da disciplina não
	adiantava, porque eu não conseguia fazer uma abordagem precisa,
	ficando bastante perdido, ainda mais com mais disciplinas.
	A questão não é dizer que uma disciplina é mais importante que
	a outra em termo de aprendizado, mas o curso atual de sistemas
	de informação, obriga você a priorizar as disciplinas,
	infelizmente, devido ao fluxograma e suas matérias
	pré-requisitos.
	Por fim, existe um contexto a cada nova turma.
	Acho que uma pesquisa inicial de perfil de turma, no inicio
	da disciplina, ajude a melhor conciliar a metodologia, mesmo
	a UFBA demonstrando o que muitos estudantes acham que
	``ela não está ligando para quem trabalha''.}
	}
	\end{itemize}

\item{Semestre 2017.1}
	\begin{itemize}
	\discursivaProblema{1}{\ProblemaG}%
	{
	\item{Achei esse problema muito extenso, gostaria de um
	\textit{feedback} para saber se realmente fiz corretamente,
	ainda mais que foi individual, não sei se conseguir
	entender realmente.}
	\item{O problema trazia pessoas que tinham acabado de
	ser resgatadas de um acidente, mas ao invés de passar
	mensagem às suas famílias, o que seria o mais lógico
	racional e irracionalmente, eles focam em enviar mensagem
	criptografada pro trabalho de um sistema criado.}
	}
	{
	\item{É interessante que as equipes sejam formadas e
	mantidas ao longo da disciplina e que recomendem esse
	grupo fixo e que alguém pegue telefone nome e email de
	todos, para facilitar posteriormente.}
	\item{Houve bastante dificuldade de entendimento
	do que era requerido.
	O problema deixou as possibilidades de resposta tão
	abertas que o grupo se perdeu.
	Nenhuma outra disciplina utiliza esse método,
	então é necessária uma maior atenção aos alunos na
	interpretação do que se requer.}
	}

        \discursivaProblema{2}{\ProblemaB}%
	{
	\item{Não ficou claro para mim o objetivo em relação
	com o escopo da disciplina}
	\item{Creio que apenas este problema em si não despertou
	a motivação necessária e tempo hábil, levando em consideração
	a exigência de outras matérias e período de primeiras
	avaliações para a conclusão de um bom trabalho.
	A construção da máquina de estados é maravilhosa,
	mas ainda um pouco confusa quando requer uma grande
	quantidade de estados.}
	\item{Tive mais facilidade com esse do que o anterior}
	\item{Foi bastante chocante a diferença de nível pro problema anterior.
	Isso dificultou o aprendizado.}
	\item{O problema foi bom, só que em alguns momentos as
	sessões se tornam cansativas e as pessoas ficam dispersas.}
	\item{Talvez a forma com que o problema que iremos submeter,
	seja a deficiência do método de aprendizado.
	Nem todos grupos chegam a uma solução que seja clara para
	todos os componentes.
	Existem componentes muito mais avançados do que outros na
	questão do desenvolvimento da solução e essa diferença não
	deixa claro o ``produto bruto'' das reuniões.
	As respostas do grupo não ficam claras de forma geral.}
	}
	{
	\item{É um método muito bom para atividades e desenvolvimento em grupo.
	Para submeter atividades individuais só favorece aqueles que tem
	conhecimento mais avançado e conseguem sozinhos resolver o problema
	desde o inicio, pois das reuniões alguns alunos não extraem
	as informações, tornando as reuniões de certa forma, indiferentes.}
	\item{O método é enriquecedor sem o processo massante de alguns
	aprendizados, a aula me desperta e faz ter \textit{insights} para a
	resolução de alguns problemas que antes não conseguiria identificar.}
	\item{Ficamos um pouco perdidos sobre o que fazer,
	o que era/como fazer autômato.
	Faltou uma aula teórica mais demonstrativa antes, de como fazer
	funcionar um autômato.}
	\item{Acredito que o quadro não fique responsável apenas
	para uma pessoa, faz com que as outras não se sintam
	intrusas do grupo e queiram participar de fato.}
	\item{Outro ponto importante é que as vezes não existe a
	cooperação por parte dos colegas em relação as
	funções (mesa, quadro...), sendo assim muitas vezes uma pessoa
	repete a função várias vezes e outros pegam o ``bonde'',
	e não se comprometem com a disciplina, dizendo que é por
	causa do trabalho, família, enfim de modo geral não
	se compromete com a matéria e com a equipe que ele
	faz parte. }
	}
        \discursivaProblema{3}{\ProblemaC}%
	{
	\item{A grande dificuldade talvez esteja em já pensar
	em soluções sem ter o mínimo de conhecimento sobre como resolver.
	Isso dificulta a ação de similar o que o problema quer passar,
	só sendo possível nas últimas aulas, quando o limite de tempo
	já está prestes a estourar.}
	\item{Achei que o problema teve uma dificuldade média de solução,
	mas foi de difícil interpretação.
	O texto continha muitas metáforas desnecessárias
	(balança, via secundária para desvio) que desviavam
	a atenção do problema principal e atrapalhavam sua solução.
	A dificuldade média se deve a não ter sido fácil para a
	maioria entender como as proporções 3:5 e 2:5 seriam atendidas,
	mas depois que o segredo foi desvendado,
	vimos que o autômato era bastante simples.}
	\item{Foi o problema mais complicado, mas também o que se aprende mais.}
	}
	{
	\item{OK.}
	\item{Funcionou muito bem para o meu aprendizado,
	pois senti mais vontade de participar das aulas e me senti
	em muitos momentos desafiado a expor claramente meu pensamento, 
	a escutar com paciência a ideia do outro e a solucionar o problema.
	Porém, não vi o mesmo acontecer com outras pessoas.
	Algumas delas tinham sempre uma postura passiva ou
	desinteressada, o que em alguns momentos era
	frustrante, mesmo para mim.
	Fico a refletir se isso aconteceria por esse não ser
	o melhor método de aprendizagem para essa pessoas,
	ou por razões pessoais como falta de motivação ou
	interesse com o curso ou disciplina.
	Talvez a metodologia tenha até ajudado essas pessoas
	de alguma forma, mas quanto a isso não posso afirmar.}
	}
        \discursivaProblema{4}{\ProblemaD}%
	{
	\item{É um problema real, então foi interessante lidar com ele.}
	\item{Nesse problema específico, tive dificuldades em conseguir
	entender o que era pedido, visto que não tinha nenhum conhecimento
	prévio sobre MT, o que criou uma barreira para o aprendizado,
	ao invés de incentivar a aprendizagem.}
	\item{Não sobre o problema, mas sobre a formação da equipe.
	Teoricamente teriam as mesmas pessoas, portanto, mesma quantidade
	de membros na equipe.
	Mas não sei se foi impressão minha mas aparentava ter menos pessoas
	nas sessões tutoriais.
	Nesse problema minha equipe teve um desempenho melhor no
	entendimento e resolução do problema.
	Acredito que o fato de ter menos gente influenciou em
	as discussões acontecerem de forma mais otimizada,
	já que todos participavam.}
	}
	{
	\item{É bom para aprender entre colegas.}
	}

        \discursivaProblema{5}{\ProblemaE}%
	{
	\item{Difícil, mas interessante.}
	}
	{
	\item{Como nunca havia participado desse método,
	sinto que fiquei pra trás por não acompanhar os
	estudos atentamente desde o início.}
	}
        
	\discursivaProblema{6}{\ProblemaF}%
	{
	\item{N=NP? Pergunta de 1 milhão de dólares.
	Quando resolvida irá solucionar o problema do
	caixeiro viajante.
	Como o assunto é puramente matemático não invocou
	muita vontade de resolver.}
	\item{Tive um pouco de dificuldade para entender.}
	}
	{
	\item{Não me adaptei ao método.}
	\item{Não consegui chegar a uma reposta que eu tivesse certeza,
	mesmo olhando todos os materiais disponíveis e extras,
	o método PBL traz isso de procurarmos as respostas,
	mas se eu não consigo encontrar acaba me desmotivandoe
	nas sessões tutoriais mesmo com dicas,
	eu me sinto um pouco perdida no conteúdo.}
	}

	\discursivaDisciplina{concluintes}%
	{
	\item{Eu não me adequei ao método.
	Não consegui absorvê-lo.
	Por isso perdi na disciplina. :(
	É um método interessante e inteligente, mas não é para mim.}
	\item{Foi interessante e desafiador ter contato com um
	método tão diferente do usual.
	A maioria das sessões teve como efeito me motivar para
	solucionar os problemas e assim aprender mais profundamente
	o conteúdo da disciplina, porém, tal efeito não durou
	até o final do semestre, e com certeza não se estendeu
	a todos os estudantes matriculadas na disciplina.
	Nas rodas de discussão sempre houveram alguns alunos
	desinteressados e a cada aula o número de desinteressados
	aumentava, o que tornava difícil proegredir com as discussões.
	Além disso, o cansaço causado pela faculdade e o
	aumento da complexidade e dificuldade dos desafios
	contribuiram para que as sessões perdessem
	sua produtividade.}
	\item{A metodologia é perfeita pra aprendizagem na matéria.}
	}
	{
	\item{Disciplina que não entendi muito bem.
	As aulas práticas me deram uma luz, sem elas,
	minhas notas seriam mais baixas do que foram.
	Consegui aprender apenas quando estava no grupo.}
	\item{A pessoa fica meio perdida se faltar alguma
	aula de sessão tutorial, comprometendo as outras.
	Percebi colegas se atrapalhando muito
	ao fim da matéria.
	A presença se torna algo muito importante mais do
	que nos outros cursos pois se a pessoa faltar num
	dia perderia o assunto e no outro perderia a atividade.
	Talvez o assunto devesse ser revisado no dia da
	atividade em grupo.}
	\item{Foco na parte final do conteúdo, uma vez que
	é do de mais difícil compreensão}
	}

	\end{itemize}

\end{itemize}
