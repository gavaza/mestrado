\tituloProblema{1}{Problema}
Os alunos da disciplina de Introdução as Linguagens Formais e Teoria da Computação normalmente
utilizam os computadores disponíveis em um dos laboratórios de informática da UFBA para
realização das atividades da disciplina.

Acontece que em um dia, os computadores de um destes laboratórios
de informática da UFBA começaram a apresentar um comportamento inesperado.

A primeira pessoa a relatar esse problema foi a Drissa.
Ela sentou de frente ao computador, como normalmente faz, mas ao digitar o próprio nome notou
que na tela estava escrito Qaxllo.
Ela tomou um susto sem entender o que estava acontecendo, então, como é amiga do Lucas, resolveu falar
com ele.
O Lucas estava sentado no computador ao lado, então ele tentou digitar Drissa no computador
que estava e percebeu que na tela apareceu Prsaaq.
O Lucas também tentou escrever o próprio nome e viu escrito na tela Dxwqa.

Então, Drissa e Lucas resolveram falar do problema para toda a turma, uma vez que não entenderam o
que estava acontecendo.
Em pouco tempo foi percebido que todos os computadores daquele laboratório estavam afetados
pelo problema.

A turma conversou rapidamente e percebeu que talvez este problema não seja muito difícil de resolver, então,
não precisariam esperar pelo atendimento do STI da UFBA, desta forma, resolveram fazer uma reunião para
discutir este problema e encontrar uma solução.

Quando todos já estavam saindo do laboratório para discutir o problema, o Angelmário, bastante atencioso,
alertou que não poderiam sair sem antes coletar mais informações para resolução do problema.
Então, assim foi feito, a Deuana e o Rodrigo foram os responsáveis por coletar o máximo de informações
possíveis.
Eles digitaram os nomes de todos os alunos da turma em cada um dos computadores e fizeram uma tabela de
correspondência com o que era apresentado na tela.
Enquanto isso a Taiane disse que iria verificar se todos os cabos estavam
corretamente encaixados, ela constatou que todos os cabos estavam encaixados nos lugares certos.

\tituloProblema{2}{Produto}
Uma foto que demonstre que o problema está solucionado, bem como, um relatório que descreve os passos
para identificação e solução do problema, além de anexar todos os arquivos que julgar necessário.

\tituloProblema{3}{Cronograma}

1 sessão tutorial.

\tituloProblema{4}{Recursos para aprendizagem}
Não foram indicados recursos adicionais de aprendizagem.
