\tituloProblema{1}{Problema}
Os estudantes da disciplina de Introdução as Linguagens Formais e Teoria da Computação da UFBA
souberam na primeira aula da disciplina que será utilizada uma metodologia de ensino baseada
em problemas, então, eles resolveram pesquisar um pouco sobre a metodologia.

Os estudantes se reuniram para conversar sobre as coisas que tinham pesquisado.
Um dos estudantes disse que saber inferir informações implícitas em problemas é muito relevante
para que consigam obter bons resultados.
Outro estudante resolveu contar uma história e os demais ficaram atentos nos mínimos detalhes.

O estudante que conta a história diz que tudo começou na festa de aniversário da Alice.
Não, não a Alice do País das Maravilhas, mas uma amiga dele chamada de Alice:\\

--- Que tal preparar-nos umas tortas saborosas?---perguntou o Rei de Copas à Rainha de Copas num dia fresco de verão.

--- Fazer as tortas sem pimenta?---perguntou a Rainha.

--- Pimenta!---exclamou o Rei, incrédulo---Quer dizer que você usa pimenta em suas tortas?

--- Não muita---respondeu a Rainha.

--- E suponho que ela tem sido roubada!

--- É claro!---disse a Rainha---Encontre a pimenta e, quando descobrir quem a roubou, corte-lhe...

--- Vamos, vamos!---disse o Rei.

--- Bem, a pimenta tinha que ser encontrada, é claro. Agora, como todos vocês sabem, as pessoas que roubam pimenta nunca dizem a verdade.

--- O quê?!---disse Alice (não a Alice do País das Maravilhas, mas a Alice dessa festa)---Nunca ouvi falar disso antes!

--- Não ouviu?---perguntei-lhe, com falsa surpresa.

--- É claro que não! E tem mais, não acredito que ninguém mais tenha ouvido! Algum de vocês ouviu falar disso antes?

Todas as crianças abanaram a cabeça negativamente.

--- Bem,---disse eu---para fins desta história, vamos presumir que as pessoas que roubam pimenta nunca dizem a verdade.

--- Está bem---disse Alice, meio relutante.

--- Então, continuando a história, o suspeito mais óbvio era a cozinheira da Duquesa.

No julgamento, ela fez apenas uma declaração:

--- Eu sei quem roubou a pimenta!

--- Supondo que as pessoas que roubam a pi\-men\-ta sempre mentem, a cozinheira é culpada ou inocente?\\

PORTANTO, QUEM ROUBOU A PIMENTA?\\

Bem, os suspeitos seguintes do Rei foram a Lebre de Março, o Chapeleiro Louco e o Leirão. Os soldados foram mandados às casas deles, mas nenhuma pimenta foi encontrada. Mesmo assim, eles poderiam estar escondendo-a em algum lugar, de modo que foram detidos, com base nos princípios gerais.

No julgamento, a Lebre de Março afirmou que o Chapeleiro era inocente e o Chapeleiro afirmou que o Leirão era inocente. O Leirão resmungou uma declaração qualquer enquanto dormia, mas ela não foi registrada.

Como se constatou, nenhum inocente fizera uma afirmação falsa, e (como estamos lembrados) as pessoas que roubam pimenta nunca fazem afirmações verdadeiras. Além disso, a pimenta foi roubada por apenas uma criatura. Qual dos três é o culpado, se é que foi um deles?\\

ENTÃO, QUEM ROUBOU A PIMENTA?\\

--- Ora, ora, esse é realmente um caso difícil---disse o Rei.

Os suspeitos seguintes, curiosamente, foram o Grifo, a Falsa Tartaruga e a Lagosta.

No julgamento, o Grifo afirmou que a Falsa Tartaruga era inocente, e a Falsa Tartaruga disse que a Lagosta era culpada.
Mais uma vez, nenhum inocente mentiu e nenhum culpado disse a verdade.

Quem roubou a pimenta?

\tituloProblema{2}{Produto}
Um relatório no modelo de artigos da SBC para discutir se a cozinheira da história é culpada ou
inocente, se a Lebre de Março, o Chapeleiro Louco, o Leirão, o Grifo, a Falsa Tartaruga ou Lagosta
roubaram a pimenta.
É necessário detalhar cada inferência utilizada, por exemplo, ao inocentar ou acusar alguém, mostrar
quais as inferências e fatos utilizados para tal conclusão.

\tituloProblema{3}{Cronograma}
1 sessão tutorial.

\tituloProblema{4}{Recursos para aprendizagem}
Não foram indicados recursos adicionais de aprendizagem.

\tituloProblema{$*$}{Referências}
Este problema é baseado no texto do Raymond Smullyan extraído do livro ``Alice no País dos Enigmas''.
