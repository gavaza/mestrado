\tituloProblema{1}{Problema}
Um certo colecionador de arte dispõe de partituras e gravações de
músicas, as quais ele precisa catalogar e processar mediante computador.
O colecionador em questão é muito exigente com o tratamento do seu material e
garante que para cada música na sua coleção, a partitura corresponde exatamente com
a gravação correspondente.

Algumas das partituras que o colecionador possui são para apenas um
instrumento.
Outras são para vários instrumentos.
Cada partitura para um instrumento pode ser vista como uma sequência de
símbolos musicais.
No caso das partituras para vários instrumentos, elas podem ser vistas
como sequências, mas de símbolos compostos: cada elemento da sequência é formado por um número finito
de símbolos musicais.
Cada música pode ter um número arbitrário de símbolos.

Veja a seguir exemplos de partituras.

Um instrumento:\\
\begin{abc}[name=um]
X:1
M:2/4
K:G
[L:1/4] G/A/ | B d | d B | c c |
\end{abc}

Note que cada nota pode ser vista de forma independente como um símbolo (escrito em um pentagrama). No exemplo a seguir, correspondente a uma partitura para três instrumentos, cada um deles possui uma representação similar à vista acima: Cada instrumento possui o seu próprio pentagrama e em cada um deles, os símbolos são superpostos em cada tempo.

Três instrumentos:\\
\begin{abc}[name=tres]
X: 1
M: 3/4
L: 1/4
V:1 name="Flauta" clef=treble
V:2 name="Baixo" clef=treble
V:3 name="Piano" clef=treble
K:Gm
%
[V:1] |: b d' b/g/ | ^f d a
[V:2] |: G B G | d ^f d 
[V:3] |: G/A/ b  B | d A c
\end{abc}

No caso das partituras para vários instrumentos, elas
também podem ser vistas como sequências, mas de símbolos
compostos: cada elemento da sequência é formado
por um número finito de símbolos musicais, sendo um
em cada instrumento.

Apesar de bastante exigente, este colecionador não entende muito bem sobre linguagens,
mas ele gostaria de ter um sistema automatizado que pudesse criar novas composições
a partir das que já possui.

O colecionador ficou sabendo que os alunos desta turma de Introdução as Linguagens Formais
e Teoria da Computação da UFBA estão estudando sobre linguagens formais, então decidiu pedir
ajuda para que eles idealizassem o sistema.

Ele disponibilizou toda a coleção e listou algumas restrições que o sistema
deve atender para que ele fique satisfeito.

\begin{enumerate}
\item O colecionador quer que ao menos duas músicas sejam utilizadas como base para gerar uma nova música,
onde apenas símbolos existentes nas músicas utilizadas como base podem ser utilizadas para uma nova composição;
\item O colecionador gostaria que o sistema fosse capaz de criar músicas para uma quantidade variada de instrumentos;
\item Em ao menos $10\%$ e no máximo $20\%$ das superposições dos símbolos, isto é, quando a nova composição for para mais de um instrumento,
o colecionador gostaria de ouvir a mesma nota em todos os instrumentos da nova composição;
\item O colecionador quer que pequenos trechos das músicas utilizadas como base estejam
presentes nas novas composições;
\item O colecionador quer ter certeza que as novas composições contemplam o máximo
de operações existentes em linguagens formais estudadas pelos alunos, então pediu para
que eles assim fizessem, além de incluir detalhadamente ao menos um exemplo de cada operação
no relatório.
\end{enumerate}

\tituloProblema{2}{Produto}
Uma nova coleção de músicas com no mínimo 10 composições geradas conforme as exigências do colecionador
para demonstrar que é capaz de desenvolver o sistema idealizado, bem como, um relatório no modelo de
artigos da SBC que descreva com o máximo de detalhes a idealização de funcionamento do sistema, onde deverá
constar quais as operações executadas pelo sistema para criação das músicas e incluir ao menos
um exemplo de cada uma das operações sob linguagens formais utilizadas pelo sistema idealizado.

\tituloProblema{3}{Cronograma}
4 sessões tutoriais e 1 aula expositiva.

\tituloProblema{4}{Recursos para aprendizagem}
RAMOS, M. V. M.; JOSÉ NETO, J.; VEGA, I. S. \textbf{Linguagens Formais: Teoria, Modelagem e Implementação}. Editora Bookman, 2009.\\

\noindent
MENEZES, Paulo Blauth. \textbf{Linguagens formais e autômatos}. 6. ed. Bookman, 2011.\\

\noindent
https://hudlac.wordpress.com/notacao-musical-abc\\

\noindent
http://abcnotation.com

\tituloProblema{$*$}{Referências}
Este problema é baseado nas notas de aula do professor Martin Musicante\footnote{\scriptsize{https://sigaa.ufrn.br/sigaa/public/docente/portal.jsf?siape=1221251}} do DIMAP - UFRN.
