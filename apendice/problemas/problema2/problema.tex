\subsection{Problema}

A empresa \textit{Refrigerantes e Salgados S.A.} resolveu desenvolver novas soluções
para as suas máquinas de vender refrigerantes e salgados
baseadas em autômatos finitos.

A ideia surgiu quando um dos representantes de vendas da empresa estava conversando
com um grupo de alunos da disciplina de Teoria da Computação da UFBA.
Eles perceberam que os autômatos finitos e expressões regulares
atendem as necessidades das máquinas e que são
de simples configuração.

%O representante de vendas resolveu que iria fornecer alguns materiais aos alunos para
%que eles idealizassem as máquinas.

%Foi entregue aos alunos sensores capazes de identificar as moedas e notas brasileiras.
%Os alunos receberam também um sistema capaz de ser programado exatamente como um
%autômato finito

O representante de vendas falou aos alunos que soube que a equipe de tecnologia
da empresa utiliza uma ferramenta chamada de JFLAP\footnote{http://www.jflap.org}
para testar outras soluções que utilizam autômatos finitos.

As máquinas devem ser configuradas para vender no mínimo três produtos e
receber notas de R\$ 2,00 e R\$ 5,00.

O representante de vendas disse aos alunos que para ajudar na divulgação
das novas máquinas irá realizar uma promoção, onde as máquinas deverão
considerar aleatoriamente a possibilidade de dar um troco de R\$ 2,00
como prêmio para o caso de o valor inserido pelo cliente ser exatamente
o preço do produto selecionado.

O representante de vendas soube da equivalência entre autômatos finitos
e expressões regulares, logo solicitou que seja construída uma expressão
regular geral para representar à máquina.

\subsection{Produto}
Um arquivo com um autômato finito que contenha a máquina de vender
refrigerantes e salgados de forma que a equipe de tecnologia da
empresa \textit{Refrigerantes e Salgados S.A.} possa testar, ou seja, no JFLAP,
a expressão regular, bem como, um relatório no modelo de
artigos da SBC que descreva com o máximo de detalhes a idealização de funcionamento do
sistema da máquina de vender refrigerantes e salgados, onde deverá
constar quais as operações executadas pelo sistema para receber o pagamento pelo cliente e entregar o
produto escolhido e ao menos 2 exemplos de funcionamento, tanto com autômatos
como com expressões regulares.
\subsection{Cronograma}

2 sessões tutoriais e 1 aula expositiva.

\subsection{\large{Recursos para aprendizagem}}

RAMOS, M. V. M.; JOSÉ NETO, J.; VEGA, I. S. \textbf{Linguagens Formais: Teoria, Modelagem e Implementação}. Editora Bookman, 2009.\\

\noindent
MENEZES, Paulo Blauth. \textbf{Linguagens formais e autômatos}. 6. ed. Bookman, 2011.\\

\subsection*{Referências}

Este problema é baseado nas notas de aula do professor Martin Musicante\footnote{\scriptsize{https://sigaa.ufrn.br/sigaa/public/docente/portal.jsf?siape=1221251}} do DIMAP - UFRN.
