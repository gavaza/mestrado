\tituloProblema{1}{Problema}
Um dos problemas que afetam as rodovias brasileiras são os buracos.
Muitos são os motivos que podem causar este problema, sendo a alternância de chuva e sol
um dos motivos que pode acelerar o desgaste do asfaltamento e consequentemente gerar buracos.
Outro motivo é o tráfego intenso de veículos, em particular a frequência com que veículos
pesados empregam as estradas.

Na estrada de Aratu, em Salvador, alguns buracos foram identificados pelos
técnicos da Superintendência de Infraestrutura de Transportes da Bahia (SIT-BA).
Os técnicos desta Superintendência acreditam que
podem reduzir a pressão sobre o asfaltamento, desta forma, melhorar o
problema dos buracos, se conseguirem aplicar um controle de
tráfego.

A ideia é separar os veículos em categorias leve e pesado, onde os
veículos até $6$ (seis) toneladas são leves e acima deste peso são pesados.

Os técnicos informaram que a proporção atual de veículos que trafegam na
estrada durante um dia é de $3$ (três) veículos pesados
para cada $5$ (cinco) veículos leves.

O controle de tráfego a ser utilizado deverá, a cada dia, gradativamente,
reduzir a proporção de veículos pesados trafegando na estrada até que a média
ao final do dia seja no máximo $1$ (um) veículo pesado
para cada $5$ (cinco) veículos leves.

Você pode considerar que existe uma balança na entrada da estrada para realizar a pesagem
dos veículos.
Também pode considerar que há uma rota alternativa mais longa em que os veículos
só serão encaminhados por este caminho para atender as restrições
do controle de tráfego, uma vez que se deseja obter o máximo de vasão
possível na estrada principal obedecendo as restrições.

\tituloProblema{2}{Produto}
(i) Autômatos de Pilha necessários para utilização no controle de tráfego descrito;
(ii) um relatório no modelo de artigos da SBC que descreva com o máximo de detalhes a idealização de funcionamento deste
sistema de controle de tráfego;
e (iii) construir gramáticas livre de contexto equivalentes aos autômatos construídos na solução.
As dificuldades ou não possibilidade de construção de gramática para algum dos autômatos construídos
deverá constar no relatório mencionado no item (ii).

\tituloProblema{3}{Cronograma}
4 sessões tutoriais e 1 aula expositiva.

\tituloProblema{4}{Recursos para aprendizagem}

RAMOS, M. V. M.; JOSÉ NETO, J.; VEGA, I. S. \textbf{Linguagens Formais: Teoria, Modelagem e Implementação}. Editora Bookman, 2009.\\

\noindent
MENEZES, Paulo Blauth. \textbf{Linguagens formais e autômatos}. 6. ed. Bookman, 2011.\\
