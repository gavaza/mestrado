\tituloProblema{1}{Problema}
Um dos problemas que afetam as rodovias brasileiras são os buracos.
Muitos são os motivos que podem causar este problema, sendo a alternância de chuva e sol
um dos motivos que pode acelerar o desgaste do asfaltamento e consequentemente gerar buracos.
Outro motivo é o tráfego intenso de veículos, em particular o peso com que veículos
pesados empregam nas estradas durante o pernoite.

Na estrada de Aratu, em Salvador, alguns buracos foram identificados pelos
técnicos da Superintendência de Infraestrutura de Transportes da Bahia.
Os técnicos desta Superintendência acreditam que
podem reduzir a pressão sobre o asfaltamento, desta forma, melhorar o
problema dos buracos, se conseguirem aplicar uma melhoria no controle de tráfego
da rodovia.

Os veículos são categorizados em: leves, pesados e muito pesados.
Veículos de até $6$ (seis) toneladas são leves;
acima de $6$ (seis) e abaixo de $10$ (dez) toneladas são pesados;
e de $10$ (dez) ou mais toneladas são muito pesados.

Já existem sensores nas entradas e nas saídas da estrada que permitem identificar a
categoria dos veículos que entram e saem da rodovia.

Os técnicos esperam uma solução que, a partir das informações dos
sensores, seja capaz de realizar a contagem de quantos veículos
de cada categoria passaram a noite estacionados na estrada (pernoite) e também apontar
a categoria que teve mais veículos no pernoite.

\tituloProblema{2}{Produto}
(i) Um arquivo com uma máquina de Turing que contenha a solução para o
problema; e (ii) um relatório no modelo de artigos da SBC que descreva com o máximo
de detalhes a idealização de funcionamento do sistema de controle de tráfego com a
melhoria de identificar quais as categorias que estão passando a noite na estrada.

\tituloProblema{3}{Cronograma}
2 sessões tutoriais e 2 aulas expositivas.

\tituloProblema{4}{Recursos para aprendizagem}

RAMOS, M. V. M.; JOSÉ NETO, J.; VEGA, I. S. \textbf{Linguagens Formais: Teoria, Modelagem e Implementação}. Editora Bookman, 2009.\\

\noindent
MENEZES, Paulo Blauth. \textbf{Linguagens formais e autômatos}. 6. ed. Bookman, 2011.\\

\noindent
HOPCROFT, John E.; ULLMAN, Jeffrey D.; MOTWANI, Rajeev. \textbf{Introdução à teoria de autômatos, linguagens e computação}. Editora Campus, 2002.\\

\noindent
SIPSER, Michael. \textbf{Introdução à teoria da computação}. Thomson Learning, 2007.
