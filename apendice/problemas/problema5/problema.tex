\subsection{Problema}
Todos os compiladores da linguagem C tomam como entrada um código fonte de um
programa e produzem como saída um código executável.
Os compiladores podem ser bastante sofisticados. Por exemplo, eles
podem detectar erros de sintaxe, podem também indicar alguns erros de
semântica (como o uso de ``='' em uma situação que normalmente pede-se ``==''),
eles podem otimizar o código resultante em termos de tempo e/ou memória consumidos.

O quanto inteligente um compilador pode ser?
Por exemplo, ele poderia detectar que todos os valores de entrada negativos
levam o seguinte programa a entrar em um \textit{loop} infinito?

\lstinputlisting{apendice/problemas/problema5/codigo1.c}

Talvez compiladores não sejam assim tão sofisticados, mas será
que seriam capazes de responder uma questão geral, como:
\\ \textbf{``Um programa P pára quando
lhe é fornecida uma determinada entrada I?''}

Suponha que o compilador possa responder tal questão, então poderíamos extrair
do programa a parte que responde a esta pergunta e
encapsular o segmento de código numa função chamada \textit{Halts}.


%Talvez compiladores não sejam assim tão sofisticados, mas será que poderiam responder
%a uma questão geral, como ``Um programa P pára quando lhe é fornecida uma determinada entrada I?'' ?
%Suponha que o compilador é capaz disto, então poderíamos extrair do programa a parte que responde a esta pergunta e
%encapsular o segmento de código numa função chamada Halts.
Uma vez que tanto o programa e a sua
entrada são simplesmente uma sequência de caracteres, \textit{Halts} pode ser
especificado como se segue:

\lstinputlisting{apendice/problemas/problema5/halts.c}

Agora vamos escrever algo meio estranho, um novo programa com o seguinte código:

\lstinputlisting{apendice/problemas/problema5/diagonal.c}

\subsection{Produto}
Um relatório no modelo de artigos da SBC que descreva com o máximo
de detalhes o funcionamento do programa \textit{diagonal} recebendo como entrada
o código fonte do próprio programa \textit{diagonal}.
No relatório você deverá (i) caracterizar este problema como de decisão;
(ii) justificar se este é um problema decidível ou indecidível; e (iii)
idealizar o funcionamento deste problema com a formalização Máquina de Turing.

\subsection{Cronograma}

2 sessões tutoriais e 1 aula expositiva.

\subsection{\large{Recursos para aprendizagem}}

\noindent
HOPCROFT, John E.; ULLMAN, Jeffrey D.; MOTWANI, Rajeev. \textbf{Introdução à teoria de autômatos, linguagens e computação}. Editora Campus, 2002.\\

\noindent
SIPSER, Michael. \textbf{Introdução à teoria da computação}. Thomson Learning, 2007.\\

\noindent
GREENLAW, Raymond; Hoover, H. James. \textbf{Fundamentals of the Theory of Computation - Principles and Practice}. Morgan Kaufmann Publishers, 1998.\\

%\section*{Referências}

%Este problema é baseado nas notas de aula do professor Martin Musicante\footnote{\scriptsize{https://sigaa.ufrn.br/sigaa/public/docente/portal.jsf?siape=1221251}} do DIMAP - UFRN.
