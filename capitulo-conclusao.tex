\xchapter{Conclusões}{} %sem preambulo
\label{sec-conclusao}
% É recomendável utilizar `\acresetall' no início de cada capítulo para reiníciar o contator de referências às siglas.
\acresetall
Como foi observado no desenvolvimento deste trabalho, sobretudo
no Capítulo~\ref{cap-revisao}, 
os pesquisadores estão preocupados em ter formas mais
efetivas de motivar os estudantes e construir conhecimento
em disciplinas de Teoria da Computação.
A opção normalmente utilizada é a inclusão de ferramentas
no contexto das aulas expositivas tradicionais.

A utilização da abordagem baseadas em problemas
em uma disciplina teórica, com todas as
particularidades que foram destacadas ao longo desta
dissertação, é um dos diferenciais deste trabalho.
Ainda é possível citar que este trabalho foi realizado
utilizando uma infraestrutura tradicional, onde foram
realizadas apenas algumas adaptações para que fosse
possível aplicar a metodologia, portanto, este
trabalho é facilmente replicável no sentido de que não exige
grandes intervenções de infraestrutura ou de recursos.

Os resultados apresentados neste trabalho são oportunidades
para argumentação na defesa de que é possível a utilização
da abordagem baseada em problemas para motivação de
estudantes e construção de conhecimento em disciplinas
teóricas, assim, podemos afirmar que existe viabilidade para
o objetivo de aumentar a motivação e aprendizagem, para redução
de evasão e reprovação.

Outra oportunidade deixada por este trabalho é
a lista extensa de trabalhos futuros, onde alguns estão
apresentados na Seção~\ref{sec-trab-futuros}.

\section{Trabalhos futuros}
\label{sec-trab-futuros}
As avaliações sobre motivação dos estudantes
foram realizadas, neste trabalho, nas perspectivas
destes, portanto, se faz recomendado um estudo futuro
mais detalhado sobre a relação entre o índice
de evasão e motivação de estudantes.

A utilização da abordagem com acompanhamento
para aplicações de ações tempestivas com base nas
percepções dos estudantes pode ser um trabalho futuro
que pode trazer consolidação na utilização da
abordagem, uma vez que permitir aos educadores evoluir
o andamento da disciplina conforme o contexto,
melhorando ainda mais os resultados.

A utilização da abordagem baseada em problemas em
outras disciplinas teóricas, outras instituições
e perfis diferentes de estudantes
é um trabalho futuro quase que obrigatório, uma vez
os resultados deste trabalho precisam ter mais
replicações.

Realizar um trabalho de avaliação de qualidade e
construção dos problemas pode permitir uma melhor
identificação do alinhamento dos conceitos nos
problemas, por este motivo, é justificado este
trabalho futuro.

A realização de trabalhos comparativos são também quase
que obrigatórios para que as vantagens da utilização
da metodologia baseada em problemas e as oportunidades
de melhorias sejam explicitadas.

A percepção de outros educadores sobre os problemas
construídos é um outro trabalho possível.
A principal justificativa para este trabalho seria
ajudar a quebrar a resistência em educadores para
utilização de uma ``nova'' abordagem.

É possível pensar em um trabalho de acompanhamento,
um \textit{tracking}, com estudantes que participaram
da metodologia baseada em problemas, para avaliação
da abordagem no contexto de formação dos estudantes.



