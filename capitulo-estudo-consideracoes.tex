\section{Considerações finais}
\label{sec-consideracoes-estudo}
Neste capítulo foi descrito o contexto de execução onde foi aplicada a
metodologia PBL em avaliação no estudo deste trabalho.

Os nove problemas construídos para o estudo foram descritos brevemente neste capítulo.

Como foi destacado, dois destes problemas foram utilizados
apenas como ferramenta para exemplificar a metodologia para os estudantes,
um para cada semestre.
Os desempenho dos estudantes não foi contabilizado para avaliação
e os dados não foram considerados para o estudo deste trabalho.

Outro destaque, é que um dos problemas foi utilizado apenas como ferramenta
de avaliação dos estudantes e foi replicado em ambos os semestres.
As percepções dos estudantes para estas replicações não foram
consideradas no estudo deste trabalho.

O primeiro problema de estudo no semestre 2016.1 foi ``\ProblemaA'', enquanto no
semestre 2017.1 foi o ``\ProblemaG''.
Embora este trabalho não se proponha a realizar estudos comparativos, é importante
estar atento que questões comparativas surgem naturalmente, então, nesse contexto,
esse tipo de replicação pode ajudar a explicitar possibilidades para
estudos futuros, neste caso, surge a questão com relação as modificações de
percepções dos estudantes com a utilização de problemas distintos
para tratar de um mesmo conteúdo.

Os quatro problemas restantes foram replicados em ambos os semestres.

Este trabalho construiu quatro pares de réplicas e mais duas
réplicas, portanto, este estudo contém um total de dez
réplicas.
