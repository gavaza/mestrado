\subsection{Contexto educacional}
\label{sec-contexto-educacional}
% curso
Os cursos da área de informática na Universidade Federal da Bahia
utilizam majoritariamente a abordagem tradicional, onde
algumas disciplinas específicas contam com carga horária específica
para laboratórios práticos.
Existem algumas poucas iniciativas de novas abordagens como é
o caso deste estudo.

Apesar de ser uma disciplina situada no início do
curso, entre os estudantes participantes deste estudo,
é grande a quantidade de estudantes que já abandonaram uma
outra disciplina do curso.
A disciplina deste estudo também possui historicamente um índice
elevado de evasão.

% papeis dos estudantes
O primeiro passo a cada sessão tutorial é a escolha dentre os
estudantes de três voluntários.
Um voluntário é responsável por realizar o registro da discussão
da sessão no quadro branco, esse
participante é denominado \textit{relator de quadro}.
Um segundo voluntário é responsável por realizar o registro
das discussões e disponibilizar um documento consolidado para todos
os participantes logo após
a sessão, sendo denominado \textit{relator de mesa}.
A discussão é conduzida por um terceiro voluntário,
denominado \textit{coordenador da sessão}, que administra
as intervenções dos participantes, permitindo que esses
tenham espaço para se posicionar.
Existe um rodízio nos papéis a cada sessão para que todos
os estudantes tenham oportunidade de desempenhar todos
os papéis descritos acima.

% o tutor
As sessões tutorias foram executadas com dois tutores.
Os tutores são os principais responsáveis por facilitar o andamento
das sessões tutoriais, zelando pelos objetivos de aprendizagem.
Durante as sessões tutoriais as intervenções dos tutores
foram mínimas, apenas em casos extremos de
afastamento dos conceitos estudados no problema existiu intervenção dos tutores.

% como decorre a sessão tutorial
Os estudantes são apresentados ao problema na primeira reunião, durante
a qual eles discutem um primeiro esboço geral do problema com seus
colegas de equipe em supervisão dos tutores.
A sessão segue com argumentações, exposição de ideias,
questionamentos e levantamento de fatos.
Nos minutos finais da reunião, os estudantes propõe metas de estudos para
que apresentem ao longo da discussão das próximas sessões tutoriais.
As metas apresentadas pelos estudantes são verificadas pelos tutores
para que estejam em um nível adequado, isto é, plausíveis e exequíveis
dentro do prazo, uma vez que metas não
alcançáveis para o período poderia desestimular os estudantes, assim
como metas muito fáceis de alcançar também poderia levar
ao mesmo problema.

% ambiente virtual de aprendizagem (AVA)
Foi disponibilizado aos estudantes um ambiente virtual de aprendizagem
institucional, que eles foram incentivados a utilizar ao longo
da disciplina.
A condução da disciplina estimulou os estudantes a
utilizarem este ambiente virtual que para eles serviu como um
espaço para continuidade das discussões sobre os problemas
e conceitos em fóruns, além de repositório para armazenar
os conteúdos produzidos.
O documento consolidado que é produzido pelo relator de mesa
durante a sessão tutorial é disponibilizado em um espaço
do ambiente virtual.
Para este estudo, o ambiente de virtual serviu
também como ferramenta para obtenção dos dados,
como está detalhado no
capítulo~\ref{cap-resultados}.
