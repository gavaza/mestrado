\section{Instrumentos de pesquisa}
\label{sec-instrumentos-pesquisa}
Todos os dados são recebidos em formulários
disponibilizados para os participantes no ambiente virtual.
No ambiente virtual, foi escolhida a opção de não
obter registro do participante nos formulários, com o objetivo
de garantir o anonimato das respostas.

Os dados recebidos não foram processados ou analisados
tempestivamente, assim, está garantido que não foram
realizadas mudanças na condução da metodologia
por influências dos dados em pesquisa.

Importante em pesquisa científica, a assinatura do termo
de consentimento livre e esclarecido pelos participantes foi digital,
isto é, assinado no ambiente virtual.
O termo utilizado está disponível
no Apêndice~\ref{termo-ciencia}.

Além de facilitar o pós-processamento dos dados, a opção de
formulário digital em contrapartida a formulário impresso, foi
realizada com o objetivo de dissociar para o participante
uma falsa impressão de obrigatoriedade em participar do experimento,
que pode ocorrer com a utilização de formulário impresso disponibilizado
logo após a realização da atividade.

Uma vez registrados no ambiente virtual, estes dados são consolidados
para utilização durante o processo de análise de dados.

A Seção~\ref{form-percepcoes} descreve
o conteúdo do formulário apresentado aos participantes para obter
informações sobre as percepções destes sobre a replicação de um problema;
a Seção~\ref{form-disciplinas} descreve
o conteúdo do formulário apresentado aos participantes para obter
informações sobre as percepções destes em uma visão geral sobre a disciplina.

\subsection{Formulário de percepções de problema}
\label{form-percepcoes}
A avaliação das hipóteses descritas na Seção~\ref{sec-hipoteses} é realizada neste estudo com informações que foram extraídas dos dados do formulário ``percepções de problema'' que é descrito nesta seção.

Para cada problema os estudantes foram convidados a responder um formulário de percepções de problema.
Neste formulário foram apresentados aos participantes 32 itens entre questões, afirmações e espaço aberto.
Os cinco primeiros itens foram questões de caracterização de perfil, como
idade e sexo.
Os próximos vinte e quatro itens foram afirmativas sobre as percepções dos participantes sobre a abordagem, o
problema, o tutor e auto avaliação.
Uma questão para o participante apresentar uma nota geral entre 0 e 10 para o problema.
Um espaço aberto para incluir considerações adicionais sobre
as percepções sobre o problema especificamente e um espaço aberto para considerações
sobre a utilização da abordagem.

No caso das afirmativas, cada afirmação sobre o tema ao qual se desejava obter informações
foi obtida em uma escala Likert com cinco itens,
sendo as opções ``concordo'', ``concordo parcialmente'', ``indiferente'',
``discordo parcialmente'' e ``discordo''.

O questionário referente a percepção de participante sobre problema está disponível no
Apêndice~\ref{form-problema}.

\subsection{Formulário de percepções da disciplina}
\label{form-disciplinas}
O formulário ``percepções da disciplina'' é utilizado como uma ferramenta adicional para obtenção de dados sobre uma visão geral do estudante com relação a disciplina.

Para obter dados sobre as percepções da disciplina, os participantes foram caracterizados em \textit{desistente}
se por algum motivo desistiu da disciplina antes da conclusão, sendo assim, foi reprovado por
falta ou realizou trancamento, e \textit{concluinte} que concluiu a disciplina independente
de aprovação ou reprovação por conceito.

Aos participantes foi apresentado um formulário de acordo com a sua condição de desistente
ou concluinte.
Em ambos os casos, o formulário apresentou 24 itens aos participantes entre questões,
afirmações e espaço aberto.
No caso do participante desistente o formulário foi focado em obter indicações
que ajudassem a trazer informações sobre o motivo da desistência, enquanto para o participante concluinte
o foco foi em trazer indicações que ajudassem a identificar benefícios da abordagem na percepção
do participante.
Os cinco primeiros itens foram questões de caracterização de perfil, como
idade e sexo.
Os próximos dezessete itens foram afirmativas sobre as percepções dos participantes em
relação ao foco do formulário respondido.
Um espaço aberto para incluir considerações adicionais.

No caso das afirmativas, cada afirmação sobre o tema ao qual se desejava obter informações
foi obtida em uma escala Likert com cinco itens, como as da Seção~\ref{form-percepcoes}.

O questionário referente a percepção sobre disciplina de participante
concluinte está disponível no Apêndice~\ref{form-disciplina-concluinte}
e de participante desistente está disponível no
Apêndice~\ref{form-disciplina-desistente}.
