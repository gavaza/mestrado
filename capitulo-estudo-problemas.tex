\section{Problemas}
\label{sec-problemas-listas}
Os problemas foram construídos seguindo as diretivas
apresentadas no Capítulo~\ref{cap-revisao}.
Nesta seção os problemas são apresentados com uma breve
descrição do contexto e objetivos de aprendizagem.
Os textos dos problemas estão disponíveis no
Apêndice~\ref{cap-problemas-textos}.

\subsection{\ProblemaA}
O problema menciona um colecionador musical que deseja que
novas músicas sejam criadas utilizando como base as músicas
de uma biblioteca fornecida.

O objetivo desse problema é trabalhar os conceitos de
Linguagens Formais por meio de uma equivalência dos conceitos
de notas musicais com os conceitos de símbolos e cadeias.

\subsection{\ProblemaB}
O problema menciona uma empresa que deseja construir uma máquina
de vender refrigerantes.

O objetivo desse problema é trabalhar os conceitos de Autômatos
Finitos e introduzir o conceito de não determinismo.

\subsection{\ProblemaC}
\label{problema3}
O problema menciona a situação de buracos em uma estrada e
solicita aos estudantes que realizem um balanceamento da proporção
entre veículos leves e pesados.

O objetivo desse problema é trabalhar os conceitos de Autômatos
Finitos com Pilha.

\subsection{\ProblemaD}
O problema é uma extensão do descrito na Seção~\ref{problema3}.
Neste problema, o estudante consegue identificar que apenas uma pilha
é insuficiente para realizar o controle da proporção proposta para este
problema.

O objetivo desse problema é trabalhar os conceitos de Máquinas de Turing.

\subsection{\ProblemaE}
O problema relata sobre a capacidade dos compiladores
em identificar erros e otimizar códigos e convida os
estudantes a discutirem o quanto um compilador
pode ser inteligente.

O objetivo desse problema é trabalhar a teoria da
computabilidade, especificamente, utiliza o Problema
da Parada para exemplificar.

\subsection{\ProblemaF}
O problema apresenta uma discussão sobre as questões
de classes de problema P \textit{versus} NP.

O objetivo desse problema é trabalhar a teoria da complexidade
computacional, especificamente, utiliza as questões de
P \textit{versus} NP.

\subsection{\ProblemaG}
O problema menciona um grupo de estudantes programadores que
estão naufragados em uma ilha deserta e precisam submeter
um projeto de desenvolvimento. Ao serem resgatados devem
discutir como utilizar um telégrafo para realizar
a transmissão.

O objetivo desse problema é trabalhar os conceitos de
Linguagens Formais por meio de uma equivalência com
o código Morse.

\subsection{\ProblemaH}
\problemaExemplo{Este problema relata falhas que acontecem
em computadores de um laboratório
de informática da universidade.}

\subsection{\ProblemaI}
\problemaExemplo{Este problema relata uma história
de um roubo de pimenta para que os estudantes realizem inferências.}
