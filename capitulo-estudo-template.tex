\newcommand{\descricaoSemestre}[6]{
Neste semestre foi utilizado uma abordagem híbrida com abordagem tradicional de ensino
e aprendizagem, e metodologia PBL.

A carga horária de #1 horas da disciplina no semestre foi dividida em:
#2 horas para a realização de sessões tutoriais de PBL em que foi conduzida a abordagem;
#3 horas para a realização de aulas expositivas em que o educador apresentou
conceitos da disciplina;
e #4 horas para a realização de avaliações tradicionais.

A experiência aconteceu em uma sala de aula tradicional equipada com um
quadro branco{\ifthenelse{\equal{#5}{1}}{.}{ e com quadros adicionais
feitos com papel metro colados nas paredes da sala.}}
Nas sessões tutorias os estudantes foram
distribuídos{\ifthenelse{\equal{#5}{1}}{ em semicírculo de
forma que todos foram capazes de enxergar o quadro e
tiveram interação entre eles facilitada.}{ em grupos de
cinco até dez participantes.
Os grupos formaram semicírculos de forma que todos foram
capazes de enxergar o quadro adicional referente ao grupo
ao qual foi designado e tiveram interação dentro do grupo
facilitada.}}

% perfil dos participantes
Para este semestre foram #6 estudantes inscritos na disciplina.}

\newcommand{\descricaoSemestreProblemas}[4]{% problemas utilizados

Foram escolhidos #1 problemas, para este semestre, para trabalhar o
conteúdo da disciplina em conjunto com a abordagem tradicional.
No que se refere a abordagem tradicional, os conteúdos foram trabalhados
com os estudantes em aulas expositivas utilizando quadro e apresentações
pelo educador.
Ao longo deste semestre, os estudantes {\ifthenelse{\equal{#2}{1}}
{construíram uma apresentação, um seminário, }
{construíram um projeto de desenvolvimento, }
com conteúdos relacionados aos
compiladores, como por exemplo, analisadores léxico, sintático e semântico,
otimizações de código, árvores de derivação}.

Para avaliação de conceito dos estudantes foram atribuídas notas para
o {\ifthenelse{\equal{#2}{1}}{seminário}{projeto de desenvolvimento}},
para os problemas e duas avaliações escritas individuais e sem consultas.
No caso dos problemas, a nota considerou assiduidade e participação dos
estudantes nas discussões das sessões tutoriais e o produto produzido
como solução para o problema.

O problema ``#3'' foi utilizado como demonstração
aos estudantes do funcionamento da metodologia PBL e por não abordar
especificamente nenhum conteúdo da disciplina não foi considerado para
efeitos de avaliação dos estudantes, nem foi considerado para
efeitos de análise deste estudo.

Para este semestre os problemas aplicados para
avaliação dos estudantes foram:
\begin{enumerate}
\item{{\ifthenelse{\equal{#4}{1}}{\ProblemaA}{\ProblemaG}}}
\item{\ProblemaB}
\item{\ProblemaC}
\item{\ProblemaD}
\item{\ProblemaE}
\item{\ProblemaF}
\end{enumerate}

Um detalhe a considerar é que embora as discussões referente ao sexto problema
tenha seguido a metodologia PBL, ao estudante foi solicitado responder
uma lista de exercícios referente as questões apresentadas ao longo
do texto ao invés da produção de um produto solução para o problema.
Esse problema não é considerado para efeitos análises e resultados
deste estudo.
}

\newcommand{\problemaExemplo}[1]{
Este é um problema construído apenas para exemplificar
a metodologia PBL para os estudantes.
#1

O objetivo deste problema é mostrar aos estudantes
a condução com a metodologia.
}
