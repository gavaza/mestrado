% incluir template
\newcommand{\descricaoSemestre}[6]{
Neste semestre foi utilizado uma abordagem híbrida com abordagem tradicional de ensino
e aprendizagem, e metodologia \ac{PBL}.

A carga horária de #1 horas da disciplina no semestre foi dividida em:
#2 horas para a realização de sessões tutoriais de \ac{PBL} em que foi conduzida a abordagem;
#3 horas para a realização de aulas expositivas em que o educador apresentou
conceitos da disciplina;
e #4 horas para a realização de avaliações tradicionais.

A experiência aconteceu em uma sala de aula tradicional equipada com um
quadro branco{\ifthenelse{\equal{#5}{1}}{.}{ e com quadros adicionais
feitos com papel metro colados nas paredes da sala.}}
Nas sessões tutorias os estudantes foram
distribuídos{\ifthenelse{\equal{#5}{1}}{ em semicírculo de
forma que todos foram capazes de enxergar o quadro e
tiveram interação entre eles facilitada.}{ em grupos de
cinco até dez participantes.
Os grupos formaram semicírculos de forma que todos foram
capazes de enxergar o quadro adicional referente ao grupo
ao qual foi designado e tiveram interação dentro do grupo
facilitada.}}

% perfil dos participantes
Para este semestre foram #6 estudantes inscritos na disciplina.}

\newcommand{\descricaoSemestreProblemas}[4]{% problemas utilizados

Foram escolhidos #1 problemas, para este semestre, para trabalhar o
conteúdo da disciplina em conjunto com a abordagem tradicional.
No que se refere a abordagem tradicional, os conteúdos foram trabalhados
com os estudantes em aulas expositivas utilizando quadro e apresentações
pelo educador.
Ao longo deste semestre, os estudantes {\ifthenelse{\equal{#2}{1}}
{construíram uma apresentação, um seminário, }
{construíram um projeto de desenvolvimento, }
com conteúdos relacionados aos
compiladores, como por exemplo, analisadores léxico, sintático e semântico,
otimizações de código, árvores de derivação}.

Para avaliação de conceito dos estudantes foram atribuídas notas para
o {\ifthenelse{\equal{#2}{1}}{seminário}{projeto de desenvolvimento}},
para os problemas e duas avaliações escritas individuais e sem consultas.
No caso dos problemas, a nota considerou assiduidade e participação dos
estudantes nas discussões das sessões tutoriais e o produto produzido
como solução para o problema.

O problema ``#3'' foi utilizado como demonstração
aos estudantes do funcionamento da metodologia \ac{PBL} e por não abordar
especificamente nenhum conteúdo da disciplina não foi considerado para
efeitos de avaliação dos estudantes, nem foi considerado para
efeitos de análise deste estudo.

Para este semestre os problemas aplicados para
avaliação dos estudantes foram:
\begin{enumerate}
\item{{\ifthenelse{\equal{#4}{1}}{\ProblemaA}{\ProblemaG}}}
\item{\ProblemaB}
\item{\ProblemaC}
\item{\ProblemaD}
\item{\ProblemaE}
\item{\ProblemaF}
\end{enumerate}

Um detalhe a considerar é que embora as discussões referente ao sexto problema
tenha seguido a metodologia \ac{PBL}, ao estudante foi solicitado responder
uma lista de exercícios referente as questões apresentadas ao longo
do texto ao invés da produção de um produto solução para o problema.
Esse problema não é considerado para efeitos análises e resultados
deste estudo.
}

\newcommand{\problemaExemplo}[1]{
Este é um problema construído apenas para exemplificar
a metodologia \ac{PBL} para os estudantes.
#1

O objetivo deste problema é mostrar aos estudantes
a condução com a metodologia.
}

\xchapter{Metodologia de estudo}{`` (...) ensinar não é transferir conhecimento, mas criar as possibilidades para a sua própria produção ou a sua construção.'' Paulo Freire, 2003} %sem preambulo
\label{cap-estudo}
% É recomendável utilizar `\acresetall' no início de cada capítulo para reiníciar o contator de referências às siglas.
\acresetall
Este capítulo descreve a experiência em dois semestres onde foi aplicada a abordagem \ac{PBL}
em conjunto com a abordagem tradicional, sendo assim, uma abordagem híbrida.
Nesse capítulo são descritos os problemas, os instrumentos de pesquisa e a
experiência.

% TODO: escrever uma pequena introdução
No curso de Sistemas da Informação da \ac{UFBA} existe
a disciplina ``Introdução as linguagens Formais e Teoria da Computação'',
que possui uma carga teórica extensa, que além dos tópicos de Linguagens Formais e Teoria
da Computação, também inclui tópicos de Complexidade e de Compiladores.
Essa disciplina também é elegível aos estudantes dos cursos de bacharelado
interdisciplinar, uma modalidade de curso oferecida pela universidade
em que o estudante tem uma formação geral antes de escolher
uma formação específica.
Na reserva de matrícula, a maioria das vagas desta disciplina são ofertadas aos
estudantes do curso de Sistema de Informação, sendo estes, assim, maioria no curso.
Nos dois semestres em que o estudo foi aplicado, não houve mudança nesse
contexto mencionado.

A disciplina tem carga horária de 68 horas conduzida ao longo de um semestre, sendo duas
aulas por semana com 2 horas aula cada, sendo cada hora aula 50 minutos.
É ofertada aos estudantes anualmente, no primeiro semestre de cada ano.

Na experiência, foi utilizado a oportunidade como critério de seleção de participantes, onde todos os
estudantes regularmente inscritos na disciplina participaram da abordagem.

A Seção~\ref{sec-experiencia} descreve a experiência em sala de aula;
a Seção~\ref{sec-instrumentos-pesquisa} descreve os instrumentos de pesquisa;
a Seção~\ref{sec-problemas-listas} lista os problemas utilizados
na aplicação da abordagem com uma breve
descrição;
e, por fim, a Seção~\ref{sec-consideracoes-estudo} apresenta
considerações sobre a aplicação da abordagem.

\section{Experiência em sala de aula}
\label{sec-experiencia}

No estudo foi utilizada uma metodologia de pesquisa descritiva na forma
de pesquisa de opinião.
Essa é uma pesquisa de satisfação dos estudantes
realizada no contexto que é descrito a seguir.

A metodologia realiza replicações de execução de problemas
PBL em dois semestres, onde as características
estão descritas na Seção \ref{sec-exp-2016} e na Seção \ref{sec-exp-2017}.

\subsection{Contexto educacional}
\label{sec-contexto-educacional}
% curso
Os cursos da área de informática na Universidade Federal da Bahia
utilizam majoritariamente a abordagem tradicional, onde
algumas disciplinas específicas contam com carga horária específica
para laboratórios práticos.
Existem algumas poucas iniciativas de novas abordagens como é
o caso deste estudo.

Apesar de ser uma disciplina situada no início do
curso, entre os estudantes participantes deste estudo,
é grande a quantidade de estudantes que já abandonaram uma
outra disciplina do curso.
A disciplina deste estudo também possui historicamente um índice
elevado de evasão.

% papeis dos estudantes
O primeiro passo a cada sessão tutorial é a escolha dentre os
estudantes de três voluntários.
Um voluntário é responsável por realizar o registro da discussão
da sessão no quadro branco, esse
participante é denominado \textit{relator de quadro}.
Um segundo voluntário é responsável por realizar o registro
das discussões e disponibilizar um documento consolidado para todos
os participantes logo após
a sessão, sendo denominado \textit{relator de mesa}.
A discussão é conduzida por um terceiro voluntário,
denominado \textit{coordenador da sessão}, que administra
as intervenções dos participantes, permitindo que esses
tenham espaço para se posicionar.
Existe um rodízio nos papéis a cada sessão para que todos
os estudantes tenham oportunidade de desempenhar todos
os papéis descritos acima.

% o tutor
Os tutores são os principais responsáveis por facilitar o andamento
das sessões tutoriais, zelando pelos objetivos de aprendizagem.
Durante as sessões tutoriais as intervenções dos tutores
foram mínimas, apenas em casos extremos de
afastamento dos conceitos estudados no problema existiu intervenção dos tutores.

% como decorre a sessão tutorial
Os estudantes são apresentados ao problema na primeira reunião, durante
a qual eles discutem um primeiro esboço geral do problema com seus
colegas de equipe em supervisão dos tutores.
A sessão segue com argumentações, exposição de ideias,
questionamentos e levantamento de fatos.
Nos minutos finais da reunião, os estudantes propõe metas de estudos para
que apresentem ao longo da discussão das próximas sessões tutoriais.
As metas apresentadas pelos estudantes são verificadas pelos tutores
para que estejam em um nível adequado, isto é, plausíveis e exequíveis
dentro do prazo, uma vez que metas não
alcançáveis para o período poderia desestimular os estudantes, assim
como metas muito fáceis de alcançar também poderia levar
ao mesmo problema.

% ambiente virtual de aprendizagem (AVA)
Foi disponibilizado aos estudantes um ambiente virtual de aprendizagem
institucional, que eles foram incentivados a utilizar ao longo
da disciplina.
A condução da disciplina estimulou os estudantes a
utilizarem este ambiente virtual que para eles serviu como um
espaço para continuidade das discussões sobre os problemas
e conceitos em fóruns, além de repositório para armazenar
os conteúdos produzidos.
O documento consolidado que é produzido pelo relator de mesa
durante a sessão tutorial é disponibilizado em um espaço
do ambiente virtual.
Para este estudo, o ambiente de virtual serviu
também como ferramenta para obtenção dos dados,
como está detalhado no
capítulo~\ref{cap-resultados}.


\subsection{Semestre 2016.1}
\label{sec-exp-2016}
\descricaoSemestre{68}{26}{32}{10}{1}{26}
\descricaoSemestreProblemas{seis}{1}{\ProblemaH}{1}

\subsection{Semestre 2017.1}
\label{sec-exp-2017}
\descricaoSemestre{68}{26}{32}{10}{2}{50}
\descricaoSemestreProblemas{seis}{2}{\ProblemaI}{2}

\section{Instrumentos de pesquisa}
\label{sec-instrumentos-pesquisa}
Todos os dados são recebidos em formulários
disponibilizados para os participantes no ambiente virtual.
No ambiente virtual, foi escolhida a opção de não
obter registro do participante nos formulários, com o objetivo
de garantir o anonimato das respostas.

Os dados recebidos não foram processados ou analisados
tempestivamente, assim, está garantido que não foram
realizadas mudanças na condução da abordagem
por influências dos dados em pesquisa.

Importante em pesquisa científica, a assinatura do termo
de consentimento livre e esclarecido pelos participantes foi digital,
isto é, assinado no ambiente virtual.
O termo utilizado está disponível
no Apêndice~\ref{termo-ciencia}.

Além de facilitar o pós-processamento dos dados, a opção de
formulário digital em contrapartida a formulário impresso foi
realizada com o objetivo de dissociar para o participante
uma falsa impressão de obrigatoriedade em participar do experimento,
que pode ocorrer com a utilização de formulário impresso disponibilizado
logo após a realização da atividade.

Uma vez registrados no ambiente virtual, estes dados são consolidados
para utilização durante o processo de análise de dados.

A Seção~\ref{form-percepcoes} descreve
o conteúdo do formulário apresentado aos participantes para obter
informações sobre as percepções desses sobre a aplicação de um problema;
a Seção~\ref{form-disciplinas} descreve
o conteúdo do formulário apresentado aos participantes para obter
informações sobre as percepções destes em uma visão geral sobre a disciplina.

\subsection{Formulário de percepções de problema}
\label{form-percepcoes}
Para cada problema os estudantes foram convidados a responder um formulário de percepções de problema.
Neste formulário, foram apresentados aos participantes 32 itens entre questões, afirmações e espaço aberto.
Os cinco primeiros itens foram questões de caracterização de perfil, como
idade e sexo.
Os próximos vinte e quatro itens foram afirmativas sobre as percepções dos participantes sobre a abordagem, o
problema, o tutor e auto avaliação.
Uma questão para o participante apresentar uma nota geral entre 0 e 10 para o problema.
Um espaço aberto para incluir considerações adicionais sobre
as percepções sobre o problema especificamente e um espaço aberto para considerações
sobre a utilização da abordagem.

No caso das afirmativas, cada afirmação sobre o tema ao qual se desejava obter informações
foi obtida em uma escala Likert com cinco itens,
sendo as opções ``concordo'', ``concordo parcialmente'', ``indiferente'',
``discordo parcialmente'' e ``discordo''.

O questionário referente a percepção de participante sobre problema está disponível no
Apêndice~\ref{form-problema}.

\subsection{Formulário de percepções da disciplina}
\label{form-disciplinas}
O formulário ``percepções da disciplina'' é utilizado como uma ferramenta adicional para obtenção de dados sobre uma visão geral do estudante com relação a disciplina.

Para obter dados sobre as percepções da disciplina, os participantes foram caracterizados em \textit{desistente}
se por algum motivo desistiu da disciplina antes da conclusão, sendo assim, foi reprovado por
falta ou realizou trancamento, e \textit{concluinte} que concluiu a disciplina independente
de aprovação ou reprovação por conceito.

Aos participantes foi apresentado um formulário de acordo com a sua condição de desistente
ou concluinte.
Em ambos os casos, o formulário apresentou 24 itens aos participantes entre questões,
afirmações e espaço aberto.
No caso do participante desistente o formulário foi focado em obter indicações
que ajudassem a trazer informações sobre o motivo da desistência, enquanto para o participante concluinte
o foco foi em trazer indicações que ajudassem a identificar benefícios da abordagem na percepção
do participante.
Os cinco primeiros itens foram questões de caracterização de perfil, como
idade e sexo.
Os próximos dezessete itens foram afirmativas sobre as percepções dos participantes em
relação ao foco do formulário respondido.
Um espaço aberto para incluir considerações adicionais.

No caso das afirmativas, cada afirmação sobre o tema ao qual se desejava obter informações
foi obtida em uma escala Likert com cinco itens, como as da Seção~\ref{form-percepcoes}.

O questionário referente a percepção sobre disciplina de participante
concluinte está disponível no Apêndice~\ref{form-disciplina-concluinte}
e de participante desistente está disponível no
Apêndice~\ref{form-disciplina-desistente}.

\section{Problemas}
\label{sec-problemas-listas}
Os problemas foram construídos seguindo as diretivas
apresentadas no Capítulo~\ref{cap-revisao}.
Nesta seção os problemas são apresentados com uma breve
descrição do contexto e objetivos de aprendizagem.
Os textos dos problemas estão disponíveis no
Apêndice~\ref{cap-problemas-textos}.

\subsection{\ProblemaA}
O problema menciona um colecionador musical que deseja que
novas músicas sejam criadas utilizando como base as músicas
de uma biblioteca fornecida.

O objetivo desse problema é trabalhar os conceitos de
Linguagens Formais por meio de uma equivalência dos conceitos
de notas musicais com os conceitos de símbolos e cadeias.

\subsection{\ProblemaB}
O problema menciona uma empresa que deseja construir uma máquina
de vender refrigerantes.

O objetivo desse problema é trabalhar os conceitos de Autômatos
Finitos e introduzir o conceito de não determinismo.

\subsection{\ProblemaC}
\label{problema3}
O problema menciona a situação de buracos em uma estrada e
solicita aos estudantes que realizem um balanceamento da proporção
entre veículos leves e pesados.

O objetivo desse problema é trabalhar os conceitos de Autômatos
Finitos com Pilha.

\subsection{\ProblemaD}
O problema é uma extensão do descrito na Seção~\ref{problema3}.
Neste problema, o estudante consegue identificar que apenas uma pilha
é insuficiente para realizar o controle da proporção proposta para este
problema.

O objetivo desse problema é trabalhar os conceitos de Máquinas de Turing.

\subsection{\ProblemaE}
O problema relata sobre a capacidade dos compiladores
em identificar erros e otimizar códigos e convida os
estudantes a discutirem o quanto um compilador
pode ser inteligente.

O objetivo desse problema é trabalhar a teoria da
computabilidade, especificamente, utiliza o Problema
da Parada para exemplificar.

\subsection{\ProblemaF}
O problema apresenta uma discussão sobre as questões
de classes de problema P \textit{versus} NP.

O objetivo desse problema é trabalhar a teoria da complexidade
computacional, especificamente, utiliza as questões de
P \textit{versus} NP.

\subsection{\ProblemaG}
O problema menciona um grupo de estudantes programadores que
estão naufragados em uma ilha deserta e precisam submeter
um projeto de desenvolvimento. Ao serem resgatados devem
discutir como utilizar um telégrafo para realizar
a transmissão.

O objetivo desse problema é trabalhar os conceitos de
Linguagens Formais por meio de uma equivalência com
o código Morse.

\subsection{\ProblemaH}
\problemaExemplo{Este problema relata falhas que acontecem
em computadores de um laboratório
de informática da universidade.}

\subsection{\ProblemaI}
\problemaExemplo{Este problema relata uma história
de um roubo de pimenta para que os estudantes realizem inferências.}

\section{Considerações finais}
\label{sec-consideracoes-estudo}
Neste capítulo foi descrito o contexto de execução onde foi aplicada a
metodologia PBL em avaliação no estudo deste trabalho.

Os nove problemas construídos para o estudo foram descritos brevemente neste capítulo.

Como foi destacado, dois destes problemas foram utilizados
apenas como ferramenta para exemplificar a metodologia para os estudantes,
um para cada semestre.
Os desempenho dos estudantes não foi contabilizado para avaliação
e os dados não foram considerados para o estudo deste trabalho.

Outro destaque, é que um dos problemas foi utilizado apenas como ferramenta
de avaliação dos estudantes e foi replicado em ambos os semestres.
As percepções dos estudantes para estas replicações não foram
consideradas no estudo deste trabalho.

O primeiro problema de estudo no semestre 2016.1 foi ``\ProblemaA'', enquanto no
semestre 2017.1 foi o ``\ProblemaG''.
Embora este trabalho não se proponha a realizar estudos comparativos, é importante
estar atento que questões comparativas surgem naturalmente.
Nesse contexto, esse tipo de replicação pode ajudar a explicitar possibilidades para
estudos futuros, neste caso, surge a questão com relação as modificações de
percepções dos estudantes com a utilização de problemas distintos
para tratar de um mesmo conteúdo.

Os quatro problemas restantes foram replicados em ambos os semestres.

Este trabalho construiu quatro pares de réplicas e mais duas
réplicas, portanto, este estudo contém um total de dez
réplicas.

