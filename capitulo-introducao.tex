\newcommand{\publicacaoTemplate}[2]{%
\textbf{#1} : #2}


\xchapter{Introdução}{Às vezes [o sonho] parece ser muito mais que função cerebral não explicada, mas um presente de Deus.} %sem preambulo
% É recomendável utilizar `\acresetall' no início de cada capítulo para reiníciar o contator de referências às siglas.
\acresetall
\label{cap-introducao}
A disciplina de Teoria da Computação é uma parte muito importante
na formação dos estudantes dos cursos da área de Computação, uma 
vez que este é um conhecimento que permite aos estudantes
entenderem as bases de sustentação da área de Computação.
Na disciplina de Teoria da Computação os tópicos tratados
possuem um alto nível de abstração e exigem bastante esforço
por parte dos estudantes para que possam acompanhar o
andamento.

Muitas são as dificuldades para que os estudantes se sintam
motivados e consigam construir o entendimento dos conceitos.
Os educadores em Ciência da Computação enfrentam muitas questões no que diz
respeito a como conduzir uma disciplina de forma que a construção do conhecimento
por parte dos estudantes seja satisfatória.
Isto não é um caso particular do ensino de Computação,
mas é necessário considerar que se trata de
conhecimento relativamente novo, e que só muito
recentemente está presente em instituições de ensino universitário.
Devemos considerar que ainda não existem metodologias e procedimentos
suficientemente testados para o ensino de Computação.
Além disto, há a natureza intrinsecamente multidisciplinar e
o desenvolvimento contínuo da área.
Embora seja possível perceber um contínuo desenvolvimento também em
outras áreas do conhecimento, em Computação é notória a robustez e
continuidade do progresso produzido nas últimas décadas.

\section{Motivação}
% contextualização
Ao mesmo tempo em que é crescente o número
de estudantes universitários, é grande a
evasão da universidade.
Os altos índices de desistências nos cursos
superiores são evidências da necessidade de
compreensão das variáveis motivacionais
dos estudantes.

% importância da motivação
A motivação do estudante é um determinante crítico 
do  nível  e  da  qualidade  da  aprendizagem
e do desempenho no contexto
escolar.
Desta forma, muitos trabalhos tem relacionado o desempenho
acadêmico dos estudantes com
a motivação~\cite{zenorini2011motivaccao,rufini2011estudo}

% classificação
A motivação para aprendizagem dos estudantes
tem sido classificada em intrínseca e
extrínseca.
A motivação é intrínseca quando a motivação
está na atividade em si e extrínseca quando
a motivação está na recompensa recebida pelo estudante
pela execução da
atividade~\cite{rufini2011estudo,neves2007escala}.

% entendimento dos estudantes
Os estudantes precisam entender o significado e importância
de estudar as disciplinas, assim, o educador
precisa identificar os interesses
dos estudantes~\cite{angeli2011relaccao}.


Ao verificar as grades curriculares dos cursos de Computação,
serão encontradas algumas atividades práticas em laboratórios,
sobretudo para o ensino de disciplinas com tópicos de programação e
algumas atividades práticas como complemento à carga horária teórica,
constituída essencialmente na exposição de conhecimento pelo educador
aos estudantes.

A abordagem pedagógica tradicional não parece ser suficiente para enfrentar
os desafios do ensino e aprendizagem das disciplinas
em Computação.
Apesar de tudo, o ensino de Computação, em geral, segue a abordagem
tradicional, com educadores atuando como transmissores de conhecimento
enquanto estudantes são receptores.
Este tipo de abordagem prioriza a memorização de conteúdos para
que os estudantes sejam capazes de replicar procedimentos,
normalmente não são desenvolvidas habilidades críticas e
necessárias para resolução de problemas do cotidiano.

A inexistência de ligação direta entre os conteúdos
e a sua vida cotidiana é normalmente uma das queixa dos estudantes
que experimentam apenas abordagens pedagógicas onde não
há este tipo de contextualização.
Ainda que o educador possa em alguns momentos tentar
utilizar gatilhos para contextualizar, para o
estudante o foco ainda estará essencialmente
na memorização dos conteúdos para futura
replicação.
Assim, nesse contexto, a aprendizagem se torna monótona
e desinteressante para os estudantes.


\section{Proposta}
\label{sec-proposta}
Utilizar uma abordagem baseada em problemas em uma disciplina
introdutória de Teoria da Computação na \ac{UFBA} e
verificar as percepções dos estudantes sobre
motivação e aprendizagem.

\section{Hipóteses}
\label{sec-hipoteses}
Este trabalho se propõe validar as hipóteses a seguir.

\begin{enumerate}
\item{\label{h1ref} \hatexto;}
\item{\label{h2ref} \hbtexto;}
\item{\label{h3ref} \hctexto.}
\end{enumerate}

\section{Objetivos}
\label{sec-objetivos}

\subsection{Objetivo geral}
Aumentar a motivação e aprendizagem dos estudantes de Computação no estudo
de disciplinas teóricas, como Teoria da Computação, para reduzir a evasão e reprovação,
por meio de uma metodologia baseada em problemas.

\subsection{Objetivos específicos}
\begin{enumerate}
\item{\label{oe1ref} \oeatexto;}
\item{\label{oe2ref} \oebtexto;}
\item{\label{oe3ref} \oectexto;}
\item{\label{oe4ref} \oedtexto;}
\item{\label{oe5ref} \oeetexto;}
\item{\label{oe6ref} \oeftexto;}
\item{\label{oe7ref} \oegtexto.}
\end{enumerate}

\section{Justificativa}
A forma de ensino e aprendizagem de Computação pode
ser determinante para o sucesso na formação dos
estudantes e também no que diz respeito a inclusão
e pensamento crítico, esses motivos servem de estímulo
para que os educadores pesquisem sobre intervenções e
abordagens de ensino e aprendizagem.

A utilização de abordagens que obtiveram
resultados positivos em outras áreas do conhecimento é uma
alternativa que surge quase que naturalmente quando
nos deparamos com desafios existentes em contextos
em que uma determinada abordagem um pouco mais difundida
não atende algum objetivo.

Existem inúmeras abordagens pedagógicas que poderiam ser
candidatas utilizando os critérios mencionados,
mas entre as mais promissoras que serviriam
as nossas necessidades, sobretudo na possibilidade
de desenvolver habilidades de pensamento crítico, se
pode destacar a abordagem baseada em problemas.

No caso da abordagem baseada em problemas, entre as
várias qualidades, podemos destacar que o processo
possui como grande objetivo ``ensinar a aprender'' para
que os estudantes possam assumir a
responsabilidade pela aprendizagem.
Na abordagem baseada em problemas os estudantes
são capacitados para as investigações e a aprendizagem
dos conceitos será consequência das investigações.

Ao assumir a responsabilidade pela aprendizagem, os
estudantes podem construir um conhecimento mais sólido
e amplo, investigar outras linhas de estudo e ter opiniões
críticas sobre os conceitos.
Assim, é esperado que estes estudantes possuam uma
formação que permita a eles trazer contribuições
relevantes para a ciência e para a sociedade.

A aplicação da abordagem baseada em problemas
para disciplinas teóricas não é difundida, desta forma,
este estudo traz resultados novos para este contexto
específico.
Este estudo também discute questões sobre a utilização
de uma abordagem de ensino e aprendizagem alternativa
para a disciplina teórica de Teoria da Computação.

\section{Publicações}
\label{sec-publicacoes}
Alguns dos resultados parciais desta dissertação já
foram publicados em conferência e revista científica,
como pode ser observado abaixo:
\begin{itemize}
\item{\textbf{Uma experiência de aplicação de uma
abordagem baseada em problemas no ensino
de Teoria da Computação em sala de aula tradicional}.
O trabalho relatou a experiência de aplicação de \ac{PBL}
para o ensino da disciplina introdutória de Teoria da
Computação utilizando uma infraestrutura básica na
\ac{UFBA} no semestre 2016.1.
Os resultados dessa experiência mostraram que os estudantes participantes
tiveram boas percepções sobre a abordagem utilizada, o que confirma o
grande potencial para aplicação de \ac{PBL} no ensino de
disciplinas teóricas como é o caso da disciplina introdutória de
Teoria da Computação.
Os resultados extraídos dos formulários respondido pelos estudantes
foram analisados na ótica da percepção
destes estudantes sobre os problemas.
Os dados foram consolidados e analisados em médias
de favorabilidade~\cite{gavaza2017}.}


\end{itemize}

Alguns dos resultados parciais desta dissertação já
foram submetidos para revisão em conferência e revista
científica, como pode ser observado abaixo:
\begin{itemize}
\item{\textbf{Avaliação de percepção de estudantes sobre
motivação e aprendizagem em uma disciplina de Teoria da
Computação com \ac{PBL}.}
O trabalho relatou a experiência de aplicação da abordagem
\ac{PBL}, em uma turma da disciplina de Teoria da Computação, na
\ac{UFBA} no semestre 2017.1.
A metodologia utilizou pesquisa descritiva na forma de pesquisa
de opinião para avaliar as percepções dos estudantes sobre
a motivação e a aprendizagem.
Os dados foram analisados por favorabilidade para cinco
replicações de problemas.
Os resultados da experiência mostraram que os estudantes
participantes tiveram boas percepções sobre a abordagem \ac{PBL},
o que confirma o potencial para utilização de \ac{PBL}
no ensino e aprendizagem da disciplina de Teoria
da Computação.}

\end{itemize}

Os últimos resultados obtidos nesta dissertação estão
em fase de organização para submissão como segue abaixo:

\begin{itemize}
\item{\publicacaoTemplate{Replicação}{Apresentar
resultados das replicações no semestre 2017.1};}
\item{\publicacaoTemplate{Comparativo}{Realizar comparativo em relação
ao tamanho do grupo tutorial};}
\item{\publicacaoTemplate{Comparativo}{Realizar comparativo em relação
a quantidade de tutores para a quantidade de estudantes};}
\item{\publicacaoTemplate{Detalhamento}{Detalhar especificamente
sobre as questões de estudo}.}
\end{itemize}

\section{Organização do trabalho}
Este trabalho está organizado em capítulos, seções e subseções.
Em cada capítulo que segue será inicialmente
apresentada uma pequena introdução do capítulo e um detalhamento
das seções do capítulo.

Além deste primeiro capítulo que apresenta uma introdução geral
para o trabalho;
o Capítulo~\ref{cap-revisao} contém a revisão bibliográfica;
o Capítulo~\ref{cap-estudo} descreve a metodologia da experiência em sala de aula;
o Capítulo~\ref{cap-resultados} apresenta os resultados obtidos nas pesquisas, bem
como discussão destes resultados;
e, por fim, o Capítulo~\ref{sec-conclusao} apresenta a conclusão e perspectivas
para trabalhos futuros.
