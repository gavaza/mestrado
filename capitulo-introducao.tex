\newcommand{\publicacaoTemplate}[2]{%
\textbf{#1} : #2}


\xchapter{Introdução}{} %sem preambulo
% É recomendável utilizar `\acresetall' no início de cada capítulo para reiníciar o contator de referências às siglas.
\acresetall
\label{cap-introducao}
Os educadores em Ciência da Computação enfrentam muitas questões no que diz
respeito a como conduzir uma disciplina de forma que a construção do conhecimento
por parte dos estudantes seja satisfatória.
Isto não é um caso particular do ensino de Computação,
mas é necessário considerar que se trata de
conhecimento relativamente novo, e que só muito
recentemente está presente em instituições de ensino universitário.
Devemos considerar que ainda não existem metodologias e procedimentos
suficientemente testados para o ensino de Computação.
Além disto, há a natureza intrinsecamente multidisciplinar e
o desenvolvimento contínuo da área.
Embora seja possível perceber um contínuo desenvolvimento também em
outras áreas do conhecimento, em Computação é notória a robustez e
continuidade do progresso produzido nas últimas décadas.

O ensino de Computação em geral segue a abordagem tradicional,
com educadores atuando como transmissores de conhecimento
enquanto estudantes são receptores.
Ao verificar as grades curriculares dos cursos de Computação,
serão encontradas algumas atividades práticas em laboratórios,
sobretudo para o ensino de disciplinas com tópicos de programação e
algumas atividades práticas como complemento à carga horária teórica,
constituída essencialmente na exposição de conhecimento pelo educador
aos estudantes.

A forma de ensinar Computação pode ser determinante
para o sucesso também no que diz respeito a
inclusão, esse é um motivo a mais de estímulo para
que os educadores pesquisem na área.

A disciplina de Teoria da Computação é uma parte muito importante
na formação dos estudantes dos cursos da área de Computação, uma 
vez que este é uma conhecimento que permite aos estudantes
entenderem em quais bases a computação está sustentada.
Na disciplina de Teoria da Computação os tópicos tratados
possuem um alto nível de abstração e exigem bastante esforço
por parte dos estudantes para que possam acompanhar o
andamento.
Muitas são as dificuldades para que os estudantes se sintam
motivados e consigam construir de forma satisfatória o entendimento
dos conceitos.
Nesse contexto, a disciplina possui um alto nível de evasão e
de reprovação.

\section{Proposta}
\label{sec-proposta}
Utilizar uma abordagem baseada em problemas em uma disciplina
introdutória de Teoria da Computação na Universidade Federal
da Bahia e verificar as percepções dos estudantes sobre a sua
motivação e aprendizagem.

\section{Hipóteses}
\label{sec-hipoteses}
Este trabalho se propõe validar as hipóteses a seguir.

\begin{enumerate}
\item{\label{h1ref} \hatexto;}
\item{\label{h2ref} \hbtexto;}
\item{\label{h3ref} \hctexto.}
\end{enumerate}

\section{Objetivos}
\label{sec-objetivos}

\subsection{Objetivo geral}
Aumentar a motivação e aprendizagem dos estudantes de Computação no estudo
de disciplinas teóricas, como Teoria da Computação, para reduzir a evasão e reprovação,
por meio de uma metodologia baseada em problemas.

\subsection{Objetivos específicos}
\begin{enumerate}
\item{\label{oe1ref} \oeatexto;}
\item{\label{oe2ref} \oebtexto;}
\item{\label{oe3ref} \oectexto;}
\item{\label{oe4ref} \oedtexto;}
\item{\label{oe5ref} \oeetexto;}
\item{\label{oe6ref} \oeftexto;}
\item{\label{oe7ref} \oegtexto.}
\end{enumerate}

\section{Justificativa}

\section{Publicações}
\label{sec-publicacoes}
Alguns dos resultados parciais desta dissertação já
foram publicados em conferência e revista científica,
como pode ser observado abaixo:
\begin{itemize}
\item{\textbf{Uma experiência de aplicação de uma
abordagem baseada em problemas no ensino
de Teoria da Computação em sala de aula tradicional}.
O trabalho relatou a experiência de aplicação de \ac{PBL}
para o ensino da disciplina introdutória de Teoria da
Computação utilizando uma infraestrutura básica na
\ac{UFBA} no semestre 2016.1.
Os resultados dessa experiência mostraram que os estudantes participantes
tiveram boas percepções sobre a abordagem utilizada, o que confirma o
grande potencial para aplicação de \ac{PBL} no ensino de
disciplinas teóricas como é o caso da disciplina introdutória de
Teoria da Computação.
Os resultados extraídos dos formulários respondido pelos estudantes
foram analisados na ótica da percepção
destes estudantes sobre os problemas.
Os dados foram consolidados e analisados em médias
de favorabilidade~\cite{gavaza2017}.}


\end{itemize}
Os últimos resultados obtidos nesta dissertação estão
em fase de organização para submissão como segue abaixo:

\begin{itemize}
\item{\publicacaoTemplate{Replicação}{Apresentar os
resultados para a replicação no semestre 2017.1};}
\item{\publicacaoTemplate{Comparativo}{Realizar comparativo em relação
ao tamanho do grupo tutorial};}
\item{\publicacaoTemplate{Comparativo}{Realizar comparativo em relação
a quantidade de tutores para a quantidade de estudantes};}
\item{\publicacaoTemplate{Detalhamento}{Detalhar especificamente
sobre as questões de estudo}.}
\end{itemize}

\section{Organização do trabalho}
Este trabalho está organizado em capítulos, seções e subseções.
Em cada capítulo que segue será inicialmente
apresentada uma pequena introdução do capítulo e um detalhamento
das seções do capítulo.

Além deste primeiro capítulo que apresenta uma introdução geral
para o trabalho;
o Capítulo~\ref{cap-revisao} contém a revisão bibliográfica;
o Capítulo~\ref{cap-estudo} descreve a metodologia da experiência em sala de aula;
o Capítulo~\ref{cap-resultados} apresenta os resultados obtidos nas pesquisas, bem
como discussão destes resultados;
e, por fim, o Capítulo~\ref{sec-conclusao} apresenta a conclusão e perspectivas
para trabalhos futuros.
