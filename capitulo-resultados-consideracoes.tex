\section{Considerações finais}
\label{sec-consideracoes-resultados}
% resumo básico
Neste capítulo foram apresentados os resultados do estudo no contexto de execução onde foi aplicada a
metodologia PBL em dez replicações de problemas em duas turmas de estudantes e apresentada a discussão
das hipóteses deste estudo.

% dificuldade em obter respostas
Uma das principais dificuldades deste trabalho foi adesão
por partes dos estudantes em responder os formulários,
como pode ser observado calculando a razão entre
a quantidade de respostas obtidas no formulário de
replicação dos problemas e a quantidade de estudantes
na turma.
A adesão ficou entre $14\%$ e $46\%$, na
quinta replicação do semestre 2017.1 e na primeira
replicação do semestre 2016.1, respectivamente.
Neste cálculo foi considerado a quantidade de estudantes
que iniciaram no semestre, portanto, não foi considerado os
estudantes que eventualmente já haviam desistido
da disciplina.
Acreditamos que a adesão poderia ser melhor se existisse
institucionalmente, desde o ingresso dos estudantes,
ampla divulgação aos estudantes da importância das pesquisas
científicas na construção do conhecimento e que os resultados,
como se propõe este estudo, podem trazer novas oportunidades
para o ensino.

Como foi destacado pelos revisores do nosso trabalho \cite{gavaza2017},
outra questão que devemos observar é que, na metodologia de pesquisa
que adotamos, pode existir uma maior possibilidade
de adesão em responder as pesquisas por partes
dos estudantes ``mais interessados'', portanto, poderiam
estes apresentar uma maior tendência a aceitação.
Embora entendamos a possibilidade desta relação, observamos que
neste estudo também existiram respostas em sentido
contrário, explicitamos um exemplo na discussão da
replicação do problema ``\ProblemaG'', no semestre 2017.1,
na Seção~\ref{sec-2017-p1}, onde exibimos a existência
de participantes que apresentaram discordância integral
para afirmativas.

% não comparativo
O experimental deste estudo não foi construído como comparativo
entre metodologias, portanto, não há indicação, nesse
contexto, para a utilização da metologia deste estudo
em substituição a alguma outra metodologia, inclusive uma
metodologia tradicional.

Além de uma infraestrutura adicional, para um estudo conter
aplicações de abordagens distintas em mais de uma turma,
se faz necessário construir ou investigar critérios de
comparabilidade, que também não foi o propósito deste estudo,
assim, a maioria das afirmativas utilizadas neste estudo
estão focadas em especificidades da metodologia PBL.

% resultados negativos

% resultados positivos
