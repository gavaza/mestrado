\section{Discussão das hipóteses}
A avaliação das hipóteses as quais este estudo se propõe,
utiliza os resultados apresentados neste capítulo.

As hipóteses são avaliadas para validade utilizando os mesmos
parâmetros conservadores utilizados na avaliação dos objetivos
específicos, descrito na Seção~\ref{sec-avaliacao-hipoteses}.

% HG1
\AvaliacaoHipotese{\hatexto}{h1ref}{}{``\LikertPX'' (X)}{}{}{}{}{}

\AprovacaoObjetivo{h}{não}{3}{0}{}
\AprovacaoHipoteseResultado{não}{\ProblemaC}{}{}{2016.1}{\ProblemaB}{\ProblemaE}{}{2017.1}
\ObjetivoFavorabilidadeDestaque{}{\ProblemaD}{2016.1}{87,5}{}{}
\ObjetivoFavorabilidadeDestaqueContinuidade{2017.1}{70,0}{}

Uma das formas que pode ser utilizada para manter a motivação
dos estudantes é utilizar uma abordagem de ensino e
aprendizagem que estes gostem.
É possível citar como por exemplo as abordagens que utilizam
jogos que tentam se justificar utilizando o contexto de que os
jovens normalmente gostam de participar de jogos, portanto,
a utilização de uma metodologia com jogos poderia motivar estes
estudantes.
Mas não devemos considerar que existe uma relação direta
entre os estudantes gostarem de uma abordagem de ensino
e aprendizagem com se sentirem motivados, por este motivo,
o resultado para essa hipótese possui uma relevância própria
nesse contexto.

% HG2
\AvaliacaoHipotese{\hbtexto}{h2ref}{hg}{``\LikertPF'' (F)}{}{}{}{}{}

\AprovacaoObjetivo{h}{não}{1}{0}{}
\AprovacaoHipoteseResultado{não}{\ProblemaE}{}{}{2017.1}{}{}{}{}
\ObjetivoFavorabilidadeDestaque{}{\ProblemaD}{2016.1}{100,0}{37,5}{}
\ObjetivoFavorabilidadeDestaqueContinuidade{2017.1}{70,0}{}

O excelente resultado afirma o potencial da metodologia
na percepção dos estudantes no que diz respeito
a aprendizagem no contexto em que foi aplicado.

% HG3
\AvaliacaoHipotese{\hctexto}{h3ref}{hg}{``\LikertPE'' (E)}{}{}{}{}{}

\AprovacaoObjetivo{h}{}{todas}{0}{}
\ObjetivoFavorabilidadeDestaque{}{\ProblemaA}{2016.1}{92,9}{42,9}{}
\ProblemaSemReplica{2017.1}{\ProblemaA}
\ObjetivoFavorabilidadeDestaqueOutra{}{\ProblemaG}{2017.1}{91,7}{25,0}
\ProblemaSemReplica{2016.1}{\ProblemaG}
\ObjetivoFavorabilidadeDestaqueOutra{}{\ProblemaB}{2016.1}{91,7}{50,0}
\ObjetivoFavorabilidadeDestaqueContinuidade{2017.1}{77,0}{}
\AprovacaoObjetivoResultado{}{}{}{}{}{}{}{}{}

No que diz respeito a motivação dos estudantes,
no contexto em que foi aplicado,
a metodologia obteve excelente resultado.
