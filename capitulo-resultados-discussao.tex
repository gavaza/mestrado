\section{Avaliação e discussão das hipóteses}
\label{sec-avaliacao-hipoteses}
Para avaliação das hipóteses as quais este estudo se propõe, utilizamos
os resultados apresentados neste capítulo.
Inicialmente avaliamos e discutimos os resultados para as hipóteses
específicas e depois os resultados para as hipóteses gerais.

Para a avaliação das hipóteses deste estudo, uma vez que a
metodologia utilizada exige que alguns critérios sejam
arbitrados, preferimos ser bastante conservadores
na definição dos parâmetros arbitrados.
No caso de uma hipótese como apenas uma afirmação, consideramos
a validade da hipótese se a afirmação recebe
resultado positivo dentro dos critérios mencionados na
Seção~\ref{sec-ref-graficos}, ou seja, $70\%$ de favorabilidade,
sendo favoráveis as respostas de concordância.
No caso de mais de uma afirmativa, consideramos a validade
da hipótese em caso de resultado positivo para todas
as afirmativas, dentro dos mesmos critérios de favorabilidade.

\subsection{Hipóteses específicas}
Destacamos que a avaliação das hipóteses foi realizada para
cada replicação de problema, ou seja, é possível concluir pela
validade da hipótese em um replicação de problema, enquanto pode
não ser possível concluir pela validade em outra replicação
de problema.

% HE1
\AvaliacaoHipotese{\heatexto}{he1ref}{he}{``\LikertPQ'' (Q)}{}{}{}{}{}

\AprovacaoHipotese{}{}{todas}{0}{}
\HipoteseFavorabilidadeDestaque{}{\ProblemaC}{2017.1}{100,0}{}{}
\HipoteseFavorabilidadeDestaqueContinuidade{2016.1}{100,0}{}
\HipoteseFavorabilidadeDestaqueDiferencas{2016.1}{25,0}{2017.1}{100,0}
\HipoteseFavorabilidadeDestaqueOutra{}{\ProblemaD}{2016.1}{100}{}
\HipoteseFavorabilidadeDestaqueContinuidade{2017.1}{90,0}{}
\AprovacaoHipoteseResultado{}{}{}{}{}{}{}{}{}

% HE2
\AvaliacaoHipotese{\hebtexto}{he2ref}{he}{``\LikertPM'' (M)}{}{}{}{}{}

\AprovacaoHipotese{}{não}{3}{0}{}
\AprovacaoHipoteseResultado{não}{\ProblemaB}{\ProblemaC}{\ProblemaE}{2016.1}{}{}{}{}
Em nosso trabalho \cite{gavaza2017}, que utilizamos apenas dados do
semestre 2016.1, apesar de não termos definidos parâmetros
de avaliação, percebemos que a percepção dos estudantes com relação
a motivação para o trabalho em equipe também não recebeu as melhores
avaliações.
No semestre 2016.1, como descrito na Seção ~\ref{sec-exp-2016}, foi utilizado
apenas um quadro branco, assim foi formado apenas um grupo com todos
os estudantes.
Para o semestre 2017.1, como descrito na Seção ~\ref{sec-exp-2017}, foram
utilizados quadros adicionais e foram formados grupos menores,
de até dez participantes cada.
\AprovacaoHipotese{}{}{todas}{0}{2017.1}
\HipoteseFavorabilidadeDestaque{}{\ProblemaE}{}{100,0}{85,7}{}
% escrever sobre a redução no grupo tutorial e qual a conclusão

% HE3
\AvaliacaoHipotese{\hectexto}{he3ref}{he}{``\LikertPT'' (T)}{}{}{}{}{}

\AprovacaoHipotese{}{}{todas}{0}{}
\HipoteseFavorabilidadeDestaque{}{\ProblemaC}{2016.1}{100,0}{50,0}{\ProblemaD}
\HipoteseFavorabilidadeDestaqueContinuidade{2017.1}{90,9}{80,0}
\HipoteseFavorabilidadeDestaqueOutra{}{\ProblemaG}{2017.1}{91,7}{91,7}
\ProblemaSemReplica{2016.1}{}
\AprovacaoHipoteseResultado{}{}{}{}{}{}{}{}{}

% HE4
\AvaliacaoHipotese{\hedtexto}{he4ref}{he}{``\LikertPC'' (C)}{``\LikertPD'' (D)}
{``\LikertPG'' (G)}{``\LikertPK'' (K)}{``\LikertPO'' (O)}{``\LikertPP'' (P)}

\HipoteseNaoAtende{\ProblemaG}{2017.1}{(C)}{(D)}{(O)}{}{}{}{}
\ProblemaSemReplica{2016.1}{\ProblemaG}
\HipoteseNaoAtende{\ProblemaB}{2017.1}{(C)}{(D)}{(P)}{}{}{}{}
\HipoteseAtende{\ProblemaB}{2016.1}{p}
\AprovacaoHipotese{}{não}{2}{1}{}
\AprovacaoHipoteseResultado{não}{\ProblemaG}{\ProblemaB}{}{2017.1}{}{}{}{}

% HE5
\AvaliacaoHipotese{\heetexto}{he5ref}{he}{``\LikertPU'' (U)}{``\LikertPV'' (V)}
{``\LikertPW'' (W)}{}{}{}

\HipoteseNaoAtende{\ProblemaG}{2017.1}{(W)}{}{}{}{}{}{}
\ProblemaSemReplica{2016.1}{\ProblemaG}
\HipoteseNaoAtende{\ProblemaB}{2017.1}{(W)}{}{}{}{}{}{}
\HipoteseAtende{\ProblemaB}{2016.1}{p}
\HipoteseNaoAtende{\ProblemaC}{2017.1}{(W)}{}{}{}{}{}{}
\HipoteseAtende{\ProblemaC}{2016.1}{p}
\HipoteseNaoAtende{\ProblemaE}{2017.1}{(W)}{}{}{}{}{}{}
\HipoteseAtende{\ProblemaE}{2016.1}{p}
A favorabilidade não foi atingida em 4 das 5 replicações no
semestre 2017.1.

% HE6
\AvaliacaoHipotese{\heftexto}{he6ref}{he}{``\LikertPI'' (I)}{}{}{}{}{}

\AprovacaoHipotese{}{}{todas}{0}{}
\HipoteseFavorabilidadeDestaque{}{\ProblemaC}{2017.1}{100,0}{90,9}{}
\HipoteseFavorabilidadeDestaqueContinuidade{2016.1}{75,0}{}
\MaisDestaque{4}{5}{2017.1}{85,8}{\ProblemaE}
\AprovacaoHipoteseResultado{}{}{}{}{}{}{}{}{}

% HE7
\AvaliacaoHipotese{\hegtexto}{he7ref}{he}{``\LikertPA'' (A)}{``\LikertPB'' (B)}{}{}{}{}

\HipoteseNaoAtende{\ProblemaE}{2016.1}{(B)}{}{}{}{}{}{}
\HipoteseAtende{\ProblemaE}{2017.1}{p}
\HipoteseNaoAtende{\ProblemaG}{2017.1}{(B)}{}{}{}{}{}{}
\ProblemaSemReplica{2017.1}{\ProblemaG}
\AprovacaoHipoteseResultado{não}{\ProblemaE}{}{}{2016.1}{\ProblemaG}{}{}{2017.1}

\subsection{Hipóteses gerais}

% HG1
\AvaliacaoHipotese{\hgatexto}{hg1ref}{hg}{``\LikertPX'' (X)}{}{}{}{}{}

\AprovacaoHipotese{}{não}{3}{0}{}
\AprovacaoHipoteseResultado{não}{\ProblemaC}{}{}{2016.1}{\ProblemaB}{\ProblemaE}{}{2017.1}
\HipoteseFavorabilidadeDestaque{}{\ProblemaD}{2016.1}{87,5}{}{}
\HipoteseFavorabilidadeDestaqueContinuidade{2017.1}{70,0}{}

% HG2
\AvaliacaoHipotese{\hgbtexto}{hg2ref}{hg}{``\LikertPF'' (F)}{}{}{}{}{}

\AprovacaoHipotese{}{não}{1}{0}{}
\AprovacaoHipoteseResultado{não}{\ProblemaE}{}{}{2017.1}{}{}{}{}
\HipoteseFavorabilidadeDestaque{}{\ProblemaD}{2016.1}{100,0}{37,5}{}
\HipoteseFavorabilidadeDestaqueContinuidade{2017.1}{70,0}{}

% HG3
\AvaliacaoHipotese{\hgctexto}{hg3ref}{hg}{``\LikertPE'' (E)}{}{}{}{}{}

\AprovacaoHipotese{}{}{todas}{0}{}
\HipoteseFavorabilidadeDestaque{}{\ProblemaA}{2016.1}{92,9}{42,9}{}
\ProblemaSemReplica{2017.1}{\ProblemaA}
\HipoteseFavorabilidadeDestaqueOutra{}{\ProblemaG}{2017.1}{91,7}{25,0}
\ProblemaSemReplica{2016.1}{\ProblemaG}
\HipoteseFavorabilidadeDestaqueOutra{}{\ProblemaB}{2016.1}{91,7}{50,0}
\HipoteseFavorabilidadeDestaqueContinuidade{2017.1}{77,0}{}
\AprovacaoHipoteseResultado{}{}{}{}{}{}{}{}{}

\subsection{Síntese das hipóteses específicas}
A Tabela~\ref{tabela-ref-hipoteses} apresenta
uma síntese das avaliações das hipóteses específicas.
\begin{table}[h]
\caption{Síntese de avaliação das hipóteses específicas}
\label{tabela-ref-hipoteses}
\begin{tabular}{p{6.5cm}|c|c|c|c|c|c|c|c|c|c}

\hfil\multirow{2}{*}{Hipótese}\hfill & \multicolumn{5}{c|}{2016.1} &
\multicolumn{5}{c}{2017.1} \\

\cline{2-11}
	 & PA & PB & PC & PD & PE &
           PG & PB & PC & PD & PE \\
\hline
% HE1
\allOK{\heatexto}
\hline
% HE2
\noAllOKA{\hebtexto}{1}{0}{0}{1}{0}
\noAllOKB{1}{1}{1}{1}{1}
\hline
% HE3
\allOK{\hectexto}
\hline
% HE4
\noAllOKA{\hedtexto}{1}{1}{1}{1}{1}
\noAllOKB{0}{0}{1}{1}{1}
\hline
% HE5
\noAllOKA{\heetexto}{1}{1}{1}{1}{1}
\noAllOKB{0}{0}{0}{1}{0}
\hline
% HE6
\allOK{\heftexto}
\hline
% HE7
\noAllOKA{\hegtexto}{1}{1}{1}{1}{0}
\noAllOKB{0}{1}{1}{1}{1}
\hline
\end{tabular}
\legendaTabelaSintese
\end{table}


É possível perceber que apenas o problema ``\ProblemaD'' obteve
validação para todas as hipóteses específicas em duas replicações, conforme os
critérios definidos, e que o
problema ``\ProblemaA'' obteve validação para todas as
hipóteses específicas na única replicação, conforme os
critérios definidos.

Existem dois principais resultados negativos facilmente
perceptíveis na síntese de avaliação das hipóteses
específicas.
O resultado para a hipótese em relação ao trabalho em equipe
para as replicações no semestre 2016.1 e para a hipótese
em relação contribuição positiva dos tutores nas
replicações do semestre 2017.1.
Os dois principais resultados negativos são justificáveis.

No caso da hipótese em relação ao trabalho em equipe
para as replicações no semestre 2016.1 o tamanho
do grupo tutorial é o principal argumento
para este resultado.

Entendemos que a divisão de grupos tutoriais menores
nas replicações do semestre 20171.1 contribuiu
positivamente para o trabalho em equipe dos
estudantes e consequentemente para a percepção destes
sobre este trabalho, assim, sugerimos que os
grupos tutoriais devem possuir entre cinco e dez estudantes,
como em nossa replicação do semestre 2017.1, para aumentar
as possibilidades dos resultados estarem dentro de critérios
semelhantes aos nossos.

No caso da hipótese em relação contribuição positiva dos tutores nas
replicações do semestre 2017.1 a quantidade de tutores para o total
de estudantes na turma é a justificativa.

Como não foi adotada nenhuma ação com relação ao aumento da
quantidade total de estudantes em relação ao semestre 2016.1,
de 26 para 50, quase o dobro, entendemos que os tutores não
puderam se aproximar mais efetivamente dos estudantes
nas replicações do semestre 2017.1.
Nossa experiência sugere que o ideal é disponibilizar um
tutor para facilitar a sessão tutorial para entre dez e
quinze estudantes.

As hipóteses específicas em relação a percepção dos estudantes sobre
a metodologia estimular a autoaprendizagem e sobre percepção de
conexões entre o conhecimento e o problema entendemos ser
resultados pontuais, que podem ser melhorados com adequações nos problemas
em que as hipóteses não foram validadas, uma vez que na maioria das replicações
as hipóteses específicas em questão foram validadas dentro dos critérios definidos.

\subsection{Síntese das hipóteses gerais}
A Tabela~\ref{tabela-ref-hipoteses-gerais} apresenta
uma síntese das avaliações das hipóteses gerais.
\begin{table}[h]
\caption{Síntese de avaliação das hipóteses gerais}
\label{tabela-ref-hipoteses-gerais}
\begin{tabular}{p{6.5cm}|c|c|c|c|c|c|c|c|c|c}

\hfil\multirow{2}{*}{Hipótese}\hfill & \multicolumn{5}{c|}{2016.1} &
\multicolumn{5}{c}{2017.1} \\

\cline{2-11}
	 & PA & PB & PC & PD & PE &
           PG & PB & PC & PD & PE \\
\hline
% HG1
\noAllOKA{\hgatexto}{1}{1}{0}{1}{1}
\noAllOKB{1}{0}{1}{1}{0}
\hline
% HG2
\noAllOKA{\hgbtexto}{1}{1}{1}{1}{1}
\noAllOKB{1}{1}{1}{1}{0}
\hline
% HG3
\allOK{\hgctexto}
\hline
\end{tabular}
\legendaTabelaSintese
\end{table}


No caso das hipóteses gerais, é possível perceber que apenas
o problema ``\ProblemaD'' obteve validação para todas as hipóteses
em duas replicações, conforme os critérios definidos, e que o
problema ``\ProblemaA'' obteve validação para todas as
hipóteses na única replicação, conforme os
critérios definidos.

Para as replicações pontuais onde a hipótese geral sobre
a aprovação da metodologia por parte dos estudantes não foi validada,
dentro dos critérios definidos, é possível que algumas adequações
nestes problemas sejam suficientes para
validação da hipótese.

Uma maior contribuição positiva por parte dos tutores pode ajudar
com relação a aprendizagem dos estudantes, assim, entendemos que
a hipótese geral sobre o problema ser capaz
de cumprir com os objetivos de aprendizagem pode
ser beneficiada com essa contribuição.
