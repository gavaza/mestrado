\section{Avaliação e discussão das hipóteses}
\label{sec-avaliacao-hipoteses}
Para avaliação das hipóteses as quais este estudo se propõe, utilizamos
os resultados apresentados neste capítulo.
Inicialmente avaliamos e discutimos os resultados para as hipóteses
específicas e depois os resultados para as hipóteses gerais.

Para a avaliação das hipóteses deste estudo, uma vez que a
metodologia utilizada exige que alguns critérios sejam
arbitrados, preferimos ser bastante conservadores
na definição dos parâmetros arbitrados.
No caso de uma hipótese como apenas uma afirmação, consideramos
a validade da hipótese se a afirmação recebe
resultado positivo dentro dos critérios mencionados na
Seção~\ref{sec-ref-graficos}, ou seja, $70\%$ de favorabilidade,
sendo favoráveis as respostas de concordância.
No caso de mais de uma afirmativa, consideramos a validade
da hipótese em caso de resultado positivo para todas
as afirmativas, dentro dos mesmos critérios de favorabilidade.

Destacamos que a avaliação das hipóteses foi realizada para
cada replicação de problema, ou seja, é possível concluir pela
validade da hipótese em um replicação de problema, enquanto pode
não ser possível concluir pela validade em outra replicação
de problema.

% HE1
\AvaliacaoHipotese{\heatexto}{he1ref}{he}{``\LikertPQ'' (Q)}{}{}{}{}{}

\AprovacaoHipotese{}{}{todas}{0}{}
\HipoteseFavorabilidadeDestaque{}{\ProblemaC}{2017.1}{100,0}{}{}
\HipoteseFavorabilidadeDestaqueContinuidade{2016.1}{100,0}{}
\HipoteseFavorabilidadeDestaqueDiferencas{2016.1}{25,0}{2017.1}{100,0}
\HipoteseFavorabilidadeDestaqueOutra{}{\ProblemaD}{2016.1}{100}{}
\HipoteseFavorabilidadeDestaqueContinuidade{2017.1}{90,0}{}
\AprovacaoHipoteseResultado{}{}{}{}{}{}{}{}{}

% HE2
\AvaliacaoHipotese{\hebtexto}{he2ref}{he}{``\LikertPM'' (M)}{}{}{}{}{}

\AprovacaoHipotese{}{não}{3}{0}{}
\AprovacaoHipoteseResultado{não}{\ProblemaB}{\ProblemaC}{\ProblemaE}{2016.1}{}{}{}{}
Em nosso trabalho \cite{gavaza2017}, que utilizamos apenas dados do
semestre 2016.1, apesar de não termos definidos parâmetros
de avaliação, percebemos que a percepção dos estudantes com relação
a motivação para o trabalho em equipe também não recebeu as melhores
avaliações.
No semestre 2016.1, como descrito na Seção ~\ref{sec-exp-2016}, foi utilizado
apenas um quadro branco, assim foi formado apenas um grupo com todos
os estudantes.
Para o semestre 2017.1, como descrito na Seção ~\ref{sec-exp-2017}, foram
utilizados quadros adicionais e foram formados grupos menores,
de até dez participantes cada.
\AprovacaoHipotese{}{}{todas}{0}{2017.1}
\HipoteseFavorabilidadeDestaque{}{\ProblemaE}{}{100,0}{85,7}{}
% escrever sobre a redução no grupo tutorial e qual a conclusão

% HE3
\AvaliacaoHipotese{\hectexto}{he3ref}{he}{``\LikertPT'' (T)}{}{}{}{}{}

\AprovacaoHipotese{}{}{todas}{0}{}
\HipoteseFavorabilidadeDestaque{}{\ProblemaC}{2016.1}{100,0}{50,0}{\ProblemaD}
\HipoteseFavorabilidadeDestaqueContinuidade{2017.1}{90,9}{80,0}
\HipoteseFavorabilidadeDestaqueOutra{}{\ProblemaG}{2017.1}{91,7}{91,7}
\ProblemaSemReplica{2016.1}{}
\AprovacaoHipoteseResultado{}{}{}{}{}{}{}{}{}

% HE4
\AvaliacaoHipotese{\hedtexto}{he4ref}{he}{``\LikertPC'' (C)}{``\LikertPD'' (D)}
{``\LikertPG'' (G)}{``\LikertPK'' (K)}{``\LikertPO'' (O)}{``\LikertPP'' (P)}

\HipoteseNaoAtende{\ProblemaG}{2017.1}{(C)}{(D)}{(O)}{}{}{}{}
\ProblemaSemReplica{2016.1}{\ProblemaG}
\HipoteseNaoAtende{\ProblemaB}{2017.1}{(C)}{(D)}{(P)}{}{}{}{}
\HipoteseAtende{\ProblemaB}{2016.1}{p}
\AprovacaoHipotese{}{não}{2}{1}{}
\AprovacaoHipoteseResultado{não}{\ProblemaG}{\ProblemaB}{}{2017.1}{}{}{}{}

% HE5
\AvaliacaoHipotese{\heetexto}{he5ref}{he}{``\LikertPU'' (U)}{``\LikertPV'' (V)}
{``\LikertPW'' (W)}{}{}{}

\HipoteseNaoAtende{\ProblemaG}{2017.1}{(W)}{}{}{}{}{}{}
\ProblemaSemReplica{2016.1}{\ProblemaG}
\HipoteseNaoAtende{\ProblemaB}{2017.1}{(W)}{}{}{}{}{}{}
\HipoteseAtende{\ProblemaB}{2016.1}{p}
\HipoteseNaoAtende{\ProblemaC}{2017.1}{(W)}{}{}{}{}{}{}
\HipoteseAtende{\ProblemaC}{2016.1}{p}
\HipoteseNaoAtende{\ProblemaE}{2017.1}{(W)}{}{}{}{}{}{}
\HipoteseAtende{\ProblemaE}{2016.1}{p}
A favorabilidade não foi atingida em 4 das 5 replicações no
semestre 2017.1.

% HE6
\AvaliacaoHipotese{\heftexto}{he6ref}{he}{``\LikertPI'' (I)}{}{}{}{}{}

\AprovacaoHipotese{}{}{todas}{0}{}
\HipoteseFavorabilidadeDestaque{}{\ProblemaC}{2017.1}{100,0}{90,9}{}
\HipoteseFavorabilidadeDestaqueContinuidade{2016.1}{75,0}{}
\MaisDestaque{4}{5}{2017.1}{85,8}{\ProblemaE}
\AprovacaoHipoteseResultado{}{}{}{}{}{}{}{}{}

% HE7
\AvaliacaoHipotese{\hegtexto}{he7ref}{he}{``\LikertPA'' (A)}{``\LikertPB'' (B)}{}{}{}{}

\HipoteseNaoAtende{\ProblemaE}{2016.1}{(B)}{}{}{}{}{}{}
\HipoteseAtende{\ProblemaE}{2017.1}{p}
\HipoteseNaoAtende{\ProblemaG}{2017.1}{(B)}{}{}{}{}{}{}
\ProblemaSemReplica{2017.1}{\ProblemaG}
\AprovacaoHipoteseResultado{não}{\ProblemaE}{}{}{2016.1}{\ProblemaG}{}{}{2017.1}

% HG1
\AvaliacaoHipotese{\hgatexto}{hg1ref}{hg}{``\LikertPX'' (X)}{}{}{}{}{}

\AprovacaoHipotese{}{não}{3}{0}{}
\AprovacaoHipoteseResultado{não}{\ProblemaC}{}{}{2016.1}{\ProblemaB}{\ProblemaE}{}{2017.1}
\HipoteseFavorabilidadeDestaque{}{\ProblemaD}{2016.1}{87,5}{}{}
\HipoteseFavorabilidadeDestaqueContinuidade{2017.1}{70,0}{}

% HG2
\AvaliacaoHipotese{\hgbtexto}{hg2ref}{hg}{``\LikertPF'' (F)}{}{}{}{}{}

\AprovacaoHipotese{}{não}{1}{0}{}
\AprovacaoHipoteseResultado{não}{\ProblemaE}{}{}{2017.1}{}{}{}{}
\HipoteseFavorabilidadeDestaque{}{\ProblemaD}{2016.1}{100,0}{37,5}{}
\HipoteseFavorabilidadeDestaqueContinuidade{2017.1}{70,0}{}

% HG3
\AvaliacaoHipotese{\hgctexto}{hg3ref}{hg}{``\LikertPE'' (E)}{}{}{}{}{}

\AprovacaoHipotese{}{}{todas}{0}{}
\HipoteseFavorabilidadeDestaque{}{\ProblemaA}{2016.1}{92,9}{42,9}{}
\ProblemaSemReplica{2017.1}{\ProblemaA}
\HipoteseFavorabilidadeDestaqueOutra{}{\ProblemaG}{2017.1}{91,7}{25,0}
\ProblemaSemReplica{2016.1}{\ProblemaG}
\HipoteseFavorabilidadeDestaqueOutra{}{\ProblemaB}{2016.1}{91,7}{50,0}
\HipoteseFavorabilidadeDestaqueContinuidade{2017.1}{77,0}{}
\AprovacaoHipoteseResultado{}{}{}{}{}{}{}{}{}


A Tabela~\ref{tabela-ref-hipoteses} apresenta
uma síntese da avaliação das hipóteses específicas.

\begin{table}[h]
\caption{Síntese de avaliação das hipóteses específicas}
\label{tabela-ref-hipoteses}
\begin{tabular}{p{6.5cm}|c|c|c|c|c|c|c|c|c|c}

\hfil\multirow{2}{*}{Hipótese}\hfill & \multicolumn{5}{c|}{2016.1} &
\multicolumn{5}{c}{2017.1} \\

\cline{2-11}
	 & PA & PB & PC & PD & PE &
           PG & PB & PC & PD & PE \\
\hline
% HE1
\allOK{\heatexto}
\hline
% HE2
\noAllOKA{\hebtexto}{1}{0}{0}{1}{0}
\noAllOKB{1}{1}{1}{1}{1}
\hline
% HE3
\allOK{\hectexto}
\hline
% HE4
\noAllOKA{\hedtexto}{1}{1}{1}{1}{1}
\noAllOKB{0}{0}{1}{1}{1}
\hline
% HE5
\noAllOKA{\heetexto}{1}{1}{1}{1}{1}
\noAllOKB{0}{0}{0}{1}{0}
\hline
% HE6
\allOK{\heftexto}
\hline
% HE7
\noAllOKA{\hegtexto}{1}{1}{1}{1}{0}
\noAllOKB{0}{1}{1}{1}{1}
\hline
\end{tabular}
\legendaTabelaSintese
\end{table}

