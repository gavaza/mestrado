\newcommand{\contribuicoesQualitativas}[7]{%
{\ifthenelse{\equal{#5}{}}{O problema ``#1'' foi
replicado apenas no semestre #2 e}{
Na replicação do problema ``#1'' no semestre #2}}
foram recebidas contribuições discursivas%
{\ifthenelse{\equal{#3}{zero}}{}{ de #3
participante{\ifthenelse{\equal{#3}{um}}{}{s}}
sobre o problema{\ifthenelse{\equal{#4}{zero}}{}{ e}}}}%
{\ifthenelse{\equal{#4}{zero}}{}{ de #4
participante{\ifthenelse{\equal{#4}{um}}{}{s}}
sobre a metodologia PBL}}.%
{\ifthenelse{\equal{#5}{}}{}{
No semestre #5
{\ifthenelse{\equal{#6}{zero}}{{\ifthenelse{\equal{#7}{zero}}{não }{}}{}}{}}%
foram recebidas contribuições discursivas%
{\ifthenelse{\equal{#6}{zero}}{}{ de #6
participante{\ifthenelse{\equal{#6}{um}}{}{s}}
sobre o problema{\ifthenelse{\equal{#7}{zero}}{}{ e}}}}%
{\ifthenelse{\equal{#7}{zero}}{}{ de #7
participante{\ifthenelse{\equal{#7}{um}}{}{s}}
sobre a metodologia PBL}}.}}}

\newcommand{\contribuicoesQualitativasDisciplina}[5]{%
Entre os #1 estudantes #2 do semestre #3
{\ifthenelse{\equal{#4}{zero}}{{\ifthenelse{\equal{#5}{zero}}{não }{}}{}}{}}%
foram recebidas contribuições discursivas%
{\ifthenelse{\equal{#4}{zero}}{}{ de #4
participante{\ifthenelse{\equal{#4}{um}}{}{s}}
sobre a metodologia PBL{\ifthenelse{\equal{#5}{zero}}{}{ e}}}}%
{\ifthenelse{\equal{#4}{zero}}{}{ de #5
participante{\ifthenelse{\equal{#5}{um}}{}{s}}
sobre a disciplina}}.%
}

\section{Respostas qualitativas}
\label{sec-sem-quali}

\subsection{\ProblemaA}
\contribuicoesQualitativas{\ProblemaA}{2016.1}{dois}{quatro}{}{}{}

Em relação ao problema ambos os participantes mencionam a complexidade.
Um participante diz que há dependência com os conceitos de música
e outro acredita que a complexidade está relacionada com este ser
o primeiro problema ao qual foram submetidos.

A metodologia foi avaliada positivamente, sendo
``interessante'', ``útil'', ``colaborativa'' e ``soluções rápidas''
as qualidaades mais mencionadas.
Foi destacada também que a metodologia precisa de
aprimoramentos.

\subsection{\ProblemaB}
\contribuicoesQualitativas{\ProblemaB}{2016.1}{dois}{dois}{2017.1}{seis}{cinco}

Para este problema os participantes mencionaram a não especificação clara
do problema, sobretuto os participantes do semestre 2016.1.
Apesar de os participantes terem recebido
uma explicação sobre a metodologia e terem participado
do problema anterior com a metodologia PBL, é necessário considerarmos
que os participantes estão envolvidos em um curso tradicional
em que as atividades que executam nas outras disciplinas são normalmente
especificações, ao contrário do que é recomendado nos princípios citados
em \cite{dolmans1997seven} e que foram utilizados para
a construção deste problema.

Ainda sobre o problema, no caso do semestre 2017.1 a dificuldade
do problema é o assunto mais mencionado.
Embora exista divergência entre os participantes
sobre o acréscimo de dificuldade, a maioria menciona que este
problema é mais díficil que o anterior, no caso, ``\ProblemaG''.

Com relação a metodologia, os participantes do semestre 2016.1
mencionam a necessidade de mais retorno por parte dos tutores,
embora as avaliações referentes ao tutores tenham sido
os melhores avaliados quantitativamente, como destacamos
na Seção~\ref{sec-sem-2016} na apresentação dos resultados
da replicação deste problema.

Os participantes do semestre 2017.1
apresentaram algumas avaliações positivas para a metodologia
e alguns pontos de atenção no que diz respeito ao
comportamento necessário dos estudantes para o bom
andamento da metodologia.
As palavras ``desperta'', ``enriquecedor'', ``bom'' e ``\textit{insights}''
foram mencionadas positivamente em relação a metodologia e ``perdidos''
foi a palavra mencionada negativamente em relação a metodologia.
Foi mencionado que das reuniões alguns estudantes não conseguem
extrair as informações necessárias e que faltou aula teórica
de como fazer o autômato antes da aplicação do problema.

O compromisso dos estudantes é necessário para
o andamento de uma disciplina, independente de qual seja
o método utilizado.
No caso do PBL, as interações entre os envolvidos no processo
se fazem extremamente necessárias, nesse contexto,
foi citado por três participantes do semestre 2017.1 as dificuldades
com relação ao trabalho em equipe.

\subsection{\ProblemaC}
\contribuicoesQualitativas{\ProblemaC}{2016.1}{um}{um}{2017.1}{três}{dois}

\subsection{\ProblemaD}
\contribuicoesQualitativas{\ProblemaD}{2016.1}{um}{zero}{2017.1}{três}{um}

\subsection{\ProblemaE}
\contribuicoesQualitativas{\ProblemaE}{2016.1}{um}{um}{2017.1}{um}{um}

\subsection{\ProblemaF}
\contribuicoesQualitativas{\ProblemaF}{2017.1}{dois}{dois}{2016.1}{zero}{zero}

\subsection{\ProblemaG}
\contribuicoesQualitativas{\ProblemaG}{2017.1}{dois}{dois}{}{}{}


\subsection{Concluintes}
\contribuicoesQualitativasDisciplina{11}{concluintes}{2016.1}{dois}{dois}

\contribuicoesQualitativasDisciplina{}{concluintes}{2017.1}{três}{três}

\subsection{Desistentes}
\contribuicoesQualitativasDisciplina{15}{desistentes}{2016.1}{um}{um}

\contribuicoesQualitativasDisciplina{}{desistentes}{2017.1}{zero}{zero}
