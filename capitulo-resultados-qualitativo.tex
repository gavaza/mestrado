\newcommand{\contribuicoesQualitativas}[7]{%
{\ifthenelse{\equal{#5}{}}{O problema ``#1'' foi
replicado apenas no semestre #2 e}{
Na aplicação do problema ``#1'' no semestre #2}}
foram recebidas contribuições discursivas%
{\ifthenelse{\equal{#3}{zero}}{}{ de #3
participante{\ifthenelse{\equal{#3}{um}}{}{s}}
sobre o problema{\ifthenelse{\equal{#4}{zero}}{}{ e}}}}%
{\ifthenelse{\equal{#4}{zero}}{}{ de #4
participante{\ifthenelse{\equal{#4}{um}}{}{s}}
sobre a abordagem \ac{PBL}}}.%
{\ifthenelse{\equal{#5}{}}{}{
No semestre #5
{\ifthenelse{\equal{#6}{zero}}{{\ifthenelse{\equal{#7}{zero}}{não }{}}{}}{}}%
foram recebidas contribuições discursivas%
{\ifthenelse{\equal{#6}{zero}}{}{ de #6
participante{\ifthenelse{\equal{#6}{um}}{}{s}}
sobre o problema{\ifthenelse{\equal{#7}{zero}}{}{ e}}}}%
{\ifthenelse{\equal{#7}{zero}}{}{ de #7
participante{\ifthenelse{\equal{#7}{um}}{}{s}}
sobre a abordagem \ac{PBL}}}.}}}

\newcommand{\contribuicoesQualitativasDisciplina}[5]{%
Entre os #1 estudantes #2 do semestre #3
{\ifthenelse{\equal{#4}{zero}}{{\ifthenelse{\equal{#5}{zero}}{não }{}}{}}{}}%
foram recebidas contribuições discursivas%
{\ifthenelse{\equal{#4}{zero}}{}{ de #4
participante{\ifthenelse{\equal{#4}{um}}{}{s}}
sobre a abordagem \ac{PBL}{\ifthenelse{\equal{#5}{zero}}{}{ e}}}}%
{\ifthenelse{\equal{#4}{zero}}{}{ de #5
participante{\ifthenelse{\equal{#5}{um}}{}{s}}
sobre a disciplina}}.%
}

\section{Discussão qualitativa}
\label{sec-sem-quali}

As contribuições dos participantes nas respostas discursivas do
formulário de percepção de problema e do formulário de percepções da disciplina,
que foram descritos no Capítulo~\ref{cap-estudo}, foram analisadas dentro do
contexto da bibliografia e da análise quantitativa das respostas
apresentadas para as afirmativas nestes formulários para construção
da discussão qualitativa apresentada nessa seção.

\subsection{\ProblemaA}
\contribuicoesQualitativas{\ProblemaA}{2016.1}{dois}{quatro}{}{}{}

Em relação ao problema ambos os participantes mencionam a complexidade.
Um participante diz que há dependência com os conceitos de música
e outro acredita que a complexidade está relacionada com este ser
o primeiro problema ao qual foram submetidos.

A abordagem foi avaliada positivamente, sendo
``interessante'', ``útil'', ``colaborativa'' e ``soluções rápidas''
as qualidades mais mencionadas.
Foi destacada também que a abordagem precisa de
aprimoramentos.

\subsection{\ProblemaB}
\contribuicoesQualitativas{\ProblemaB}{2016.1}{dois}{dois}{2017.1}{seis}{cinco}

Para este problema os participantes mencionaram a não especificação clara
do problema, sobretudo os participantes da aplicação do semestre 2016.1.
Apesar de os participantes terem recebido
uma explicação sobre a abordagem e terem participado
do problema anterior com a abordagem \ac{PBL}, é necessário considerarmos
que os participantes estão envolvidos em um curso tradicional
em que as atividades que executam nas outras disciplinas são normalmente
especificações, ao contrário do que é recomendado nos princípios
citados \citeonline{dolmans1997seven} e que foram utilizados para
a construção deste problema.

Ainda sobre o problema, no caso da aplicação do semestre 2017.1 a dificuldade
do problema é o assunto mais mencionado.
Embora exista divergência entre os participantes
sobre o acréscimo de dificuldade, a maioria menciona que este
problema é mais difícil que o anterior, no caso, ``\ProblemaG''.

Com relação a abordagem, os participantes da aplicação do semestre 2016.1
mencionam a necessidade de mais retorno por parte dos tutores,
embora as avaliações referentes ao tutores tenham sido
os melhores avaliados quantitativamente, como destacamos
na Seção~\ref{sec-sem-2016} na apresentação dos resultados
da aplicação deste problema.

Os participantes na aplicação deste problema no semestre 2017.1
apresentaram algumas avaliações positivas para a abordagem
e alguns pontos de atenção no que diz respeito ao
comportamento necessário dos estudantes para o bom
andamento da abordagem.
As palavras ``desperta'', ``enriquecedor'', ``bom'' e ``\textit{insights}''
foram mencionadas positivamente em relação a abordagem e ``perdidos''
foi a palavra mencionada negativamente em relação a abordagem.
Foi mencionado que das reuniões alguns estudantes não conseguem
extrair as informações necessárias e que faltou aula teórica
de como fazer o autômato antes da aplicação do problema.

O compromisso dos estudantes é necessário para
o andamento de uma disciplina, independente de qual seja
o método utilizado.
No caso do \ac{PBL}, as interações entre os envolvidos no processo
se fazem extremamente necessárias.
Nesse contexto,
foi citado por três participantes na aplicação do semestre 2017.1 as dificuldades
com relação ao trabalho em equipe.

\subsection{\ProblemaC}
\contribuicoesQualitativas{\ProblemaC}{2016.1}{um}{um}{2017.1}{três}{dois}

Na aplicação deste problema no semestre 2016.1
foi destacado o equilíbrio entre tempo e a dificuldade
que também é observado nos resultados quantitativos
na favorabilidade da afirmativa ``\LikertPL'' (L)
e mencionado uma maior participação
dos estudantes nesta aplicação.

No caso da aplicação no semestre 2017.1 a dificuldade
do problema é o assunto mais mencionado, onde o problema foi
qualificado com um dificuldade entre média e difícil.
Novamente apontado sobre o texto, acreditamos que o
envolvimento dos participantes em um curso
tradicional pode criar expectativas de que o texto
seja mais ``especificação''.

Sobre a abordagem \ac{PBL}, na aplicação do semestre 2017.1,
é mencionado a vontade em participar e o desafio trazido
pelo problema.
Também foi mencionado que outros estudantes não
demonstravam interesse e que isto de certa forma
contamina os estudantes interessados.

\subsection{\ProblemaD}
\contribuicoesQualitativas{\ProblemaD}{2016.1}{um}{zero}{2017.1}{três}{um}

Na aplicação do semestre 2016.1, sobre o problema, foi mencionado
sobre a necessidade de construção mais simples nas aulas
expositivas para melhor entendimento do problema.
Também foi mencionado sobre as inúmeras possibilidades, em consonância
com os resultados quantitativos onde essa característica foi
percebida, em algum nível, por todos os participantes.

Para a aplicação do problema no semestre 2017.1, foi mencionado
que este problema trata de um assunto real, assim como
os dados quantitativos, que podem ser observados
na afirmativa ``\LikertPH'' (H).
Também foi citado que a redução na quantidade de membros no grupo
tutorial, ocasionada pela ausência de estudantes, contribuiu
positivamente para as discussões.

Percebemos que alguns estudantes possuem um perfil mais passivo
de aprendizagem ou possuem outras obrigações que os impedem
de se dedicar em buscar conhecimento, nesse contexto,
um dos participantes, na aplicação do problema no
semestre 2017.1, menciona a ausência de conhecimento
sobre o assunto como barreira para a aprendizagem.
Este problema foi construído seguindo
os princípios citados em \citeonline{dolmans1997seven},
portanto, considerando os conhecimentos prévios
dos estudantes, mas como foi considerado um perfil
genérico para estes, uma vez que não foi
previamente verificado os seus perfis, alguns estudantes
podem não ser exatamente contemplados.
A qualidade da abordagem \ac{PBL} em estimular os estudantes
em buscar o conhecimento, como mencionado no trabalho
\citeonline{savery2015overview}, é mais efetivo quando
existe essa aderência de perfil com o problema.
O resultado quantitativo, referente a motivação dos
participantes (E), é excelente para esta
aplicação, assim, em análise conjunta com o
resultado qualitativo entendemos que o perfil
do participante em questão pode não ter sido
atingido pelo problema.

\subsection{\ProblemaE}
\contribuicoesQualitativas{\ProblemaE}{2016.1}{um}{um}{2017.1}{um}{um}

A dificuldade do problema foi mencionada em ambas as aplicações, mas
também foi mencionado como interessante.
Com relação a disciplina, foi mencionado, por participante da
aplicação do semestre 2016.1, que o problema deveria ser
mais extenso no conteúdo.
No caso de uma disciplina de Teoria da Computação, entendemos
que a construção de um problema com mais conceitos se
torna ainda mais complexa, em todos os sentidos, sobretudo
para conciliar as recomendações
no trabalho \citeonline{dolmans1997seven}.

\subsection{\ProblemaF}
\contribuicoesQualitativas{\ProblemaF}{2017.1}{dois}{dois}{2016.1}{zero}{zero}

Para este problema foi mencionada a dificuldade e a falta de estímulos
para resolver por se tratar de um assunto mais matemático.
Os conceitos deste problema exigem um nível de abstração bem elevado
dos estudantes, e por maiores que sejam os esforços na construção
do problema não é simples construir um problema capaz de cumprir com
os objetivos de aprendizagem e que ``escondam'' as dificuldades
em meio ao problema.

No que diz respeito à disciplina, é destacado ser desmotivante
discutir em relação um problema em aberto da área de Computação.
Embora para este problema não tenha sido realizada uma aplicação
completa da abordagem \ac{PBL}, um participante mencionou que não
conseguiu se adaptar, neste caso, entendemos que possa estar
comentando também em relação a outros problemas.

\subsection{\ProblemaG}
\contribuicoesQualitativas{\ProblemaG}{2017.1}{dois}{dois}{}{}{}

Foi mencionado sobre o problema ser extenso e
apresentar uma situação diferente das que as pessoas fariam em caso
real.
No caso da situação apresentar uma situação real (H), o resultado
quantitativo é apenas razoável, o que está de acordo
com o que foi mencionado pelo participante, assim, para utilizar
este problema em outras aplicações é interessante realizar adequações
da situação mencionada.

\subsection{Percepção dos concluintes}
\contribuicoesQualitativasDisciplina{11}{concluintes}{2016.1}{dois}{dois}
\contribuicoesQualitativasDisciplina{32}{concluintes}{2017.1}{três}{três}

Os comentários em relação a abordagem \ac{PBL} foram na maioria elogiosos.
Foi mencionado que a abordagem ajuda no processo de aprendizagem da
disciplina, além de interessante e motivadora.

A aprendizagem mais profunda dos conteúdos e a necessidade de mais
esforço por parte dos estudantes também foram destacadas.
Acreditamos que nesse contexto, o objetivo de fazer com que
os estudantes assumam também a responsabilidade pela a
sua aprendizagem fica explicitada.
Apesar disto, verificamos que algumas afirmativas relacionadas
com estímulos para desenvolvimento de habilidades para autoaprendizagem
não foram das melhores avaliadas em todas as aplicações,
como por exemplo, buscar outras referências de estudo (D),
estudar individualmente fora das sessões \ac{PBL} (O) e buscar
conhecimento por motivação (P), na aplicação dos
problemas ``\ProblemaG'' e ``\ProblemaB'' no semestre 2017.1.

A importância do grupo e das sessões tutoriais para a aprendizagem
também é mencionada em resposta discursiva de participante concluinte
e reforça essa característica da abordagem \ac{PBL}, como citado
em \citeonline{van2000motivation}.
Foi mencionado que a abordagem \ac{PBL} exige um trabalho em equipe por
parte dos estudantes e que alguns destes estudantes não participavam
durante as discussões, como consequência isto dificultava o
andamento de sessões.
De fato, como mencionado nos
trabalhos \citeonline{savery2015overview} e \citeonline{albanese2010problem}
e baseados nas percepções durante as aplicações, a eficiência da abordagem
também depende da colaboração entre os participantes.
Também percebemos a existência de estudantes que não estavam
completamente compromissados com o andamento da discussão e que
por vezes dificultavam o andamento das sessões tutoriais, sendo necessária
intervenção por parte dos tutores para retormar o foco das discussões.

Destacamos também comentário sobre a importância da assiduidade dos
estudantes para o desempenho destes.
Em nossa estratégia, como mencionamos na descrição da experiência
em sala de aula, no Capítulo~\ref{cap-estudo}, a assiduidade faz
parte da avaliação dos estudantes em cada uma das aplicações dos
problemas e serve também como uma ferramenta para que estes
percebam mais diretamente a importância mencionada.

Também foi mencionado como útil para a abordagem a criação de máquinas
para a abordagem \ac{PBL}, entretanto, não ficou claro em qual contexto
o estudante surge tal percepção por parte do estudante,
uma vez que foi percebido, em praticamente todas as aplicações, que
existiram tentativas de criação das máquinas por partes
dos estudantes.
Acreditamos que alguns estudantes, por estarem em um curso
tradicional, podem de certa forma estar bastante conectados
à imagem do educador tradicional como um repositório de
respostas ao ser confrontados com uma nova abordagem.
Na abordagem \ac{PBL}, o educador, no papel de tutor, é o facilitador
do processo, como mencionado em diversos trabalhos, como
\citeonline{hmelo2004problem} e \citeonline{savery2015overview}.
Em nossas percepções durante as aplicações, percebemos que
parece ser a melhor estrategia para manter a discussão, ter um
espaço para a manifestações com o máximo de liberdade possível,
ainda que inicialmente possa parecer que a intervenção
de alguns estudante não vá contribuir para
a discussão.
O tutor deve intervir, sem contrapor, dentro das possibilidades,
apenas se percebe que a maioria dos estudantes estão se afastando
dos objetivos de aprendizagem.

Um dos participantes menciona que entende não ter se adequado
à abordagem \ac{PBL} e este seria o motivo da reprovação por conceito,
apesar de ter destacado que a abordagem é ``interessante'' e ``inteligente''.
A nossa percepção, obtida na execução deste trabalho, é que a aplicação
da abordagem \ac{PBL} pode ser beneficiada com um mapeamento menos genérico
do perfil dos participantes, principalmente em cursos em que a maioria
das disciplinas utiliza uma abordagem tradicional para aprendizagem, que
é a situação da instituição de ensino onde replicamos este trabalho.

Alguns estudantes citaram a necessidade de mais aulas
expositivas ou de revisões de conteúdo para aplicação
da abordagem, que provavelmente surge também pelo
contexto deles estarem em um curso
predominantemente tradicional.
A abordagem \ac{PBL} se propõe ao inverso disto, entregando
aos estudantes autonomia maior para a construção
do conhecimento.
Entendemos que o mapeamento menos genérico do perfil
dos participantes antes da introdução da metologia \ac{PBL}
e um acompanhamento tempestivo das percepções dos estudantes
e de como estão evoluindo nas habilidades de buscar o conhecimento
pode ser útil para um balanceamento mais efetivo com
relação a carga de aulas expositivas e sessões tutoriais
em um contexto como o que aplicamos a abordagem.

Os comentários sobre a disciplina com a utilização da abordagem \ac{PBL}
apresentados nas respostas discursivas por participantes concluintes
é semelhante às nossas percepções.
Esse resultado está de acordo com os resultados do
trabalho \citeonline{sockalingam2011student}, mas não só sobre a eficácia
dos problemas para a abordagem, mas também no que diz respeito
à percepção geral em relação ao andamento da disciplina
com a abordagem \ac{PBL}.

\subsection{Percepção dos desistentes}
\contribuicoesQualitativasDisciplina{15}{desistentes}{2016.1}{um}{um}
\contribuicoesQualitativasDisciplina{18}{desistentes}{2017.1}{zero}{zero}

A contextualização da desistência dos estudantes é extremamente útil
para que os educadores e as instituições consigam desenvolver um
plano de ação mais efetivo contra a evasão.

Os dados das afirmativas, as questões discursivas e o perfil,
ainda que bem genérico, dos estudantes em estudo, indicam que
eles possuem dificuldades diversas para o estudo além
das horas da disciplina.

O estudante desistente que participou discursivamente mencionou a sua
principal dificuldade com a disciplina como sendo conciliar o
sua atividade profissional e demais atividades com
a demanda da disciplina.
A abordagem \ac{PBL} foi destacada como ``interessante''
pelo participante e ele, apesar da desistência,
menciona que gostou da disciplina.

A caracterização mais detalhada do perfil dos estudantes
antes da aplicação da abordagem \ac{PBL} também é mencionada
na contribuição do estudante desistente, entendemos
que muitos são os benefícios que um mapeamento mais detalhado
pode trazer para a aplicação da abordagem \ac{PBL}, como
mencionamos na discussão das respostas discursivas dos estudantes
concluintes.
