\newcommand{\contribuicoesQualitativas}[7]{%
{\ifthenelse{\equal{#5}{}}{O problema ``#1'' foi
replicado apenas no semestre #2 e}{
Na replicação do problema ``#1'' no semestre #2}}
foram recebidas contribuições discursivas%
{\ifthenelse{\equal{#3}{zero}}{}{ de #3
participante{\ifthenelse{\equal{#3}{um}}{}{s}}
sobre o problema{\ifthenelse{\equal{#4}{zero}}{}{ e}}}}%
{\ifthenelse{\equal{#4}{zero}}{}{ de #4
participante{\ifthenelse{\equal{#4}{um}}{}{s}}
sobre a metodologia PBL}}.%
{\ifthenelse{\equal{#5}{}}{}{
No semestre #5
{\ifthenelse{\equal{#6}{zero}}{{\ifthenelse{\equal{#7}{zero}}{não }{}}{}}{}}%
foram recebidas contribuições discursivas%
{\ifthenelse{\equal{#6}{zero}}{}{ de #6
participante{\ifthenelse{\equal{#6}{um}}{}{s}}
sobre o problema{\ifthenelse{\equal{#7}{zero}}{}{ e}}}}%
{\ifthenelse{\equal{#7}{zero}}{}{ de #7
participante{\ifthenelse{\equal{#7}{um}}{}{s}}
sobre a metodologia PBL}}.}}}

\newcommand{\contribuicoesQualitativasDisciplina}[5]{%
Entre os #1 estudantes #2 do semestre #3
{\ifthenelse{\equal{#4}{zero}}{{\ifthenelse{\equal{#5}{zero}}{não }{}}{}}{}}%
foram recebidas contribuições discursivas%
{\ifthenelse{\equal{#4}{zero}}{}{ de #4
participante{\ifthenelse{\equal{#4}{um}}{}{s}}
sobre a metodologia PBL{\ifthenelse{\equal{#5}{zero}}{}{ e}}}}%
{\ifthenelse{\equal{#4}{zero}}{}{ de #5
participante{\ifthenelse{\equal{#5}{um}}{}{s}}
sobre a disciplina}}.%
}

\section{Respostas discursivas}
\label{sec-sem-quali}

\subsection{\ProblemaA}
\contribuicoesQualitativas{\ProblemaA}{2016.1}{dois}{quatro}{}{}{}

Em relação ao problema ambos os participantes mencionam a complexidade.
Um participante diz que há dependência com os conceitos de música
e outro acredita que a complexidade está relacionada com este ser
o primeiro problema ao qual foram submetidos.

A metodologia foi avaliada positivamente, sendo
``interessante'', ``útil'', ``colaborativa'' e ``soluções rápidas''
as qualidades mais mencionadas.
Foi destacada também que a metodologia precisa de
aprimoramentos.

\subsection{\ProblemaB}
\contribuicoesQualitativas{\ProblemaB}{2016.1}{dois}{dois}{2017.1}{seis}{cinco}

Para este problema os participantes mencionaram a não especificação clara
do problema, sobretudo os participantes da replicação do semestre 2016.1.
Apesar de os participantes terem recebido
uma explicação sobre a metodologia e terem participado
do problema anterior com a metodologia PBL, é necessário considerarmos
que os participantes estão envolvidos em um curso tradicional
em que as atividades que executam nas outras disciplinas são normalmente
especificações, ao contrário do que é recomendado nos princípios citados
em \cite{dolmans1997seven} e que foram utilizados para
a construção deste problema.

Ainda sobre o problema, no caso da replicação do semestre 2017.1 a dificuldade
do problema é o assunto mais mencionado.
Embora exista divergência entre os participantes
sobre o acréscimo de dificuldade, a maioria menciona que este
problema é mais difícil que o anterior, no caso, ``\ProblemaG''.

Com relação a metodologia, os participantes da replicação do semestre 2016.1
mencionam a necessidade de mais retorno por parte dos tutores,
embora as avaliações referentes ao tutores tenham sido
os melhores avaliados quantitativamente, como destacamos
na Seção~\ref{sec-sem-2016} na apresentação dos resultados
da replicação deste problema.

Os participantes na replicação deste problema no semestre 2017.1
apresentaram algumas avaliações positivas para a metodologia
e alguns pontos de atenção no que diz respeito ao
comportamento necessário dos estudantes para o bom
andamento da metodologia.
As palavras ``desperta'', ``enriquecedor'', ``bom'' e ``\textit{insights}''
foram mencionadas positivamente em relação a metodologia e ``perdidos''
foi a palavra mencionada negativamente em relação a metodologia.
Foi mencionado que das reuniões alguns estudantes não conseguem
extrair as informações necessárias e que faltou aula teórica
de como fazer o autômato antes da aplicação do problema.

O compromisso dos estudantes é necessário para
o andamento de uma disciplina, independente de qual seja
o método utilizado.
No caso do PBL, as interações entre os envolvidos no processo
se fazem extremamente necessárias, nesse contexto,
foi citado por três participantes na replicação do semestre 2017.1 as dificuldades
com relação ao trabalho em equipe.

\subsection{\ProblemaC}
\contribuicoesQualitativas{\ProblemaC}{2016.1}{um}{um}{2017.1}{três}{dois}

Na replicação deste problema no semestre 2016.1
foi destacado o equilíbrio entre tempo e a dificuldade
que também é observado nos resultados quantitativos
na favorabilidade da afirmativa ``\LikertPL'' (L)
e mencionado uma maior participação
dos estudantes nesta replicação.

No caso da replicação no semestre 2017.1 a dificuldade
do problema é o assunto mais mencionado, onde o problema foi
qualificado com um dificuldade entre média e difícil.
Novamente apontado sobre o texto, acreditamos que o
envolvimento dos participantes em um curso
tradicional pode criar expectativas de que o texto
seja mais ``especificação''.

Sobre a metodologia PBL, na replicação do semestre 2017.1,
é mencionado a vontade em participar e o desafio trazido
pelo problema.
Também foi mencionado que outros estudantes não
demonstravam interesse e que isto de certa forma
contamina os estudantes interessados.

\subsection{\ProblemaD}
\contribuicoesQualitativas{\ProblemaD}{2016.1}{um}{zero}{2017.1}{três}{um}

Na replicação do semestre 2016.1, sobre o problema, foi mencionado
sobre a necessidade de construção mais simples nas aulas
expositivas para melhor entendimento do problema.
Também foi mencionado sobre as inúmeras possibilidades, em consonância
com os resultados qualitativos onde essa característica foi
percebida, em algum nível, por todos os participantes.

Para a replicação do problema no semestre 2017.1, foi mencionado
que este problema trata de um assunto real, assim como
os dados quantitativos, que podem ser observados
na afirmativa ``\LikertPH'' (H).
Também foi citado que a redução na quantidade de membros no grupo
tutorial, ocasionada pela ausência de estudantes, contribuiu
positivamente para as discussões.

Percebemos que alguns estudantes possuem um perfil mais passivo
de aprendizagem ou possuem outras obrigações que os impedem
de se dedicar em buscar conhecimento, nesse contexto,
um dos participantes, na replicação do problema no
semestre 2017.1, menciona a ausência de conhecimento
sobre o assunto como barreira para a aprendizagem.
Este problema foi construído seguindo
os princípios citados em \cite{dolmans1997seven},
portanto, considerando os conhecimentos prévios
dos estudantes, mas como foi considerado um perfil
genérico para estes, uma vez que não foi
traçado previamente os seus perfis, alguns estudantes
podem não ser exatamente contemplados.
A qualidade da metodologia PBL em estimular os estudantes
em buscar o conhecimento, como mencionado no trabalho
\cite{savery2015overview} é mais efetivo quando
existe essa aderência de perfil com o problema.
O resultado quantitativo, referente a motivação dos
participantes (E), é excelente para esta
replicação, assim, em análise conjunta com o
resultado qualitativo entendemos que o perfil
do participante em questão pode não ter sido
atingido pelo problema.

\subsection{\ProblemaE}
\contribuicoesQualitativas{\ProblemaE}{2016.1}{um}{um}{2017.1}{um}{um}

A dificuldade do problema foi mencionada em ambas as replicações, mas
também foi mencionado como interessante.
Com relação a disciplina, foi mencionado, por participante da
replicação do semestre 2016.1, que o problema deveria ser
mais extenso no conteúdo.
No caso de uma disciplina de Teoria da Computação, entendemos
que a construção de um problema com mais conceitos se
torna ainda mais complexa, em todos os sentidos, sobretudo
para conciliar as recomendações
no trabalho \cite{dolmans1997seven}.

\subsection{\ProblemaF}
\contribuicoesQualitativas{\ProblemaF}{2017.1}{dois}{dois}{2016.1}{zero}{zero}

Para este problema foi mencionada a dificuldade e a falta de estímulos
para resolver por ser um assunto mais matemático.
Os conceitos deste problema exigem um nível de abstração bem elevado
dos estudantes, e por maiores que sejam os esforços na construção
do problema não é simples construir um problema capaz de cumprir com
os objetivos de aprendizagem e que ``escondam'' as dificuldades
em meio ao problema.

No que diz respeito a disciplina, é destacado que a dificuldade
em sentir que possui uma resposta é um desmotivante.
Embora para este problema não tenha sido realizada uma replicação
completa da metodologia PBL, um participante mencionou que não
conseguiu se adaptar, neste caso, entendemos que possa estar
comentando também em relação a outros problemas.

\subsection{\ProblemaG}
\contribuicoesQualitativas{\ProblemaG}{2017.1}{dois}{dois}{}{}{}

Foi mencionado sobre o problema ser extenso e
apresentar uma situação diferente das que as pessoas fariam em caso
real.
O resultado quantitativo em relação ao tempo (L)
foi muito bem avaliado nesta replicação, então, entendemos que
para este caso pode ser uma percepção isolada.
No caso da situação apresentar uma situação real (H), o resultado
quantitativo é apenas razoável, o que está de acordo
com o que foi mencionado pelo participante, assim, para utilizar
este problema em outras replicações é interessante realizar adequações
da situação mencionada.

\subsection{Concluintes}
\contribuicoesQualitativasDisciplina{11}{concluintes}{2016.1}{dois}{dois}
\contribuicoesQualitativasDisciplina{}{concluintes}{2017.1}{três}{três}

Os comentários em relação a metodologia PBL foram na maioria elogiosos.
Foi mencionado que a metodologia ajuda no processo de aprendizagem da
disciplina, além de interessante e motivadora.

A aprendizagem mais profunda dos conteúdos e a necessidade de mais
esforço por parte dos estudantes também foram destacadas.
Acreditamos que nesse contexto, o objetivo de fazer com que
os estudantes assumam também a responsabilidade pela a
sua aprendizagem fica explicitada.
Apesar disto, verificamos que algumas afirmativas relacionadas
com estímulos para desenvolvimento de habilidades para autoaprendizagem
não foram das melhores avaliadas em todas as replicações,
como por exemplo, buscar outras referências de estudo (D),
estudar individualmente fora das sessões PBL (O) e buscar
conhecimento por motivação (P), na replicação dos
problemas ``\ProblemaG'' e ``\ProblemaB'' no semestre 2017.1.

Foi mencionado que alguns estudantes durantes as discussões
não participavam e que isto dificultava o andamento da sessão.
De fato, como mencionado nos
trabalhos \cite{savery2015overview} e \cite{albanese2010problem}
e baseados nas percepções durante as replicações, a eficiência da metodologia
também depende da colaboração entre os participantes.

Não está claro em qual contexto o estudante menciona como útil
para a metodologia PBL a criação de máquinas, uma vez que
foi percebido, em praticamente todas as replicações, que
existiram tentativas de criação das máquinas por partes
dos estudantes.
Acreditamos que alguns estudantes, por estarem em um curso
tradicional, podem de certa forma estarem bastante conectados
à imagem do professor tradicional como um repositório de
respostas.
Na metodologia PBL, o professor, no papel de tutor, é o facilitador
do processo, como mencionado em diversos trabalhos, como
\cite{hmelo2004problem} e \cite{savery2015overview}.
Em nossas percepções durante as replicações, percebemos que
parece ser a melhor estratégia para manter a discussão, ter um
espaço para a manifestações com o máximo de liberdade possível,
ainda que inicialmente possa parecer que a intervenção
de alguns estudante não vá contribuir para
a discussão.
O tutor deve intervir, sem contrapor, dentro das possibilidades,
apenas se percebe que a maioria dos estudantes estão se afastando
dos objetivos de aprendizagem.

Um dos participantes menciona que entende não ter se adequado
a metodologia PBL e este seria o motivo da reprovação por conceito,
apesar de ter destacado que a metodologia é ``interessante'' e ``inteligente''.
A nossa percepção, obtida na execução deste trabalho, é que a replicação
da metodologia PBL pode ser beneficiada com um mapeamento menos genérico
do perfil dos participantes, principalmente em cursos em que a maioria
das disciplinas utiliza uma abordagem tradicional para aprendizagem, que
é a situação da instituição de ensino onde replicamos este trabalho.

\subsection{Desistentes}
\contribuicoesQualitativasDisciplina{15}{desistentes}{2016.1}{um}{um}
\contribuicoesQualitativasDisciplina{x}{desistentes}{2017.1}{zero}{zero}
