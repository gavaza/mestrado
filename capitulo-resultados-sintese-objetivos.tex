\subsection{Síntese dos objetivos específicos}
A Tabela~\ref{tabela-ref-objetivos} apresenta
uma síntese das avaliações dos objetivos específicos.
\begin{table}[h]
\caption{Síntese de avaliação dos objetivos específicos}
\label{tabela-ref-objetivos}
\begin{tabular}{p{6.5cm}|c|c|c|c|c|c|c|c|c|c}

\hfil\multirow{2}{*}{Objetivo}\hfill & \multicolumn{5}{c|}{2016.1} &
\multicolumn{5}{c}{2017.1} \\

\cline{2-11}
	 & PA & PB & PC & PD & PE &
           PG & PB & PC & PD & PE \\
\hline
% HE1
\allOK{\oeatexto}
\hline
% HE2
\noAllOKA{\oebtexto}{1}{0}{0}{1}{0}
\noAllOKB{1}{1}{1}{1}{1}
\hline
% HE3
\allOK{\oectexto}
\hline
% HE4
\noAllOKA{\oedtexto}{1}{1}{1}{1}{1}
\noAllOKB{0}{0}{1}{1}{1}
\hline
% HE5
\noAllOKA{\oeetexto}{1}{1}{1}{1}{1}
\noAllOKB{0}{0}{0}{1}{0}
\hline
% HE6
\allOK{\oeftexto}
\hline
% HE7
\noAllOKA{\oegtexto}{1}{1}{1}{1}{0}
\noAllOKB{0}{1}{1}{1}{1}
\hline
\end{tabular}
\legendaTabelaSintese
\end{table}


É possível perceber que apenas o problema ``\ProblemaD'' obteve
avaliação positiva para todos os objetivos específicos em duas
replicações, conforme os critérios definidos, e que o
problema ``\ProblemaA'' obteve avaliação positiva para todos as
objetivos específicos na única replicação, conforme os
critérios definidos.

Existem dois principais resultados negativos facilmente
perceptíveis na síntese de avaliação dos objetivos
específicos.
O resultado para o objetivo específico em relação ao trabalho em equipe
para as replicações no semestre 2016.1 e para o objetivo
específico em relação contribuição positiva dos tutores nas
replicações do semestre 2017.1.
Os dois principais resultados negativos são justificáveis.

No caso do objetivo específico em relação ao trabalho em equipe
para as replicações no semestre 2016.1 o tamanho
do grupo tutorial é o principal argumento
para este resultado.

Entendemos que a divisão de grupos tutoriais menores
nas replicações do semestre 20171.1 contribuiu
positivamente para o trabalho em equipe dos
estudantes e consequentemente para a percepção destes
sobre este trabalho, assim, sugerimos que os
grupos tutoriais devem possuir entre cinco e dez estudantes,
como em nossa replicação do semestre 2017.1, para aumentar
as possibilidades dos resultados estarem dentro de critérios
semelhantes aos nossos.

No caso do objetivo específico em relação contribuição positiva
dos tutores nas replicações do semestre 2017.1 a quantidade de
tutores para o total de estudantes na turma é a justificativa.

Como não foi adotada nenhuma ação com relação ao aumento da
quantidade total de estudantes em relação ao semestre 2016.1,
de 26 para 50, quase o dobro, entendemos que os tutores não
puderam se aproximar mais efetivamente dos estudantes
nas replicações do semestre 2017.1.
Nossa experiência sugere que o ideal é disponibilizar um
tutor para facilitar a sessão tutorial para entre dez e
quinze estudantes.

Os objetivos específicos em relação à percepção dos estudantes sobre
a metodologia estimular a autoaprendizagem e sobre percepção de
conexões entre o conhecimento e o problema entendemos ser
resultados pontuais, que podem ser melhorados com adequações nos problemas
onde os resultados não foram positivos, uma vez que na maioria das replicações
os objetivos específicos em questão foram avaliados positivamente
dentro dos critérios definidos.
