\newcommand{\resultadoTurma}[8]{%
No semestre #1, entre os #2 estudantes matriculados que
iniciaram a disciplina, #3 estudantes foram reprovados por
falta, #4 estudantes utilizaram o processo legal de trancamento
e desistiram da disciplina antes da conclusão, assim, uma
evasão aproximada de $#5\%$.

Entre #6 estudantes concluintes, #7 estudantes foram aprovados
e #8 estudantes foram reprovados por conceito.
}

\newcommand{\perfilProblema}[4]{
O problema ``#1'' foi o Problema #2 aplicado no semestre #3.
Para esta aplicação foram #4 participantes a responder o questionário
de pesquisa sobre a percepção do problema.}

% caracterização de perfil dos participantes
\newcommand{\perfilParticipanteBase}[6]{com a idade média do
conjunto de participantes de #1 anos, onde o mais jovem possui
#2 anos e o mais velho possui #3 anos.
A maioria dos participantes está cursando #6,
onde {\ifthenelse{\equal{#4}{0}}{nenhum dos}{{\ifthenelse{\equal{#4}{100,0}}{todos os}{#4\% dos}}}} participantes
já desistiu desta disciplina de estudo antes e {\ifthenelse{\equal{#5}{0}}{nenhum dos}{{\ifthenelse{\equal{#5}{100,0}}{todos os}{#5\% dos}}}} participantes
já desistiu de outras disciplinas.}


\newcommand{\perfilParticipante}[7]{Este conjunto de participantes
é predominantemente constituído de
participantes do sexo masculino, sendo apenas uma participante do
sexo feminino#7, \perfilParticipanteBase{#1}{#2}{#3}{#4}{#5}{#6}}

\newcommand{\perfilParticipanteA}[7]{Este conjunto de participantes
tem maioria de participantes do sexo masculino,
sendo do sexo feminino#7 participantes, \perfilParticipanteBase{#1}{#2}{#3}{#4}{#5}{#6}}

\newcommand{\perfilParticipanteB}[7]{Este conjunto
tem participantes do sexo masculino e feminino em #7 proporção,
\perfilParticipanteBase{#1}{#2}{#3}{#4}{#5}{#6}}

\newcommand{\perfilParticipanteC}[7]{Este conjunto
tem participantes apenas do sexo masculino,
\perfilParticipanteBase{#1}{#2}{#3}{#4}{#5}{#6}}

\newcommand{\perfilParticipanteD}[5]{O participante
é do sexo #1, tem #2 anos, está cursando o #3 semestre%
{\ifthenelse{\equal{#4}{}}{}{
#4 desistiu da disciplina de estudo antes e #5
desistiu de outros disciplinas antes}}.}

% gráfico de percepção dos participantes
\newcommand{\figuraPercepcaoParticipante}[3]{

A Figura~\ref{percep-#1} apresenta um gráfico com os resultados referentes
as percepções dos participantes na aplicação do
Problema #2 no semestre #3.

\begin{figure}[!htb]
\centering
\includegraphics[scale=0.22]{question-#1.eps}
\caption{Percepções dos participantes do semestre #3 sobre o Problema #2}
\label{percep-#1}
\end{figure}
}

% gráfico de percepção dos participantes para a disciplina
\newcommand{\figuraPercepcaoParticipanteDisciplina}[3]{

A Figura~\ref{percep-#1} apresenta um gráfico com os resultados referentes
a percepção geral dos participantes #2 do semestre #3 sobre a disciplina.

\begin{figure}[!htb]
\centering
\includegraphics[scale=0.22]{#1.eps}
\caption{Percepções dos participantes #2 do semestre #3 sobre a disciplina}
\label{percep-#1}
\end{figure}
}

% gráficos de avaliação dos participantes
\newcommand{\figuraPercepcaoParticipanteNotasBase}[3]{

A Figura~\ref{aval-#1} apresenta um gráfico com a
avaliação dos participantes para o Problema #2 aplicado no semestre #3.

\begin{figure}[!htb]
\centering
\includegraphics[scale=0.18]{notas-#1.eps}
\caption{Avaliação dos participantes do semestre #3 para o Problema #2}
\label{aval-#1}
\end{figure}}

\newcommand{\figuraPercepcaoParticipanteNotas}[7]{
\figuraPercepcaoParticipanteNotasBase{#1}{#2}{#3}
Podemos observar que a maioria dos participantes atribuíram
notas altas para o Problema #2 no semestre #3, assim, #7 $#4\%$ das notas
foram maiores ou iguais a $7,00$ e nenhuma nota foi menor que $#5$, com uma média
de $#6$.}

\newcommand{\figuraPercepcaoParticipanteNotasA}[7]{
Apesar de nas afirmações de percepções o Problema #2 no semestre
#3 não ter obtido os melhores resultados, no que diz respeito
as notas para avaliação dos problemas pelo participante,
#7 $#4\%$ das notas foram maiores ou iguais a $7,00$, com
uma média de $#6$.}

% gráficos de avaliação dos participantes para a disciplina
\newcommand{\figuraPercepcaoParticipanteDisciplinaNotasBase}[3]{

A Figura~\ref{aval-#1} apresenta um gráfico com a
avaliação dos participantes #2 do semestre #3 sobre a disciplina.

\begin{figure}[!htb]
\centering
\includegraphics[scale=0.18]{notas-#1.eps}
\caption{Avaliação dos participantes #2 do semestre #3 sobre a disciplina}
\label{aval-#1}
\end{figure}}

\newcommand{\figuraPercepcaoParticipanteDisciplinaNotas}[7]{
\figuraPercepcaoParticipanteDisciplinaNotasBase{#1}{#2}{#3}
Podemos observar que a maioria dos participantes #2 atribuíram
notas altas para avaliar a disciplina no semestre #3, onde,%
{\ifthenelse{\equal{#7}{}}{}{ #7}}
{\ifthenelse{\equal{#4}{100,0}}{todas as}{$#4\%$ das}}
notas foram maiores ou iguais a $7,00$%
{\ifthenelse{\equal{#5}{}}{}{ e nenhuma nota foi menor que $#5$}},
com uma média de $#6$.}

\newcommand{\figuraPercepcaoParticipanteDisciplinaNotasA}[7]{
\figuraPercepcaoParticipanteDisciplinaNotasBase{#1}{#2}{#3}
Podemos observar que os participantes #2 não atribuíram
notas altas para avaliar a disciplina no semestre #3, onde,%
{\ifthenelse{\equal{#7}{}}{}{ #7}}
{\ifthenelse{\equal{#4}{100,0}}{todas as}{$#4\%$ das}}
notas foram maiores ou iguais a $7,00$%, 
{\ifthenelse{\equal{#5}{}}{}{, mas nenhuma nota foi menor que $#5$}},
com uma média de $#6$.}

\newcommand{\SemFiguraPercepcaoParticipanteDisciplinaNotas}[2]{
Como existe dados para apenas um participante #1
em avaliação sobre a disciplina no semestre #2, não foi
construído gráfico.}

\newcommand{\AvaliacaoHipoteseBase}[3]{Para avaliação da hipótese ``#1'', referenciada na
Seção~\ref{sec-hipoteses} como item~\ref{#2} de hipóteses
{\ifthenelse{\equal{#3}{he}}{específicas}{gerais}}, foi
considerada a favorabilidade para os resultados
de percepção dos estudantes para}

\newcommand{\AvaliacaoHipotese}[9]{\AvaliacaoHipoteseBase{#1}{#2}{#3}
\AfirmativasSeparador{#4}{#5}{#6}{#7}{#8}{#9}{}{}{}.}

\newcommand{\AprovacaoHipotese}[5]{Ao analisar os resultados obtidos para as respostas
dos participantes {\ifthenelse{\equal{#5}{}}{}{do semestre #5 }}para
a{\ifthenelse{\equal{#4}{1}}{s}{}}
afirmativa{\ifthenelse{\equal{#4}{1}}{s}{}} de avaliação
da hipótese, identificamos que a favorabilidade
para #1a{\ifthenelse{\equal{#4}{1}}{s}{}}
afirmativa{\ifthenelse{\equal{#4}{1}}{s}{}}
{\ifthenelse{\equal{#2}{}}{}{#2 }}atende{\ifthenelse{\equal{#4}{1}}{m}{}} os critérios
previamente definidos em {\ifthenelse{\equal{#3}{1}}{apenas uma replicação}{#3 replicações}}.}

\newcommand{\AprovacaoHipoteseResultado}[9]{Os resultados
{\ifthenelse{\equal{#1}{}}{}{#1 }}permitem concluir
sobre a validade da hipótese com os critérios
definidos{\ifthenelse{\equal{#2}{}}{}{ {\ifthenelse{\equal{#3}{}}
{para o problema ``#2''}
{{\ifthenelse{\equal{#4}{}}
{para os problemas ``#2'' e ``#3''}
{para os problemas ``#2'', ``#3'' e ``#4''}}}}}}%
{\ifthenelse{\equal{#5}{}}{}{ na replicação do semestre #5}}%
{\ifthenelse{\equal{#6}{}}{}{%
{\ifthenelse{\equal{#7}{}}%
{ e para o problema ``#6''}
{\ifthenelse{\equal{#8}{}}
{ e para os problemas ``#6'' e ``#7''}
{ e para os problemas ``#6'', ``#7'' e ``#8''}}} no semestre #9}}.}

\newcommand{\HipoteseFavorabilidade}[4]{Para a afirmativa ``#1''
na replicação do problema ``#2'' no
semestre #3 a percepção de favorabilidade,
nos critérios definidos, ficou em $#4\%$.}

\newcommand{\HipoteseFavorabilidadeDestaque}[6]{O{\ifthenelse{\equal{#6}{}}{}{s}}
principa{\ifthenelse{\equal{#6}{}}{l}{is}} destaque{\ifthenelse{\equal{#6}{}}{}{s}} para esta avaliação
est{\ifthenelse{\equal{#6}{}}{á}{ão}}{\ifthenelse{\equal{#1}{}}{}{ na afirmativa``#1''}} na
replicação do problema ``#2''%
{\ifthenelse{\equal{#6}{}}{}{ e do problema ``#6''}}%
{\ifthenelse{\equal{#3}{}}{}{ no semestre #3}} que a percepção de favorabilidade,
nos critérios definidos, {\ifthenelse{\equal{#6}{}}{teve um}{tiveram}}
atingimento {\ifthenelse{\equal{#4}{100,0}}{integral}{de $#4\%$}}%
{\ifthenelse{\equal{#5}{}}{}{\ifthenelse{\equal{#5}{100,0}}{, com consenso dos
participantes sobre a concordância}{, sendo $#5\%$ a
concordância integral}}}.}

\newcommand{\HipoteseFavorabilidadeDestaqueContinuidade}[3]{No
semestre #1 a percepção de favorabilidade para a afirmação em
destaque, para {\ifthenelse{\equal{#3}{}}{este mesmo problema}{os mesmos problemas}},
{\ifthenelse{\equal{#3}{}}{teve um}{tiveram, respectivamente,}}
atingimento {\ifthenelse{\equal{#2}{100,0}}{integral}{de $#2\%$}}%
{\ifthenelse{\equal{#3}{}}{}{{\ifthenelse{\equal{#3}{100,0}}{ também integral}{ e $#3\%$}}}}.}

\newcommand{\HipoteseFavorabilidadeDestaqueDiferencas}[4]{No
caso do semestre #1 {\ifthenelse{\equal{#2}{100,0}}
{não houve concordância em partes}{a concordância em partes foi de $#2\%$}},
enquanto no semestre #3 {\ifthenelse{\equal{#4}{100,0}}
{não houve concordância em partes}{a concordância em partes foi de $#4\%$}}.}

\newcommand{\HipoteseFavorabilidadeDestaqueOutra}[5]{Outro destaque
para esta avaliação está{\ifthenelse{\equal{#1}{}}{}{na afirmativa``#1''}} na
replicação do problema ``#2'' no
semestre #3 que a percepção de favorabilidade,
nos critérios definidos, teve um
atingimento {\ifthenelse{\equal{#4}{100,0}}{integral}{de $#4\%$}}%
{\ifthenelse{\equal{#5}{}}{}%
{{\ifthenelse{\equal{#5}{#4}}{, sem concordância em partes}%
{, sendo de $#5\%$ a concordância em partes}}}}.}

\newcommand{\ProblemaSemReplica}[2]{O problema%
{\ifthenelse{\equal{#2}{}}{}{ ``#2''}}
não foi replicado no semestre #1.}

\newcommand{\HipoteseNaoAtende}[9]{%
Na replicação do problema ``#1''%
{\ifthenelse{\equal{#2}{}}{}{, no semestre ``#2'',}}
a favorabilidade não atinge os critérios para
\AfirmativasSeparador{#3}{#4}{#5}{#6}{#7}{#8}{#9}{}{}.}

\newcommand{\AfirmativasSeparador}[9]{%
a{\ifthenelse{\equal{#2}{}}{}{s}}
afirmativa{\ifthenelse{\equal{#2}{}}{}{s}}
#1%
\Separador{#1}{#2}{#3}{#4}{#5}{#6}{#7}{#8}{#9}}

\newcommand{\Separador}[9]{%
{\ifthenelse{\equal{#2}{}}{}{{\ifthenelse{\equal{#3}{}}{ e #2}%
{, #2{\ifthenelse{\equal{#4}{}}{ e #3}{%
{, #3{\ifthenelse{\equal{#5}{}}{ e #4}{%
{, #4{\ifthenelse{\equal{#6}{}}{ e #5}{%
{, #5{\ifthenelse{\equal{#7}{}}{ e #6}{%
{, #6{\ifthenelse{\equal{#8}{}}{ e #7}{%
{, #7{\ifthenelse{\equal{#9}{}}{ e #8}{, #8  e #9}}}}}}}}}}}}}}}}}}}}}}

\newcommand{\HipoteseAtende}[3]{%
Na replicação do problema ``#1''%
{\ifthenelse{\equal{#2}{}}{}{, no semestre #2,}}
a favorabilidade atinge os critérios para
{\ifthenelse{\equal{#3}{s}}{a afirmativa}{todas as afirmativas}}.}

\newcommand{\MaisDestaque}[5]{%
Também é destaque que em #1 das #2 replicações o atingimento da
favorabilidade foi integral no semestre #3, sendo o resultado menos
favorável de $#4\%$ no problema ``#5''.}

\newcommand{\QtdParticipantes}[4]{%
No semestre #1, entre os #2 estudantes que
{\ifthenelse{\equal{#3}{0}}{desistiram da}{concluíram a}}
disciplina, {\ifthenelse{\equal{#4}{1}}%
{apenas um participante respondeu}%
{foram #4 participantes a responder}}
o questionário de pesquisa sobre
a percepção geral da disciplina.}

\newcommand{\AprovacaoDisciplina}[7]{%
{\ifthenelse{\equal{#6}{}}{No}{Para o semestre #6, os #7, no}}
que diz respeito a #3 condução da disciplina (#5),
{\ifthenelse{\equal{#1}{100,0}}{todos os}%
{{\ifthenelse{\equal{#1}{0,0}}{nenhum dos}{$#1\%$ dos}}}}
participantes{\ifthenelse{\equal{#2}{}}{}{ #2}}
mencionaram, em algum nível,
{\ifthenelse{\equal{#2}{desistentes}}%
{desistência por desagrado}{satisfação}} com #4.}

\newcommand{\AprovacaoDisciplinaA}[3]{Para
{\ifthenelse{\equal{#1}{100,0}}{todos os}%
{{\ifthenelse{\equal{#1}{0,0}}{nenhum dos}{$#1\%$ dos}}}}
participantes a consideração foi que, em algum nível, #2 (#3).}

\newcommand{\AprovacaoDisciplinaB}[5]{%
Foi expressado o desejo de, em algum nível, #2 (#3) por
{\ifthenelse{\equal{#1}{100,0}}{todos os}%
{{\ifthenelse{\equal{#1}{0,0}}{nenhum dos}{$#1\%$ dos}}}}
participantes%
{\ifthenelse{\equal{#5}{}}{}{,{\ifthenelse{\equal{#5}{0,0}}{ sem nenhuma
contrariedade declarada}{onde #4 $#5\%$ discorda, em
algum nível, desta opção}}}}.}

\newcommand{\AprovacaoDisciplinaC}[2]{%
A maioria dos participantes responderam que gostaram de
ter a presença (CB) e a participação (CC) em sala de aula
como critérios de avaliação sendo, respectivamente,
$#1\%$ e $#2\%$ os que concordaram, em algum nível, com as afirmações.}

\newcommand{\AprovacaoDisciplinaD}[3]{%
{\ifthenelse{\equal{#1}{100,0}}{Todos os}%
{{\ifthenelse{\equal{#1}{0,0}}{Nenhum dos}{}}}}
participantes {{\ifthenelse{\equal{#1}{0,0}}{diz}{dizem}}} gostar de #2 (#3).}

\newcommand{\AprovacaoDisciplinaE}[3]{%
Dizem gostar de #2 (#3) $#1\%$ dos participantes.}

\newcommand{\AprovacaoDisciplinaF}[4]{%
A maioria dos participantes
entende que{\ifthenelse{\equal{#4}{}}{}{ #4}} #2
(#3), onde
{\ifthenelse{\equal{#1}{100,0}}{todos os}%
{{\ifthenelse{\equal{#1}{0,0}}{nenhum dos}{$#1\%$}}}}
participantes concordaram, em algum nível, com a afirmação.}

\newcommand{\AprovacaoDisciplinaG}[5]{%
No semestre #1, #2 entre os #3 participantes
que concluíram a disciplina consideraram a
atividade profissional deles durante o semestre (#5),
para $#4\%$ destes foi possível, em algum nível,
conciliar este trabalho com a disciplina.}

\newcommand{\falseH}{$\bigodot$}
\newcommand{\trueH}{\checkmark}


\newcommand{\allOK}[1]{%
#1 & \trueH & \trueH & \trueH &
\trueH & \trueH & \trueH & \trueH &
\trueH & \trueH & \trueH \\
}

\newcommand{\noAllOKA}[6]{%
#1 & {\ifthenelse{\equal{#2}{0}}{\falseH}{\trueH}} &
{\ifthenelse{\equal{#3}{0}}{\falseH}{\trueH}} &
{\ifthenelse{\equal{#4}{0}}{\falseH}{\trueH}} &
{\ifthenelse{\equal{#5}{0}}{\falseH}{\trueH}} &
{\ifthenelse{\equal{#6}{0}}{\falseH}{\trueH}} &
}

\newcommand{\noAllOKB}[5]{%
{\ifthenelse{\equal{#1}{0}}{\falseH}{\trueH}} &
{\ifthenelse{\equal{#2}{0}}{\falseH}{\trueH}} &
{\ifthenelse{\equal{#3}{0}}{\falseH}{\trueH}} &
{\ifthenelse{\equal{#4}{0}}{\falseH}{\trueH}} &
{\ifthenelse{\equal{#5}{0}}{\falseH}{\trueH}}\\
}
