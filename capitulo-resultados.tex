% incluir template de discussões
\newcommand{\resultadoTurma}[8]{%
No semestre #1, entre os #2 estudantes matriculados que
iniciaram a disciplina, #3 estudantes foram reprovados por
falta, #4 estudantes utilizaram o processo legal de trancamento
e desistiram da disciplina antes da conclusão, assim, uma
evasão aproximada de $#5\%$.

Entre #6 estudantes concluintes, #7 estudantes foram aprovados
e #8 estudantes foram reprovados por conceito.
}

\newcommand{\perfilProblema}[4]{
O problema ``#1'' foi o Problema #2 aplicado no semestre #3.
Para esta aplicação foram #4 participantes a responder o questionário
de pesquisa sobre a percepção do problema.}

% caracterização de perfil dos participantes
\newcommand{\perfilParticipanteBase}[6]{com a idade média do
conjunto de participantes de #1 anos, em que o mais jovem possui
#2 anos e o mais velho possui #3 anos.
A maioria dos participantes está cursando #6,
sendo que {\ifthenelse{\equal{#4}{0}}{nenhum dos}{{\ifthenelse{\equal{#4}{100,0}}{todos os}{#4\% dos}}}} participantes
já desistiu desta disciplina de estudo antes e {\ifthenelse{\equal{#5}{0}}{nenhum dos}{{\ifthenelse{\equal{#5}{100,0}}{todos os}{#5\% dos}}}} participantes
já desistiu de outras disciplinas.}


\newcommand{\perfilParticipante}[7]{Este conjunto
é predominantemente constituído de
participantes do sexo masculino, sendo apenas uma participante do
sexo feminino#7, \perfilParticipanteBase{#1}{#2}{#3}{#4}{#5}{#6}}

\newcommand{\perfilParticipanteA}[7]{Este conjunto de participantes
tem maioria de participantes do sexo masculino,
sendo do sexo feminino#7 participantes, \perfilParticipanteBase{#1}{#2}{#3}{#4}{#5}{#6}}

\newcommand{\perfilParticipanteB}[7]{Este conjunto
tem participantes do sexo masculino e feminino em #7 proporção,
\perfilParticipanteBase{#1}{#2}{#3}{#4}{#5}{#6}}

\newcommand{\perfilParticipanteC}[7]{Este conjunto
tem participantes apenas do sexo masculino,
\perfilParticipanteBase{#1}{#2}{#3}{#4}{#5}{#6}}

\newcommand{\perfilParticipanteD}[5]{O participante
é do sexo #1, tem #2 anos, está cursando o #3 semestre%
{\ifthenelse{\equal{#4}{}}{}{
#4 desistiu da disciplina de estudo antes e #5
desistiu de outros disciplinas antes}}.}

% gráfico de percepção dos participantes
\newcommand{\figuraPercepcaoParticipante}[3]{

A Figura~\ref{percep-#1} apresenta um gráfico com os resultados referentes
as percepções dos participantes na aplicação do
Problema #2 no semestre #3.

\begin{figure}[!htb]
\centering
\includegraphics[scale=0.22]{question-#1.eps}
\caption{Percepções dos participantes do semestre #3 sobre o Problema #2}
\label{percep-#1}
\end{figure}
}

% gráfico de percepção dos participantes para a disciplina
\newcommand{\figuraPercepcaoParticipanteDisciplina}[3]{

A Figura~\ref{percep-#1} apresenta um gráfico com os resultados referentes
a percepção geral dos participantes #2 do semestre #3 sobre a disciplina.

\begin{figure}[!htb]
\centering
\includegraphics[scale=0.22]{#1.eps}
\caption{Percepções dos participantes #2 do semestre #3 sobre a disciplina}
\label{percep-#1}
\end{figure}
}

% gráficos de avaliação dos participantes
\newcommand{\figuraPercepcaoParticipanteNotasBase}[3]{

A Figura~\ref{aval-#1} apresenta um gráfico com a
avaliação dos participantes para o Problema #2 aplicado no semestre #3.

\begin{figure}[!htb]
\centering
\includegraphics[scale=0.18]{notas-#1.eps}
\caption{Avaliação dos participantes do semestre #3 para o Problema #2}
\label{aval-#1}
\end{figure}}

\newcommand{\figuraPercepcaoParticipanteNotas}[7]{
\figuraPercepcaoParticipanteNotasBase{#1}{#2}{#3}
Podemos observar que a maioria dos participantes atribuíram
notas altas para o Problema #2 no semestre #3, assim, #7 $#4\%$ das notas
foram maiores ou iguais a $7,00$ e nenhuma nota foi menor que $#5$, com uma média
de $#6$.}

\newcommand{\figuraPercepcaoParticipanteNotasA}[7]{
\figuraPercepcaoParticipanteNotasBase{#1}{#2}{#3}
Apesar de nas afirmações de percepções o Problema #2 no semestre
#3 não ter obtido os melhores resultados, no que diz respeito
as notas para avaliação dos problemas pelo participante,
#7 $#4\%$ das notas foram maiores ou iguais a $7,00$, com
uma média de $#6$.}

% gráficos de avaliação dos participantes para a disciplina
\newcommand{\figuraPercepcaoParticipanteDisciplinaNotasBase}[3]{

A Figura~\ref{aval-#1} apresenta um gráfico com a
avaliação dos participantes #2 do semestre #3 sobre a disciplina.

\begin{figure}[!htb]
\centering
\includegraphics[scale=0.18]{notas-#1.eps}
\caption{Avaliação dos participantes #2 do semestre #3 sobre a disciplina}
\label{aval-#1}
\end{figure}}

\newcommand{\figuraPercepcaoParticipanteDisciplinaNotas}[7]{
\figuraPercepcaoParticipanteDisciplinaNotasBase{#1}{#2}{#3}
Podemos observar que a maioria dos participantes #2 atribuíram
notas altas para avaliar a disciplina no semestre #3, sendo que%
{\ifthenelse{\equal{#7}{}}{}{ #7}}
{\ifthenelse{\equal{#4}{100,0}}{todas as}{$#4\%$ das}}
notas foram maiores ou iguais a $7,00$%
{\ifthenelse{\equal{#5}{}}{}{ e nenhuma nota foi menor que $#5$}},
com uma média de $#6$.}

\newcommand{\figuraPercepcaoParticipanteDisciplinaNotasA}[7]{
\figuraPercepcaoParticipanteDisciplinaNotasBase{#1}{#2}{#3}
Podemos observar que os participantes #2 não atribuíram
notas altas para avaliar a disciplina no semestre #3, sendo que%
{\ifthenelse{\equal{#7}{}}{}{ #7}}
{\ifthenelse{\equal{#4}{100,0}}{todas as}{$#4\%$ das}}
notas foram maiores ou iguais a $7,00$%, 
{\ifthenelse{\equal{#5}{}}{}{, mas nenhuma nota foi menor que $#5$}},
com uma média de $#6$.}

\newcommand{\SemFiguraPercepcaoParticipanteDisciplinaNotas}[2]{
Como existe dados para apenas um participante #1
em avaliação sobre a disciplina no semestre #2, não foi
construído gráfico.}

\newcommand{\AvaliacaoObjetivoBase}[3]{Para avaliação do objetivo específico ``#1'',
referenciada na Seção~\ref{sec-objetivos} como item~\ref{#2},
foi considerada a favorabilidade para os resultados
de percepção dos estudantes para}

\newcommand{\AvaliacaoHipoteseBase}[3]{Para avaliação da hipótese ``#1'',
referenciada na Seção~\ref{sec-hipoteses} como item~\ref{#2},
foi considerada a favorabilidade para os resultados
de percepção dos estudantes para}

\newcommand{\AvaliacaoObjetivo}[9]{\AvaliacaoObjetivoBase{#1}{#2}{#3}
\AfirmativasSeparador{#4}{#5}{#6}{#7}{#8}{#9}{}{}{}.}

\newcommand{\AvaliacaoHipotese}[9]{\AvaliacaoHipoteseBase{#1}{#2}{#3}
\AfirmativasSeparador{#4}{#5}{#6}{#7}{#8}{#9}{}{}{}.}

\newcommand{\AprovacaoObjetivo}[5]{Ao analisar os resultados obtidos para as respostas
dos participantes {\ifthenelse{\equal{#5}{}}{}{do semestre #5 }}para
a{\ifthenelse{\equal{#4}{1}}{s}{}}
afirmativa{\ifthenelse{\equal{#4}{1}}{s}{}} de avaliação
{\ifthenelse{\equal{#1}{h}}{da hipótese}{do objetivo}},
identificamos que a favorabilidade
para a{\ifthenelse{\equal{#4}{1}}{s}{}}
afirmativa{\ifthenelse{\equal{#4}{1}}{s}{}}
{\ifthenelse{\equal{#2}{}}{}{#2 }}atende{\ifthenelse{\equal{#4}{1}}{m}{}} os critérios
previamente definidos em {\ifthenelse{\equal{#3}{1}}{apenas uma replicação}{#3 replicações}}.}

\newcommand{\AprovacaoObjetivoResultado}[9]{Os resultados
{\ifthenelse{\equal{#1}{}}{}{#1 }}permitem concluir
sobre o resultado positivo para o objetivo com os critérios
definidos{\ifthenelse{\equal{#2}{}}{}{ {\ifthenelse{\equal{#3}{}}
{para o problema ``#2''}
{{\ifthenelse{\equal{#4}{}}
{para os problemas ``#2'' e ``#3''}
{para os problemas ``#2'', ``#3'' e ``#4''}}}}}}%
{\ifthenelse{\equal{#5}{}}{}{ na replicação do semestre #5}}%
{\ifthenelse{\equal{#6}{}}{}{%
{\ifthenelse{\equal{#7}{}}%
{ e para o problema ``#6''}
{\ifthenelse{\equal{#8}{}}
{ e para os problemas ``#6'' e ``#7''}
{ e para os problemas ``#6'', ``#7'' e ``#8''}}} no semestre #9}}.}

\newcommand{\AprovacaoHipoteseResultado}[9]{Os resultados
{\ifthenelse{\equal{#1}{}}{}{#1 }}permitem concluir
sobre a validade da hipótese com os critérios
definidos{\ifthenelse{\equal{#2}{}}{}{ {\ifthenelse{\equal{#3}{}}
{para o problema ``#2''}
{{\ifthenelse{\equal{#4}{}}
{para os problemas ``#2'' e ``#3''}
{para os problemas ``#2'', ``#3'' e ``#4''}}}}}}%
{\ifthenelse{\equal{#5}{}}{}{ na replicação do semestre #5}}%
{\ifthenelse{\equal{#6}{}}{}{%
{\ifthenelse{\equal{#7}{}}%
{ e para o problema ``#6''}
{\ifthenelse{\equal{#8}{}}
{ e para os problemas ``#6'' e ``#7''}
{ e para os problemas ``#6'', ``#7'' e ``#8''}}} no semestre #9}}.}

\newcommand{\ObjetivoFavorabilidade}[4]{Para a afirmativa ``#1''
na replicação do problema ``#2'' no
semestre #3 a percepção de favorabilidade,
nos critérios definidos, ficou em $#4\%$.}

\newcommand{\ObjetivoFavorabilidadeDestaque}[6]{O{\ifthenelse{\equal{#6}{}}{}{s}}
principa{\ifthenelse{\equal{#6}{}}{l}{is}} destaque{\ifthenelse{\equal{#6}{}}{}{s}} para esta avaliação
est{\ifthenelse{\equal{#6}{}}{á}{ão}}{\ifthenelse{\equal{#1}{}}{}{ na afirmativa``#1''}} na
replicação do problema ``#2''%
{\ifthenelse{\equal{#6}{}}{}{ e do problema ``#6''}}%
{\ifthenelse{\equal{#3}{}}{}{ no semestre #3}} que a percepção de favorabilidade,
nos critérios definidos, {\ifthenelse{\equal{#6}{}}{teve um}{tiveram}}
atingimento {\ifthenelse{\equal{#4}{100,0}}{integral}{de $#4\%$}}%
{\ifthenelse{\equal{#5}{}}{}{\ifthenelse{\equal{#5}{100,0}}{, com consenso dos
participantes sobre a concordância}{, sendo $#5\%$ a
concordância integral}}}.}

\newcommand{\ObjetivoFavorabilidadeDestaqueContinuidade}[3]{No
semestre #1 a percepção de favorabilidade para a afirmação em
destaque, para {\ifthenelse{\equal{#3}{}}{este mesmo problema}{os mesmos problemas}},
{\ifthenelse{\equal{#3}{}}{teve um}{tiveram, respectivamente,}}
atingimento {\ifthenelse{\equal{#2}{100,0}}{integral}{de $#2\%$}}%
{\ifthenelse{\equal{#3}{}}{}{{\ifthenelse{\equal{#3}{100,0}}{ também integral}{ e $#3\%$}}}}.}

\newcommand{\ObjetivoFavorabilidadeDestaqueDiferencas}[4]{No
caso do semestre #1 {\ifthenelse{\equal{#2}{100,0}}
{não houve concordância em partes}{a concordância em partes foi de $#2\%$}},
enquanto no semestre #3 {\ifthenelse{\equal{#4}{100,0}}
{não houve concordância em partes}{a concordância em partes foi de $#4\%$}}.}

\newcommand{\ObjetivoFavorabilidadeDestaqueOutra}[5]{Outro destaque
para esta avaliação está{\ifthenelse{\equal{#1}{}}{}{na afirmativa``#1''}} na
replicação do problema ``#2'' no
semestre #3 que a percepção de favorabilidade,
nos critérios definidos, teve um
atingimento {\ifthenelse{\equal{#4}{100,0}}{integral}{de $#4\%$}}%
{\ifthenelse{\equal{#5}{}}{}%
{{\ifthenelse{\equal{#5}{#4}}{, sem concordância em partes}%
{, sendo de $#5\%$ a concordância em partes}}}}.}

\newcommand{\ProblemaSemReplica}[2]{O problema%
{\ifthenelse{\equal{#2}{}}{}{ ``#2''}}
não foi replicado no semestre #1.}

\newcommand{\ObjetivoNaoAtende}[9]{%
Na replicação do problema ``#1''%
{\ifthenelse{\equal{#2}{}}{}{, no semestre ``#2'',}}
a favorabilidade não atinge os critérios para
\AfirmativasSeparador{#3}{#4}{#5}{#6}{#7}{#8}{#9}{}{}.}

\newcommand{\AfirmativasSeparador}[9]{%
a{\ifthenelse{\equal{#2}{}}{}{s}}
afirmativa{\ifthenelse{\equal{#2}{}}{}{s}}
#1%
\Separador{#1}{#2}{#3}{#4}{#5}{#6}{#7}{#8}{#9}}

\newcommand{\Separador}[9]{%
{\ifthenelse{\equal{#2}{}}{}{{\ifthenelse{\equal{#3}{}}{ e #2}%
{, #2{\ifthenelse{\equal{#4}{}}{ e #3}{%
{, #3{\ifthenelse{\equal{#5}{}}{ e #4}{%
{, #4{\ifthenelse{\equal{#6}{}}{ e #5}{%
{, #5{\ifthenelse{\equal{#7}{}}{ e #6}{%
{, #6{\ifthenelse{\equal{#8}{}}{ e #7}{%
{, #7{\ifthenelse{\equal{#9}{}}{ e #8}{, #8  e #9}}}}}}}}}}}}}}}}}}}}}}

\newcommand{\ObjetivoAtende}[3]{%
Na replicação do problema ``#1''%
{\ifthenelse{\equal{#2}{}}{}{, no semestre #2,}}
a favorabilidade atinge os critérios para
{\ifthenelse{\equal{#3}{s}}{a afirmativa}{todas as afirmativas}}.}

\newcommand{\MaisDestaque}[5]{%
Também é destaque que em #1 das #2 replicações o atingimento da
favorabilidade foi integral no semestre #3, sendo o resultado menos
favorável de $#4\%$ no problema ``#5''.}

\newcommand{\QtdParticipantes}[4]{%
No semestre #1, entre os #2 estudantes que
{\ifthenelse{\equal{#3}{0}}{desistiram da}{concluíram a}}
disciplina, {\ifthenelse{\equal{#4}{1}}%
{apenas um participante respondeu}%
{foram #4 participantes a responder}}
o questionário de pesquisa sobre
a percepção geral da disciplina.}

\newcommand{\AprovacaoDisciplina}[7]{%
{\ifthenelse{\equal{#6}{}}{No}{Para o semestre #6, os #7, no}}
que diz respeito a #3 condução da disciplina (#5),
{\ifthenelse{\equal{#1}{100,0}}{todos os}%
{{\ifthenelse{\equal{#1}{0,0}}{nenhum dos}{$#1\%$ dos}}}}
participantes{\ifthenelse{\equal{#2}{}}{}{ #2}}
{\ifthenelse{\equal{#1}{0,0}}{mencionou}{mencionaram}}%
, em algum nível,
{\ifthenelse{\equal{#2}{desistentes}}%
{desistência por desagrado}{satisfação}} com #4.}

\newcommand{\AprovacaoDisciplinaA}[3]{Para
{\ifthenelse{\equal{#1}{100,0}}{todos os}%
{{\ifthenelse{\equal{#1}{0,0}}{nenhum dos}{$#1\%$ dos}}}}
participantes a consideração foi que, em algum nível, #2 (#3).}

\newcommand{\AprovacaoDisciplinaB}[5]{%
Foi expresso o desejo de, em algum nível, #2 (#3) por
{\ifthenelse{\equal{#1}{100,0}}{todos os}%
{{\ifthenelse{\equal{#1}{0,0}}{nenhum dos}{$#1\%$ dos}}}}
participantes%
{\ifthenelse{\equal{#5}{}}{}{,{\ifthenelse{\equal{#5}{0,0}}{ sem nenhuma
contrariedade declarada}{sendo que #4 $#5\%$ discorda, em
algum nível, desta opção}}}}.}

\newcommand{\AprovacaoDisciplinaC}[2]{%
A maioria dos participantes responderam que gostaram de
ter a presença (CB) e a participação (CC) em sala de aula
como critérios de avaliação sendo, respectivamente,
$#1\%$ e $#2\%$ os que concordaram, em algum nível, com as afirmações.}

\newcommand{\AprovacaoDisciplinaD}[3]{%
{\ifthenelse{\equal{#1}{100,0}}{Todos os}%
{{\ifthenelse{\equal{#1}{0,0}}{Nenhum dos}{}}}}
participantes {{\ifthenelse{\equal{#1}{0,0}}{diz}{dizem}}} gostar de #2 (#3).}

\newcommand{\AprovacaoDisciplinaE}[3]{%
Dizem gostar de #2 (#3) $#1\%$ dos participantes.}

\newcommand{\AprovacaoDisciplinaF}[4]{%
A maioria dos participantes
entende que{\ifthenelse{\equal{#4}{}}{}{ #4}} #2
(#3), sendo que
{\ifthenelse{\equal{#1}{100,0}}{todos os}%
{{\ifthenelse{\equal{#1}{0,0}}{nenhum dos}{$#1\%$}}}}
participantes
{{\ifthenelse{\equal{#1}{0,0}}{concordou}{concordaram}}},
em algum nível, com a afirmação.}

\newcommand{\AprovacaoDisciplinaH}[4]{%
A maioria dos participantes
entende que{\ifthenelse{\equal{#4}{}}{}{ #4}} #2
(#3), sendo que apenas
{\ifthenelse{\equal{#1}{100,0}}{todos os}%
{{\ifthenelse{\equal{#1}{0,0}}{nenhum dos}{$#1\%$}}}}
participantes 
{{\ifthenelse{\equal{#1}{0,0}}{concordou}{concordaram}}},
em algum nível, com a afirmação.}

\newcommand{\AprovacaoDisciplinaG}[5]{%
No semestre #1, #2 entre os #3 participantes
que concluíram a disciplina consideraram a
atividade profissional deles durante o semestre (#5),
para $#4\%$ destes foi possível, em algum nível,
conciliar este trabalho com a disciplina.}

\newcommand{\falseH}{$\bigodot$}
\newcommand{\trueH}{\checkmark}


\newcommand{\allOK}[1]{%
#1 & \trueH & \trueH & \trueH &
\trueH & \trueH & \trueH & \trueH &
\trueH & \trueH & \trueH \\
}

\newcommand{\noAllOKA}[6]{%
#1 & {\ifthenelse{\equal{#2}{0}}{\falseH}{\trueH}} &
{\ifthenelse{\equal{#3}{0}}{\falseH}{\trueH}} &
{\ifthenelse{\equal{#4}{0}}{\falseH}{\trueH}} &
{\ifthenelse{\equal{#5}{0}}{\falseH}{\trueH}} &
{\ifthenelse{\equal{#6}{0}}{\falseH}{\trueH}} &
}

\newcommand{\noAllOKB}[5]{%
{\ifthenelse{\equal{#1}{0}}{\falseH}{\trueH}} &
{\ifthenelse{\equal{#2}{0}}{\falseH}{\trueH}} &
{\ifthenelse{\equal{#3}{0}}{\falseH}{\trueH}} &
{\ifthenelse{\equal{#4}{0}}{\falseH}{\trueH}} &
{\ifthenelse{\equal{#5}{0}}{\falseH}{\trueH}}\\
}

\newcommand{\legendaTabelaSintese}{%
\\

PA $\Rightarrow$ ``\ProblemaA''\\
PB $\Rightarrow$ ``\ProblemaB''\\
PC $\Rightarrow$ ``\ProblemaC''\\
PD $\Rightarrow$ ``\ProblemaD''\\
PE $\Rightarrow$ ``\ProblemaE''\\
PG $\Rightarrow$ ``\ProblemaG''\\
}


\xchapter{Resultados}{Será que alguns sonhos devem ser apenas sonhos?} %sem preambulo
\label{cap-resultados}
% É recomendável utilizar `\acresetall' no início de cada capítulo para reiníciar o contator de referências às siglas.
\acresetall

O objetivo deste capítulo é apresentar os resultados das experiências
mencionadas no Capítulo~\ref{cap-estudo} para discutir as
hipóteses de estudo.
Os resultados são apresentados em forma de gráficos e
breves discussões sobre detalhes percebidos durante as análises.

Os dados consolidados que são mencionados nos instrumentos de pesquisa
do Capítulo~\ref{cap-estudo}, foram processados com
\textit{scripts} de filtragem e manipulação
de dados\footnote{C, Bash e GnuPlot}.
Desta forma,
foram consolidados em informações que estão
apresentadas e serão discutidas neste capítulo.

Este capítulo apresenta os resultados quantitativos
dos estudos e uma discussão qualitativa.

As respostas para as afirmativas e as notas atribuídas
pelos participantes foram utilizadas para a construção
dos resultados quantitativos deste estudo.

Como mencionado no Capítulo~\ref{cap-estudo}, os participantes
tiveram a oportunidade de responder questões abertas tanto no
formulário de percepção sobre o problema quanto no formulário
de percepção sobre a disciplina.
Com estes dados foram construídos os resultados qualitativos
exibidos neste trabalho.
Os dados utilizados para a construção dos resultados
qualitativos está disponível no Apêndice~\ref{apendice-qualitativo}.

A Seção~\ref{sec-ref-graficos} apresenta considerações relevantes para o
entendimento dos gráficos e demais informações apresentadas neste capítulo;
as Seções~\ref{sec-sem-2016} e~\ref{sec-sem-2017}
apresenta os resultados quantitativos para o semestre 2016.1 e 2017.1, respectivamente;
a Seção~\ref{sec-avaliacao-hipoteses} apresenta discussão dos resultados
quantitativos com avaliação das hipóteses de estudo deste trabalho;
a Seção~\ref{sec-sem-quali} apresenta uma discussão qualitativa com as respostas
discursivas;
e, por fim, a Seção~\ref{sec-consideracoes-resultados} apresenta
considerações referente aos resultados e hipóteses
deste estudo.

\section{Considerações sobre os gráficos e resultados}
\label{sec-ref-graficos}

Como a participação nas pesquisas foi opcional, a partir deste ponto, 
o termo  ``participante'' é utilizado para designar apenas 
o estudante que respondeu a pesquisa mencionada na discussão,
não devendo ser confundido com o termo ``estudante'', que a partir deste ponto
é utilizado para designar o participante da abordagem, independente
de participação na pesquisa.

A Tabela~\ref{tabela-ref-graficos} apresenta o significado das barras
nos gráficos referentes as percepções dos participantes para as
afirmações mencionadas na Seção~\ref{form-percepcoes} sobre
o problema.

\begin{table}[h]
\caption{Referências para os gráficos de percepção de participante sobre problema}
\label{tabela-ref-graficos}
\begin{tabular}{c|p{14.6cm}}
Legenda & Afirmativa respondida pelo participante \\
\hline
A & \LikertPA \\
\hline
B & \LikertPB \\
\hline
C & \LikertPC \\
\hline
D & \LikertPD \\
\hline
E & \LikertPE \\
\hline
F & \LikertPF \\
\hline
G & \LikertPG \\
\hline
H & \LikertPH \\
\hline
I & \LikertPI \\
\hline
J & \LikertPJ \\
\hline
K & \LikertPK \\
\hline
L & \LikertPL \\
\hline
M & \LikertPM \\
\hline
N & \LikertPN \\
\hline
O & \LikertPO \\
\hline
P & \LikertPP \\
\hline
Q & \LikertPQ \\
\hline
R & \LikertPR \\
\hline
S & \LikertPS \\
\hline
T & \LikertPT \\
\hline
U & \LikertPU \\
\hline
V & \LikertPV \\
\hline
W & \LikertPW \\
\hline
X & \LikertPX \\
\end{tabular}
\end{table}


Um exemplo de leitura do gráfico é a barra \textbf{H} que apresenta
os resultados para a afirmação ``\LikertPH'', que deseja verificar
a percepção do participante se o problema apresenta realidade
e atualidade.

Para cada aplicação de problema foram construídos gráficos
dos resultados das respostas dos participantes para as afirmativas
de percepção e para a nota atribuída ao problema pelos
participantes em uma avaliação geral.


Nos gráficos sobre as percepções dos participantes
para afirmativas, a disposição nas barras
foi construída para facilitar a leitura por nível de favorabilidade.
A favorabilidade pode ser lida por porcentagem de satisfação,
em $y_1$ à esquerda, ou adicionalmente, por porcentagem de
insatisfação em $y_2$ à direta.
No caso de analisar os resultados em uma perspectiva conservadora ao máximo,
é possível considerar como favorável apenas a concordância
integral, isto é, o participante respondeu ``concordo'' para
a afirmação.
A consideração de concordância em parte como favorável,
isto é, também considerar favorável que
o participante respondeu ``concordo em partes'' para a afirmação,
apresenta um resultado menos conservador em relação ao caso anterior.
E assim por diante, com a consideração da resposta `indiferente'', que
representará o caso de favorabilidade onde não há de alguma forma de
discordância, etc.

Ao longo da discussão, ao mencionar textualmente uma afirmativa de
algum dos formulários, foi incluído entre parênteses a qual barra o texto
está se referindo, por exemplo, ``gosto da abordagem PBL (X)''.

Nos gráficos para os resultados das notas atribuídas pelo participante,
existem duas projeções, sendo uma projeção de barras com
a leitura em $y_1$ à esquerda e uma projeção em curva
em $y_2$ à direita.
Ambas as projeções utilizam $x$ na parte de baixo para as notas
inteiras atribuídas de $0$ a $10$.
Em $y_1$ a leitura tem o significado de quantas respostas, valor absoluto,
que foram atribuídas para uma determinada nota, ou seja,
o tamanho da barra.
No caso de $y_2$ o significado é do percentil igual ou superior
a um determinado valor, isto é,
qual a porcentagem da distribuição das barras que é igual ou superior
a uma determinada nota.

Para as discussões e conclusões apresentadas neste trabalho
sobre percepção foi considerada que uma afirmação obteve
resultado favorável ou positivo
se obteve $70\%$ de satisfação,
sendo favoráveis as percepções com algum nível de concordância,
isto é, o participante respondeu ``concordo''
ou ``concordo parcialmente''.


Nas seções a seguir que detalham os resultados de percepção sobre o problema, 
serão discutidos apenas os pontos principais observados,
positivos e negativos, em torno da favorabilidade para o
resultado positivo dentro dos critérios
definidos no parágrafo anterior.
A discordância integral nas afirmativas será destacada para que
os possíveis pontos de melhoria estejam mais explicitados em meio
aos resultados positivos.


A Tabela~\ref{tabela-ref-graficos2} apresenta
o significado das barras nos gráficos referentes às percepções
dos participantres para as afirmações mencionadas
na Seção~\ref{form-disciplinas} sobre a aplicação da disciplina,
para estudante \textit{concluinte} e
a Tabela~\ref{tabela-ref-graficos3} apresenta
para estudante \textit{desistente}.

\begin{table}[h]
\caption{Referências para os gráficos de percepção de estudante \textit{concluinte} sobre a disciplina}
\label{tabela-ref-graficos2}
\begin{tabular}{c|p{14.6cm}}
Legenda & Afirmativa respondida pelo participante \\
\hline
CA & \LikertCA\\
\hline
CB & \LikertCB\\
\hline
CC & \LikertCC\\
\hline
CD & \LikertCD\\
\hline
CE & \LikertCE\\
\hline
CF & \LikertCF\\
\hline
CG & \LikertCG\\
\hline
CH & \LikertCH\\
\hline
CI & \LikertCI\\
\hline
CJ & \LikertCJ\\
\hline
CK & \LikertCK\\
\hline
CL & \LikertCL\\
\hline
CM & \LikertCM\\
\hline
CN & \LikertCN\\
\end{tabular}
\end{table}

\begin{table}[h]
\caption{Referências para os gráficos de percepção de estudante \textit{desistente} sobre a disciplina}
\label{tabela-ref-graficos3}
\begin{tabular}{c|p{14.6cm}}
Legenda & Afirmativa respondida pelo participante \\
\hline
DA & \LikertDA\\
\hline
DB & \LikertDB\\
\hline
DC & \LikertDC\\
\hline
DD & \LikertDD\\
\hline
DE & \LikertDE\\
\hline
DF & \LikertDF\\
\hline
DG & \LikertDG\\
\hline
DG1 & \LikertDGa\\
\hline
DH & \LikertDH\\
\hline
DI & \LikertDI\\
\hline
DJ & \LikertDJ\\
\hline
DO & \LikertDO\\
\hline
DO1 & \LikertDOa\\
\end{tabular}
\end{table}


Para cada semestre de aplicação, foram construídos gráficos
de resultados para as respostas dos participantes para as afirmativas
de percepção e para a nota atribuída para a disciplina pelos participantes
em uma avaliação geral.
%Para cada característica de participante, concluinte
%ou desistente, foram construídos quatro gráficos
%de resultados.
%Dois gráficos que exibem os resultados para as respostas
%dos participantes para as afirmativas de percepção,
%para os semestre 2016.1 e 2017.1, enquanto os outros dois gráficos
%exibem os resultados da nota geral atribuída pelos participantes.

Os artifícios utilizados para exibição dos gráficos e
discussão da percepção sobre a disciplina são os mesmos
utilizados nos gráficos e discussão da percepção sobre
o problema.
A disposição nas barras, a referência ao mencionar textualmente
uma afirmativa, valor absoluto e percentil nos gráfico da
nota geral etc.

\section{Semestre 2016.1}
\label{sec-sem-2016}
\resultadoTurma{2016.1}{26}{9}{6}{57,7}{11}{8}{3}

\subsection{Problema 1 -- \ProblemaA}
\perfilProblema{\ProblemaA}{1}{2016.1}{14}
\perfilParticipante{27,9}{20}{51}{35,7}{71,4}{entre o terceiro e o quinto semestre}{}
\figuraPercepcaoParticipante{s1p1}{1}{2016.1}

É possível destacar que para todas as questões a favorabilidade
foi superior aos $70\%$. A favorabilidade foi superior aos $90\%$
para 12 afirmações apresentadas.
Apenas 6 afirmações receberam ao menos uma discordância integral.

A afirmação sobre ideias alternativas no caminho do problema (B)
foi a que teve a maior percepção de concordância por partes dos
participantes.
Entendemos que este resultado é explicado pelo problema
ter apresentado os conceitos de músicas, de forma que
os participantes precisaram realizar relacionamentos entre
os conceitos deste tema com os conceitos de linguagens
formais, e que os participantes perceberam que a depender do
nível de abstração poderá existir um caminho diferente.


A necessidade de recorrer aos materiais de apoio (D),
existência de informações suficientes no texto (K) e
sobre o participante gostar da metodologia baseada
em problemas (X) foram as afirmações com
as menores favorabilidades.
Embora possa parecer existir alguma contradição
sobre os resultados das duas primeiras, vale
ressaltar que os dados referente
a contribuição da sessão tutorial (Q) obtiveram
alta favorabilidade.

\figuraPercepcaoParticipanteNotas{s1p1}{1}{2016.1}{80}{5,00}{7,29}{quase}

\subsection{Problema 2 -- \ProblemaB}
\perfilProblema{\ProblemaB}{2}{2016.1}{12}
\perfilParticipante{28,5}{20}{51}{25,0}{75,0}{entre o terceiro e o quinto semestre}{}
\figuraPercepcaoParticipante{s1p2}{2}{2016.1}

Para 5 afirmações, na percepção dos estudantes, a favorabilidade foi integral.
Em 15 afirmações a favorabilidade superou os $90\%$.
Em apenas uma das afirmações a favorabilidade ficou pouco abaixo dos $70\%$, onde
$66,7\%$ dos participante acreditam que o problema estimulou
o trabalho em grupo (M).
Apenas 5 afirmações receberam ao menos uma discordância integral.

A contribuição dos tutores para a evolução do problema (U) foi a
afirmação melhor avaliada pelos participantes, assim como as demais
afirmações referentes aos tutores (W) e (V) também foram bem avaliadas,
indicando que os participantes aprovaram a participação
dos tutores na condução deste problema.

Para a construção do produto deste problema, os participantes
precisaram se reunir em equipe além das sessões tutoriais, desta forma,
acreditamos que as dificuldades referentes a se reunir fora das sessões tutorias
para trabalhar em grupo (M) estão também representados neste resultado.
Também observamos uma justificativa para este resultado ao analisar a favorabilidade para
a afirmação referente a percepção dos participantes com relação a quantidade
apropriada de estudantes em cada grupo tutorial (S).

\figuraPercepcaoParticipanteNotas{s1p2}{2}{2016.1}{70}{6,00}{7,17}{mais de}

\subsection{Problema 3 -- \ProblemaC}
\perfilProblema{\ProblemaC}{3}{2016.1}{8}
\perfilParticipante{31,4}{20}{51}{28,6}{85,7}{até o quinto semestre}{ para cada três participantes}
\figuraPercepcaoParticipante{s1p3}{3}{2016.1}

Para 7 afirmações, na percepção dos estudantes, a favorabilidade foi integral.
Em três afirmações a favorabilidade ficou abaixo dos $70\%$, mas apenas 2
afirmações receberam ao menos uma discordância integral.

A contribuição das sessões tutorias para o processo
de resolução do problema (Q) obteve a melhor avaliação pelos
participantes.
Entendemos que essa afirmação é bem avaliada sempre que
os participantes conseguem perceber uma informação de
grande relevância para a solução durante a sessão
tutorial.
Neste caso, a ideia de como manipular a pilha
de forma a manter a proporção exigida pelo problema
surgiu durante uma sessão tutorial.

No que diz respeito a quantidade de pessoas em
cada grupo tutorial (S) ser a afirmação a obter a pior
avaliação a justificativa pode ser obtida quando
é observada em conjunto com a afirmação referente ao
trabalho em equipe (M) que também esteve entre
as piores avaliações para este problema, neste caso,
ambos resultados são explicados pela quantidade de
participantes nas sessões tutoriais.

\figuraPercepcaoParticipanteNotas{s1p3}{3}{2016.1}{90}{6,00}{7,50}{quase}

\subsection{Problema 4 -- \ProblemaD}
\perfilProblema{\ProblemaD}{4}{2016.1}{8}
\perfilParticipante{31,5}{20}{51}{25,0}{87,5}{até o quinto semestre}{}
\figuraPercepcaoParticipante{s1p4}{4}{2016.1}

Para 12 afirmações, na percepção dos estudantes, a favorabilidade foi integral.
Em todas as afirmações a favorabilidade ficou acima dos $70\%$.
Em 4 afirmações foi recebida ao menos uma discordância integral.

A máquina de Turing foi o principal objetivo de aprendizagem para o Problema 4
aplicado no semestre 2016.1, sendo que para os participantes a afirmação
melhor avaliada diz respeito a utilidade dos conhecimentos aprendidos para 
o profissional da área de Computação (I).
Este problema foi muito bem avaliado pelos participantes para todas
as afirmações, mas cabe destacar os resultados para as afirmações que
dizem respeito a necessidade de aprender novos conhecimentos (G) e
motivação dos participantes para resolver o problema (E).

\figuraPercepcaoParticipanteNotas{s1p4}{4}{2016.1}{90}{5,00}{7,50}{quase}

\subsection{Problema 5 -- \ProblemaE}
\perfilProblema{\ProblemaE}{5}{2016.1}{7}
\perfilParticipante{32,3}{20}{51}{28,6}{71,4}{até o quinto semestre}{}
\figuraPercepcaoParticipante{s1p5}{5}{2016.1}

Para 6 afirmações, na percepção dos estudantes, a favorabilidade foi integral.
Em apenas duas afirmações a favorabilidade ficou abaixo dos $70\%$.
Em 9 afirmações foi recebida ao menos uma discordância integral, desta forma,
indicando vários pontos de atenção para a abordagem deste problema
em outras situações.

Este também foi um problema em que os participantes se reuniram em grupo para
construir uma solução, assim, como no Problema 2 no semestre 2016.1, a
afirmação pior avaliada também foi referente ao trabalho em equipe (M) que
indicamos que o resultado também inclui as dificuldades em se reunir além
das sessões tutoriais.

Ideias alternativas (B), utilização de referências bibliográficas (C),
recorrer a materiais não indicados nas referências (D), motivação
para resolver o problema (E), o tempo para resolução (L), a quantidade de
pessoas no grupo da sessão tutorial (S), utilidade do problema para o
processo de ensino e aprendizagem (T), \textit{feedback} dos
tutores (W), além da afirmação sobre o trabalho em equipe (M)
mencionado no parágrafo anterior, foram as afirmações com ao menos
uma discordância integral.
O mais possível é que este resultados sejam justificados pela sobrecarga
adicional que os estudantes possuem no fim do semestre, momento em que
este problema foi aplicado.
Ainda assim, não se deve desconsiderar que estes são pontos
de atenção claros, mas que mesmo em outras metodologias, com pesquisa
semelhante, os resultados da sobrecarga estariam também explicitados.

\figuraPercepcaoParticipanteNotasA{s1p5}{5}{2016.1}{70}{}{7,00}{pouco mais de}

\section{Semestre 2017.1}
\label{sec-sem-2017}
\resultadoTurma{2017.1}{50}{x}{y}{z}{a}{14}{b}


\subsection{Problema 1 -- \ProblemaG}
\label{sec-2017-p1}
\perfilProblema{\ProblemaG}{1}{2017.1}{12}
\perfilParticipante{26,1}{19}{44}{18,2}{72,7}
{até o terceiro semestre}{ para cada seis participantes}
\figuraPercepcaoParticipante{s2p1}{1}{2017.1}

Para 7 afirmações apresentadas a favorabilidade ficou um pouco abaixo dos $70\%$, mas
para 10 afirmações foi superior aos $90\%$.
Em 11 afirmações foi recebida ao menos uma discordância integral.

Neste problema com o objetivo de aprendizagem foi Linguagens
Formais, sendo consenso de percepção dos participantes que
este problema apresenta um conhecimento útil para um profissional
da área de Computação (I).

É possível destacar o excelente resultado para a afirmação que diz respeito a
utilidade do problema no processo de ensino e aprendizagem (T).
A explicação para este resultado está no fato de que os estudantes
conseguiram realizar bem as correlações entre os conceitos
do código Morse, apresentado no problema, com os conceitos de
linguagens formais, que são os objetivos de aprendizagem para
este problema.

Em uma análise mais detalhada dos dados foi possível
detectar que a discordância integral está concentrada
em 3 participantes, onde apenas um destes apresentou
5 discordâncias integrais.
Estes números além de indicar pontos possíveis de
priorização, também podem indicar a necessidade
de um nivelamento maior dos estudantes para a aplicação
da metodologia.

\figuraPercepcaoParticipanteNotas{s2p1}{1}{2017.1}{90}{5,00}{8,25}{quase}

\subsection{Problema 2 -- \ProblemaB}
\perfilProblema{\ProblemaB}{2}{2017.1}{13}
\perfilParticipanteA{24,3}{19}{38}{8,3}{63,6}
{até o terceiro semestre}{ quatro para cada dez}
\figuraPercepcaoParticipante{s2p2}{2}{2017.1}

A favorabilidade ficou um pouco abaixo dos $70\%$ em 6 afirmações apresentadas,
sendo superior aos $80\%$ em 12 afirmações.
Em 14 afirmações foi recebida ao menos uma discordância integral.

Neste problema o objetivo de aprendizagem principal foi
Autômatos Finitos, sendo consenso de percepção dos participantes que
este problema apresenta um conhecimento útil para um profissional
da área de Computação (I) e que é necessário aprender novos
conhecimentos (G).

Apenas um dos participantes respondeu com discordância integral para
8 afirmações, assim, também se faz necessário considerar a
necessidade de nivelamento do grupo, como mencionado
na Seção~\ref{sec-2017-p1}.

Apesar de obter favorabilidade muito próxima aos $70\%$, a afirmação
sobre o participante gostar da metodologia PBL (X) teve como
destaque negativo receber a maior discondância integral nessa
replicação.

\figuraPercepcaoParticipanteNotas{s2p2}{2}{2017.1}{60}{5,00}{7,31}{pouco mais de}

\subsection{Problema 3 -- \ProblemaC}
\perfilProblema{\ProblemaC}{3}{2017.1}{11}
\perfilParticipanteA{25,1}{19}{38}{0}{63,6}
{até o terceiro semestre}{ quatro para cada dez}
\figuraPercepcaoParticipante{s2p3}{3}{2017.1}

A favorabilidade ficou um pouco abaixo dos $70\%$ em apenas uma afirmação,
sendo superior aos $80\%$ para 21 destas afirmações.
Em apenas 2 afirmações foi recebida ao menos uma discordância integral.

Para este caso a percepção dos participantes
sobre a relevância das sessões tutorias no processo
de resolução do problema (Q), assim como no semestre
anterior, obteve favorabilidade integral, com o destaque de
que para este semestre todas as respostas recebidas foram de
total concordância.

A afirmativa pior avaliada diz respeito a percepção dos participantes
sobre o \textit{feedback} dos tutores a cada sessão tutorial (W).
Para este caso, apesar do resultado ainda ser alto, ficou como
um ponto de atenção aos tutores.

A Figura~\ref{aval-s2p3} apresenta o gráfico da
avaliação do participantes para o Problema 3 aplicado no semestre 2017.1.

\begin{figure}[!htb]
\centering
\includegraphics[scale=0.18]{notas-s2p3.eps}
\caption{Avaliação dos participantes do semestre 2017.1 para o Problema 3}
\label{aval-s2p3}
\end{figure}

Assim como no semestre anterior, este problema foi muito bem avaliado
pelos participantes, assim, todas as notas atribuídas foram maiores
ou iguais $7,00$ e uma média de $8,18$.

Entre os problemas que foram replicados em ambos os semestres, este
notadamente é o que apresentou as melhores avaliações pelos participantes
em todos os critérios, evidenciando o potencial da metodologia em uma
disciplina teórica com a utilização de um problema do ``mundo real''.

\subsection{Problema 4 -- \ProblemaD}
\perfilProblema{\ProblemaD}{4}{2017.1}{10}
\perfilParticipanteB{25,7}{19}{38}{11,1}{88,9}
{até o terceiro semestre}{igual}
\figuraPercepcaoParticipante{s2p4}{4}{2017.1}

Para todas as afirmativas a percepção dos participantes obteve favorabilidade
superior aos $70\%$. Em 5 afirmações foi recebida ao menos uma discordância
integral.

O destaque positivo para este problema foi a favorabilidade em relação
a necessidade de recorrer a materiais fora da bibliográfia básica (D).
Este é uma característica de estudo muito relevante no desenvolvimento
do estudante que passa a ver a necessidade de explorar fontes diversas
de informação para construir o seu conhecimento.

Como ponto de atenção, apesar de a favorabilidade ser igual aos $70\%$ para
a percepção do participante sobre a adequação do tempo disponível para desenvolver
a solução (L), houve uma quantidade relevante de discordância integral para esta afirmativa.

Este foi outro problema que foi bem avaliado em todos os critérios em ambos
os semestres.
\figuraPercepcaoParticipanteNotas{s2p4}{4}{2017.1}{80}{6,00}{7,31}{quase}

\subsection{Problema 5 -- \ProblemaE}
\perfilProblema{\ProblemaE}{5}{2017.1}{7}
\perfilParticipanteB{27,0}{20}{38}{20,0}{100,0}
{até o terceiro semestre}{semelhante}
\figuraPercepcaoParticipante{s2p5}{5}{2017.1}

Em 3 afirmações a favorabilidade ficou abaixo dos $70\%$ e 5 afirmações
receberam ao menos uma discordância integral.

Em contraste com o resultado que o mesmo problema teve com
os participantes do semestre 2016.1, onde foi a afirmação com a
pior favorabilidade, para os participantes neste semestre de
2017.1, a percepção de que o problema estimula o
trabalho em grupo (M) foi a afirmação melhor avaliada
para o Problema 5.
A diferença de resultado é justificada ao observar
a percepção dos participantes com relação a quantidade
apropriada de estudantes em cada grupo tutorial (S).

A afirmação com a menor favorabilidade diz respeito
a percepção de cumprimento dos objetivos de aprendizagem
pelo participante (F).
Neste ponto cabe destacar que ao apresentar um resultado
negativo para esta afirmação, uma vez que com outros
problemas dentro do mesmo grupo tiveram favorabilidade
melhores, se pode questionar o problema e não
a metodologia.
Também não se pode deixar de considerar o nível de abstração
mais elevado para o entendimento dos conceitos deste
problema em relação aos demais.

\figuraPercepcaoParticipanteNotas{s2p5}{5}{2017.1}{70}{5,00}{7,43}{pouco mais de}


\subsection{Percepção dos concluintes}
\QtdParticipantes{2016.1}{11}{1}{6}
\perfilParticipante{28,8}{20}{49}{50,0}{83,3}{até o quarto semestre}{}
\figuraPercepcaoParticipanteDisciplina{concluintes-2016}{concluintes}{2016.1}
\AprovacaoDisciplina{83,3}{}{aprovação da metodologia para a}%
{a metodologia}{CA}{2016.1}{concluintes}
\AprovacaoDisciplinaC{66,7}{83,3}
\AprovacaoDisciplinaD{100,0}{ter várias avaliações}{CD}
\AprovacaoDisciplinaE{83,3}{avaliações em equipe}{CE}
\AprovacaoDisciplina{100,0}{}{avaliação dos tutores na}%
{os tutores}{CF}{}{}
\AprovacaoDisciplinaG{2016.1}{4}{6}{75,0}{CG}
\AprovacaoDisciplinaA{100,0}{entenderam a metodologia \ac{PBL}}{CH}
\AprovacaoDisciplinaF{16,7}{é um problema falar em público}{CI}{não}
\AprovacaoDisciplinaB{100,0}{experimentar outras abordagens além das aulas tradicionais}{CJ}{}{}
\AprovacaoDisciplinaA{83,3}{a metodologia \ac{PBL} ajudou nas avaliações escritas}{CK}
\AprovacaoDisciplinaA{83,3}{a metodologia \ac{PBL} ajudou a entender melhor os conceitos}{CL}
\AprovacaoDisciplinaF{66,7}{que há um balanceamento satisfatório na distribuição de carga
horária entre sessões tutoriais e aulas expositivas}{CM}{}
\AprovacaoDisciplinaB{83,3}{experimentar outras disciplinas com \ac{PBL}}{CN}{}{0,0}

\figuraPercepcaoParticipanteDisciplinaNotas{2016-concluintes}{concluintes}{2016.1}{100,0}{}{7,83}{}

\QtdParticipantes{2017.1}{32}{1}{14}
\perfilParticipanteB{22,75}{20}{28}{16,7}{58,3}{até o terceiro semestre}{semelhante}
\figuraPercepcaoParticipanteDisciplina{concluintes-2017}{concluintes}{2017.1}
\AprovacaoDisciplina{100,0}{}{aprovação da metodologia para a}%
{a metodologia}{CA}{2017.1}{concluintes}
\AprovacaoDisciplinaC{75,0}{66,7}
\AprovacaoDisciplinaE{75,0}{ter várias avaliações}{CD}
\AprovacaoDisciplinaE{83,3}{avaliações em equipe}{CE}
\AprovacaoDisciplina{100,0}{}{avaliação dos tutores na}%
{os tutores}{CF}{}{}
\AprovacaoDisciplinaG{2017.1}{9}{14}{55,6}{CG}
\AprovacaoDisciplinaA{75,0}{entenderam a metodologia \ac{PBL}}{CH}
\AprovacaoDisciplinaF{25,0}{é um problema falar em público}{CI}{não}
\AprovacaoDisciplinaB{91,7}{experimentar outras abordagens além das aulas tradicionais}{CJ}%
{apenas}{8,3}
\AprovacaoDisciplinaA{66,7}{a metodologia \ac{PBL} ajudou nas avaliações escritas}{CK}
\AprovacaoDisciplinaA{75,0}{a metodologia \ac{PBL} ajudou a entender melhor os conceitos}{CL}
\AprovacaoDisciplinaF{66,7}{que há um balanceamento satisfatório na distribuição de carga
horária entre sessões tutoriais e aulas expositivas}{CM}{}
\AprovacaoDisciplinaB{66,7}{experimentar outras disciplinas com \ac{PBL}}{CN}{apenas}{8,3}

\figuraPercepcaoParticipanteDisciplinaNotas{2017-concluintes}{concluintes}{2017.1}{80}{5,00}{7,83}{mais de}

\input{capitulo-resultados-desistentes.tex}

\section{Avaliações dos objetivos específicos}
\label{sec-avaliacao-hipoteses}
Para avaliação dos objetivos específicos as quais este estudo se propõe,
utilizamos os resultados apresentados neste capítulo.

Para a avaliação dos objetivos específicos deste estudo, uma vez que a
metodologia utilizada exige que alguns critérios sejam
arbitrados, preferimos ser bastante conservadores
na definição dos parâmetros arbitrados.
No caso de um objetivo com apenas uma afirmação, consideramos
como avaliação positiva se a afirmação recebe
resultado positivo dentro dos critérios mencionados na
Seção~\ref{sec-ref-graficos}, ou seja, $70\%$ de favorabilidade,
sendo favoráveis as respostas de concordância.
No caso de mais de uma afirmativa, consideramos
como avaliação positiva em caso de resultado positivo para todas
as afirmativas, dentro dos mesmos critérios de favorabilidade.

Destacamos que a avaliação dos objetivos específicos foi realizada para
cada aplicação de problema, ou seja, é possível concluir com
uma avaliação positiva em uma aplicação de problema, enquanto pode
não ser possível concluir com uma avaliação positiva em outra aplicação
de problema.

% OE1
\AvaliacaoObjetivo{\oeatexto}{oe1ref}{oe}{``\LikertPQ'' (Q)}{}{}{}{}{}

\AprovacaoObjetivo{}{}{todas}{0}{}
\ObjetivoFavorabilidadeDestaque{}{\ProblemaC}{2017.1}{100,0}{}{}
\ObjetivoFavorabilidadeDestaqueContinuidade{2016.1}{100,0}{}
\ObjetivoFavorabilidadeDestaqueDiferencas{2016.1}{25,0}{2017.1}{100,0}
\ObjetivoFavorabilidadeDestaqueOutra{}{\ProblemaD}{2016.1}{100}{}
\ObjetivoFavorabilidadeDestaqueContinuidade{2017.1}{90,0}{}
\AprovacaoObjetivoResultado{}{}{}{}{}{}{}{}{}

% OE2
\AvaliacaoObjetivo{\oebtexto}{oe2ref}{oe}{``\LikertPM'' (M)}{}{}{}{}{}

\AprovacaoObjetivo{}{não}{3}{0}{}
\AprovacaoObjetivoResultado{não}{\ProblemaB}{\ProblemaC}{\ProblemaE}{2016.1}{}{}{}{}
Em nosso trabalho \cite{gavaza2017}, que utilizamos apenas dados do
semestre 2016.1, apesar de não termos definidos parâmetros
de avaliação, percebemos que a percepção dos estudantes com relação
a motivação para o trabalho em equipe também não recebeu as melhores
avaliações.
No semestre 2016.1, como descrito na Seção ~\ref{sec-exp-2016}, foi utilizado
apenas um quadro branco, assim foi formado apenas um grupo com todos
os estudantes.
Para o semestre 2017.1, como descrito na Seção ~\ref{sec-exp-2017}, foram
utilizados quadros adicionais e foram formados grupos menores,
de até dez participantes cada.
\AprovacaoObjetivo{}{}{todas}{0}{2017.1}
\ObjetivoFavorabilidadeDestaque{}{\ProblemaE}{}{100,0}{85,7}{}
% escrever sobre a redução no grupo tutorial e qual a conclusão

% OE3
\AvaliacaoObjetivo{\oectexto}{oe3ref}{oe}{``\LikertPT'' (T)}{}{}{}{}{}

\AprovacaoObjetivo{}{}{todas}{0}{}
\ObjetivoFavorabilidadeDestaque{}{\ProblemaC}{2016.1}{100,0}{50,0}{\ProblemaD}
\ObjetivoFavorabilidadeDestaqueContinuidade{2017.1}{90,9}{80,0}
\ObjetivoFavorabilidadeDestaqueOutra{}{\ProblemaG}{2017.1}{91,7}{91,7}
\ProblemaSemReplica{2016.1}{}
\AprovacaoObjetivoResultado{}{}{}{}{}{}{}{}{}

% OE4
\AvaliacaoObjetivo{\oedtexto}{oe4ref}{}{``\LikertPC'' (C)}{``\LikertPD'' (D)}
{``\LikertPG'' (G)}{``\LikertPK'' (K)}{``\LikertPO'' (O)}{``\LikertPP'' (P)}

\ObjetivoNaoAtende{\ProblemaG}{2017.1}{(C)}{(D)}{(O)}{}{}{}{}
\ProblemaSemReplica{2016.1}{\ProblemaG}
\ObjetivoNaoAtende{\ProblemaB}{2017.1}{(C)}{(D)}{(P)}{}{}{}{}
\ObjetivoAtende{\ProblemaB}{2016.1}{p}
\AprovacaoObjetivo{}{não}{2}{1}{}
\AprovacaoObjetivoResultado{não}{\ProblemaG}{\ProblemaB}{}{2017.1}{}{}{}{}

% OE5
\AvaliacaoObjetivo{\oeetexto}{oe5ref}{oe}{``\LikertPU'' (U)}{``\LikertPV'' (V)}
{``\LikertPW'' (W)}{}{}{}

\ObjetivoNaoAtende{\ProblemaG}{2017.1}{(W)}{}{}{}{}{}{}
\ProblemaSemReplica{2016.1}{\ProblemaG}
\ObjetivoNaoAtende{\ProblemaB}{2017.1}{(W)}{}{}{}{}{}{}
\ObjetivoAtende{\ProblemaB}{2016.1}{p}
\ObjetivoNaoAtende{\ProblemaC}{2017.1}{(W)}{}{}{}{}{}{}
\ObjetivoAtende{\ProblemaC}{2016.1}{p}
\ObjetivoNaoAtende{\ProblemaE}{2017.1}{(W)}{}{}{}{}{}{}
\ObjetivoAtende{\ProblemaE}{2016.1}{p}
A favorabilidade não foi atingida em 4 das 5 aplicações no
semestre 2017.1.

% OE6
\AvaliacaoObjetivo{\oeftexto}{oe6ref}{oe}{``\LikertPI'' (I)}{}{}{}{}{}

\AprovacaoObjetivo{}{}{todas}{0}{}
\ObjetivoFavorabilidadeDestaque{}{\ProblemaC}{2017.1}{100,0}{90,9}{}
\ObjetivoFavorabilidadeDestaqueContinuidade{2016.1}{75,0}{}
\MaisDestaque{4}{5}{2017.1}{85,8}{\ProblemaE}
\AprovacaoObjetivoResultado{}{}{}{}{}{}{}{}{}

% OE7
\AvaliacaoObjetivo{\oegtexto}{oe7ref}{oe}{``\LikertPA'' (A)}{``\LikertPB'' (B)}{}{}{}{}

\ObjetivoNaoAtende{\ProblemaE}{2016.1}{(B)}{}{}{}{}{}{}
\ObjetivoAtende{\ProblemaE}{2017.1}{p}
\ObjetivoNaoAtende{\ProblemaG}{2017.1}{(B)}{}{}{}{}{}{}
\ProblemaSemReplica{2017.1}{\ProblemaG}
\AprovacaoObjetivoResultado{não}{\ProblemaE}{}{}{2016.1}{\ProblemaG}{}{}{2017.1}

\subsection{Síntese dos objetivos específicos}
A Tabela~\ref{tabela-ref-objetivos} apresenta
uma síntese das avaliações dos objetivos específicos.
\input{capitulo-resultados-objetivos}

É possível perceber que apenas o problema ``\ProblemaD'' obteve
avaliação positiva para todos os objetivos específicos em duas
replicações, conforme os critérios definidos, e que o
problema ``\ProblemaA'' obteve avaliação positiva para todos as
objetivos específicos na única replicação, conforme os
critérios definidos.

Existem dois principais resultados negativos facilmente
perceptíveis na síntese de avaliação dos objetivos
específicos.
O resultado para o objetivo específico em relação ao trabalho em equipe
para as replicações no semestre 2016.1 e para o objetivo
específico em relação contribuição positiva dos tutores nas
replicações do semestre 2017.1.
Os dois principais resultados negativos são justificáveis.

No caso do objetivo específico em relação ao trabalho em equipe
para as replicações no semestre 2016.1 o tamanho
do grupo tutorial é o principal argumento
para este resultado.

Entendemos que a divisão de grupos tutoriais menores
nas replicações do semestre 20171.1 contribuiu
positivamente para o trabalho em equipe dos
estudantes e consequentemente para a percepção destes
sobre este trabalho, assim, sugerimos que os
grupos tutoriais devem possuir entre cinco e dez estudantes,
como em nossa replicação do semestre 2017.1, para aumentar
as possibilidades dos resultados estarem dentro de critérios
semelhantes aos nossos.

No caso do objetivo específico em relação contribuição positiva
dos tutores nas replicações do semestre 2017.1 a quantidade de
tutores para o total de estudantes na turma é a justificativa.

Como não foi adotada nenhuma ação com relação ao aumento da
quantidade total de estudantes em relação ao semestre 2016.1,
de 26 para 50, quase o dobro, entendemos que os tutores não
puderam se aproximar mais efetivamente dos estudantes
nas replicações do semestre 2017.1.
Nossa experiência sugere que o ideal é disponibilizar um
tutor para facilitar a sessão tutorial para entre dez e
quinze estudantes.

Os objetivos específicos em relação a percepção dos estudantes sobre
a metodologia estimular a autoaprendizagem e sobre percepção de
conexões entre o conhecimento e o problema entendemos ser
resultados pontuais, que podem ser melhorados com adequações nos problemas
onde os resultados não foram positivos, uma vez que na maioria das replicações
os objetivos específicas em questão foram avaliados positivamente
dentro dos critérios definidos.


\newcommand{\contribuicoesQualitativas}[7]{%
{\ifthenelse{\equal{#5}{}}{O problema ``#1'' foi
replicado apenas no semestre #2 e}{
Na replicação do problema ``#1'' no semestre #2}}
foram recebidas contribuições discursivas%
{\ifthenelse{\equal{#3}{zero}}{}{ de #3
participante{\ifthenelse{\equal{#3}{um}}{}{s}}
sobre o problema{\ifthenelse{\equal{#4}{zero}}{}{ e}}}}%
{\ifthenelse{\equal{#4}{zero}}{}{ de #4
participante{\ifthenelse{\equal{#4}{um}}{}{s}}
sobre a metodologia PBL}}.%
{\ifthenelse{\equal{#5}{}}{}{
No semestre #5
{\ifthenelse{\equal{#6}{zero}}{{\ifthenelse{\equal{#7}{zero}}{não }{}}{}}{}}%
foram recebidas contribuições discursivas%
{\ifthenelse{\equal{#6}{zero}}{}{ de #6
participante{\ifthenelse{\equal{#6}{um}}{}{s}}
sobre o problema{\ifthenelse{\equal{#7}{zero}}{}{ e}}}}%
{\ifthenelse{\equal{#7}{zero}}{}{ de #7
participante{\ifthenelse{\equal{#7}{um}}{}{s}}
sobre a metodologia PBL}}.}}}

\newcommand{\contribuicoesQualitativasDisciplina}[5]{%
Entre os #1 estudantes #2 do semestre #3
{\ifthenelse{\equal{#4}{zero}}{{\ifthenelse{\equal{#5}{zero}}{não }{}}{}}{}}%
foram recebidas contribuições discursivas%
{\ifthenelse{\equal{#4}{zero}}{}{ de #4
participante{\ifthenelse{\equal{#4}{um}}{}{s}}
sobre a metodologia PBL{\ifthenelse{\equal{#5}{zero}}{}{ e}}}}%
{\ifthenelse{\equal{#4}{zero}}{}{ de #5
participante{\ifthenelse{\equal{#5}{um}}{}{s}}
sobre a disciplina}}.%
}

\section{Respostas discursivas}
\label{sec-sem-quali}

\subsection{\ProblemaA}
\contribuicoesQualitativas{\ProblemaA}{2016.1}{dois}{quatro}{}{}{}

Em relação ao problema ambos os participantes mencionam a complexidade.
Um participante diz que há dependência com os conceitos de música
e outro acredita que a complexidade está relacionada com este ser
o primeiro problema ao qual foram submetidos.

A metodologia foi avaliada positivamente, sendo
``interessante'', ``útil'', ``colaborativa'' e ``soluções rápidas''
as qualidades mais mencionadas.
Foi destacada também que a metodologia precisa de
aprimoramentos.

\subsection{\ProblemaB}
\contribuicoesQualitativas{\ProblemaB}{2016.1}{dois}{dois}{2017.1}{seis}{cinco}

Para este problema os participantes mencionaram a não especificação clara
do problema, sobretudo os participantes da replicação do semestre 2016.1.
Apesar de os participantes terem recebido
uma explicação sobre a metodologia e terem participado
do problema anterior com a metodologia PBL, é necessário considerarmos
que os participantes estão envolvidos em um curso tradicional
em que as atividades que executam nas outras disciplinas são normalmente
especificações, ao contrário do que é recomendado nos princípios citados
em \cite{dolmans1997seven} e que foram utilizados para
a construção deste problema.

Ainda sobre o problema, no caso da replicação do semestre 2017.1 a dificuldade
do problema é o assunto mais mencionado.
Embora exista divergência entre os participantes
sobre o acréscimo de dificuldade, a maioria menciona que este
problema é mais difícil que o anterior, no caso, ``\ProblemaG''.

Com relação a metodologia, os participantes da replicação do semestre 2016.1
mencionam a necessidade de mais retorno por parte dos tutores,
embora as avaliações referentes ao tutores tenham sido
os melhores avaliados quantitativamente, como destacamos
na Seção~\ref{sec-sem-2016} na apresentação dos resultados
da replicação deste problema.

Os participantes na replicação deste problema no semestre 2017.1
apresentaram algumas avaliações positivas para a metodologia
e alguns pontos de atenção no que diz respeito ao
comportamento necessário dos estudantes para o bom
andamento da metodologia.
As palavras ``desperta'', ``enriquecedor'', ``bom'' e ``\textit{insights}''
foram mencionadas positivamente em relação a metodologia e ``perdidos''
foi a palavra mencionada negativamente em relação a metodologia.
Foi mencionado que das reuniões alguns estudantes não conseguem
extrair as informações necessárias e que faltou aula teórica
de como fazer o autômato antes da aplicação do problema.

O compromisso dos estudantes é necessário para
o andamento de uma disciplina, independente de qual seja
o método utilizado.
No caso do PBL, as interações entre os envolvidos no processo
se fazem extremamente necessárias, nesse contexto,
foi citado por três participantes na replicação do semestre 2017.1 as dificuldades
com relação ao trabalho em equipe.

\subsection{\ProblemaC}
\contribuicoesQualitativas{\ProblemaC}{2016.1}{um}{um}{2017.1}{três}{dois}

Na replicação deste problema no semestre 2016.1
foi destacado o equilíbrio entre tempo e a dificuldade
que também é observado nos resultados quantitativos
na favorabilidade da afirmativa ``\LikertPL'' (L)
e mencionado uma maior participação
dos estudantes nesta replicação.

No caso da replicação no semestre 2017.1 a dificuldade
do problema é o assunto mais mencionado, onde o problema foi
qualificado com um dificuldade entre média e difícil.
Novamente apontado sobre o texto, acreditamos que o
envolvimento dos participantes em um curso
tradicional pode criar expectativas de que o texto
seja mais ``especificação''.

Sobre a metodologia PBL, na replicação do semestre 2017.1,
é mencionado a vontade em participar e o desafio trazido
pelo problema.
Também foi mencionado que outros estudantes não
demonstravam interesse e que isto de certa forma
contamina os estudantes interessados.

\subsection{\ProblemaD}
\contribuicoesQualitativas{\ProblemaD}{2016.1}{um}{zero}{2017.1}{três}{um}

Na replicação do semestre 2016.1, sobre o problema, foi mencionado
sobre a necessidade de construção mais simples nas aulas
expositivas para melhor entendimento do problema.
Também foi mencionado sobre as inúmeras possibilidades, em consonância
com os resultados qualitativos onde essa característica foi
percebida, em algum nível, por todos os participantes.

Para a replicação do problema no semestre 2017.1, foi mencionado
que este problema trata de um assunto real, assim como
os dados quantitativos, que podem ser observados
na afirmativa ``\LikertPH'' (H).
Também foi citado que a redução na quantidade de membros no grupo
tutorial, ocasionada pela ausência de estudantes, contribuiu
positivamente para as discussões.

Percebemos que alguns estudantes possuem um perfil mais passivo
de aprendizagem ou possuem outras obrigações que os impedem
de se dedicar em buscar conhecimento, nesse contexto,
um dos participantes, na replicação do problema no
semestre 2017.1, menciona a ausência de conhecimento
sobre o assunto como barreira para a aprendizagem.
Este problema foi construído seguindo
os princípios citados em \cite{dolmans1997seven},
portanto, considerando os conhecimentos prévios
dos estudantes, mas como foi considerado um perfil
genérico para estes, uma vez que não foi
traçado previamente os seus perfis, alguns estudantes
podem não ser exatamente contemplados.
A qualidade da metodologia PBL em estimular os estudantes
em buscar o conhecimento, como mencionado no trabalho
\cite{savery2015overview} é mais efetivo quando
existe essa aderência de perfil com o problema.
O resultado quantitativo, referente a motivação dos
participantes (E), é excelente para esta
replicação, assim, em análise conjunta com o
resultado qualitativo entendemos que o perfil
do participante em questão pode não ter sido
atingido pelo problema.

\subsection{\ProblemaE}
\contribuicoesQualitativas{\ProblemaE}{2016.1}{um}{um}{2017.1}{um}{um}

A dificuldade do problema foi mencionada em ambas as replicações, mas
também foi mencionado como interessante.
Com relação a disciplina, foi mencionado, por participante da
replicação do semestre 2016.1, que o problema deveria ser
mais extenso no conteúdo.
No caso de uma disciplina de Teoria da Computação, entendemos
que a construção de um problema com mais conceitos se
torna ainda mais complexa, em todos os sentidos, sobretudo
para conciliar as recomendações
no trabalho \cite{dolmans1997seven}.

\subsection{\ProblemaF}
\contribuicoesQualitativas{\ProblemaF}{2017.1}{dois}{dois}{2016.1}{zero}{zero}

Para este problema foi mencionada a dificuldade e a falta de estímulos
para resolver por ser um assunto mais matemático.
Os conceitos deste problema exigem um nível de abstração bem elevado
dos estudantes, e por maiores que sejam os esforços na construção
do problema não é simples construir um problema capaz de cumprir com
os objetivos de aprendizagem e que ``escondam'' as dificuldades
em meio ao problema.

No que diz respeito a disciplina, é destacado que a dificuldade
em sentir que possui uma resposta é um desmotivante.
Embora para este problema não tenha sido realizada uma replicação
completa da metodologia PBL, um participante mencionou que não
conseguiu se adaptar, neste caso, entendemos que possa estar
comentando também em relação a outros problemas.

\subsection{\ProblemaG}
\contribuicoesQualitativas{\ProblemaG}{2017.1}{dois}{dois}{}{}{}

Foi mencionado sobre o problema ser extenso e
apresentar uma situação diferente das que as pessoas fariam em caso
real.
O resultado quantitativo em relação ao tempo (L)
foi muito bem avaliado nesta replicação, então, entendemos que
para este caso pode ser uma percepção isolada.
No caso da situação apresentar uma situação real (H), o resultado
quantitativo é apenas razoável, o que está de acordo
com o que foi mencionado pelo participante, assim, para utilizar
este problema em outras replicações é interessante realizar adequações
da situação mencionada.

\subsection{Concluintes}
\contribuicoesQualitativasDisciplina{11}{concluintes}{2016.1}{dois}{dois}
\contribuicoesQualitativasDisciplina{}{concluintes}{2017.1}{três}{três}

Os comentários em relação a metodologia PBL foram na maioria elogiosos.
Foi mencionado que a metodologia ajuda no processo de aprendizagem da
disciplina, além de interessante e motivadora.

A aprendizagem mais profunda dos conteúdos e a necessidade de mais
esforço por parte dos estudantes também foram destacadas.
Acreditamos que nesse contexto, o objetivo de fazer com que
os estudantes assumam também a responsabilidade pela a
sua aprendizagem fica explicitada.
Apesar disto, verificamos que algumas afirmativas relacionadas
com estímulos para desenvolvimento de habilidades para autoaprendizagem
não foram das melhores avaliadas em todas as replicações,
como por exemplo, buscar outras referências de estudo (D),
estudar individualmente fora das sessões PBL (O) e buscar
conhecimento por motivação (P), na replicação dos
problemas ``\ProblemaG'' e ``\ProblemaB'' no semestre 2017.1.

Foi mencionado que alguns estudantes durantes as discussões
não participavam e que isto dificultava o andamento da sessão.
De fato, como mencionado nos
trabalhos \cite{savery2015overview} e \cite{albanese2010problem}
e baseados nas percepções durante as replicações, a eficiência da metodologia
também depende da colaboração entre os participantes.

Não está claro em qual contexto o estudante menciona como útil
para a metodologia PBL a criação de máquinas, uma vez que
foi percebido, em praticamente todas as replicações, que
existiram tentativas de criação das máquinas por partes
dos estudantes.
Acreditamos que alguns estudantes, por estarem em um curso
tradicional, podem de certa forma estarem bastante conectados
à imagem do professor tradicional como um repositório de
respostas.
Na metodologia PBL, o professor, no papel de tutor, é o facilitador
do processo, como mencionado em diversos trabalhos, como
\cite{hmelo2004problem} e \cite{savery2015overview}.
Em nossas percepções durante as replicações, percebemos que
parece ser a melhor estratégia para manter a discussão, ter um
espaço para a manifestações com o máximo de liberdade possível,
ainda que inicialmente possa parecer que a intervenção
de alguns estudante não vá contribuir para
a discussão.
O tutor deve intervir, sem contrapor, dentro das possibilidades,
apenas se percebe que a maioria dos estudantes estão se afastando
dos objetivos de aprendizagem.

Um dos participantes menciona que entende não ter se adequado
a metodologia PBL e este seria o motivo da reprovação por conceito,
apesar de ter destacado que a metodologia é ``interessante'' e ``inteligente''.
A nossa percepção, obtida na execução deste trabalho, é que a replicação
da metodologia PBL pode ser beneficiada com um mapeamento menos genérico
do perfil dos participantes, principalmente em cursos em que a maioria
das disciplinas utiliza uma abordagem tradicional para aprendizagem, que
é a situação da instituição de ensino onde replicamos este trabalho.

\subsection{Desistentes}
\contribuicoesQualitativasDisciplina{15}{desistentes}{2016.1}{um}{um}
\contribuicoesQualitativasDisciplina{x}{desistentes}{2017.1}{zero}{zero}


\section{Considerações finais}
\label{sec-consideracoes-resultados}
% resumo básico
Neste capítulo foram apresentados os resultados do estudo no contexto de execução onde foi aplicada a
metodologia PBL em dez replicações de problemas em duas turmas de estudantes e apresentada a discussão
das hipóteses deste estudo.

% dificuldade em obter respostas
Uma das principais dificuldades deste trabalho foi adesão
por partes dos estudantes em responder os formulários,
como pode ser observado calculando a razão entre
a quantidade de respostas obtidas no formulário de
replicação dos problemas e a quantidade de estudantes
na turma.
A adesão ficou entre $14\%$ e $46\%$, na
quinta replicação do semestre 2017.1 e na primeira
replicação do semestre 2016.1, respectivamente.
Neste cálculo foi considerado a quantidade de estudantes
que iniciaram no semestre, portanto, não foi considerado os
estudantes que eventualmente já haviam desistido
da disciplina.
Acreditamos que a adesão poderia ser melhor se existisse
institucionalmente, desde o ingresso dos estudantes,
ampla divulgação aos estudantes da importância das pesquisas
científicas na construção do conhecimento e que os resultados,
como se propõe este estudo, podem trazer novas oportunidades
para o ensino.

Como foi destacado pelos revisores do nosso trabalho \cite{gavaza2017},
outra questão que devemos observar é que, na metodologia de pesquisa
que adotamos, pode existir uma maior possibilidade
de adesão em responder as pesquisas por partes
dos estudantes ``mais interessados'', portanto, poderiam
estes apresentar uma maior tendência a aceitação.
Embora entendamos a possibilidade desta relação, observamos que
neste estudo também existiram respostas em sentido
contrário, explicitamos um exemplo na discussão da
replicação do problema ``\ProblemaG'', no semestre 2017.1,
na Seção~\ref{sec-2017-p1}, onde exibimos a existência
de participantes que apresentaram discordância integral
para afirmativas.

% não comparativo
O experimental deste estudo não foi construído como comparativo
entre metodologias, portanto, não há indicação, nesse
contexto, para a utilização da metologia deste estudo
em substituição a alguma outra metodologia, inclusive uma
metodologia tradicional.

Além de uma infraestrutura adicional, para um estudo conter
aplicações de abordagens distintas em mais de uma turma,
se faz necessário construir ou investigar critérios de
comparabilidade, que também não foi o propósito deste estudo,
assim, a maioria das afirmativas utilizadas neste estudo
estão focadas em especificidades da metodologia PBL.

% resultados negativos

% resultados positivos

