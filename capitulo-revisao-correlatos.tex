\section{Trabalhos relacionados}
\label{sec-revisao-relacionados}
A motivação e a aprendizagem dos estudantes dos cursos
de Computação são algumas das diversas motivações que tem
estimulado os educadores e pesquisadores de ensino de Computação
a buscar alternativas pedagógicas para ensino e aprendizagem
das disciplinas e temas do curso.

A disciplina de \ac{IHC} está presente em
alguns currículos de Computação com carga horária
própria ou compondo a carga horária de Engenharia de Software.
É uma disciplina que também possui uma grande quantidade de conceitos
a trabalhar e uma clara multidisciplinaridade.

Para a disciplina de \ac{IHC} o trabalho de \cite{pelissoni2003proposta}
utiliza uma abordagem baseada em projetos.
Neste trabalho os conceitos teóricos e técnicas
sobre o desenvolvimento de \textit{interfaces} é construído
pelos estudantes a partir da experiência prática
de construção e avaliação de projeto.
Na experiência os estudantes são agrupados para construir e
avaliar um projeto para a disciplina.
Embora não esteja explicitado em números, está destacado que nas
primeiras fases do projeto a motivação dos estudantes é baixa,
mas que é crescente à medida que cresce o envolvimento com
o projeto.
Apesar de ter descrito um bom resultado para
uma disciplina em que há um grande volume de conteúdo,
que também é o caso da disciplina de Teoria da Computação,
se faz necessário considerar que no caso de conceitos de
Teoria da Computação o nível de abstração exigido dos estudantes
é mais elevado, assim, a opção de conduzir a disciplina com apenas
um projeto em todo o curso exige considerar as influências
que podem ocorrer na motivação dos estudantes.
Além disso, também é necessário considerar a dificuldade para o
educador construir um único projeto capaz de abordar todos os
conceitos de Teoria da Computação.

Os temas das disciplinas teóricas em Computação podem
se tornar mais interessante se for utilizada uma combinação
de diferentes abordagens~\cite{chesnevar2004didactic}.
A proposta de \cite{chesnevar2004didactic} é a utilizar
abordagens construtivistas, para que estudantes deixem
de aplicar o conhecimento mecanicamente e possam ter
uma construção de conhecimento que permita ter pensamento
crítico sobre este conhecimento.
Não está explicito, mas foi realizada uma pesquisa
de percepção com os estudantes, que apresentou elevado índice
de satisfação com abordagens construtivistas, mas não
apresentou resultados com a quantificação da satisfação.

É possível contribuir para o processo de
motivação e aprendizagem dos estudantes se
eles obtiverem respostas
tempestivamente.
Utilizando este argumento, o
trabalho \cite{vieira2003language} construiu
uma ferramenta de aprendizagem denominada
de \textit{Language Emulator} onde os estudantes
podem utilizar expressões regulares, gramáticas
regulares e autômatos finitos.
Apesar de aprovação por $95\%$ dos
participantes, apenas alguns dos conceitos de Teoria
da Computação estão contemplados pela ferramenta.

Os jogos educacionais são propostas comuns para abordagem
pedagógica, e algumas são as propostas que utilizam
desta alternativa para motivação e aprendizagem
de estudantes em disciplinas de Computação.
Os aspectos lúdico e interativo dos jogos educacionais
pode ser uma boa alternativa para auxiliar na aprendizagem
e motivação dos estudantes~\cite{silva2010automata}.

O trabalho \cite{leite2014montanha} propõe utilizar
um jogo educacional com um ambiente para correção
automática de exercícios de Teoria da Computação.
Apesar de bem avaliada pelos estudantes, a ferramenta
contempla apenas uma parte dos temas da disciplina
de Teoria da Computação.

A utilização de um jogo educacional também é a
proposta descrita em \cite{de2011jogo}
como abordagem pedagógica para motivação dos estudantes
e aprendizagem dos conteúdos da disciplina de Teoria da
Computação.
Para os estudantes foi disponibilizado um aplicativo
móvel que permite a eles terem acesso, em qualquer lugar,
em qualquer tempo, de forma facilitada, aos conteúdos
da disciplina de Teoria da Computação, isso permitiria
maior imersão dos estudantes, entretanto, o jogo
descrito neste trabalho parece uma lista de
exercícios com uma única história.
A motivação dos estudantes pode ser prejudicada
em situações em que não são fornecidos caminhos
alternativos para construção do conhecimento,
além disto, a capacidade crítica dos estudantes
sobre os conceitos também pode não ser explorada
ou desenvolvida.

O curso de Engenharia da Computação na
Universidade Estadual de Feira de Santana utiliza
uma abordagem com PBL desde a criação do curso em
2003.
O curso possui dez módulos curriculares em que são
aplicados a metodologia PBL, entre eles podem ser citados
\textit{Estruturas de Dados} e
\textit{Engenharia de Software}.
Apesar de consolidado o método PBL na
Universidade Estadual de Feira de Santana,
não há aplicação da metodologia PBL para a disciplina de
Teoria da Computação~\cite{dospensamento}.
