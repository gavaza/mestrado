\section{Motivação}
\label{sec-revisao-motivacao}
% contextualização
Ao mesmo tempo em que é crescente o número
de estudantes universitários, é grande a
evasão da universidade.
Os altos índices de desistências nos cursos
superiores são evidências da necessidade de
compreensão das variáveis motivacionais
dos estudantes.

% importância da motivação
A motivação do estudante é um determinante crítico 
do  nível  e  da  qualidade  da  aprendizagem
e do desempenho no contexto
escolar.
Desta forma, muitos trabalhos tem relacionado o desempenho
acadêmico dos estudantes com
a motivação~\cite{zenorini2011motivaccao,rufini2011estudo}

% classificação
A motivação para aprendizagem dos estudantes
tem sido classificada em intrínseca e
extrínseca.
A motivação é intrínseca quando a motivação
está na atividade em si e extrínseca quando
a motivação está na recompensa recebida pelo estudante
pela execução da
atividade~\cite{rufini2011estudo,neves2007escala}.

% entendimento dos estudantes
Os estudantes precisam entender o significado e importância
de estudar as disciplinas, assim, o educador
precisa identificar os interesses
dos estudantes~\cite{angeli2011relaccao}.
