No ensino de Computação, a aplicação de PBL ainda é restrita, e geralmente
acontece por iniciativa própria de alguns educadores,
quase sempre isoladamente tentando melhorar a
absorção do conteúdo de uma disciplina
pelos estudantes~\cite{wood2003problem, o2012practical}.

A Computação é uma área do conhecimento onde naturalmente existem
diversos problemas possíveis de construir, portanto, possível para
aplicação de PBL.
Nas disciplinas de programação e de engenharia de software, onde os
cursos são concebidos essencialmente para ensinar os estudantes
a resolver problemas, é ainda mais explícita a possibilidade
de construção de problemas para utilização da abordagem
PBL~\cite{fee2010teaching}.

Na revisão do currículo de Ciências da Computação da ACM em 2008
foram descritas seis habilidades esperadas dos estudantes
de graduação: perspectiva de nível de sistema,
apreciação da interação entre teoria e prática,
familiaridade com temas e princípios comuns,
experiência significativa em projetos,
atenção para o pensamento formal e
adaptabilidade.
A metodologia PBL pode ser utilizada para atingir
todos as seis habilidades, sendo que para as três primeiras
é necessário construção ou seleção de problemas com este objetivo e 
as três últimas são intrínsecas ao processo da abordagem baseada
em problemas~\cite{cassel2008computer}.
