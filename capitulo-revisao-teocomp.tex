\section{Teoria da Computação}
\label{sec-revisao-teocomp}
% origem (história)
Nos anos 1930 os matemáticos lógicos começaram a explorar
o significado de computação~\cite{sipser2006introduction}.

% o que é
Tradicionalmente a Teoria da Computação contém três áreas centrais
do conhecimento: complexidade, computabilidade e autômatos.

% complexidade
A complexidade é área do conhecimento que estuda a complexidade
dos problemas no contexto computacional.
Quando pensamos em atividades, tarefas ou problemas computacionais,
uma variedade destes podem ser construídos.
Alguns destes, podem ser trivialmente fáceis, como por exemplo,
realizar a classificação de uma lista de números inteiros baseado
em algum critério de seleção que seja computacionalmente fácil.
Outros podem ser mais difíceis do ponto de vista computacional,
uma vez que exigem maiores esforços para obter uma resposta, um exemplo seria
realizar a conciliação de uma lista de atividades para uma lista
de agendas, de forma, a atender algum critério
ótimo~\cite{sipser2006introduction}.

A teoria da complexidade formaliza a intuição de dificuldade descrita acima
utilizando modelos matemáticos de computação que estudam problemas
computacionais, nesse contexto, e quantificam os recursos necessários
para resolvê-los, sendo tempo e armazenamento exemplos de recursos
quantificáveis.

% computabilidade
Os matemáticos Kurt G{\"o}del, Alan Turing e Alonzo Church descobriram que
certos tipos de problemas básicos não podem ser
resolvidos por computadores.
Um exemplo seria verificar computacionalmente se um programa em questão
sempre irá parar e apresentar uma resposta para qualquer que seja
as entradas possíveis~\cite{sipser2006introduction}.

A teoria da computabilidade está bastante relacionada com a teoria da complexidade.
Enquanto a teoria da complexidade classifica problemas por dificuldade, a teoria
da computabilidade classifica problemas por capacidade de resolver em sistemas
computacionais~\cite{sipser2006introduction}.

% autômatos
A teoria dos autômatos é um formalismo matemático consolidado na área de
Ciência da Computação que possui diversas aplicações como análise de sintaxe,
verificação de \textit{software}, linguística e padrões
de reconhecimento~\cite{pin2011elements}.
As máquinas construídas com o formalismo da teoria dos autômatos são ferramentas
utilizadas nas construções das teorias da computabilidade
e complexidade.

% contexto
As escolas de Computação, portanto, integram nas grades dos seus cursos disciplinas
para trabalhar os conhecimentos de Teoria da Computação.
