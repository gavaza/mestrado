\xchapter{Revisão bibliográfica}{} %sem preambulo
\label{cap-revisao}
% É recomendável utilizar `\acresetall' no início de cada capítulo para reiníciar o contator de referências às siglas.
\acresetall

Este capítulo apresenta a revisão bibliográfica para este trabalho.

A Seção~\ref{sec-revisao-pbl} apresenta uma revisão sobre a abordagem \ac{PBL};
a Seção~\ref{sec-revisao-teocomp} apresenta os conceitos básicos da disciplina
de Teoria da Computação;
a Seção~\ref{sec-revisao-pbl-computacao} apresenta algumas experiências da
abordagem \ac{PBL} em disciplinas de Teoria da Computação;
a Seção~\ref{sec-revisao-motivacao} apresenta uma breve revisão sobre
estudos de motivação de estudantes;
a Seção~\ref{sec-revisao-relacionados} apresenta alguns dos trabalhos
relacionados;
e, por fim, a Seção~\ref{sec-revisao-consideracoes} apresenta
considerações sobre a aplicação da metodologia.

\section{Problem Based Learning}
\label{sec-revisao-pbl}
% origem
A metodologia de Aprendizagem Baseada em Problemas, em
inglês \textit{Problem Based Learning} (PBL),
foi criada na Universidade McMaster nos anos 1960.
A metodologia \ac{PBL} tem origem no ensino de disciplinas e cursos de medicina,
mas é também utilizada amplamente em disciplinas de enfermagem,
odontologia, artes, arquitetura, arqueologia, engenharias, direitos
etc.~\cite{albanese2010problem, amos1998problem}.

% definição
A metodologia \ac{PBL} pode ser definida como uma abordagem educacional
construtivista ativa que usa problemas como contexto para que os estudantes
adquiram conhecimentos sobre os conceitos. O foco de aprendizagem está
nos estudantes, que são capacitados para que assumam a responsabilidade pela
aprendizagem que acontece nas interações dinâmicas
entre eles, sendo assim, os educadores exercem um papel
de facilitador~\cite{dolmans2005problem, albanese2010problem,
amos1998problem, forsythe2002problem}.

% o que é / como é
Na metodologia, os estudantes trabalham em pequenos grupos colaborativos
e aprendem o que precisam saber para resolver um problema.
O educador atua como um facilitador para orientar a aprendizagem
dos estudantes através do ciclo
de aprendizagem.
A metodologia é uma estratégia pedagógica que situa a aprendizagem em
complexos contextos de resolução de problemas.
Fornece aos estudantes oportunidades para pensar nos relacionamentos
entre os fatos em um contexto de um problema específico em questão.
Os estudantes são obrigados a se perguntar o que precisam saber.
O \ac{PBL} oferece o potencial para ajudar os estudantes a se tornarem
pensadores reflexivos e flexíveis que podem usar o conhecimento
para agir~\cite{hmelo2004problem}.

% eficácia da metodologia
Na metodologia \ac{PBL} o problema é uma ferramenta que busca fornecer
aos estudantes motivações para que eles alcancem os
objetivos de aprendizagem.
Para que a aprendizagem seja efetiva é importante considerar os objetivos
dos estudantes.
Então, sua aprendizagem poderá ser mais eficaz se os cenários utilizados
nos problemas são baseados em situações desencadeantes para
aprendizagem~\cite{wood2003problem, o2012practical, amos1998problem}.

A utilização da metodologia \ac{PBL} não apenas permite desenvolvimento
técnico dos participantes, mas também os capacitam em outros
atributos relevantes da formação, como capacidade de pensar
criticamente, comunicação, trabalho em equipe,
analisar e resolver problemas complexos do mundo real,
autodidatismo, compartilhamento de conhecimento e informações,
argumentação e respeito às
divergências~\cite{wood2003problem, savery2015overview}.

% objetivos da metodologia
O principal objetivo da metodologia é manter a motivação
dos estudantes.
Essa motivação ocorre quando os estudantes trabalham em
uma tarefa motivada por seus próprios interesses, desafiantes
ou tem a sensação de satisfação~\cite{hmelo2004problem}.

% ferramentas utilizadas
Em termos de ferramentas, a metodologia utiliza ferramentas simples,
um quadro branco estruturado com listas de fatos,
ideias (hipóteses), problemas de aprendizagem (questões)
e planos de ação (metas) para ajudar a estruturar a
resolução de problemas dos estudantes e a
aprendizagem~\cite{hmelo2004problem}. 

% o que são os problemas
Os problemas na metodologia se referem aos materiais de instrução apresentados
aos estudantes para desencadear seus processos de aprendizagem.
Os problemas são frequentemente apresentados em formato de texto,
às vezes com imagens e simulações de computador.
Eles geralmente descrevem situações ou fenômenos estabelecidos
em contextos da vida real e exige que os estudantes expliquem ou
resolvam~\cite{hmelo2004problem}.

% pra que serve os problemas
Ainda que as habilidades em resolver problemas possam ser benéficas
para a metodologia baseada em problemas, a resolução de problemas em si não é o
principal objetivo da metodologia.
Os problemas são na metodologia ferramentas para estimular e
favorecer a compreensão dos conceitos pelos
estudantes~\cite{wood2003problem, amos1998problem}.

Os problemas também servem para reativar o conhecimento prévio dos
estudantes, bem como estimular as suas habilidades de aprendizagem
autodirigidas em um processo, na maioria das vezes com colegas,
de explicar, entender e resolver o problema em
questão.
Os problemas permitem que os estudantes forneçam evidências
e raciocínio para pontos de vista e ações.
As informações ficam retidas na memória de longo
prazo~\cite{des1999delphi, azer2012twelve}.

% qualidade dos problemas
A qualidade de construção dos problemas é crítico para o
sucesso da metodologia \ac{PBL}, uma vez que é através destes
que os educadores conduzem a
aprendizagem.
Isso implica que a qualidade de aprendizagem dos estudantes
pode ser melhorada com controle de qualidade dos
problemas~\cite{santos2009analisa,des1999delphi,dolmans1997seven}.

Os criadores de problemas precisam considerar cuidadosamente
os objetivos de aprendizagem pretendidos de um problema
e formular o problema de forma que seja claro e
orientado para os objetivos
de aprendizagem~\cite{sockalingam2011student}.

Apesar da relevância atribuída a qualidade dos problemas para o
sucesso da metodologia, não há instrumentos suficientemente
validados para medir sua
qualidade~\cite{des1999delphi,sockalingam2012assessing}.

Os problemas devem ser construídos de forma a
%Para construção de problemas é necessário
%considerar o nível de
%conhecimento prévio dos estudantes, e claro, 
direcionar os estudantes
para um ou mais objetivos de educacionais.
Geralmente os problemas são construídos baseados em conhecimentos,
experiências e princípios teóricos de aprendizagem e
cognição~\cite{des1999delphi,dolmans1997seven}.
O trabalho \cite{dolmans1997seven} consolida em sete
princípios que devem ser considerados na construção
de problemas:

\begin{enumerate}
\item{O problema deve considerar o conhecimento
prévio dos estudantes porque influencia bastante na
natureza e na quantidade de novas informações que
eles podem processar, uma vez que possuir
uma base de conhecimento a respeito do assunto
discutido no problema irá motivar eles sobre os
detalhes do problema;}
\item{O problema deve sugerir várias dicas
que estimulem os estudantes para discussões
e buscar explicações. O problema não deve
conter tantas dicas ao ponto de o trabalho
para os estudantes ser filtrar as sem relevância,
além disto, conteúdo sem sentido irá distrair
os estudantes;}
\item{O problema deve apresentar um contexto relevante,
apresentar os conceitos como desafios motivantes aos
estudantes, tentando situar a aprendizagem em contextos
semelhantes aos que os estudantes enfrentam em suas
situações na vida real presente ou futura;}
\item{O problema deve apresentar conceitos básicos
relevantes no contexto mais específico.
Isso permite aos estudantes a oportunidade
de integrar o conhecimento mais básico para obter melhores
respostas no contexto mais específico;}
%Permitir que os estudantes gerem hipóteses e discutam
%estas hipóteses~\cite{albanese2010problem, azer2012twelve}.
\item{O problema deve ser ``mal estruturado'', assim como
são os problemas do mundo real, isto é, não
devem ser descritos como especificações ou apresentar
todos os parâmetros, de forma a permitir questões em
aberto para discussão entre os participantes.
Assim, o problema não deve conter as questões do problema
explicitamente e nem referências para uma solução,
para não prejudicar os estudantes na
autoaprendizagem;}
%A capacidade de identificar o problema e definir os
%parâmetros para o desenvolvimento de uma solução é uma
%habilidade desenvolvida em problemas que apresentam
%tal qualidade.
%É necessário considerar também que os participantes
%são menos motivados e envolvidos no caso de problemas
%``bem estruturados''~\cite{savery2015overview}.
\item{O problema deve estar sustentado uma discussão
sobre possíveis soluções e facilitar que os estudantes
explorem alternativas para melhorar o seu interesse
no assunto;}
\item{O problema deve confrontar os estudantes com
os seus objetivos acadêmicos, isso é capaz de
fazer o estudante dedique tempo ao problema.}
\end{enumerate}

Utilizar um formato adequado poderá ajudar os estudantes
a compreender melhor e ajudar na motivação no processo
de resolução de problemas, resultando em uma
melhor aprendizagem.
A escolha do formato do problema pode se basear na análise
das necessidades dos estudantes, no estilo de aprendizagem
e na adequação do formato aos objetivos de
aprendizagem pretendidos.
Em vez de descrição textual apenas, o problema pode
utilizar analogias, comparações ou apresentar imagens,
gráficos etc.~\cite{sockalingam2011student}.

O projeto de construção de problemas adequados é um desafio, uma
vez que um problema ineficaz resulta em falhas no processo de
aprendizagem~\cite{azer2012twelve,amos1998problem,
des1999delphi,kukkamalla2011designing}.

% dificuldade para construir problemas
\suprimir{
TODO: dificuldades na construção dos problemas.
FONTE: azer2012twelve, amos1998problem, des1999delphi
}

% papel do tutor
O educador não é considerado o principal repositório
de conhecimento, é considerado facilitador da aprendizagem
colaborativa.
Na metodologia o educador exerce um papel de tutor, orientando
o processo de aprendizagem através de questionamentos abertos
destinados a fazer com que os estudantes tornem o seu pensamento
visível e que todos os estudantes se mantenham envolvidos
no processo do grupo~\cite{hmelo2004problem}.

É também importante para o sucesso da metodologia \ac{PBL} o
papel do tutor que orienta o processo de aprendizagem
e conduz um debate detalhado na experiência de
aprendizagem~\cite{savery2015overview}.

O tutor ajuda os estudantes a aprender as habilidades cognitivas
necessárias para solucionar problemas e colaborar.
Como os estudos são autodirigidos, com os estudantes gerenciando
seus objetivos de aprendizagem e estratégias
para resolver os problemas mal estruturados do \ac{PBL}, eles
também adquirem as habilidades necessárias para a
aprendizagem ao longo da vida~\cite{hmelo2004problem}.

% ciclo PBL
\suprimir{
TODO: o ciclo PBL
FONTE: azer2012twelve, albanese2010problem, forsythe2002problem
}

O ciclo \ac{PBL} normalmente é executada seguindo sete
passos~\cite{de2014aprendizado,conrado2014aprendizagem}.

\begin{enumerate}
\item{Identificar o problema;}
\item{Definir o problema;}
\item{Formular hipóteses;}
\item{Resumir as hipóteses;}
\item{Propor estudos de investigação;}
\item{Realizar estudos individuais;}
\item{Rediscutir o problema (voltar para o primeiro item).}
\end{enumerate}

Os cinco primeiros passos, normalmente, acontecem durante o período
em que os estudantes possuem contato com os tutores, este período é
denominado de sessão tutorial ou reunião tutorial.
É na primeira sessão tutorial que os estudantes
são apresentados para um cenário problemático.
Eles formulam e analisam o problema identificando os fatos
relevantes do cenário (passo 1 do ciclo \ac{PBL}).
Este passo de identificação de fato ajuda os estudantes
a representar o problema (passo 2 do ciclo \ac{PBL}).
Os estudantes formulam as primeira hipóteses sobre o
problema (passo 3 do ciclo \ac{PBL}).
À medida que os estudantes compreendem o problema melhor, eles detalham
as hipóteses sobre possíveis soluções (passo 4 do ciclo \ac{PBL}).
Uma parte importante desse ciclo é a identificação de deficiências
de conhecimento em relação ao problema (passo 5 do ciclo \ac{PBL}).
Os estudantes realizam estudos próprios para obtenção de
conhecimentos destas deficiências
(passo 6 do ciclo \ac{PBL})~\cite{hmelo2004problem}.

O tutor inicialmente fornece aos estudantes
pequenos problemas iniciais e gradualmente adicionam
complexidade aos novos problemas, removendo orientações
específicas, assim, os problemas apresentam cada vez mais
características dos problemas encontrados no mundo
real~\cite{fee2010teaching}.

Os estudantes, neste processo gradativo de incremento
na complexidade e aproximação da realidade, desenvolvem
as habilidades para enfrentar estas mudanças.
Mas é de se esperar que este processo pode não
funcionar perfeitamente em uma sala de aula
tradicional com todas as dificuldades que conhecemos,
por este motivo, a aplicação pura da abordagem pedagógica
pode ser insuficiente, assim, surgem
desafios~\cite{fee2010teaching}.

% variações no ciclo PBL
\suprimir{
TODO: variações existentes na metodologia.
FONTE: azer2012twelve, albanese2010problem
}

Os estudantes podem julgar necessário e decidir
pela realização de estudos em grupo,
fora do período de uma sessão tutorial, para
os estudos que tem o objetivo de construir
conhecimento nas deficiências identificadas.
Essa é uma variação possível na execução
do ciclo \ac{PBL}.

Outra adaptação comum que pode ser realizada
é com relação ao primeiro passo do ciclo \ac{PBL},
uma vez que os estudantes podem ao fim
da primeira sessão terem fechado
entendimento sobre o que é o problema.
Existe também a possibilidade de que os
conhecimentos adquiridos não construa nenhuma
nova hipótese para os estudantes, assim, o segundo
e terceiro passo do ciclo \ac{PBL} também podem ser
suprimidos.

Em nossa experiência percebemos que para o
fim da primeira sessão tutorial é muito importante
que os estudantes sejam capazes de entender
o que é o problema e ter formulado algumas
hipóteses, ainda que não tenham avançado
significativamente nos demais passos.
Para as demais sessões, a percepção
que temos em nossa experiência é de
que os estudantes devem ter mais
liberdade para passear entre os passos,
não seguindo exatamente a ordem destes passos.
Em todas as sessões tutoriais os estudantes
devem ser capazes de propor estudos
para investigação, os tutores devem
estar atentos a discussão para
eventualmente propor e também para
adequar as propostas de investigação
dos estudantes.

% o que é esperado da metodologia
A metodologia baseada em problemas deve capacitar os estudantes a
realizar pesquisas, integrar teoria e prática e aplicar conhecimentos
e habilidades para desenvolver uma solução viável para
um problema definido~\cite{savery2015overview}.

%São elaborados de forma a estimular os estudantes a lidar com
%problemas ligados ao mundo real e que ao mesmo tempo tenham que
%abordar os conhecimentos teóricos exigidos nos módulos específicos.
%Normalmente tentam simular o ambiente computacional encontrado nas
%empresas, estipulando metas, prazos, comportamentos em grupo,
%tarefas e dinâmicas de empreendendorismo.

% agrupamento dos estudantes
A colaboração entre os estudantes é de grande importância para o
processo de aprendizagem na metodologia \ac{PBL}.
O agrupamento dos estudantes é tanto relevante para o processo quanto
são os problemas. Assim, é necessário dimensionar os grupos de forma
que os estudantes tenham mais facilidade de agendar reuniões e que
os menos participativos possam contribuir e participar das
deliberações~\cite{savery2015overview, albanese2010problem}.

O trabalho \cite{van2000motivation}, entre outras conclusões
relevantes para o estudo da metodologia, chegou a conclusão
que com quanto melhor o grupo funcionar, melhor será o
resultando em avaliações tradicionais de aprendizagem.

% avaliação de aprendizagem
Para medir o nível de aprendizagem dos estudantes é necessário considerar
a importância não só da solução efetivamente produzida, mas também do
processo de resolução do problema.
As interações sociais entre os estudantes durante
o processo também são importantes para a
avaliação de aprendizagem~\cite{albanese2010problem}.

Os resultados obtidos com uma abordagem baseada em problemas para o ensino
produz resultados no mínimo iguais aos das abordagens tradicionais em
termos testes convencionais de conhecimento~\cite{savery2015overview}.

\suprimir{
TODO : comparação PBL x abordagem tradicional
FONTE: savery2015overview
}

% percepção dos estudantes x percepção dos tutores
Uma das conclusões do trabalho \cite{sockalingam2011student} é que
não existem diferenças significativas nas percepções dos estudantes
e dos tutores sobre a eficácia geral dos problemas.
O argumento é que apesar de desempenharem papeis distintos
no processo, além das diferenças de experiências e conhecimento,
ambos são envolvidos no processo de resolução de problemas.
Este consenso entre estudantes e tutores sugere que o \textit{feedback} de estudantes
e tutores sobre a eficácia do problema pode ser útil para melhorar
os problemas.

\section{Teoria da Computação}
\label{sec-revisao-teocomp}
% o que é
Tradicionalmente a Teoria da Computação contém três áreas centrais
do conhecimento: complexidade, computabilidade e autômatos.

% complexidade
A complexidade é área do conhecimento que estuda a complexidade
dos problemas no contexto computacional.
Quando pensamos em atividades, tarefas ou problemas computacionais,
uma variedade destes podem ser construídos.
Alguns destes, podem ser trivialmente fáceis, como por exemplo,
realizar a classificação de uma lista de números inteiros baseado
em algum critério de seleção que seja computacionalmente fácil.
Outros podem ser mais difíceis do ponto de vista computacional,
uma vez que exigem maiores esforços para obter uma resposta.
Um exemplo seria
realizar a conciliação de uma lista de atividades para uma lista
de agendas, atendendo algum critério
ótimo~\cite{sipser2006introduction}.

A teoria da complexidade formaliza a intuição de dificuldade descrita acima
utilizando modelos matemáticos de computação que estudam problemas
computacionais, nesse contexto, e quantificam os recursos necessários
para resolvê-los, sendo tempo e armazenamento exemplos de recursos
quantificáveis.

% computabilidade
Os matemáticos Kurt G{\"o}del, Alan Turing e Alonzo Church descobriram que
certos tipos de problemas básicos não podem ser
resolvidos por computadores.
Um exemplo seria verificar computacionalmente se um programa em questão
sempre irá parar e apresentar uma resposta para qualquer que seja
as entradas possíveis~\cite{sipser2006introduction}.

A teoria da computabilidade está bastante relacionada com a teoria da complexidade.
Enquanto a teoria da complexidade classifica problemas por dificuldade, a teoria
da computabilidade classifica problemas por capacidade de resolver em sistemas
computacionais~\cite{sipser2006introduction}.

% autômatos
A teoria dos autômatos é um formalismo matemático consolidado na área de
Ciência da Computação que possui diversas aplicações como análise de sintaxe,
verificação de \textit{software}, linguística e padrões
de reconhecimento~\cite{pin2011elements}.
As máquinas construídas com o formalismo da teoria dos autômatos são ferramentas
utilizadas nas construções das teorias da computabilidade
e complexidade.

% contexto
As escolas de Computação, portanto, integram nas grades dos seus cursos disciplinas
para trabalhar os conhecimentos de Teoria da Computação.

No ensino de Computação, a aplicação de PBL ainda é restrita, e geralmente
acontece por iniciativa própria de alguns educadores,
quase sempre isoladamente tentando melhorar a
absorção do conteúdo de uma disciplina
pelos estudantes~\cite{wood2003problem, o2012practical}.

A Computação é uma área do conhecimento onde naturalmente existem
diversos problemas possíveis de construir, portanto, possível para
aplicação de PBL.
Nas disciplinas de programação e de engenharia de software, onde os
cursos são concebidos essencialmente para ensinar os estudantes
a resolver problemas, é ainda mais explícita a possibilidade
de construção de problemas para utilização da abordagem
PBL~\cite{fee2010teaching}.

Na revisão do currículo de Ciência da Computação da ACM em 2008
foram descritas seis habilidades esperadas dos estudantes
de graduação: perspectiva de nível de sistema,
apreciação da interação entre teoria e prática,
familiaridade com temas e princípios comuns,
experiência significativa em projetos,
atenção para o pensamento formal e
adaptabilidade.
A metodologia PBL pode ser utilizada para atingir
todos as seis habilidades, sendo que para as três primeiras
é necessário construção ou seleção de problemas com este objetivo e 
as três últimas são intrínsecas ao processo da abordagem baseada
em problemas~\cite{cassel2008computer}.

% contextualização
Ao mesmo tempo em que é crescente o número
de estudantes universitários, é grande a
evasão.
Os altos índices de desistências nos cursos
superiores são evidências da necessidade de
compreensão das variáveis motivacionais
dos estudantes.

% importância da motivação
A motivação do estudante é um determinante crítico 
do  nível  e  da  qualidade  da  aprendizagem
e do desempenho no contexto
escolar.
Desta forma, muitos trabalhos tem relacionado o desempenho
acadêmico dos estudantes com
a motivação~\cite{zenorini2011motivaccao,rufini2011estudo}.

% entendimento dos estudantes
Os estudantes precisam entender o significado e importância
de estudar as disciplinas, assim, o educador
precisa identificar os interesses
dos estudantes~\cite{angeli2011relaccao}.

\section{Trabalhos relacionados}
\label{sec-revisao-relacionados}
A motivação e a aprendizagem dos estudantes dos cursos
de Computação são algumas das diversas motivações que tem
estimulado os educadores e pesquisadores de ensino de Computação
a buscar alternativas pedagógicas para ensino e aprendizagem
das disciplinas e temas do curso.

No ensino de Computação, a aplicação de PBL ainda é restrita, e geralmente
acontece por iniciativa própria de alguns educadores,
quase sempre isoladamente tentando melhorar a
absorção do conteúdo de uma disciplina
pelos estudantes~\cite{wood2003problem, o2012practical}.

A Computação é uma área do conhecimento onde naturalmente existem
diversos problemas possíveis de construir, portanto, possível para
aplicação de PBL.
Nas disciplinas de programação e de engenharia de software, onde os
cursos são concebidos essencialmente para ensinar os estudantes
a resolver problemas, é ainda mais explícita a possibilidade
de construção de problemas para utilização da abordagem
PBL~\cite{fee2010teaching}.

Na revisão do currículo de Ciência da Computação da ACM em 2008
foram descritas seis habilidades esperadas dos estudantes
de graduação: perspectiva de nível de sistema,
apreciação da interação entre teoria e prática,
familiaridade com temas e princípios comuns,
experiência significativa em projetos,
atenção para o pensamento formal e
adaptabilidade.
A metodologia PBL pode ser utilizada para atingir
todos as seis habilidades, sendo que para as três primeiras
é necessário construção ou seleção de problemas com este objetivo e 
as três últimas são intrínsecas ao processo da abordagem baseada
em problemas~\cite{cassel2008computer}.


A disciplina de \ac{IHC} está presente em
alguns currículos de Computação com carga horária
própria ou compondo a carga horária de Engenharia de Software.
É uma disciplina que também possui uma grande quantidade de conceitos
a trabalhar e uma clara multidisciplinaridade.

Para a disciplina de \ac{IHC} o trabalho de \cite{pelissoni2003proposta}
utiliza uma abordagem baseada em projetos.
Neste trabalho os conceitos teóricos e técnicas
sobre o desenvolvimento de \textit{interfaces} é construído
pelos estudantes a partir da experiência prática
de construção e avaliação de projeto.
Na experiência os estudantes são agrupados para construir e
avaliar um projeto para a disciplina.
Embora não esteja explicitado em números, está destacado que nas
primeiras fases do projeto a motivação dos estudantes é baixa,
mas que é crescente à medida que cresce o envolvimento com
o projeto.
Apesar de ter descrito um bom resultado para
uma disciplina em que há um grande volume de conteúdo,
que também é o caso da disciplina de Teoria da Computação,
se faz necessário considerar que no caso de conceitos de
Teoria da Computação o nível de abstração exigido dos estudantes
é mais elevado, assim, a opção de conduzir a disciplina com apenas
um projeto em todo o curso exige considerar as influências
que podem ocorrer na motivação dos estudantes.
Além disso, também é necessário considerar a dificuldade para o
educador construir um único projeto capaz de abordar todos os
conceitos de Teoria da Computação.

Os temas das disciplinas teóricas em Computação podem
se tornar mais interessante se for utilizada uma combinação
de diferentes abordagens~\cite{chesnevar2004didactic}.
A proposta de \cite{chesnevar2004didactic} é a utilizar
abordagens construtivistas, para que estudantes deixem
de aplicar o conhecimento mecanicamente e possam ter
uma construção de conhecimento que permita ter pensamento
crítico sobre este conhecimento.
Não está explicito, mas foi realizada uma pesquisa
de percepção com os estudantes, que apresentou elevado índice
de satisfação com abordagens construtivistas, mas não
apresentou resultados com a quantificação da satisfação.

É possível contribuir para o processo de
motivação e aprendizagem dos estudantes se
eles obtiverem respostas
tempestivamente.
Utilizando este argumento, o
trabalho \cite{vieira2003language} construiu
uma ferramenta de aprendizagem denominada
de \textit{Language Emulator} onde os estudantes
podem utilizar expressões regulares, gramáticas
regulares e autômatos finitos.
Apesar de aprovação por $95\%$ dos
participantes, apenas alguns dos conceitos de Teoria
da Computação estão contemplados pela ferramenta.

Os jogos educacionais são propostas comuns para abordagem
pedagógica, e algumas são as propostas que utilizam
desta alternativa para motivação e aprendizagem
de estudantes em disciplinas de Computação.
Os aspectos lúdico e interativo dos jogos educacionais
pode ser uma boa alternativa para auxiliar na aprendizagem
e motivação dos estudantes~\cite{silva2010automata}.

O trabalho \cite{leite2014montanha} propõe utilizar
um jogo educacional com um ambiente para correção
automática de exercícios de Teoria da Computação.
Apesar de bem avaliada pelos estudantes, a ferramenta
contempla apenas uma parte dos temas da disciplina
de Teoria da Computação.

A utilização de um jogo educacional também é a
proposta descrita em \cite{de2011jogo}
como abordagem pedagógica para motivação dos estudantes
e aprendizagem dos conteúdos da disciplina de Teoria da
Computação.
Para os estudantes foi disponibilizado um aplicativo
móvel que permite a eles terem acesso, em qualquer lugar,
em qualquer tempo, de forma facilitada, aos conteúdos
da disciplina de Teoria da Computação, isso permitiria
maior imersão dos estudantes, entretanto, o jogo
descrito neste trabalho parece uma lista de
exercícios com uma única história.
A motivação dos estudantes pode ser prejudicada
em situações em que não são fornecidos caminhos
alternativos para construção do conhecimento,
além disto, a capacidade crítica dos estudantes
sobre os conceitos também pode não ser explorada
ou desenvolvida.

O curso de Engenharia da Computação na
\ac{UEFS} utiliza uma abordagem com PBL desde a criação
do curso em 2003.
A escolha pela utilização da abordagem baseadas em
problemas ocorreu desde a elaboração do Projeto de curso.
A flexibilidade é uma das principais características do
curso, onde é permitido aos estudantes não só dentro
das disciplinas escolherem por caminhos diversos
para a solução dos problemas e construção do conhecimento,
mas também na forma como este estudante vai integralizar
o seu currículo para formação.
Para aplicação da abordagem \ac{PBL}, no curso da \ac{UEFS},
são utilizadas infraestruturas
específicas~\cite{dos2007aplicaccao, bittencourt2003curriculo}.

O curso de Engenharia de Computação na \ac{UEFS}
possui dez módulos curriculares em que são
aplicados a metodologia PBL, entre eles podem ser citados
\textit{Estruturas de Dados} e
\textit{Engenharia de Software}.
Apesar de consolidado o método PBL na \ac{UEFS}
e inovador no que diz respeito as outras
instituições no Brasil e no exterior,
ainda não há aplicação da metodologia PBL para
a disciplina de Teoria da Computação~\cite{dospensamento}.


\section{Considerações finais}
\label{sec-revisao-consideracoes}
