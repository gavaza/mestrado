\xchapter{Introdução}{} %sem preambulo
% É recomendável utilizar `\acresetall' no início de cada capítulo para reiníciar o contator de referências às siglas.
\acresetall
Os educadores em Ciência da Computação enfrentam muitas questões no que diz
respeito a como conduzir uma disciplina de forma que a absorção do conhecimento
por parte dos estudantes seja satisfatória.
Isto não é um caso particular do ensino de Computação,
mas é necessário considerar que se trata de
um conhecimento relativamente novo, e que só muito
recentemente está presente em instituições de ensino universitário.
Devemos considerar que ainda não existem metodologias e procedimentos
suficientemente testados para o ensino de Computação.
Além disto, há a natureza intrinsecamente multidisciplinar e
o desenvolvimento contínuo da área.
Embora seja possível perceber um contínuo desenvolvimento também em
outras áreas do conhecimento, em Computação é notória a robustez e
continuidade do progresso produzido nas últimas décadas.

O ensino de Computação em geral segue a abordagem tradicional,
com educadores atuando como transmissores de conhecimento
enquanto estudantes são receptores.
Ao verificar as grades curriculares dos cursos de Computação,
serão encontradas algumas atividades práticas em laboratórios,
sobretudo para o ensino de disciplinas com tópicos de programação e
algumas atividades práticas como complemento à carga horária teórica,
constituída essencialmente na exposição de conhecimento pelo educador
aos estudantes.

A forma de ensinar Computação pode ser determinante
para o sucesso também no que diz respeito a
inclusão, esse é um motivo a mais de estímulo para
que os educadores pesquisem na área.

\section{Proposta}
Utilizar uma abordagem baseada em problemas em uma disciplina
introdutória de Teoria da Computação na Universidade Federal da Bahia.

\section{Hipóteses}
Este trabalho se propõe a validar hipóteses gerais e específicas.

\subsection{Hipóteses gerais}
\begin{enumerate}
\item{A metodologia é bem avalidada pelos estudantes nesse contexto;}
\item{Em disciplinas teóricas, como é o caso da disciplina de Teoria da Computação, os problemas
são capazes de cumprir com os objetivos de aprendizagem na percepção dos estudantes;}
\item{Os estudantes se sentem motivados com a utilização da metodologia.}
\end{enumerate}

\subsection{Hipóteses específicas}
\begin{enumerate}
\item{Na percepção dos estudantes as sessões tutoriais contribuem no processo de solução dos problemas;}
\item{Na percepção dos estudantes os problemas construídos são capazes de motivar o trabalho em grupo;}
\item{Na percepção dos estudantes os problemas são úteis no processo de ensino e aprendizagem;}
\item{Na percepção dos estudantes a metodologia estimula a autoaprendizagem.}
\end{enumerate}

\section{Publicações}
\section{Organização do trabalho}
