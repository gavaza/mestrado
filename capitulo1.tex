\xchapter{Introdução}{} %sem preambulo
% É recomendável utilizar `\acresetall' no início de cada capítulo para reiníciar o contator de referências às siglas.
\acresetall
\label{cap-introducao}
Os educadores em Ciência da Computação enfrentam muitas questões no que diz
respeito a como conduzir uma disciplina de forma que a absorção do conhecimento
por parte dos estudantes seja satisfatória.
Isto não é um caso particular do ensino de Computação,
mas é necessário considerar que se trata de
um conhecimento relativamente novo, e que só muito
recentemente está presente em instituições de ensino universitário.
Devemos considerar que ainda não existem metodologias e procedimentos
suficientemente testados para o ensino de Computação.
Além disto, há a natureza intrinsecamente multidisciplinar e
o desenvolvimento contínuo da área.
Embora seja possível perceber um contínuo desenvolvimento também em
outras áreas do conhecimento, em Computação é notória a robustez e
continuidade do progresso produzido nas últimas décadas.

O ensino de Computação em geral segue a abordagem tradicional,
com educadores atuando como transmissores de conhecimento
enquanto estudantes são receptores.
Ao verificar as grades curriculares dos cursos de Computação,
serão encontradas algumas atividades práticas em laboratórios,
sobretudo para o ensino de disciplinas com tópicos de programação e
algumas atividades práticas como complemento à carga horária teórica,
constituída essencialmente na exposição de conhecimento pelo educador
aos estudantes.

A forma de ensinar Computação pode ser determinante
para o sucesso também no que diz respeito a
inclusão, esse é um motivo a mais de estímulo para
que os educadores pesquisem na área.

\section{Proposta}
\label{sec-proposta}
Utilizar uma abordagem baseada em problemas em uma disciplina
introdutória de Teoria da Computação na Universidade Federal da Bahia.

\section{Hipóteses}
\label{sec-hipoteses}
Este trabalho se propõe a validar hipóteses gerais e específicas.

\subsection{Hipóteses gerais}
\label{sec-hipoteses-gerais}
\begin{enumerate}
\item{\label{hg1ref} \hgatexto;}
\item{\label{hg2ref} \hgbtexto;}
\item{\label{hg3ref} \hgctexto.}
\end{enumerate}

\subsection{Hipóteses específicas}
\label{sec-hipoteses-especificas}
\begin{enumerate}
\item{\label{he1ref} \heatexto;}
\item{\label{he2ref} \hebtexto;}
\item{\label{he3ref} \hectexto;}
\item{\label{he4ref} \hedtexto;}
\item{\label{he5ref} \heetexto;}
\item{\label{he6ref} \heftexto.}
\item{\label{he7ref} \hegtexto.}
\end{enumerate}

\section{Publicações}
\label{sec-publicacoes}
Alguns dos resultados parciais desta dissertação já
foram publicados em conferência e revista científica,
como pode ser observado abaixo:
\begin{itemize}
\item{\textbf{Uma experiência de aplicação de uma
abordagem baseada em problemas no ensino
de Teoria da Computação em sala de aula tradicional}.
O trabalho relatou a experiência de aplicação de uma
abordagem de aprendizado baseado em problemas (PBL) para o ensino da
disciplina introdutória de Teoria da Computação utilizando uma
infraestrutura básica na Universidade Federal da Bahia no
semestre 2016.1.
Os resultados dessa experiência mostraram que os estudantes participantes
tiveram boas percepções sobre a abordagem utilizada, o que confirma o
grande potencial para aplicação de PBL no ensino de
disciplinas teóricas como é o caso da disciplina introdutória de
Teoria da Computação.
Os resultados extraídos dos formulários respondido pelos estudantes
foram analisados na ótica da percepção
destes estudantes sobre os problemas.
Os dados foram consolidados e analisados em médias
de favorabilidade~\cite{gavaza2017}.}


\end{itemize}
Os últimos resultados obtidos nesta dissertação estão
em fase de organização para submissão como segue abaixo:

\section{Organização do trabalho}
