\xchapter{Revisão bibliográfica}{} %sem preambulo
% É recomendável utilizar `\acresetall' no início de cada capítulo para reiníciar o contator de referências às siglas.
\acresetall
\section{Problem Based Learning}

% origem
A metodologia de Aprendizagem Baseada em Problemas, em
inglês \textit{Problem Based Learning} (PBL),
foi criada na Universidade McMaster a quase 50 anos.
A metodologia PBL tem origem no ensino de disciplinas e cursos de medicina,
mas é também utilizada amplamente em disciplinas de enfermagem,
odontologia, artes, arquitetura, arqueologia, engenharias, direitos
etc.~\cite{albanese2010problem, amos1998problem}.

% definição
A metodologia PBL pode ser definida como uma abordagem educacional
construtivista ativa que usa problemas como contexto para que os estudantes
adiquiram conhecimentos sobre os conceitos. O foco de aprendizagem está
nos estudantes, que são capacitados para que assumam a responsabilidade pela
aprendizagem que acontece nas interações dinâmicas
entre eles, sendo assim, os professores exercem um papel
de facilitador~\cite{dolmans2005problem, albanese2010problem,
amos1998problem, forsythe2002problem}.

% eficácia da metodologia
Na metodologia PBL o problema é uma ferramenta que busca fornecer
aos estudantes motivações para que eles alcancem os
objetivos de aprendizagem.
Para que a aprendizagem seja efetiva é importante considerar os objetivos
dos estudantes.
Então, sua aprendizagem poderá ser mais eficaz se os cenários utilizados
nos problemas são baseados em situações desencadeantes para
aprendizagem~\cite{wood2003problem, o2012practical, amos1998problem}.

A utilização da metodologia PBL não apenas permite desenvolvimento
técnico dos participantes, mas também os capacita em outros
atributos relevantes da formação, como capacidade de pensar
criticamente, comunicação, trabalho em equipe,
analisar e resolver problemas complexos do mundo real,
autodidatismo, compartilhamento de conhecimento e informações,
argumentação e respeito às
divergências~\cite{wood2003problem, savery2015overview}.

% pra que serve os problemas
Ainda que as habilidades em resolver problemas possam ser benefícas
para a metodologia baseada em problemas, a resolução de problemas em si não é o
principal objetivo da metodologia.
Os problemas são na metodologia ferramentas para estimular e
favorecer a compreensão dos conceitos pelos
estudantes~\cite{wood2003problem, amos1998problem}.

Os problemas também servem para reativar o conhecimento prévio dos
estudantes, bem como estimular as suas habilidades de aprendizagem
autodirigidas em um processo, na maioria das vezes com colegas,
de explicar, entender e resolver o problema em
questão.
Os problemas permitem que os estudantes forneçam evidências
e raciocínio para pontos de vista e ações.
As informações ficam retidas na memória de longo
prazo~\cite{des1999delphi, azer2012twelve}.




% qualidade dos problemas
A qualidade de construção dos problemas é crítico para o
sucesso da metodologia PBL, uma vez que é através destes
que os educadores conduzem a
aprendizagem.
Isso implica que a qualidade de aprendizagem dos estudantes
pode ser melhorada com controle de qualidade dos
problemas~\cite{santos2009analisa,des1999delphi,dolmans1997seven}.

Apesar da relevância atribuída a qualidade dos problemas para o
sucesso da metodologia, não há instrumentos suficientemente
validados para medir sua
qualidade~\cite{des1999delphi,sockalingam2012assessing}

Os problemas devem ser construídos de forma a
%Para construção de problemas é necessário
%considerar o nível de
%conhecimento prévio dos estudantes, e claro, 
direcionar os estudantes
para um ou mais objetivos de educacionais.
Geralmente os problemas são construídos baseados em conhecimentos,
experiências e princípios teóricos de aprendizagem e
cognição~\cite{des1999delphi,dolmans1997seven}.
O trabalho \cite{dolmans1997seven} consolida em sete
princípios que devem ser considerados na construção
de problemas:

\begin{enumerate}
\item{O problema deve considerar o conhecimento
prévio dos estudantes porque influencia bastante na
natureza e na quantidade de novas informações que
eles podem processar, uma vez que possuir
uma base de conhecimento a respeito do assunto
discutido no problema irá motivar eles sobre os
detalhes do problema;}
\item{O problema deve sugerir várias dicas
que estimulem os estudantes para discussões
e buscar explicações. O problema não deve
conter tantas dicas ao ponto de o trabalho
para os estudantes ser filtar as sem relevância,
além disto, conteúdo sem sentido irá distrair
os estudantes;}
\item{O problema deve apresentar um contexto relevante,
apresentar os conceitos como desafios motivantes aos
estudantes, tentando situar a aprendizagem em contextos
semelhantes aos que os alunos enfrentam em suas
situações na vida real presente ou futura;}
\item{O problema deve apresentar conceitos básicos
relevantes no contexto mais específico.
Isso permite aos estudantes a oportunidade
de integrar o conhecimento mais básico para obter melhores
respostas no contexto mais específico;}
%Permitir que os estudantes gerem hipóteses e discutam
%estas hispóteses~\cite{albanese2010problem, azer2012twelve}.
\item{O problema deve ser ``mal estruturado'', assim como
são os problemas do mundo real, isto é, não
devem ser descritos como especificações ou apresentar
todos os parâmetros, de forma a permitir questões em
aberto para discurssão entre os participantes.
Assim, o problema não deve conter as questões do problema
explicitamente e nem referências para uma solução,
para não prejudicar os estudantes na
autoaprendizagem;}
%A capacidade de identificar o problema e definir os
%parâmetros para o desenvolvimento de uma solução é uma
%habilidade desenvolvida em problemas que apresentam
%tal qualidade.
%É necessário considerar também que os participantes
%são menos motivados e envolvidos no caso de problemas
%``bem estruturados''~\cite{savery2015overview}.
\item{O problema deve estar sustentado uma discussão
sobre possíveis soluções e facilitar que os estudantes
explorem alternativas para melhorar o seu interesse
no assunto;}
\item{O problema deve confrontar os estudantes com
os seus objetivos acadêmicos, isso é capaz de
fazer o estudante dedique tempo ao problema.}
\end{enumerate}

% como devem ser os problemas

\suprimir{
TODO: qualidade dos problemas
LOCALIZAR FONTE: A qualidade de construção dos problemas é também importante,
mas não existe na literatura uma quantidade relevante
de estudos.
FONTE: des1999delphi
}

% dificuldade para construir problemas
\suprimir{
TODO: dificuldades na construção dos problemas.
FONTE: azer2012twelve, amos1998problem, des1999delphi
}

% papel do tutor
É também importante para o sucesso da metodologia PBL o
papel do tutor que orienta o processo de aprendizagem
e conduz um debate detalhado na experiência de
aprendizagem~\cite{savery2015overview}.

% ciclo PBL
\suprimir{
TODO: o ciclo PBL
FONTE: azer2012twelve, albanese2010problem, forsythe2002problem
}

% variações no ciclo PBL
\suprimir{
TODO: variações existentes na metodologia.
FONTE: azer2012twelve, albanese2010problem
}

% o que é esperado da metodologia
A metodologia baseada em problemas deve capacitar os estudantes a
realizar pesquisas, integrar teoria e prática e aplicar conhecimentos
e habilidades para desenvolver uma solução viável para
um problema definido~\cite{savery2015overview}.

%São elaborados de forma a estimular os alunos a lidar com
%problemas ligados ao mundo real e que ao mesmo tempo tenham que
%abordar os conhecimentos teóricos exigidos nos módulos específicos.
%Normalmente tentam simular o ambiente computacional encontrado nas
%empresas, estipulando metas, prazos, comportamentos em grupo,
%tarefas e dinâmicas de empreendendorismo.

% agrupamento dos estudantes
A colaboração entre os estudantes é de grande importância para o
processo de aprendizagem na metodologia PBL.
O agrupamento dos estudantes é tanto relevante para o processo quanto
são os problemas. Assim, é necessário dimensionar os grupos de forma
que os estudantes tenham mais facilidade de agendar reuniões e que
os menos participativos possam contribuir e participar das
deliberações~\cite{savery2015overview, albanese2010problem}.

% avaliação de aprendizagem
Para medir o nível de apredizagem dos estudantes é necessário considerar
a importância não só da solução efetivamente produzida, mas também do
processo de resolução do problema.
As interações sociais entre os estudantes durante
o processo também são importantes para a
avaliação de aprendizagem~\cite{albanese2010problem}.

Os resultados obtidos com uma abordagem baseada em problemas para o ensino
produz resultados no mínimo iguais aos das abordagens tradicionais em
termos testes convencionais de conhecimento~\cite{savery2015overview}.
\suprimir{
TODO : comparação PBL x abordagem tradicional
FONTE: savery2015overview
}

\section{Problem Based Learning em Computação}
No ensino de Computação, a aplicação de PBL ainda é restrita, e geralmente
acontece por iniciativa própria de alguns educadores,
quase sempre isoladamente tentando melhorar a
absorção do conteúdo de uma disciplina
pelos estudantes~\cite{wood2003problem, o2012practical}.

\section{Teoria da Computação}
\section{Trabalhos relacionados}
\section{Considerações finais}
