\xchapter{Revisão bibliográfica}{} %sem preambulo
% É recomendável utilizar `\acresetall' no início de cada capítulo para reiníciar o contator de referências às siglas.
\acresetall
\section{Problem Based Learning}
A metodologia PBL pode ser definida como uma abordagem educacional
construtivista em que o foco de aprendizagem está nos estudantes,
que são capacitados para que assumam a responsabilidade pela
aprendizagem~\cite{dolmans2005problem}.
Na metodologia PBL o problema é uma ferramenta que fornece
aos estudantes motivações para que eles alcancem os
objetivos de aprendizagem~\cite{wood2003problem, o2012practical}.
A metodologia PBL não se resume à resolução de problema em si, mas usá-lo como
ferramenta para estimular o estudante e favorecer a compreensão dos
conceitos~\cite{wood2003problem}.
\section{Problem Based Learning em Computação}
\section{Trabalhos relacionados}
\section{Considerações finais}
