\xchapter{Revisão bibliográfica}{} %sem preambulo
% É recomendável utilizar `\acresetall' no início de cada capítulo para reiníciar o contator de referências às siglas.
\acresetall
\section{Problem Based Learning}

A metodologia de Aprendizagem Baseada em Problemas, em inglês \textit{Problem Based Learning} (PBL),
foi criada na Universidade McMaster a quase 50 anos.
A metodologia PBL tem origem no ensino de disciplinas e cursos de medicina, mas é também utilizada
amplamente em disciplinas de enfermagem,
odontologia, artes, arquitetura, arqueologia, engenharias, direitos
etc.~\cite{albanese2010problem, amos1998problem}.

A metodologia PBL pode ser definida como uma abordagem educacional
construtivista ativa que usa problemas como contexto para que os estudantes
adiquiram conhecimentos sobre os conceitos. O foco de aprendizagem está
nos estudantes, que são capacitados para que assumam a responsabilidade pela
aprendizagem que acontece nas interações dinâmicas
entre eles~\cite{dolmans2005problem, albanese2010problem, amos1998problem}.

Na metodologia PBL o problema é uma ferramenta que fornece
aos estudantes motivações para que eles alcancem os
objetivos de aprendizagem.
Para que a aprendizagem seja efetiva é importante considerar os objetivos
dos estudantes.
Então, sua aprendizagem poderá ser mais eficaz se os cenários utilizados
nos problemas são baseados em situações desencadeantes para
aprendizagem~\cite{wood2003problem, o2012practical, amos1998problem}.

Os problemas permitem que os estudantes forneçam evidências
e raciocínio para pontos de vista e ações.
As informações ficam retidas na memória de longo
prazo~\cite{azer2012twelve}.

Os problemas devem tentar situar a aprendizagem em contextos que
são semelhantes aos que os alunos enfrentam em suas situações
da vida real, permintindo que gerem hipóteses e realizem discussões
destas hispóteses~\cite{azer2012twelve}.

A qualidade de construção dos problemas é também importante,
mas não existe na literatura uma quantidade relevante
de estudos.
Os problemas devem ser realistas e apresentar os conceitos como
desafios motivantes aos estudantes, tentando situar a
aprendizagem em contextos semelhantes aos que os alunos
enfrentam em suas situações na vida real.
Permitir que os estudantes gerem hipóteses e discutam
estas hispóteses~\cite{albanese2010problem, azer2012twelve}.

\suprimir{
TODO: dificuldades na construção dos problemas.
FONTE: azer2012twelve, amos1998problem
}

\suprimir{
TODO: o ciclo PBL
FONTE: azer2012twelve, albanese2010problem
}

\suprimir{
TODO: variações existentes na metodologia.
FONTE: azer2012twelve, albanese2010problem
}

Ainda que as habilidades em resolver problemas possam ser benefícas
para a metodologia PBL, a resolução de problemas em si não é o
principal objetivo da metodologia.
Os problemas são na metodologia ferramentas para estimular e
favorecer a compreensão dos conceitos pelos
estudantes~\cite{wood2003problem,amos1998problem}.

O agrupamento dos estudantes é tanto relevante para o processo quanto
são os problemas. Assim, é necessário dimensionar os grupos de forma
que os estudantes tenham mais facilidade de agendar reuniões e que
os menos participativos possam contribuir e participar das
deliberações~\cite{albanese2010problem}.

Para medir o nível de apredizagem dos estudantes é necessário considerar
a importância não só da solução efetivamente produzida, mas também do
processo de resolução do problema.
As interações sociais entre os estudantes durante
o processo também são importantes para a
avaliação de aprendizagem~\cite{albanese2010problem}.


\section{Problem Based Learning em Computação}
\section{Teoria da Computação}
\section{Trabalhos relacionados}
\section{Considerações finais}
