\xchapter{Experiência em sala de aula}{} %sem preambulo
% É recomendável utilizar `\acresetall' no início de cada capítulo para reiníciar o contator de referências às siglas.
\acresetall

Este capítulo irá descrever a experiência da condução da disciplina em um primeiro
semestre onde foi adotada apenas a abordagem tradicional de ensino e aprendizagem,
depois descreve a experiência em dois semestres subsequentes onde foi também aplicada a metologia PBL,
sendo assim, uma abordagem híbrida.

Para todos os três semestres mencionados foi utilizado a disciplina que apresenta os
conceitos introdutórios de Teoria da Computação para estudantes de graduação no turno noturno
na Universidade Federal da Bahia.

A disciplina tem carga horária de 68 horas conduzida ao longo de um semestre, sendo duas
aulas por semana com 2 horas cada.
É ofertada aos estudantes anualmente, no primeiro semestre de cada ano.

Na experiência foi utilizado a oportunidade como critério de seleção de participantes, onde todos os
estudantes regularmente inscritos na disciplina participaram da abordagem.

Para cada semestre da disciplina é disponilizado aos estudantes um ambiente virtual de aprendizagem,
que eles são incentivados a utilizar ao longo da disciplina.
Nesse ambiente virtual algumas das utilizações dos estudantes foram discutir sobre os conceitos
em fóruns, apresentar soluções para os problemas quando na abordagem com PBL e responder
questionários de percepção sobre a disciplina.

\section{Semestre 2015.1}
Neste semestre foi utilizado apenas a abordagem tradicional de ensino e aprendizagem, onde o educador
com ajuda de recursos como quadro e projetor apresenta os conceitos para os estudantes.

Para este semestre foram 24 estudantes inscritos na disciplina.


\section{Semestre 2016.1}

Neste semestre foi utilizado uma abordagem híbrida com abordagem tradicional de ensino
e aprendizagem, e metodologia PBL.

A carga horária de 68 horas da disciplina no semestre foi dividida em:
26 horas para a realização de sessões tutoriais de PBL em que foi conduzida a abordagem;
32 horas para a realização de aulas expositivas em que o educador apresentou conceitos da disciplina;
e 10 horas para a realização de avaliações tradicionais.

A experiência aconteceu em uma sala de aula tradicional equipada com um
quadro branco.
Nas sessões tutorias os estudantes foram distribuídos em semicírculo de forma que todos foram capazes de
enxergar o quadro e tiveram interação entre eles facilitada.

Para este semestre foram 26 estudantes inscritos na disciplina.

\section{Semestre 2017.1}
Neste semestre foi utilizado uma abordagem híbrida com abordagem tradicional de ensino
e aprendizagem, e metodologia PBL.

Para este semestre foram 50 estudantes inscritos na disciplina.
\section{Considerações finais}
