\newcommand{\descricaoSemestre}[6]{
Neste semestre foi utilizado uma abordagem híbrida com abordagem tradicional de ensino
e aprendizagem, e metodologia PBL.

A carga horária de #1 horas da disciplina no semestre foi dividida em:
#2 horas para a realização de sessões tutoriais de PBL em que foi conduzida a abordagem;
#3 horas para a realização de aulas expositivas em que o educador apresentou
conceitos da disciplina;
e #4 horas para a realização de avaliações tradicionais.

A experiência aconteceu em uma sala de aula tradicional equipada com um
quadro branco{\ifthenelse{\equal{#5}{1}}{.}{ e com quadros adicionais
feitos com papel metro colados nas paredes da sala.}}
Nas sessões tutorias os estudantes foram
distribuídos{\ifthenelse{\equal{#5}{1}}{ em semicírculo de
forma que todos foram capazes de enxergar o quadro e
tiveram interação entre eles facilitada.}{ em grupos de
cinco até dez participantes.
Os grupos formaram semicírculos de forma que todos foram
capazes de enxergar o quadro adicional referente ao grupo
ao qual foi designado e tiveram interação dentro do grupo
facilitada.}}

% perfil dos participantes
Para este semestre foram #6 estudantes inscritos na disciplina.}

\newcommand{\descricaoSemestreProblemas}[4]{% problemas utilizados

Foram escolhidos #1 problemas, para este semestre, para trabalhar o
conteúdo da disciplina em conjunto com a abordagem tradicional.
No que se refere a abordagem tradicional, os conteúdos foram trabalhados
com os estudantes em aulas expositivas utilizando quadro e apresentações
pelo professor.
Ao longo deste semestre, os estudantes {\ifthenelse{\equal{#2}{1}}
{construíram uma apresentação, um seminário, }
{construíram um projeto de desenvolvimento, }
com conteúdos relacionados aos
compiladores, como por exemplo, analisadores léxico, sintático e semântico,
otimizações de código, árvores de derivação}.

Para avaliação de conceito dos estudantes foram atribuídas notas para
o {\ifthenelse{\equal{#2}{1}}{seminário}{projeto de desenvolvimento}},
para os problemas e duas avaliações escritas individuais e sem consultas.
No caso dos problemas, a nota considerou assiduidade e participação dos
estudantes nas discussões das sessões tutoriais e o produto produzido
como solução para o problema.

O problema ``#3'' foi utilizado como demonstração
aos estudantes do funcionamento da metodologia PBL e por não abordar
especificamente nenhum conteúdo da disciplina não foi considerado para
efeitos de avaliação dos estudantes, nem foi considerado para
efeitos de análise deste estudo.

Para este semestre os problemas aplicados para
avaliação dos estudantes foram:
\begin{enumerate}
\item{{\ifthenelse{\equal{#4}{1}}{Coleção de músicas}{O Código Morse}}}
\item{A máquina de vender refrigerantes}
\item{O controle de tráfego (com Autômatos Finitos com Pilha)}
\item{O controle de tráfego (com Máquina de Turing)}
\item{Uma função bastante curiosa}
\item{P \textit{versus} NP (P = NP?)}
\end{enumerate}

Um detalhe a considerar é que embora as discussões referente ao sexto problema
tenha seguido a metodologia PBL, ao estudante foi solicitado responder
uma lista de exercícios referente as questões apresentadas ao longo
do texto ao invés da produção de um produto solução para o problema.
Esse problema não é considerado para efeitos análises e resultados
deste estudo.
}

\xchapter{Experiência em sala de aula}{} %sem preambulo
\label{cap-experiencia}
% É recomendável utilizar `\acresetall' no início de cada capítulo para reiníciar o contator de referências às siglas.
\acresetall

Este capítulo descreve a experiência em dois semestres onde foi aplicada a metologia PBL
em conjunto com a abordagem tradicional, sendo assim, uma abordagem híbrida.

Para os dois semestres mencionados foi utilizada a disciplina que apresenta os
conceitos introdutórios de Teoria da Computação para estudantes de graduação de cursos de computação
no turno noturno na Universidade Federal da Bahia.

A disciplina tem carga horária de 68 horas conduzida ao longo de um semestre, sendo duas
aulas por semana com 2 horas cada.
É ofertada aos estudantes anualmente, no primeiro semestre de cada ano.

Na experiência foi utilizado a oportunidade como critério de seleção de participantes, onde todos os
estudantes regularmente inscritos na disciplina participaram da abordagem.

\section{Contexto educacional}
% curso
Os cursos da área de informática na Universidade Federal da Bahia
utilizam majoritariamente a abordagem tradicional, onde
algumas disciplinas específicas contam com carga horária específica
para laboratórios práticos.
Existem algumas poucas iniciativas de novas abordagens como é
o caso deste estudo.

Apesar de ser uma disciplina situada no início do
curso, entre os estudantes participantes deste estudo,
é grande a quantidade de estudantes que já abandonaram uma
outra disciplina do curso.
A disciplina deste estudo também possui historicamente um índice
elevado de evasão.



% papeis dos estudantes
O primeiro passo a cada sessão tutorial é a escolha dentre os
estudantes de três voluntários.
Um voluntário é responsável por realizar o registro da discussão
da sessão no quadro branco, esse
participante é denominado \textit{relator de quadro}.
Um segundo voluntário é responsável por realizar o registro
das discussões e disponibilizar um documento consolidado para todos
os participantes logo após
a sessão, sendo denominado \textit{relator de mesa}.
A discussão é conduzida por um terceiro voluntário,
denominado \textit{coordenador da sessão}, que administra
as intervenções dos participantes, permitindo que esses
tenham espaço para se posicionar.
Existe um rodízio nos papéis a cada sessão para que todos
os estudantes tenham oportunidade de desempenhar todos
os papéis descritos acima.

% o tutor
Os tutores são os principais responsáveis por facilitar o andamento
das sessões tutoriais, zelando pelos objetivos de aprendizagem.
Durante as sessões tutoriais as intervenções dos tutores
foram mínimas, apenas em casos extremos de
afastamento dos conceitos estudados no problema existiu intervenção dos tutores.

% como decorre a sessão tutorial
Os estudantes são apresentados ao problema na primeira reunião, durante
a qual eles discutem um primeiro esboço geral do problema com seus
colegas de equipe em supervisão dos tutores.
A sessão segue com argumentações, exposição de ideias,
questionamentos e levantamento de fatos.
Nos minutos finais da reunião, os estudantes propõe metas de estudos para
que apresentem ao longo da discussão das próximas sessões tutoriais.
As metas apresentadas pelos estudantes são verificadas pelos tutores
para que estejam em um nível adequado, isto é, plausíveis e exequíveis
dentro do prazo, uma vez que metas não
alcancáveis para o período poderia desistimular os estudantes, assim
como metas muito fáceis de alcançar também poderia levar
ao mesmo problema.

% ambiente virtual de aprendizagem (AVA)
Foi disponilizado aos estudantes um ambiente virtual de aprendizagem
institucional, que eles foram incentivados a utilizar ao longo
da disciplina.
A condução da disciplina estimulou os estudantes a
utilizarem este ambiente virtual que para eles serviu como um
espaço para continuidade das discussões sobre os problemas
e conceitos em fóruns, além de repositório para armazenar
os conteúdos produzidos.
O documento consolidado que é produzido pelo relator de mesa
durante a sessão tutorial é disponibilizado em um espaço
do ambiente virtual.
Para este estudo, o ambiente de virtual serviu
também como ferramenta para obteção dos dados,
como está detalhado no
capítulo~\ref{cap-resultados}.

\section{Problemas}
Os problemas foram construídos seguindo as diretivas
apresentadas no Capítulo~\ref{cap-revisao}.
Nesta seção os problemas são apresentados com uma breve
descrição do contexto e objetivos de aprendizagem.
Os textos dos problemas estão disponíveis no
Apêndice~\ref{cap-problemas-textos}

\subsection{Coleção de músicas}
O problema menciona um colecionador musical que deseja que
novas músicas sejam criadas utilizando como base as músicas
de uma biblioteca fornecida.
O objetivo desse problema é trabalhar os conceitos de
Linguagens Formais por meio de uma equivalência dos conceitos
de notas musicais com os conceitos de símbolos e cadeias.

\subsection{A máquina de vender refrigerantes}
O problema menciona uma empresa que deseja construir uma máquina
de vender refrigerantes.
O objetivo desse problema é trabalhar os conceitos de Autômatos
Finitos e introduzir o conceito de não determinismo.

\subsection{O controle de tráfego (com Autômatos
Finitos com Pilha)}
\label{problema3}
O problema menciona a situação de buracos em uma estrada e
solicita aos estudantes que realizem o balanceamento da proporção
entre veículos leves e pesados.
O objetivo desse problema é trabalhar os conceitos de Autômatos
Finitos com Pilha.

\subsection{O controle de tráfego (com Máquina de Turing)}
O problema é uma extensão do descrito na Seção~\ref{problema3}.
Neste problema, o estudante consegue identificar que apenas uma pilha
é insuficiente para realizar o controle da proporção proposta neste
problema.
O objetivo é trabalhar os conceitos de Máquinas de Turing.

\subsection{Uma função bastante curiosa}
O problema relata sobre a capacidade dos compiladores
em identificar erros e otimizar códigos e convida os
estudantes a discutirem o quanto um compilador
pode ser inteligente.
O objetivo é trabalhar o Problema da Parada.

\subsection{P \textit{versus} NP (P = NP?)}
O problema apresenta uma discussão sobre as questões
de classes de problema P \textit{versus} NP.
O objetivo é trabalhar as questões de complexidade.

\subsection{O defeito nos computadores}
Este é um problema construído apenas para exemplificar
a metodologia PBL para os estudantes.
Este problema relata falhas que acontecem
em computadores de um laboratório
de informática da universidade.
O objetivo deste problema é mostrar aos estudantes
a condução na metologia.

\section{Semestre 2016.1}
\descricaoSemestre{68}{26}{32}{10}{1}{26}
\descricaoSemestreProblemas{seis}{1}{O defeito nos computadores}{1}

\section{Semestre 2017.1}
\descricaoSemestre{68}{26}{32}{10}{2}{50}
\descricaoSemestreProblemas{seis}{2}{A torta e o ladrão de pimenta}{2}

\section{Considerações finais}
