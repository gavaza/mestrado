\xchapter{Experiência em sala de aula}{} %sem preambulo
% É recomendável utilizar `\acresetall' no início de cada capítulo para reiníciar o contator de referências às siglas.
\acresetall

Este capítulo descreve a experiência da condução da disciplina em um primeiro
semestre onde foi adotada apenas a abordagem tradicional de ensino e aprendizagem,
depois descreve a experiência em dois semestres subsequentes onde foi também aplicada a metologia PBL,
sendo assim, uma abordagem híbrida.

Para todos os três semestres mencionados foi utilizada a disciplina que apresenta os
conceitos introdutórios de Teoria da Computação para estudantes de graduação de cursos de computação
no turno noturno na Universidade Federal da Bahia.

A disciplina tem carga horária de 68 horas conduzida ao longo de um semestre, sendo duas
aulas por semana com 2 horas cada.
É ofertada aos estudantes anualmente, no primeiro semestre de cada ano.

Na experiência foi utilizado a oportunidade como critério de seleção de participantes, onde todos os
estudantes regularmente inscritos na disciplina participaram da abordagem.

Para cada semestre da disciplina foi disponilizado aos estudantes um ambiente virtual de aprendizagem,
que eles foram incentivados a utilizar ao longo da disciplina.
Nesse ambiente virtual algumas das utilizações dos estudantes foram discutir sobre os conceitos
em fóruns, apresentar soluções para os problemas quando na abordagem com PBL e responder
questionários de percepção sobre a disciplina.

\section{Semestre 2015.1}
Neste semestre foi utilizado apenas a abordagem tradicional de ensino e aprendizagem, onde o educador
com ajuda de recursos como quadro e projetor apresenta os conceitos para os estudantes.

Para este semestre foram 24 estudantes inscritos na disciplina.


\section{Semestre 2016.1}

Neste semestre foi utilizado uma abordagem híbrida com abordagem tradicional de ensino
e aprendizagem, e metodologia PBL.

A carga horária de 68 horas da disciplina no semestre foi dividida em:
26 horas para a realização de sessões tutoriais de PBL em que foi conduzida a abordagem;
32 horas para a realização de aulas expositivas em que o educador apresentou conceitos da disciplina;
e 10 horas para a realização de avaliações tradicionais.

A experiência aconteceu em uma sala de aula tradicional equipada com um
quadro branco.
Nas sessões tutorias os estudantes foram distribuídos em semicírculo de forma que todos foram capazes de
enxergar o quadro e tiveram interação entre eles facilitada.

Para este semestre foram 26 estudantes inscritos na disciplina.

\subsection{Problema 1}
\subsection{Problema 2}
\subsection{Problema 3}
\subsection{Problema 4}
\subsection{Problema 5}

\section{Semestre 2017.1}
Neste semestre foi utilizado uma abordagem híbrida com abordagem tradicional de ensino
e aprendizagem, e metodologia PBL.

A carga horária de 68 horas da disciplina no semestre foi dividida em:
26 horas para a realização de sessões tutoriais de PBL em que foi conduzida a abordagem;
32 horas para a realização de aulas expositivas em que o educador apresentou conceitos da disciplina;
e 10 horas para a realização de avaliações tradicionais.

A experiência aconteceu em uma sala de aula tradicional equipada com um
quadro branco e com quadros adicionais feitos com papel metro colados nas paredes da sala.
Nas sessões tutorias os estudantes foram distribuídos em grupos de cinco até dez participantes.
Os grupos formaram semicírculos de forma que todos foram capazes de
enxergar o quadro adicional referente ao grupo ao qual foi designado
e tiveram interação dentro do grupo facilitada.

Para este semestre foram 50 estudantes inscritos na disciplina.

\subsection{Problema 1}
\subsection{Problema 2}
\subsection{Problema 3}
\subsection{Problema 4}
\subsection{Problema 5}
\section{Considerações finais}
