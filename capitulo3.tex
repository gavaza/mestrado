\xchapter{Experiência em sala de aula}{} %sem preambulo
\label{cap-experiencia}
% É recomendável utilizar `\acresetall' no início de cada capítulo para reiníciar o contator de referências às siglas.
\acresetall

Este capítulo descreve a experiência da condução da disciplina em um primeiro
semestre onde foi adotada apenas a abordagem tradicional de ensino e aprendizagem,
depois descreve a experiência em dois semestres subsequentes onde foi também aplicada a metologia PBL,
sendo assim, uma abordagem híbrida.

Para os dois semestres mencionados foi utilizada a disciplina que apresenta os
conceitos introdutórios de Teoria da Computação para estudantes de graduação de cursos de computação
no turno noturno na Universidade Federal da Bahia.

A disciplina tem carga horária de 68 horas conduzida ao longo de um semestre, sendo duas
aulas por semana com 2 horas cada.
É ofertada aos estudantes anualmente, no primeiro semestre de cada ano.

Na experiência foi utilizado a oportunidade como critério de seleção de participantes, onde todos os
estudantes regularmente inscritos na disciplina participaram da abordagem.

Para cada semestre da disciplina foi disponilizado aos estudantes um ambiente virtual de aprendizagem,
que eles foram incentivados a utilizar ao longo da disciplina.
Nesse ambiente virtual algumas das utilizações dos estudantes foram discutir sobre os conceitos
em fóruns, apresentar soluções para os problemas quando na abordagem com PBL e responder
questionários de percepção sobre a disciplina.

\section{Semestre 2016.1}

Neste semestre foi utilizado uma abordagem híbrida com abordagem tradicional de ensino
e aprendizagem, e metodologia PBL.

A carga horária de 68 horas da disciplina no semestre foi dividida em:
26 horas para a realização de sessões tutoriais de PBL em que foi conduzida a abordagem;
32 horas para a realização de aulas expositivas em que o educador apresentou conceitos da disciplina;
e 10 horas para a realização de avaliações tradicionais.

A experiência aconteceu em uma sala de aula tradicional equipada com um
quadro branco.
Nas sessões tutorias os estudantes foram distribuídos em semicírculo de forma que todos foram capazes de
enxergar o quadro e tiveram interação entre eles facilitada.

% perfil dos participantes
Para este semestre foram 26 estudantes inscritos na disciplina.

\section{Semestre 2017.1}
Neste semestre foi utilizado uma abordagem híbrida com abordagem tradicional de ensino
e aprendizagem, e metodologia PBL.

A carga horária de 68 horas da disciplina no semestre foi dividida em:
26 horas para a realização de sessões tutoriais de PBL em que foi conduzida a abordagem;
32 horas para a realização de aulas expositivas em que o educador apresentou conceitos da disciplina;
e 10 horas para a realização de avaliações tradicionais.

A experiência aconteceu em uma sala de aula tradicional equipada com um
quadro branco e com quadros adicionais feitos com papel metro colados nas paredes da sala.
Nas sessões tutorias os estudantes foram distribuídos em grupos de cinco até dez participantes.
Os grupos formaram semicírculos de forma que todos foram capazes de
enxergar o quadro adicional referente ao grupo ao qual foi designado
e tiveram interação dentro do grupo facilitada.

Para este semestre foram 50 estudantes inscritos na disciplina.

\section{Contexto educacional}
% papeis dos estudantes
O primeiro passo a cada sessão tutorial é a escolha dentre os
estudantes de três voluntários.
Um voluntário é responsável por realizar o registro da discussão
da sessão no quadro branco, esse
participante é denominado \textit{relator de quadro}.
Um segundo voluntário é responsável por realizar o registro
das discussões e disponibilizar um documento consolidado para todos
os participantes logo após
a sessão, sendo denominado \textit{relator de mesa}.
A discussão é conduzida por um terceiro voluntário,
denominado \textit{coordenador da sessão}, que administra
as intervenções dos participantes, permitindo que esses
tenham espaço para se posicionar.
Existe um rodízio nos papéis a cada sessão para que todos
os estudantes tenham oportunidade de desempenhar todos
os papéis descritos acima.

% o tutor
Os tutores são os principais responsáveis por facilitar o andamento
das sessões tutoriais, zelando pelos objetivos de aprendizagem.
Durante as sessões tutoriais as intervenções dos tutores
foram mínimas, apenas em casos extremos de
afastamento dos conceitos estudados no problema existiu intervenção dos tutores.

% como decorre a sessão tutorial
Os estudantes são apresentados ao problema na primeira reunião, durante
a qual eles discutem um primeiro esboço geral do problema com seus
colegas de equipe em supervisão dos tutores.
A sessão segue com argumentações, exposição de ideias,
questionamentos e levantamento de fatos.
Nos minutos finais da reunião, os estudantes propõe metas de estudos para
que apresentem ao longo da discussão das próximas sessões tutoriais.
As metas apresentadas pelos estudantes são verificadas pelos tutores
para que estejam em um nível adequado, isto é, plausíveis e exequíveis
dentro do prazo, uma vez que metas não
alcancáveis para o período poderia desistimular os estudantes, assim
como metas muito fáceis de alcançar também poderia levar
ao mesmo problema.

% ambiente virtual de aprendizagem (AVA)
A condução da disciplina estimulou os estudantes a
utilizarem um ambiente virtual de aprendizagem
institucional que para eles serviu como um
espaço para continuidade das discussões e para
expor dúvidas, além de repositório para
armazenar os conteúdos produzidos.
O documento consolidado que é produzido pelo relator de mesa
durante a sessão tutorial é disponibilizado em um espaço
do ambiente virtual.
Para este estudo, o ambiente de virtual serviu
também como ferramenta para obteção dos dados,
como está detalhado no
capítulo~\ref{cap-resultados}.

\section{Problemas}

\subsection{Coleção de músicas}
O problema menciona um colecionador musical que deseja que
novas músicas sejam criadas utilizando como base as músicas
de uma biblioteca fornecida.
O objetivo desse problema é trabalhar os conceitos de
Linguagens Formais por meio de uma equivalência dos conceitos
de notas musicais com os conceitos de símbolos e cadeias.

\subsection{A máquina de vender refrigerantes}
O problema menciona uma empresa que deseja construir uma máquina
de vender refrigerantes.
O objetivo desse problema é trabalhar os conceitos de Autômatos
Finitos e introduzir o conceito de não determinismo.

\subsection{O controle de tráfego (com Autômatos
Finitos com Pilha)}
\label{problema3}
O problema menciona a situação de buracos em uma estrada e
solicita aos estudantes que realizem o balanceamento da proporção
entre veículos leves e pesados.
O objetivo desse problema é trabalhar os conceitos de Autômatos
Finitos com Pilha.

\subsection{O controle de tráfego (com mT)}
O problema é uma extensão do descrito na Seção~\ref{problema3}.
Neste problema, o estudante consegue identificar que apenas uma pilha
é insuficiente para realizar o controle da proporção proposta neste
problema.
O objetivo é trabalhar os conceitos de Máquinas de Turing.

\subsection{Uma função bastante curiosa}
O problema relata sobre a capacidade dos compiladores
em identificar erros e otimizar códigos e convida os
estudantes a discutirem o quanto um compilador
pode ser inteligente.
O objetivo é trabalhar o Problema da Parada.

\subsection{P \textit{versus} NP (P = NP?)}
O problema apresenta uma discussão sobre as questões
de classes de problema P \textit{versus} NP.
O objetivo é trabalhar as questões de complexidade.

\section{Considerações finais}
