\xchapter{Experiência em sala de aula}{} %sem preambulo
% É recomendável utilizar `\acresetall' no início de cada capítulo para reiníciar o contator de referências às siglas.
\acresetall

Este capítulo irá descrever detalhes da condução da disciplina num primeiro
semestre onde foi adotada apenas a abordagem tradicional de ensino e aprendizagem e
em dois semestres subsequentes onde foi também aplicada a metologia PBL, sendo assim,
uma disciplina híbrida.

Para todos os três semestres mencionados foi utilizado a disciplina que apresenta os
conceitos introdutórios de Teoria da Computação para estudantes de graduação no turno noturno
na Universidade Federal da Bahia.

A disciplina tem carga horária de 68 horas conduzida ao longo de um semestre, sendo duas
aulas por semana com 2 horas cada.
É ofertada aos estudantes anualmente, no primeiro semestre de cada ano.

Na experiência foi utilizado a oportunidade como critério de seleção de participantes, onde todos os
estudantes regularmente inscritos na disciplina participaram da abordagem.

\section{Semestre 2015.1}
Neste semestre foi utilizado apenas a abordagem tradicional de ensino e aprendizagem, onde o professor
com ajuda de recursos como quadro e projetor apresenta os conceitos para os estudantes.

Para este semestre foram 24 estudantes inscritos na disciplina.

\section{Semestre 2016.1}

Neste semestre foi utilizado uma abordagem híbrida com abordagem tradicional de ensino
e aprendizagem, e metodologia PBL.

Para este semestre foram 26 estudantes inscritos na disciplina.

\section{Semestre 2017.1}
\section{Considerações finais}
