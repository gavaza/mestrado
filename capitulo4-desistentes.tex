\subsection{Percepção dos desistentes}
\QtdParticipantes{2016.1}{15}{0}{3}
\perfilParticipanteC{24,0}{23}{25}{66,7}{100,0}{até o sétimo semestre}{}
\figuraPercepcaoParticipanteDisciplina{desistentes-2016}{desistentes}{2016.1}
\AprovacaoDisciplina{0,0}{desistentes}{aprovação da metodologia para a}%
{a metodologia}{DA}{}{}
Nenhum dos participantes apresentou como motivo para a desistência a
oposição em ter a presença como critério de avaliação (DB) e
em ter várias avaliações (DD), mas existe oposição por $33,3\%$
em ter a participação nas aulas como critério de avaliação (DC) e em ter
avaliações em equipe (DE).
Os tutores não foram motivadores da desistência (DF) para nenhum
dos desistentes.
Dificuldade de conciliar a disciplina com
o trabalho profissional (DG) e questões pessoais (DG1)
foram os motivos mais apontados para desistência,
com $66,7\%$.
\AprovacaoDisciplinaA{66,7}{entenderam a metodologia PBL}{DH}
\AprovacaoDisciplinaF{33,7}{é um problema falar em público}{DI}{não}
\AprovacaoDisciplinaB{66,7}{experimentar outras abordagens além das aulas tradicionais}{DJ}{}{0,0}
Apesar de apenas $33,3\%$ dos participantes ter respondido ter interesse
em cursar a disciplina com metodologia PBL (DO1) em outra oportunidade,
não existiu manisfestação contrária.
Para $66,7\%$ a carga horária de aulas expositivas deveria ser maior (DO).

\figuraPercepcaoParticipanteDisciplinaNotasA{2016-desistentes}{desistentes}{2016.1}{30}
{5,00}{5,67}{apenas pouco mais de}

\QtdParticipantes{2017.1}{x}{0}{1}
\perfilParticipanteD{masculino}{23}{primeiro}{}{}
\figuraPercepcaoParticipanteDisciplina{desistentes-2017}{desistentes}{2017.1}
O participante apresentou concordância apenas para a desistência ter sido motivada
por não conseguir conciliar a disciplina com o trabalho profissional
dele (DG) e também por outras questões pessoais (DG1).
O participante considera que conseguiu entender a metodologia PBL (DH),
que tem interesse em cursar a disciplina em outra oportunidade com
a metodologia PBL (DO1), com mais carga horária de aula expostivas (DO) e
que gostaria de experimentar outras abordagens de ensino (DJ).
Não menciona que falar em público seja um problema (DI).

\SemFiguraPercepcaoParticipanteDisciplinaNotas{desistente}{2017.1}

O participante desistente atribuiu nota $9,00$ como avaliação da
disciplina.
A nota atribuída reforça que como os dados de percepção demonstra,
o participante desistiu da disciplina por motivos externos, no caso,
não conseguiu conciliar o trabalho profissional que exerce com
a disciplina (DG).
