\subsection{Percepção dos desistentes}
\QtdParticipantes{2016.1}{15}{0}{3}
\perfilParticipanteC{24,0}{23}{25}{66,7}{100,0}{até o sétimo semestre}{}
\figuraPercepcaoParticipanteDisciplina{desistentes-2016}{desistentes}{2016.1}
\AprovacaoDisciplina{0,0}{desistentes}{aprovação da metodologia para a}%
{a metodologia}{CA}
Nenhum dos participantes apresentou como motivo para a desistência a
oposição em ter a presença como critério de avaliação (DB) e
em ter várias avaliações (DD), mas existe oposição por $33,3\%$
em ter a participação nas aulas como critério de avaliação (DC) e em ter
avaliações em equipe (DE).
Os tutores não foram motivadores da desistência (DF) para nenhum
dos desistentes.
Dificuldade de conciliar a disciplina com
o trabalho profissional (DG) e questões pessoais (DG1)
foram os motivos mais apontados para desistência,
com $66,7\%$.
\AprovacaoDisciplinaA{66,7}{entenderam a metodologia PBL}{DH}
\AprovacaoDisciplinaB{66,7}{experimentar outras abordagens além das aulas tradicionais}{DJ}{}{0,0}
Apesar de apenas $33,3\%$ dos participantes ter respondido ter interesse
em curso a disciplina com metodologia PBL (DO1), não existiu
manisfestação contrária.

\figuraPercepcaoParticipanteDisciplinaNotasA{2016-desistentes}{desistentes}{2016.1}{30}
{5,00}{5,67}{apenas pouco mais de}

\QtdParticipantes{2017.1}{x}{0}{1}
\perfilParticipanteD{masculino}{23}{primeiro}{}{}
\figuraPercepcaoParticipanteDisciplina{desistentes-2017}{desistentes}{2017.1}
\AprovacaoDisciplina{0,0}{desistentes}{aprovação da metodologia para a}%
{a metodologia}{CA}


\SemFiguraPercepcaoParticipanteDisciplinaNotas{desistente}{2017.1}

O participante desistente atribuiu nota $9,00$ como avaliação da
disciplina.
A nota atribuída reforça que como os dados de percepção demonstra,
o participante desistiu da disciplina por motivos externos, no caso,
não conseguiu conciliar o trabalho profissional que exerce com
a disciplina (DG).
