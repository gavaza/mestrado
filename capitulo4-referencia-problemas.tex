\begin{table}[h]
\caption{Referências para os gráficos}
\label{tabela-ref-graficos}
\begin{tabular}{c|p{14.6cm}}
Legenda & Pergunta respondida pelo participante \\
\hline
A & A construção da solução envolveu a elaboração, análise e exposição clara e racional de argumentos.\\
\hline
B & Foram identificadas ideias alternativas para uma mesma situação (ou seja, partes ou etapas individuais do problema).\\
\hline
C & Foram utilizadas, ao menos, uma das referências bibliográficas indicadas no texto do problema para estudo extraclasse.\\
\hline
D & Foi necessário recorrer a livros, vídeos, apostilas ou outros recursos não indicados no problema para chegar à solução.\\
\hline
E & O problema lhe deixou motivado para descobrir uma possível solução.\\
\hline
F & Você acredita que cumpriu com os objetivos de aprendizagem do problema.\\
\hline
G & Você teve que aprender novos conhecimentos (conceitos, habilidades ou atitudes) para chegar à solução do problema.\\
\hline
H & As situações abordadas pelo problema se aproximam de um cenário real e atual.\\
\hline
I & O conhecimento aprendido é útil para um profissional da área de Computação.\\
\hline
J & O texto do problema estava claro e bem escrito.\\
\hline
K & Havia no texto informações suficientes para direcionar a investigação.\\
\hline
L & O tempo disponibilizado para o desenvolvimento da solução foi adequado.\\
\hline
M & O problema estimulou o trabalho em grupo.\\
\hline
N & Você utilizou conhecimentos prévios (de um problema anterior ou mesmo de outro componente curricular) para chegar à solução do problema.\\
\hline
O & O problema exigiu o estudo individual de seus conteúdos fora das sessões tutoriais.\\
\hline
P & O método PBL me motiva para ir em busca do meu próprio conhecimento.\\
\hline
Q & As sessões tutoriais contribuem para o processo de resolução do problema.\\
\hline
R & Os participantes geralmente apresentam uma relação interpessoal boa e produtiva.\\
\hline
S & A quantidade de pessoas em cada Grupo Tutorial é apropriada.\\
\hline
T & Os problemas são úteis no processo de ensino-aprendizagem.\\
\hline
U & Os tutores contribuem, quando necessário, para a evolução das sessões tutoriais.\\
\hline
V & Os tutores deixam claros os critérios de avaliação do produto.\\
\hline
W & Os tutores dão feedbacks sobre o desempenho do grupo tutorial a cada sessão.\\
\hline
X & Eu gosto do método PBL.\\
\end{tabular}
\end{table}
