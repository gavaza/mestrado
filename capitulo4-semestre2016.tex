\section{Semestre 2016.1}
\label{sec-sem-2016}
\resultadoTurma{2016.1}{26}{9}{6}{57,7}{11}{8}{3}

\subsection{Problema 1 -- \ProblemaA}
\perfilProblema{\ProblemaA}{1}{2016.1}{14}
\perfilParticipante{27,9}{20}{51}{35,7}{71,4}{entre o terceiro e o quinto semestre}{}
\figuraPercepcaoParticipante{s1p1}{1}{2016.1}

É possível destacar que para todas as questões a favorabilidade
foi superior aos $70\%$. A favorabilidade foi superior aos $90\%$
para 12 afirmações apresentadas.
Apenas 6 afirmações receberam ao menos uma discordância integral.

A afirmação sobre ideias alternativas no caminho do problema (B)
foi a que teve a maior percepção de concordância por partes dos
participantes.
Entendemos que este resultado é explicado pelo problema
ter apresentado os conceitos de músicas, de forma que
os participantes precisaram realizar relacionamentos entre
os conceitos deste tema com os conceitos de linguagens
formais, e que os participantes perceberam que a depender do
nível de abstração poderá existir um caminho diferente.


A necessidade de recorrer aos materiais de apoio (D),
existência de informações suficientes no texto (K) e
sobre o participante gostar da metodologia baseada
em problemas (X) foram as afirmações com
as menores favorabilidades.
Embora possa parecer existir alguma contradição
sobre os resultados das duas primeiras, vale
ressaltar que os dados referente
a contribuição da sessão tutorial (Q) obtiveram
alta favorabilidade.

\figuraPercepcaoParticipanteNotas{s1p1}{1}{2016.1}{80}{5,00}{7,29}{quase}

\subsection{Problema 2 -- \ProblemaB}
\perfilProblema{\ProblemaB}{2}{2016.1}{12}
\perfilParticipante{28,5}{20}{51}{25,0}{75,0}{entre o terceiro e o quinto semestre}{}
\figuraPercepcaoParticipante{s1p2}{2}{2016.1}

Para 5 afirmações, na percepção dos estudantes, a favorabilidade foi integral.
Em 15 afirmações a favorabilidade superou os $90\%$.
Em apenas uma das afirmações a favorabilidade ficou pouco abaixo dos $70\%$, onde
$66,7\%$ dos participante acreditam que o problema estimulou
o trabalho em grupo (M).
Apenas 5 afirmações receberam ao menos uma discordância integral.

A contribuição dos tutores para a evolução do problema (U) foi a
afirmação melhor avaliada pelos participantes, assim como as demais
afirmações referentes aos tutores (W) e (V) também foram bem avaliadas,
indicando que os participantes aprovaram a participação
dos tutores na condução deste problema.

Para a construção do produto deste problema, os participantes
precisaram se reunir em equipe além das sessões tutoriais, desta forma,
acreditamos que as dificuldades referentes a se reunir fora das sessões tutorias
para trabalhar em grupo (M) estão também representados neste resultado.
Também observamos uma justificativa para este resultado ao analisar a favorabilidade para
a afirmação referente a percepção dos participantes com relação a quantidade
apropriada de estudantes em cada grupo tutorial (S).

\figuraPercepcaoParticipanteNotas{s1p2}{2}{2016.1}{70}{6,00}{7,17}{mais de}

\subsection{Problema 3 -- \ProblemaC}
\perfilProblema{\ProblemaC}{3}{2016.1}{8}
\perfilParticipante{31,4}{20}{51}{28,6}{85,7}{até o quinto semestre}{ para cada três participantes}
\figuraPercepcaoParticipante{s1p3}{3}{2016.1}

Para 7 afirmações, na percepção dos estudantes, a favorabilidade foi integral.
Em três afirmações a favorabilidade ficou abaixo dos $70\%$, mas apenas 2
afirmações receberam ao menos uma discordância integral.

A contribuição das sessões tutorias para o processo
de resolução do problema (Q) obteve a melhor avaliação pelos
participantes.
Entendemos que essa afirmação é bem avaliada sempre que
os participantes conseguem perceber uma informação de
grande relevância para a solução durante a sessão
tutorial, neste caso, a ideia de como manipular a pilha
de forma a manter a proporção exigida pelo problema
surgiu durante uma sessão tutorial.

No que diz respeito a quantidade de pessoas em
cada grupo tutorial (S) ser a afirmação a obter a pior
avaliação a justificativa pode ser obtida quando
é observada em conjunto com a afirmação referente ao
trabalho em equipe (M) que também esteve entre
as piores avaliações para este problema, neste caso,
ambos resultados são explicados pela quantidade de
participantes nas sessões tutoriais.

\figuraPercepcaoParticipanteNotas{s1p3}{3}{2016.1}{90}{6,00}{7,50}{quase}

\subsection{Problema 4 -- \ProblemaD}
\perfilProblema{\ProblemaD}{4}{2016.1}{8}
\perfilParticipante{31,5}{20}{51}{25,0}{87,5}{até o quinto semestre}{}
\figuraPercepcaoParticipante{s1p4}{4}{2016.1}

Para 12 afirmações, na percepção dos estudantes, a favorabilidade foi integral.
Em todas as afirmações a favorabilidade ficou acima dos $70\%$.
Em 4 afirmações foi recebida ao menos uma discordância integral.

A máquina de Turing foi o principal objetivo de aprendizagem para o Problema 4
aplicado no semestre 2016.1, sendo que para os participantes a afirmação
melhor avaliada diz respeito a utilidade dos conhecimentos aprendidos para 
o profissional da área de Computação (I).
Este problema foi muito bem avaliado pelos participantes para todas
as afirmações, mas cabe destacar os resultados para as afirmações que
dizem respeito a necessidade de aprender novos conhecimentos (G) e
motivação dos participantes para resolver o problema (E).

\figuraPercepcaoParticipanteNotas{s1p4}{4}{2016.1}{90}{5,00}{7,50}{quase}

\subsection{Problema 5 -- \ProblemaE}
\perfilProblema{\ProblemaE}{5}{2016.1}{7}
\perfilParticipante{32,3}{20}{51}{28,6}{71,4}{até o quinto semestre}{}
\figuraPercepcaoParticipante{s1p5}{5}{2016.1}

Para 6 afirmações, na percepção dos estudantes, a favorabilidade foi integral.
Em apenas duas afirmações a favorabilidade ficou abaixo dos $70\%$.
Em 9 afirmações foi recebida ao menos uma discordância integral, desta forma,
indicando vários pontos de atenção para a abordagem deste problema
em outras situações.

Este também foi um problema em que os participantes se reuniram em grupo para
construir uma solução, assim, como no Problema 2 no semestre 2016.1, a
afirmação pior avaliada também foi referente ao trabalho em equipe (M) que
indicamos que o resultado também inclui as dificuldades em se reunir além
das sessões tutoriais.

Ideias alternativas (B), utilização de referências bibliográficas (C),
recorrer a materiais não indicados nas referências (D), motivação
para resolver o problema (E), o tempo para resolução (L), a quantidade de
pessoas no grupo da sessão tutorial (S), utilidade do problema para o
processo de ensino e aprendizagem (T), \textit{feedback} dos
tutores (W), além da afirmação sobre o trabalho em equipe (M)
mencionado no parágrafo anterior, foram as afirmações com ao menos
uma discondância integral.
O mais possível é que este resultados sejam justificados pela sobrecarga
adicional que os estudantes possuem no fim do semestre, momento em que
este problema foi aplicado.
Ainda assim, não se deve desconsiderar que estes são pontos
de atenção claros, mas que mesmo em outras metodologias, com pesquisa
semelhante, os resultados da sobrecarga estariam também explicitados.

\figuraPercepcaoParticipanteNotasA{s1p5}{5}{2016.1}{70}{}{7,00}{pouco mais de}

\subsection{Percepção dos concluintes}
\figuraPercepcaoParticipanteDisciplina{concluintes-2016}{concluintes}{2016.1}

\subsection{Percepção dos desistentes}
\figuraPercepcaoParticipanteDisciplina{desistentes-2016}{desistentes}{2016.1}
