\section{Semestre 2016.1}
\label{sec-sem-2016}
\subsection{Problema 1}
Na aplicação do Problema 1 no semestre 2016.1 foram 14 participantes.

A Figura~\ref{percep-s1p1} apresenta gráfico com os resultados referente
as percepções dos participantes na aplicação do
Problema 1 no semestre 2016.1.
\begin{figure}[!htb]
\centering
\includegraphics[scale=0.22]{question-s1p1.eps}
\caption{Percepções dos participantes do semestre 2016.1 sobre o Problema 1}
\label{percep-s1p1}
\end{figure}

É possível destacar que para todas as questões a favorabilidade
foi superior aos $70\%$. A favorabilidade foi superior aos $90\%$
para 12 afirmações apresentadas.
Apenas 6 afirmações receberam ao menos uma discordância integral.

A afirmação sobre ideias alternativas no caminho do problema (B)
foi a que teve a maior percepção de concordância por partes dos
participantes.
Entendemos que este resultado é explicado pelo problema
ter apresentado os conceitos de músicas, de forma que
os participantes precisaram realizar relacionamentos entre
os conceitos deste tema com os conceitos de linguagens
formais, e que os participantes perceberam que a depender do
nível de abstração poderá existir um caminho diferente.

A necessidade de recorrer aos materiais de apoio (D),
existência de informações suficientes no texto (K) e
sobre o participante gostar da metodologia baseada
em problemas (X) foram as afirmações com
as menores favorabilidades.
Embora possa parecer existir alguma contradição
sobre os resultados das duas primeiras, vale
ressaltar que os dados referente
a contribuição da sessão tutorial (Q) obtiveram
alta favorabilidade.

A Figura~\ref{aval-s1p1} apresenta o gráfico da
avaliação do participantes para o Problema 1 aplicado no semestre 2016.1.

\begin{figure}[!htb]
\centering
\includegraphics[scale=0.18]{notas-s1p1.eps}
\caption{Avaliação dos participantes do semestre 2016.1 para o Problema 1}
\label{aval-s1p1}
\end{figure}

Como é possível observar no gráfico, a maioria dos participantes atribuíram
notas altas para o Problema 1 no semestre 2016.1, assim, quase $80\%$ das notas
foram maiores ou iguais a 7 e nenhuma nota foi menor que 5, com uma média
de $7,29$.
  
\subsection{Problema 2}
Na aplicação do Problema 2 no semestre 2016.1 foram 12 participantes.

A Figura~\ref{percep-s1p2} apresenta gráfico com os resultados referente
as percepções dos participantes na aplicação do
Problema 2 no semestre 2016.1.

\begin{figure}[!htb]
\centering
\includegraphics[scale=0.22]{question-s1p2.eps}
\caption{Percepções dos participantes do semestre 2016.1 sobre o Problema 2}
\label{percep-s1p2}
\end{figure}

A Figura~\ref{aval-s1p2} apresenta o gráfico da
avaliação do participantes para o Problema 2 aplicado no semestre 2016.1.

\begin{figure}[!htb]
\centering
\includegraphics[scale=0.18]{notas-s1p2.eps}
\caption{Avaliação dos participantes do semestre 2016.1 para o Problema 2}
\label{aval-s1p2}
\end{figure}

\subsection{Problema 3}
Na aplicação do Problema 3 no semestre 2016.1 foram 8 participantes.

A Figura~\ref{percep-s1p3} apresenta gráfico com os resultados referente
as percepções dos participantes na aplicação do
Problema 3 no semestre 2016.1.

\begin{figure}[!htb]
\centering
\includegraphics[scale=0.22]{question-s1p3.eps}
\caption{Percepções dos participantes do semestre 2016.1 sobre o Problema 3}
\label{percep-s1p3}
\end{figure}

A Figura~\ref{aval-s1p3} apresenta o gráfico da
avaliação do participantes para o Problema 3 aplicado no semestre 2016.1.

\begin{figure}[!htb]
\centering
\includegraphics[scale=0.18]{notas-s1p3.eps}
\caption{Avaliação dos participantes do semestre 2016.1 para o Problema 3}
\label{aval-s1p3}
\end{figure}

\subsection{Problema 4}
Na aplicação do Problema 4 no semestre 2016.1 foram 8 participantes.

A Figura~\ref{percep-s1p4} apresenta gráfico com os resultados referente
as percepções dos participantes na aplicação do
Problema 4 no semestre 2016.1.

\begin{figure}[!htb]
\centering
\includegraphics[scale=0.22]{question-s1p4.eps}
\caption{Percepções dos participantes do semestre 2016.1 sobre o Problema 4}
\label{percep-s1p4}
\end{figure}

A Figura~\ref{aval-s1p4} apresenta o gráfico da
avaliação do participantes para o Problema 4 aplicado no semestre 2016.1.

\begin{figure}[!htb]
\centering
\includegraphics[scale=0.18]{notas-s1p4.eps}
\caption{Avaliação dos participantes do semestre 2016.1 para o Problema 4}
\label{aval-s1p4}
\end{figure}

\subsection{Problema 5}
Na aplicação do Problema 5 no semestre 2016.1 foram 7 participantes.

A Figura~\ref{percep-s1p5} apresenta gráfico com os resultados referente
as percepções dos participantes na aplicação do
Problema 5 no semestre 2016.1.
\begin{figure}[!htb]
\centering
\includegraphics[scale=0.22]{question-s1p5.eps}
\caption{Percepções dos participantes do semestre 2016.1 sobre o Problema 5}
\label{percep-s1p5}
\end{figure}

A Figura~\ref{aval-s1p5} apresenta o gráfico da
avaliação do participantes para o Problema 5 aplicado no semestre 2016.1.

\begin{figure}[!htb]
\centering
\includegraphics[scale=0.18]{notas-s1p5.eps}
\caption{Avaliação dos participantes do semestre 2016.1 para o Problema 5}
\label{aval-s1p5}
\end{figure}

\subsection{Problema 6}
Na aplicação do Problema 6 no semestre 2016.1 foram 7 participantes.

A Figura~\ref{percep-s1p6} apresenta gráfico com os resultados referente
as percepções dos participantes na aplicação do
Problema 6 no semestre 2016.1.

\begin{figure}[!htb]
\centering
\includegraphics[scale=0.22]{question-s1p6.eps}
\caption{Percepções dos participantes do semestre 2016.1 sobre o Problema 6}
\label{percep-s1p6}
\end{figure}

A Figura~\ref{aval-s1p6} apresenta o gráfico da
avaliação do participantes para o Problema 6 aplicado no semestre 2016.1.

\begin{figure}[!htb]
\centering
\includegraphics[scale=0.18]{notas-s1p6.eps}
\caption{Avaliação dos participantes do semestre 2016.1 para o Problema 6}
\label{aval-s1p6}
\end{figure}
