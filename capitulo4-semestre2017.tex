\section{Semestre 2017.1}
\label{sec-sem-2017}

\subsection{Problema 1}
\label{sec-2017-p1}
Na aplicação do Problema 1 no semestre 2017.1 foram 12 participantes.

\begin{figure}[!htb]
\centering
\includegraphics[scale=0.22]{question-s2p1.eps}
\caption{Percepções dos participantes do semestre 2017.1 sobre o Problema 1}
\label{percep-s2p1}
\end{figure}

A Figura~\ref{percep-s2p1} apresenta gráfico com os resultados referente
as percepções dos participantes na aplicação do
Problema 1 no semestre 2017.1.


Para 7 afirmações apresentadas a favorabilidade ficou um pouco abaixo dos $70\%$, mas
foi superior aos $90\%$ para 10 afirmações apresentadas.
Em 11 afirmações foi recebida ao menos uma discordância integral.

Neste problema com o objetivo de aprendizagem principal sendo
as linguagens formais, foi consenso de percepção dos participantes que
este problema apresenta um conhecimento útil para um profissional
da área de Computação (I).

É possível destacar o excelente resultado para a afirmação que diz respeito a
utilidade do problema no processo de ensino e aprendizagem (T).
A explicação para este resultado está no fato de que os estudantes
conseguiram realizar bem as correlações entre os conceitos
do código Morse, apresentado no problema, com os conceitos de
linguagens formais, que são os objetivos de aprendizagem para
este problema.

Em uma análise mais detalhada dos dados foi possível
detectar que a discordância integral está concentrada
em 3 participantes, onde apenas um destes apresentou
5 discordâncias integrais.
Estes números além de indicar pontos possíveis de
priorização, também podem indicar a necessidade
de um nivelamento maior dos estudantes para a aplicação
da metodologia.

\begin{figure}[!htb]
\centering
\includegraphics[scale=0.18]{notas-s2p1.eps}
\caption{Avaliação dos participantes do semestre 2017.1 para o Problema 1}
\label{aval-s2p1}
\end{figure}

A Figura~\ref{aval-s2p1} apresenta o gráfico da
avaliação do participantes para o Problema 1 aplicado no semestre 2017.1.

Podemos observar que a maioria dos participantes atribuíram
notas altas para o Problema 1 no semestre 2017.1, assim, quase $90\%$ das notas
foram maiores ou iguais a $7,00$ e nenhuma nota foi menor que $5,00$, com uma média
de $8,25$.

\subsection{Problema 2}
Na aplicação do Problema 2 no semestre 2017.1 foram 13 participantes.

\begin{figure}[!htb]
\centering
\includegraphics[scale=0.22]{question-s2p2.eps}
\caption{Percepções dos participantes do semestre 2017.1 sobre o Problema 2}
\label{percep-s2p2}
\end{figure}

A Figura~\ref{percep-s2p2} apresenta gráfico com os resultados referente
as percepções dos participantes na aplicação do
Problema 2 no semestre 2017.1.


A favorabilidade ficou um pouco abaixo dos $70\%$ em 6 afirmações apresentadas,
sendo superior aos $80\%$ para 12 afirmações destas afirmações.
Em 14 afirmações foi recebida ao menos uma discordância integral.

Neste problema com o objetivo de aprendizagem principal sendo
os autômatos finitos, foi consenso de percepção dos participantes que
este problema apresenta um conhecimento útil para um profissional
da área de Computação (I) e que é necessário aprender novos
conhecimentos (G).

Apenas um dos participantes respondeu com discondância integral para
8 afirmações, assim, também se faz necessário considerar a
necessidade de nivelamento do grupo, como mencionado
na Seção~\ref{sec-2017-p1}.

Apesar de obter favorabilidade muito próximo aos $70\%$, a afirmação
sobre o participante gostar da metodologia PBL (X) teve como
destaque negativo receber a maior discondância integral nesse
teste.

\begin{figure}[!htb]
\centering
\includegraphics[scale=0.18]{notas-s2p2.eps}
\caption{Avaliação dos participantes do semestre 2017.1 para o Problema 2}
\label{aval-s2p2}
\end{figure}

A Figura~\ref{aval-s2p2} apresenta o gráfico da
avaliação do participantes para o Problema 2 aplicado no semestre 2017.1.

Para pouco mais de $60\%$ dos participantes os problemas foram
avaliados com notas iguais ou superiores a $7,00$ e nenhuma nota
foi inferior a $5,00$, com uma média de $7,31$.

\subsection{Problema 3}
Na aplicação do Problema 3 no semestre 2017.1 foram 11 participantes.

\begin{figure}[!htb]
\centering
\includegraphics[scale=0.22]{question-s2p3.eps}
\caption{Percepções dos participantes do semestre 2017.1 sobre o Problema 3}
\label{percep-s2p3}
\end{figure}

A Figura~\ref{percep-s2p3} apresenta gráfico com os resultados referente
as percepções dos participantes na aplicação do
Problema 3 no semestre 2017.1.

A favorabilidade ficou um pouco abaixo dos $70\%$ em apenas uma afirmação,
sendo superior aos $80\%$ para 21 destas afirmações.
Em apenas 2 afirmações foi recebida ao menos uma discordância integral.

Para este caso a percepção dos participantes
sobre a relevância das sessões tutorias no processo
de resolução do problema (Q), assim como no semestre
anterior, obteve favorabilidade integral, com o destaque de
que para este semestre todas as respostas recebidas foram de
total concordância.

A afirmativa pior avaliada diz respeito a percepção dos participantes
sobre o \textit{feedback} dos tutores a cada sessão tutorial (W).
Para este caso, apesar do resultado ainda ser alto, ficou como
um ponto de atenção aos tutores.

\begin{figure}[!htb]
\centering
\includegraphics[scale=0.18]{notas-s2p3.eps}
\caption{Avaliação dos participantes do semestre 2017.1 para o Problema 3}
\label{aval-s2p3}
\end{figure}

A Figura~\ref{aval-s2p3} apresenta o gráfico da
avaliação do participantes para o Problema 3 aplicado no semestre 2017.1.


Assim como no semestre anterior, este problema foi muito bem avaliado
pelos participantes, assim, todas as notas atribuídas foram maiores
ou iguais $7,00$ e uma média de $8,18$.

Entre os problemas que foram replicados em ambos os semestres, este
notadamente é o que apresentou as melhores avaliações pelos participantes
em todos os critérios, evidenciando o potencial da metodologia em uma
disciplina teórica com a utilização de um problema do ``mundo real''.

\subsection{Problema 4}
Na aplicação do Problema 4 no semestre 2017.1 foram 10 participantes.

\begin{figure}[!htb]
\centering
\includegraphics[scale=0.22]{question-s2p4.eps}
\caption{Percepções dos participantes do semestre 2017.1 sobre o Problema 4}
\label{percep-s2p4}
\end{figure}

A Figura~\ref{percep-s2p4} apresenta gráfico com os resultados referente
as percepções dos participantes na aplicação do
Problema 4 no semestre 2017.1.

Para todas as afirmativas a percepeção dos participantes obteve favorabilidade
superior aos $70\%$. Em 5 afirmações foi recebida ao menos uma discordância
integral.

O destaque positivo para este problema foi que a favorabilidade em relação
a necessidade de recorrer a materiais fora da bibliográfia básica (D).
Este é uma característica de estudo muito relevante no desenvolvimento
do estudante que passa a ver a necessidade de explorar fontes diversas
de informação para construir o seu conhecimento.

Como ponto de atenção, apesar de a favorabilidade ser igual aos $70\%$ para
a percepção do participante sobre a adequação do tempo disponível para desenvolver
a solução (L), uma quantidade relevante de discordância integral para esta afirmativa.

\begin{figure}[!htb]
\centering
\includegraphics[scale=0.18]{notas-s2p4.eps}
\caption{Avaliação dos participantes do semestre 2017.1 para o Problema 4}
\label{aval-s2p4}
\end{figure}

A Figura~\ref{aval-s2p4} apresenta o gráfico da
avaliação do participantes para o Problema 4 aplicado no semestre 2017.1.

Este foi outro problema que foi bem avaliado em todos os critérios em ambos
os semestres.
Quase $80\%$ dos particintes avaliaram o Problema 4 no semestre 2017.1 com nota igual
ou superior a $7,00$ e nenhuma nota foi inferior a $6,00$.
Os participantes atribuíram em média $7,50$.


\subsection{Problema 5}
Na aplicação do Problema 5 no semestre 2017.1 foram 7 participantes.

\begin{figure}[!htb]
\centering
\includegraphics[scale=0.22]{question-s2p5.eps}
\caption{Percepções dos participantes do semestre 2017.1 sobre o Problema 5}
\label{percep-s2p5}
\end{figure}

A Figura~\ref{percep-s2p5} apresenta gráfico com os resultados referente
as percepções dos participantes na aplicação do
Problema 5 no semestre 2017.1.

Em 3 afirmações a favorabilidade ficou abaixo dos $70\%$ e 5 afirmações
receberam ao menos uma discordância integral.

Em contraste com o resultado que o mesmo problema teve com
os participantes do semestre 2016.1, onde foi a afirmação com a
pior favorabilidade, para os participantes neste semestre de
2017.1, a percepção de que o problema estimula o
trabalho em grupo (M) foi a afirmação melhor avaliada
para o Problema 5.
A diferença de resultado é justificada ao observar
a percepção dos participantes com relação a quantidade
apropriada de estudantes em cada grupo tutorial (S).

A afirmação com a menor favorabilidade diz respeito
percepção de cumprimento dos objetivos de aprendizagem
pelo participante (F).
Neste ponto cabe destacar que ao apresentar um resultado
negativo para esta afirmação, uma vez que com outros
problemas dentro do mesmo grupo tiveram favorabilidade
melhores, se pode questionar o problema e não
a metodologia.
Também não se pode deixar de considerar o nível de abstração
mais elevado para o entendimento dos conceitos deste
problema em relação aos demais.


\begin{figure}[!htb]
\centering
\includegraphics[scale=0.18]{notas-s2p5.eps}
\caption{Avaliação dos participantes do semestre 2017.1 para o Problema 5}
\label{aval-s2p5}
\end{figure}

A Figura~\ref{aval-s2p5} apresenta o gráfico da
avaliação do participantes para o Problema 5 aplicado no semestre 2017.1.

Para pouco mais de $70\%$ dos participantes este problema foi avaliado
com uma nota igual ou superior a $7,00$ e nenhuma nota
foi inferior a $5,00$. A média para as avaliações foi de $7,43$.


\subsection{Problema 6}
Na aplicação do Problema 6 no semestre 2017.1 foram 4 participantes.

Como neste caso foi baixa a adesão, além disto a metodologia
não foi integralmente aplicada, iremos apresentar apenas
os resultados sem realizar detalhamentos.

\begin{figure}[!htb]
\centering
\includegraphics[scale=0.22]{question-s2p6.eps}
\caption{Percepções dos participantes do semestre 2017.1 sobre o Problema 6}
\label{percep-s2p6}
\end{figure}

A Figura~\ref{percep-s2p6} apresenta gráfico com os resultados referente
as percepções dos participantes na aplicação do
Problema 6 no semestre 2017.1.

\begin{figure}[!htb]
\centering
\includegraphics[scale=0.18]{notas-s2p6.eps}
\caption{Avaliação dos participantes do semestre 2017.1 para o Problema 6}
\label{aval-s2p6}
\end{figure}

A Figura~\ref{aval-s2p6} apresenta o gráfico da
avaliação do participantes para o Problema 6 aplicado no semestre 2017.1.

