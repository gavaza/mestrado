\section{Semestre 2017.1}
\label{sec-sem-2017}

\subsection{Problema 1}
Na aplicação do Problema 1 no semestre 2017.1 foram 11 participantes.

A Figura~\ref{percep-s2p1} apresenta gráfico com os resultados referente
as percepções dos participantes na aplicação do
Problema 1 no semestre 2017.1.

\begin{figure}[!htb]
\centering
\includegraphics[scale=0.22]{question-s2p1.eps}
\caption{Percepções dos participantes do semestre 2017.1 sobre o Problema 1}
\label{percep-s2p1}
\end{figure}

A Figura~\ref{aval-s2p1} apresenta o gráfico da
avaliação do participantes para o Problema 1 aplicado no semestre 2017.1.

\begin{figure}[!htb]
\centering
\includegraphics[scale=0.18]{notas-s2p1.eps}
\caption{Avaliação dos participantes do semestre 2017.1 para o Problema 1}
\label{aval-s2p1}
\end{figure}

\subsection{Problema 2}
Na aplicação do Problema 2 no semestre 2017.1 foram 12 participantes.

A Figura~\ref{percep-s2p2} apresenta gráfico com os resultados referente
as percepções dos participantes na aplicação do
Problema 2 no semestre 2017.1.

\begin{figure}[!htb]
\centering
\includegraphics[scale=0.22]{question-s2p2.eps}
\caption{Percepções dos participantes do semestre 2017.1 sobre o Problema 2}
\label{percep-s2p2}
\end{figure}

A Figura~\ref{aval-s2p2} apresenta o gráfico da
avaliação do participantes para o Problema 2 aplicado no semestre 2017.1.

\begin{figure}[!htb]
\centering
\includegraphics[scale=0.18]{notas-s2p2.eps}
\caption{Avaliação dos participantes do semestre 2017.1 para o Problema 2}
\label{aval-s2p2}
\end{figure}

\subsection{Problema 3}
Na aplicação do Problema 3 no semestre 2017.1 foram 10 participantes.

A Figura~\ref{percep-s2p3} apresenta gráfico com os resultados referente
as percepções dos participantes na aplicação do
Problema 3 no semestre 2017.1.

\begin{figure}[!htb]
\centering
\includegraphics[scale=0.22]{question-s2p3.eps}
\caption{Percepções dos participantes do semestre 2017.1 sobre o Problema 3}
\label{percep-s2p3}
\end{figure}

A Figura~\ref{aval-s2p3} apresenta o gráfico da
avaliação do participantes para o Problema 3 aplicado no semestre 2017.1.

\begin{figure}[!htb]
\centering
\includegraphics[scale=0.18]{notas-s2p3.eps}
\caption{Avaliação dos participantes do semestre 2017.1 para o Problema 3}
\label{aval-s2p3}
\end{figure}

\subsection{Problema 4}
Na aplicação do Problema 4 no semestre 2017.1 foram 9 participantes.

A Figura~\ref{percep-s2p4} apresenta gráfico com os resultados referente
as percepções dos participantes na aplicação do
Problema 4 no semestre 2017.1.

\begin{figure}[!htb]
\centering
\includegraphics[scale=0.22]{question-s2p4.eps}
\caption{Percepções dos participantes do semestre 2017.1 sobre o Problema 4}
\label{percep-s2p4}
\end{figure}

A Figura~\ref{aval-s2p4} apresenta o gráfico da
avaliação do participantes para o Problema 4 aplicado no semestre 2017.1.

\begin{figure}[!htb]
\centering
\includegraphics[scale=0.18]{notas-s2p4.eps}
\caption{Avaliação dos participantes do semestre 2017.1 para o Problema 4}
\label{aval-s2p4}
\end{figure}

\subsection{Problema 5}
Na aplicação do Problema 5 no semestre 2017.1 foram 6 participantes.

A Figura~\ref{percep-s2p5} apresenta gráfico com os resultados referente
as percepções dos participantes na aplicação do
Problema 5 no semestre 2017.1.

\begin{figure}[!htb]
\centering
\includegraphics[scale=0.22]{question-s2p5.eps}
\caption{Percepções dos participantes do semestre 2017.1 sobre o Problema 5}
\label{percep-s2p5}
\end{figure}

A Figura~\ref{aval-s2p5} apresenta o gráfico da
avaliação do participantes para o Problema 5 aplicado no semestre 2017.1.

\begin{figure}[!htb]
\centering
\includegraphics[scale=0.18]{notas-s2p5.eps}
\caption{Avaliação dos participantes do semestre 2017.1 para o Problema 5}
\label{aval-s2p5}
\end{figure}

\subsection{Problema 6}
Na aplicação do Problema 6 no semestre 2017.1 foram 3 participantes.

A Figura~\ref{percep-s2p6} apresenta gráfico com os resultados referente
as percepções dos participantes na aplicação do
Problema 6 no semestre 2017.1.

\begin{figure}[!htb]
\centering
\includegraphics[scale=0.22]{question-s2p6.eps}
\caption{Percepções dos participantes do semestre 2017.1 sobre o Problema 6}
\label{percep-s2p6}
\end{figure}

A Figura~\ref{aval-s2p6} apresenta o gráfico da
avaliação do participantes para o Problema 6 aplicado no semestre 2017.1.

\begin{figure}[!htb]
\centering
\includegraphics[scale=0.18]{notas-s2p6.eps}
\caption{Avaliação dos participantes do semestre 2017.1 para o Problema 6}
\label{aval-s2p6}
\end{figure}
