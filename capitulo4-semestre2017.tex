\section{Semestre 2017.1}
\label{sec-sem-2017}

\subsection{Problema 1}
\label{sec-2017-p1}
Na aplicação do Problema 1 no semestre 2017.1 foram 11 participantes.

A Figura~\ref{percep-s2p1} apresenta gráfico com os resultados referente
as percepções dos participantes na aplicação do
Problema 1 no semestre 2017.1.

\begin{figure}[!htb]
\centering
\includegraphics[scale=0.22]{question-s2p1.eps}
\caption{Percepções dos participantes do semestre 2017.1 sobre o Problema 1}
\label{percep-s2p1}
\end{figure}

A favorabilidade foi superior aos $90\%$ para 11 afirmações apresentadas.
Em 10 afirmações foi recebida ao menos uma discordância integral.

A afirmação melhor avaliada diz respeito a utilidade do problema no
processo de ensino e aprendizagem (T).
A explicação para este resultado está no fato de que os estudantes
conseguiram realizar bem as correlações entre os conceitos
do código Morse, apresentado no problema, com os conceitos de
linguagens formais, que são os objetivos de aprendizagem para
este problema.

Em uma análise mais detalhada dos dados foi possível
detectar que a discordância integral está concentrada
em 3 participantes, onde apenas um destes apresentou
5 discordâncias integrais.
Estes números além de indicar pontos possíveis de
priorização, também podem indicar a necessidade
de um nivelamento maior dos estudantes para a aplicação
da metodologia.

A Figura~\ref{aval-s2p1} apresenta o gráfico da
avaliação do participantes para o Problema 1 aplicado no semestre 2017.1.

\begin{figure}[!htb]
\centering
\includegraphics[scale=0.18]{notas-s2p1.eps}
\caption{Avaliação dos participantes do semestre 2017.1 para o Problema 1}
\label{aval-s2p1}
\end{figure}

Podemos observar que a maioria dos participantes atribuíram
notas altas para o Problema 1 no semestre 2017.1, assim, quase $90\%$ das notas
foram maiores ou iguais a $7$ e nenhuma nota foi menor que $5$, com uma média
de $8,25$.

\subsection{Problema 2}
Na aplicação do Problema 2 no semestre 2017.1 foram 12 participantes.

A Figura~\ref{percep-s2p2} apresenta gráfico com os resultados referente
as percepções dos participantes na aplicação do
Problema 2 no semestre 2017.1.

\begin{figure}[!htb]
\centering
\includegraphics[scale=0.22]{question-s2p2.eps}
\caption{Percepções dos participantes do semestre 2017.1 sobre o Problema 2}
\label{percep-s2p2}
\end{figure}

A favorabilidade foi superior aos $90\%$ para 8 afirmações apresentadas.
Em 13 afirmações foi recebida ao menos uma discordância integral.

Neste problema com o objetivo de aprendizagem principal sendo
os autômatos finitos, foi consenso de percepção dos participantes que
este problema apresenta um conhecimento útil para um profissional
da área de Computação (I) e que é necessário aprender novos
conhecimentos (G).

Apenas um dos participantes respondeu com discondância integral para
8 afirmações, assim, também se faz necessário considerar a
necessidade de nivelamento do grupo, como mencionado
na Seção~\ref{sec-2017-p1}.

A Figura~\ref{aval-s2p2} apresenta o gráfico da
avaliação do participantes para o Problema 2 aplicado no semestre 2017.1.

\begin{figure}[!htb]
\centering
\includegraphics[scale=0.18]{notas-s2p2.eps}
\caption{Avaliação dos participantes do semestre 2017.1 para o Problema 2}
\label{aval-s2p2}
\end{figure}

Para pouco mais de $60\%$ dos participantes os problemas foram
avaliados com notas iguais ou superiores a $7$ e nenhuma nota
foi inferior a $5$, com uma média de $7,31$.

\subsection{Problema 3}
Na aplicação do Problema 3 no semestre 2017.1 foram 10 participantes.

A Figura~\ref{percep-s2p3} apresenta gráfico com os resultados referente
as percepções dos participantes na aplicação do
Problema 3 no semestre 2017.1.

\begin{figure}[!htb]
\centering
\includegraphics[scale=0.22]{question-s2p3.eps}
\caption{Percepções dos participantes do semestre 2017.1 sobre o Problema 3}
\label{percep-s2p3}
\end{figure}

A Figura~\ref{aval-s2p3} apresenta o gráfico da
avaliação do participantes para o Problema 3 aplicado no semestre 2017.1.

\begin{figure}[!htb]
\centering
\includegraphics[scale=0.18]{notas-s2p3.eps}
\caption{Avaliação dos participantes do semestre 2017.1 para o Problema 3}
\label{aval-s2p3}
\end{figure}

\subsection{Problema 4}
Na aplicação do Problema 4 no semestre 2017.1 foram 9 participantes.

A Figura~\ref{percep-s2p4} apresenta gráfico com os resultados referente
as percepções dos participantes na aplicação do
Problema 4 no semestre 2017.1.

\begin{figure}[!htb]
\centering
\includegraphics[scale=0.22]{question-s2p4.eps}
\caption{Percepções dos participantes do semestre 2017.1 sobre o Problema 4}
\label{percep-s2p4}
\end{figure}

A Figura~\ref{aval-s2p4} apresenta o gráfico da
avaliação do participantes para o Problema 4 aplicado no semestre 2017.1.

\begin{figure}[!htb]
\centering
\includegraphics[scale=0.18]{notas-s2p4.eps}
\caption{Avaliação dos participantes do semestre 2017.1 para o Problema 4}
\label{aval-s2p4}
\end{figure}

\subsection{Problema 5}
Na aplicação do Problema 5 no semestre 2017.1 foram 6 participantes.

A Figura~\ref{percep-s2p5} apresenta gráfico com os resultados referente
as percepções dos participantes na aplicação do
Problema 5 no semestre 2017.1.

\begin{figure}[!htb]
\centering
\includegraphics[scale=0.22]{question-s2p5.eps}
\caption{Percepções dos participantes do semestre 2017.1 sobre o Problema 5}
\label{percep-s2p5}
\end{figure}

A Figura~\ref{aval-s2p5} apresenta o gráfico da
avaliação do participantes para o Problema 5 aplicado no semestre 2017.1.

\begin{figure}[!htb]
\centering
\includegraphics[scale=0.18]{notas-s2p5.eps}
\caption{Avaliação dos participantes do semestre 2017.1 para o Problema 5}
\label{aval-s2p5}
\end{figure}

\subsection{Problema 6}
Na aplicação do Problema 6 no semestre 2017.1 foram 3 participantes.

A Figura~\ref{percep-s2p6} apresenta gráfico com os resultados referente
as percepções dos participantes na aplicação do
Problema 6 no semestre 2017.1.

\begin{figure}[!htb]
\centering
\includegraphics[scale=0.22]{question-s2p6.eps}
\caption{Percepções dos participantes do semestre 2017.1 sobre o Problema 6}
\label{percep-s2p6}
\end{figure}

A Figura~\ref{aval-s2p6} apresenta o gráfico da
avaliação do participantes para o Problema 6 aplicado no semestre 2017.1.

\begin{figure}[!htb]
\centering
\includegraphics[scale=0.18]{notas-s2p6.eps}
\caption{Avaliação dos participantes do semestre 2017.1 para o Problema 6}
\label{aval-s2p6}
\end{figure}
