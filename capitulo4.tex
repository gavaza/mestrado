\xchapter{Resultados}{} %sem preambulo
% É recomendável utilizar `\acresetall' no início de cada capítulo para reiníciar o contator de referências às siglas.
\acresetall
\section{Formulário de percepções de problema}
Para cada problema os estudantes foram convidados a responder um formulário de percepções de problema.
Neste formulário foram apresentadas 32 questões aos participantes.
As cinco primeiras questões neste formulário foram de caracterização de perfil, como
idade e sexo.
As próximas vinte e cinco questões foram sobre as percepções dos participantes sobre a abordagem, o
problema, o tutor e auto avaliação.

Cada pergunta apresentou uma afirmação sobre o tema ao qual se desejava obter informações sobre as
percepções dos participantes.
A resposta do participante é obtida em uma escala Likert com cinco itens, onde para efeitos dos estudos
apresentados neste trabalho foram consideradas desfavoráveis
as respostas ``discordo plenamente'', ``discordo'' e ``indiferente'', enquanto as
respostas ``concordo'' e ``concordo plenamente'' foram consideradas favoráveis.

O participante também teve disponível um espaço aberto para incluir considereções adicionais sobre
as suas percepções sobre o problema especificamente e sobre a utilização da abordagem.

\section{Formulário de percepção da disciplina}
\section{Semestre 2015.1}
\section{Semestre 2016.1}
\section{Semestre 2017.1}
\section{Considerações finais}
