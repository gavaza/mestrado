\xchapter{Resultados}{} %sem preambulo
% É recomendável utilizar `\acresetall' no início de cada capítulo para reiníciar o contator de referências às siglas.
\acresetall

Este capítulo apresenta os resultados das experiências
mencionadas no Capítulo~\ref{cap-experiencia}.
Os resultados são apresentados em forma de gráficos com
as respectivas discussões.

As percepções dos participantes sobre os problemas e a metodologia
são recebidas em formulários no ambiente virtual.
No ambiente virtual foi escolhida a opção de não
obter registro do participante nos formulários, assim,
está garantido o anonimato das respostas.

A assinatura do termo de esclarecimento pelos participantes foi
digital, assinado no ambiente virtual.

As informações consolidadas no ambiente virtual foram
extraídas e processadas com scripts de filtragem e manipulação de dados,
desta forma, foram consolidadas em informações que estão apresentadas
e serão discutidas neste capítulo.

As Seções~\ref{form-percepcoes} e \ref{form-disciplinas} descrevem
o conteúdo dos formulários apresentados aos participantes para obter
informações sobre as percepções destes;
a Seção~\ref{sec-ref-graficos} apresenta considerações referente aos
gráficos apresentados nas seções que seguem deste capítulo;
as Seções de~\ref{sec-sem-2015} até~\ref{sec-sem-2017}
apresenta os resultados para cada semestre, um por seção;
e, por fim, a Seção~\ref{sec-consideracoes-resultados} apresenta
considerações referente aos resultados.

\section{Formulário de percepções de problema}
\label{form-percepcoes}
Para cada problema os estudantes foram convidados a responder um formulário de percepções de problema.
Neste formulário foram apresentadas 32 questões aos participantes.
As cinco primeiras questões neste formulário foram de caracterização de perfil, como
idade e sexo.
As próximas vinte e quatro questões foram sobre as percepções dos participantes sobre a abordagem, o
problema, o tutor e auto avaliação.

Cada pergunta apresentou uma afirmação sobre o tema ao qual se desejava obter informações sobre as
percepções dos participantes.
A resposta do participante é obtida em uma escala Likert com cinco itens, sendo as opções
``concordo'', ``concordo parcialmente'', ``indiferente'', ``discordo parcialmente'' e ``discordo''.

O participante também teve disponível um espaço aberto para incluir considareções adicionais sobre
as suas percepções sobre o problema especificamente e sobre a utilização da abordagem.

\section{Formulário de percepções da disciplina}
\label{form-disciplinas}
Para obter as percepções da disciplina, os participantes foram caracterizados em \textit{desistente}
se por algum motivo desistiu da disciplina antes da conclusão, sendo assim, foi reprovado por
falta ou realizou trancamento, e \textit{concluinte} que concluiu a disciplina independente
de aprovação ou reprovação por conceito.

Aos participantes foi apresentado um formulário de acordo com a sua condição de desistente
ou concluinte.
Em ambos os casos o formulário apresentou 24 questões aos participantes.
O foco do formulário apresentado ao participante desistente é identificar indicações
sobre o motivo da desistência, enquanto para o participante concluinte o foco do
formulário apresentado está em identificar benefícios da abordagem na percepção
do participante.
As cinco primeiras questões neste formulário foram de caracterização de perfil, como
idade e sexo.
As próximas dezessete questões foram sobre as percepções dos participantes sobre referente
ao foco do formulário respondido.

Cada pergunta apresentou uma afirmação sobre o tema ao qual se desejava obter informações sobre as
percepções dos participantes.
A resposta do participante é obtida em uma escala Likert com cinco itens, como as da Seção~\ref{form-percepcoes}.

O participante também teve disponível um espaço aberto para incluir considareções adicionais sobre
as suas percepções sobre o problema especificamente e sobre a utilização da abordagem.

\section{Considerações sobre os gráficos}
\label{sec-ref-graficos}
Para reduzir a quantidade de informações, por consequência melhor utilizar o
espaço, os gráficos utilizados nas seções que seguem neste capítulo
estão limpos de legendas.

A Tabela~\ref{tabela-ref-graficos} apresenta o significado das barras
e a Figura~\ref{figura-ref-graficos} apresenta a legenda (cor da barra)
nos gráficos referentes as percepções dos participantes para as
afirmações mencionadas na Seção~\ref{form-percepcoes} sobre
o problema.
Um exemplo de leitura do gráfico é a barra \textbf{H} que apresenta
os resultados para a pergunta ``As situações abordadas pelo problema
se aproximam de um cenário real e atual.'', que deseja verificar
a percepção do participante se o problema apresenta realidade
e atualidade.

A Tabela~\ref{tabela-ref-graficos2} apresenta o significado das barras
e a Figura~\ref{figura-ref-graficos} apresenta a legenda (cor da barra)
nos gráficos referentes percepções dos participantres para as
afirmações mencionadas na Seção~\ref{form-disciplinas} sobre
a aplicação da disciplina.

Para cada um dos problemas foram construídos dois gráficos de resultados.
No primeiro gráfico são exibidos resultados para as respostas
dos participantes para questões afirmativas e
no segundo gráfico são exibidos os resultados para as notas atribuídas
ao problema pelo participante em uma avaliação geral.

Nos gráficos que seguem sobre as percepeções dos participantes para os
problemas e também sobre nos sobre a disciplina, a disposição nas barras
foi construída para facilitar a leitura por nível de favorabilidade.
No caso de analisar os resultados em uma perscpectiva mais conservadora
para a favorabilidade, é possível considerar que apenas a concordância
integral, isto é, o participante respondeu ``concordo'' para
a afirmação como favorável.
A inclusão de concordância em parte, isto é, também é considerado favorável que
o participante respondeu ``concordo em partes'' para a afirmação,
apresenta um resultado menos conservador em relação ao caso anterior.
E assim por diante, com a inclusão da resposta `indiferente'', que
representará o caso de favorabilidade onde não há discordância de alguma forma,
etc.

Para as conclusões apresentadas neste trabalho será considerado
que percepções com algum nível de concordância são favoráveis, isto é,
o participante respondeu ``concordo'' ou ``concordo parcialmente''.

\begin{table}[h]
\caption{Referências para os gráficos}
\begin{tabular}{c|p{14.7cm}}
\hline
A & A construção da solução envolveu a elaboração, análise e exposição clara e racional de argumentos.\\
\hline
B & Foram identificadas ideias alternativas para uma mesma situação (ou seja, partes ou etapas individuais do problema).\\
\hline
C & Foram utilizadas, ao menos, uma das referências bibliográficas indicadas no texto do problema para estudo extraclasse.\\
\hline
D & Foi necessário recorrer a livros, vídeos, apostilas ou outros recursos não indicados no problema para chegar à solução.\\
\hline
E & O problema lhe deixou motivado para descobrir uma possível solução.\\
\hline
F & Você acredita que cumpriu com os objetivos de aprendizagem do problema.\\
\hline
G & Você teve que aprender novos conhecimentos (conceitos, habilidades ou atitudes) para chegar à solução do problema.\\
\hline
H & As situações abordadas pelo problema se aproximam de um cenário real e atual.\\
\hline
I & As situações abordadas pelo problema se aproximam de um cenário real e atual.\\
\hline
J & O texto do problema estava claro e bem escrito.\\
\hline
K & Havia no texto informações suficientes para direcionar a investigação.\\
\hline
L & O tempo disponibilizado para o desenvolvimento da solução foi adequado.\\
\hline
M & O problema estimulou o trabalho em grupo.\\
\hline
N & Você utilizou conhecimentos prévios (de um problema anterior ou mesmo de outro componente curricular) para chegar à solução do problema.\\
\hline
O & O problema exigiu o estudo individual de seus conteúdos fora das sessões tutoriais.\\
\hline
P & O método PBL me motiva para ir em busca do meu próprio conhecimento.\\
\hline
Q & As sessões tutoriais contribuem para o processo de resolução do problema.\\
\hline
R & Os participantes geralmente apresentam uma relação interpessoal boa e produtiva.\\
\hline
S & A quantidade de pessoas em cada Grupo Tutorial é apropriada.\\
\hline
T & Os problemas são úteis no processo de ensino-aprendizagem.\\
\hline
U & Os tutores contribuem, quando necessário, para a evolução das sessões tutoriais.\\
\hline
V & Os tutores deixam claros os critérios de avaliação do produto.\\
\hline
W & Os tutores dão feedbacks sobre o desempenho do grupo tutorial a cada sessão.\\
\hline
X & Eu gosto do método PBL.\\
\end{tabular}
\end{table}



\section{Semestre 2015.1}
\label{sec-sem-2015}
\section{Semestre 2016.1}
\label{sec-sem-2016}
\subsection{Problema 1}
Na aplicação do Problema 1 no semestre 2016.1 foram 14 participantes.

A Figura~\ref{percep-s1p1} apresenta gráfico com os resultados referente
as percepções dos participantes na aplicação do
Problema 1 no semestre 2016.1.
\begin{figure}[!htb]
\centering
\includegraphics[scale=0.22]{question-s1p1.eps}
\caption{Percepções dos participantes do semestre 2016.1 sobre o Problema 1}
\label{percep-s1p1}
\end{figure}

É possível destacar que para todas as questões a favorabilidade
foi superior aos $70\%$. A favorabilidade foi superior aos $90\%$
para 12 afirmações apresentadas.
Apenas 6 afirmações receberam ao menos uma discordância integral.

A afirmação sobre ideias alternativas no caminho do problema (B)
foi a que teve a maior percepção de concordância por partes dos
participantes.
Entendemos que este resultado é explicado pelo problema
ter apresentado os conceitos de músicas, de forma que
os participantes precisaram realizar relacionamentos entre
os conceitos deste tema com os conceitos de linguagens
formais, e que os participantes perceberam que a depender do
nível de abstração poderá existir um caminho diferente.

A necessidade de recorrer aos materiais de apoio (D),
existência de informações suficientes no texto (K) e
sobre o participante gostar da metodologia baseada
em problemas (X) foram as afirmações com
as menores favorabilidades.
Embora possa parecer existir alguma contradição
sobre os resultados das duas primeiras, vale
ressaltar que os dados referente
a contribuição da sessão tutorial (Q) obtiveram
alta favorabilidade.

A Figura~\ref{aval-s1p1} apresenta o gráfico da
avaliação do participantes para o Problema 1 aplicado no semestre 2016.1.

\begin{figure}[!htb]
\centering
\includegraphics[scale=0.18]{notas-s1p1.eps}
\caption{Avaliação dos participantes do semestre 2016.1 para o Problema 1}
\label{aval-s1p1}
\end{figure}

Como é possível observar no gráfico, a maioria dos participantes atribuíram
notas altas para o Problema 1 no semestre 2016.1, assim, quase $80\%$ das notas
foram maiores ou iguais a 7 e nenhuma nota foi menor que 5, com uma média
de $7,29$.
  
\subsection{Problema 2}
\subsection{Problema 3}
\subsection{Problema 4}
\subsection{Problema 5}
\section{Semestre 2017.1}
\label{sec-sem-2017}
\subsection{Problema 1}
\subsection{Problema 2}
\subsection{Problema 3}
\subsection{Problema 4}
\subsection{Problema 5}
\section{Considerações finais}
\label{sec-consideracoes-resultados}
