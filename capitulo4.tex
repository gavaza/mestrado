\xchapter{Resultados}{} %sem preambulo
% É recomendável utilizar `\acresetall' no início de cada capítulo para reiníciar o contator de referências às siglas.
\acresetall
\section{Formulário de percepções de problema}
Para cada problema os estudantes foram convidados a responder um formulário de percepções de problema.
Neste formulário foram apresentadas 32 questões aos participantes.
As cinco primeiras questões neste formulário foram de caracterização de perfil, como
idade e sexo.
As próximas vinte e cinco questões foram sobre as percepções dos participantes sobre a abordagem, o
problema, o tutor e auto avaliação.

Cada pergunta apresentou uma afirmação sobre o tema ao qual se desejava obter informações sobre as
percepções dos participantes.
A resposta do participante é obtida em uma escala Likert com cinco itens, onde para efeitos dos estudos
apresentados neste trabalho foram consideradas desfavoráveis
as respostas ``discordo plenamente'', ``discordo'' e ``indiferente'', enquanto as
respostas ``concordo'' e ``concordo plenamente'' foram consideradas favoráveis.

O participante também teve disponível um espaço aberto para incluir considareções adicionais sobre
as suas percepções sobre o problema especificamente e sobre a utilização da abordagem.

\section{Formulário de percepções da disciplina}
Para obter as percepções da disciplina, os participantes foram caracterizados em \textit{desistente}
se por algum motivo desistiu da disciplina antes da conclusão, sendo assim, foi reprovado por
falta ou realizou trancamento, e \textit{concluinte} que concluiu a disciplina independente
de aprovação ou reprovação por conceito.

Aos participantes foi apresentado um formulário de acordo com a sua condição de desistente
ou concluinte.
Em ambos os casos o formulário apresentou 24 questões aos participantes.
O foco do formulário apresentado ao participante desistente é identificar indicações
sobre o motivo da desistência, enquanto para o participante concluinte o foco do
formulário apresentado está em identificar benefícios da abordagem na percepção
do participante.
As cinco primeiras questões neste formulário foram de caracterização de perfil, como
idade e sexo.
As próximas dezessete questões foram sobre as percepções dos participantes sobre referente
ao foco do formulário respondido.

Cada pergunta apresentou uma afirmação sobre o tema ao qual se desejava obter informações sobre as
percepções dos participantes.
A resposta do participante é obtida em uma escala Likert com cinco itens, onde para efeitos dos estudos
apresentados neste trabalho foram consideradas desfavoráveis
as respostas ``discordo plenamente'', ``discordo'' e ``indiferente'', enquanto as
respostas ``concordo'' e ``concordo plenamente'' foram consideradas favoráveis.

O participante também teve disponível um espaço aberto para incluir considareções adicionais sobre
as suas percepções sobre o problema especificamente e sobre a utilização da abordagem.

\section{Semestre 2015.1}
\section{Semestre 2016.1}
\subsection{Problema 1}
\subsection{Problema 2}
\subsection{Problema 3}
\subsection{Problema 4}
\subsection{Problema 5}
\section{Semestre 2017.1}
\subsection{Problema 1}
\subsection{Problema 2}
\subsection{Problema 3}
\subsection{Problema 4}
\subsection{Problema 5}
\section{Considerações finais}
