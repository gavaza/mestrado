\xchapter{Resultados}{} %sem preambulo
% É recomendável utilizar `\acresetall' no início de cada capítulo para reiníciar o contator de referências às siglas.
\acresetall
\section{Formulário de percepções de problema}
Para cada problema os estudantes foram convidados a responder um formulário de percepções de problema.
Neste formulário foram apresentadas 32 questões aos participantes.
As cinco primeiras questões neste formulário foram de caracterização de perfil, como
idade e sexo.
As próximas vinte e cinco questões foram sobre as percepções dos participantes sobre a abordagem, o
problema, o tutor e auto avaliação.

Cada pergunta apresentou uma afirmação sobre o tema ao qual se desejava obter informações sobre as
percepções dos participantes.
A resposta do participante é obtida em uma escala Likert com cinco itens, onde para efeitos dos estudos
apresentados neste trabalho foram consideradas desfavoráveis
as respostas ``discordo plenamente'', ``discordo'' e ``indiferente'', enquanto as
respostas ``concordo'' e ``concordo plenamente'' foram consideradas favoráveis.

O participante também teve disponível um espaço aberto para incluir considareções adicionais sobre
as suas percepções sobre o problema especificamente e sobre a utilização da abordagem.

\section{Formulário de percepções da disciplina}
Para obter as percepções da disciplina, os participantes foram caracterizados em \textit{desistente}
se por algum motivo desistiu da disciplina antes da conclusão, sendo assim, foi reprovado por
falta ou realizou trancamento, e \textit{concluinte} que concluiu a disciplina independente
de aprovação ou reprovação por conceito.

Aos participantes foi apresentado um formulário de acordo com a sua condição de desistente
ou concluinte.
Em ambos os casos o formulário apresentou 24 questões aos participantes.
O foco do formulário apresentado ao participante desistente é identificar indicações
sobre o motivo da desistência, enquanto para o participante concluinte o foco do
formulário apresentado está em identificar benefícios da abordagem na percepção
do participante.
As cinco primeiras questões neste formulário foram de caracterização de perfil, como
idade e sexo.
As próximas dezessete questões foram sobre as percepções dos participantes sobre referente
ao foco do formulário respondido.

Cada pergunta apresentou uma afirmação sobre o tema ao qual se desejava obter informações sobre as
percepções dos participantes.
A resposta do participante é obtida em uma escala Likert com cinco itens, onde para efeitos dos estudos
apresentados neste trabalho foram consideradas desfavoráveis
as respostas ``discordo plenamente'', ``discordo'' e ``indiferente'', enquanto as
respostas ``concordo'' e ``concordo plenamente'' foram consideradas favoráveis.

O participante também teve disponível um espaço aberto para incluir considareções adicionais sobre
as suas percepções sobre o problema especificamente e sobre a utilização da abordagem.

\section{Referências para os gráficos}
\begin{table}[h]
\caption{Referências para os gráficos}
\label{tabela-ref-graficos}
\begin{tabular}{c|p{14.6cm}}
Legenda & Pergunta respondida pelo participante \\
\hline
A & A construção da solução envolveu a elaboração, análise e exposição clara e racional de argumentos.\\
\hline
B & Foram identificadas ideias alternativas para uma mesma situação (ou seja, partes ou etapas individuais do problema).\\
\hline
C & Foram utilizadas, ao menos, uma das referências bibliográficas indicadas no texto do problema para estudo extraclasse.\\
\hline
D & Foi necessário recorrer a livros, vídeos, apostilas ou outros recursos não indicados no problema para chegar à solução.\\
\hline
E & O problema lhe deixou motivado para descobrir uma possível solução.\\
\hline
F & Você acredita que cumpriu com os objetivos de aprendizagem do problema.\\
\hline
G & Você teve que aprender novos conhecimentos (conceitos, habilidades ou atitudes) para chegar à solução do problema.\\
\hline
H & As situações abordadas pelo problema se aproximam de um cenário real e atual.\\
\hline
I & O conhecimento aprendido é útil para um profissional da área de Computação.\\
\hline
J & O texto do problema estava claro e bem escrito.\\
\hline
K & Havia no texto informações suficientes para direcionar a investigação.\\
\hline
L & O tempo disponibilizado para o desenvolvimento da solução foi adequado.\\
\hline
M & O problema estimulou o trabalho em grupo.\\
\hline
N & Você utilizou conhecimentos prévios (de um problema anterior ou mesmo de outro componente curricular) para chegar à solução do problema.\\
\hline
O & O problema exigiu o estudo individual de seus conteúdos fora das sessões tutoriais.\\
\hline
P & O método PBL me motiva para ir em busca do meu próprio conhecimento.\\
\hline
Q & As sessões tutoriais contribuem para o processo de resolução do problema.\\
\hline
R & Os participantes geralmente apresentam uma relação interpessoal boa e produtiva.\\
\hline
S & A quantidade de pessoas em cada Grupo Tutorial é apropriada.\\
\hline
T & Os problemas são úteis no processo de ensino-aprendizagem.\\
\hline
U & Os tutores contribuem, quando necessário, para a evolução das sessões tutoriais.\\
\hline
V & Os tutores deixam claros os critérios de avaliação do produto.\\
\hline
W & Os tutores dão feedbacks sobre o desempenho do grupo tutorial a cada sessão.\\
\hline
X & Eu gosto do método PBL.\\
\end{tabular}
\end{table}


\section{Semestre 2015.1}
\section{Semestre 2016.1}
\subsection{Problema 1}
\begin{figure}[!htb]
\centering
\includegraphics[scale=0.22]{question-s1p1.eps}
\caption{Percepções dos participantes do semestre 2016.1 sobre o Problema 1}
\label{Rotulo}
\end{figure}
\begin{figure}[!htb]
\centering
\includegraphics[scale=0.18]{notas-s1p1.eps}
\caption{Avaliação dos participantes do semestre 2016.1 para o Problema 1}
\label{Rotulo}
\end{figure}
\subsection{Problema 2}
\subsection{Problema 3}
\subsection{Problema 4}
\subsection{Problema 5}
\section{Semestre 2017.1}
\subsection{Problema 1}
\subsection{Problema 2}
\subsection{Problema 3}
\subsection{Problema 4}
\subsection{Problema 5}
\section{Considerações finais}
