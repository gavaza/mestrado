% hipóteses gerais
\newcommand{\hgatexto}{A metodologia é bem avalidada pelos estudantes nesse contexto}
\newcommand{\hgbtexto}{Em disciplinas teóricas, como é o caso da disciplina de
Teoria da Computação, os problemas são capazes de cumprir com os objetivos de
aprendizagem na percepção dos estudantes}
\newcommand{\hgctexto}{Os estudantes se sentem motivados com
a utilização da metodologia}

% hipóteses específicas
\newcommand{\heatexto}{Na percepção dos estudantes as sessões tutoriais contribuem
no processo de solução dos problemas}
\newcommand{\hebtexto}{Na percepção dos estudantes os problemas construídos
são capazes de motivar o trabalho em grupo}
\newcommand{\hectexto}{Na percepção dos estudantes os problemas são úteis no
processo de ensino e aprendizagem}
\newcommand{\hedtexto}{Na percepção dos estudantes a metodologia estimula
a autoaprendizagem}
\newcommand{\heetexto}{Na percepção dos estudantes os tutores contribuem positivamente}
\newcommand{\heftexto}{Na percepção dos estudantes os problemas apresentam conhecimentos úteis para um profissional
da área de Computação}
\newcommand{\hegtexto}{Os estudantes percebem conexões entre o conhecimento e o problema}
