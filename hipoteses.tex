% hipóteses
\newcommand{\hatexto}{A abordagem é bem avaliada pelos estudantes}
\newcommand{\hbtexto}{Em disciplinas teóricas, como é o caso da disciplina de
Teoria da Computação, os problemas são capazes de cumprir com os objetivos de
aprendizagem na percepção dos estudantes}
\newcommand{\hctexto}{Os estudantes se sentem motivados com
a utilização da abordagem}

% objetivos gerais
\newcommand{\ogatexto}{Discutir a percepção dos estudantes sobre aprendizagem}
\newcommand{\ogbtexto}{Discutir sobre a motivação dos estudantes}
\newcommand{\ogctexto}{Analisar a avaliação dos estudantes para os problemas}

% objetivos específicos
\newcommand{\oeatexto}{Avaliar se as sessões tutoriais contribuem
no processo de solução dos problemas}
\newcommand{\oebtexto}{Avaliar se os problemas construídos
são capazes de motivar o trabalho em grupo}
\newcommand{\oectexto}{Avaliar utilidade dos problemas no
processo de ensino e aprendizagem}
\newcommand{\oedtexto}{Avaliar estímulo a autoaprendizagem pela abordagem}
\newcommand{\oeetexto}{Avaliar se os tutores contribuem positivamente}
\newcommand{\oeftexto}{Avaliar utilidade dos conhecimentos nos problemas para um profissional
da área de Computação}
\newcommand{\oegtexto}{Avaliar percepção de conexões entre o conhecimento e o problema}
