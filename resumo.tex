\resumo
A disciplina de Teoria da Computação é uma das bases de
sustentação da área de computação, por este motivo, de
fundamental importância na formação dos estudantes
da área.
Seu conteúdo possui um alto nível de abstração e exige
um grande esforço dos estudantes e dos educadores
no processo de ensino e aprendizagem.
A percepção por parte de estudantes de
que não estão compreendendo muito bem os conceitos e
a desmotivação destes são os principais desafios nesse processo.
Por sua vez o uso de metodologias ativas de aprendizagem, como
a abordagem \ac{PBL} tem sido apresentada
como uma alternativa interessante para reduzir problemas relativos à motivação
e aprendizagem dos estudantes.
Nesse sentido o objetivo dessa dissertação é descrever a implementação
de \ac{PBL} em turmas de Teoria da Computação no curso de Sistemas
de Informação na \ac{UFBA}.
Para avaliação dos resultados e validação das hipóteses,
esse trabalho realizou uma pesquisa descritiva na forma
de pesquisa de opinião, onde
os estudantes foram convidados a responder questionários com
caracterização de perfil, afirmativas em escala Likert e
espaço aberto relativos à abordagem adotada. 
Os resultados deste estudo sugerem que a utilização de \ac{PBL}
é promissora como alternativa para disciplinas teóricas
de Computação, como na disciplina na qual este estudo foi
aplicado, sobretudo no que diz respeito a percepção
dos estudantes sobre motivação e aprendizagem.
% Palavras-chave do resumo em Portugues
\begin{keywords}
\ac{PBL}, Teoria da Computação, abordagem pedagógica, metodologia pedagógica.
\end{keywords}
