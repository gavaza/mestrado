\resumo
A disciplina de Teoria da Computação é uma das bases de
sustentação da área de computação, por este motivo, de
fundamental importância na formação dos estudantes
da área.
A disciplina de Teoria da Computação possui
um alto nível de abstração e exige um grande esforço
dos estudantes e dos educadores no processo de
ensino e aprendizagem.
As disciplinas que possuem necessidade mais elevada de dedicação
por parte dos estudantes ou conceitos com alto de nível de abstração
são um desafio para os estudantes e também para os educadores.
A percepção por parte de estudantes de
que não estão compreendendo muito bem os conceitos e
desmotivação destes são os principais desafios para o processo
de ensino e aprendizagem.
A capacidade de construção de conhecimento é uma
questão importante na formação dos estudantes, por este
motivo, é igualmente importante tentar superar as
dificuldades.
Este trabalho realizou uma pesquisa descritiva na forma
de pesquisa de opinião dos estudantes em dois semestres
na Universidade Federal da Bahia utilizando a
metodologia PBL em uma disciplina de Teoria da
Computação.
Os estudantes foram convidados a responder questionários com
caracterização de perfil, afirmativas em escala Likert e
espaço aberto.
O estudo utilizou critérios bem conservadores para qualificação
de uma avaliação como favorável e para avaliação da validação das
hipóteses.
Os excelentes resultados, deste estudo, dentro do contexto
em que foram realizadas as avaliações, indicaram que a
metodologia PBL também poderá ser utilizada como ferramenta
no ensino e aprendizagem de disciplinas teóricas
de Computação.

% Palavras-chave do resumo em Portugues
\begin{keywords}
PBL, Teoria da Computação, metodologia.
\end{keywords}
