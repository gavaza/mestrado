\resumo

As disciplinas que possuem necessidade mais elevada de dedicação
por parte dos estudantes ou conceitos com alto de nível de abstração
são um desafio para os estudantes e também para os educadores
que convivem com a desmotivação e a percepção por parte dos
estudantes de que estes não estão compreendendo muito
bem os conceitos.
A satisfação e capacidade de construção de conhecimento
são questões importantes para uma melhor formação dos estudantes.
Este trabalho realizou uma pesquisa descritiva na forma
de pesquisa de opinião dos estudantes em dois semestres
na Universidade Federal da Bahia utilizando a
metodologia PBL em uma disciplina de Teoria da
Computação.
As opiniões objetivas dos estudantes foram utilizadas para
avaliação e discussão de hipóteses gerais e específicas e
as opiniões discursivas dos estudantes foram utilizadas para
uma discussão qualitativa.
No estudo foram utilizados critérios bem
conservador para qualificação de uma avaliação
como favorável e para avaliação da validação das
hipóteses.
O estudo obteve excelentes resultados, dentro do contexto
em que foi realizada a avaliação, indicando que a
metodologia PBL poderá ser utilizada como ferramenta
no ensino e aprendizagem de disciplinas teóricas
de Computação.

% Palavras-chave do resumo em Portugues
\begin{keywords}
PBL, Teoria da Computação, metodologia.
\end{keywords}
